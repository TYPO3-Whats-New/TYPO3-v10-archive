% ------------------------------------------------------------------------------
% TYPO3 Version 10 LTS - What's New (English Version)
%
% @author	Michael Schams <schams.net>
% @license	Creative Commons BY-NC-SA 3.0
% @link		https://typo3.org/help/documentation/whats-new/
% @language	English
% ------------------------------------------------------------------------------

\section{Verouderde/verwijderde functies}
\begin{frame}[fragile]
	\frametitle{Verouderde/verwijderde functies}

	\begin{center}\huge{\color{typo3darkgrey}\textbf{Verouderde/verwijderde functies}}\end{center}
	\begin{center}\large{\textit{De TYPO3 core opgeruimd en klaar voor de toekomst}}\end{center}

\end{frame}

% ------------------------------------------------------------------------------
% ==============================================================================
% TYPO3 Version 10.1
% ==============================================================================
% ------------------------------------------------------------------------------
% Deprecation | 88854 | T3_THIS_LOCATION
% Deprecation | 88862 | T3_RETURN_URL
% Deprecation | 89033 | jumpToUrl
% Deprecation | 88854 | jumpExt() of RecordListController

\begin{frame}[fragile]
	\frametitle{Verouderde/verwijderde functies}
	\framesubtitle{Verouderd JavaScript}

	\begin{itemize}
		\item Twee globale JavaScript-variabelen zjin als verouderd gemarkeerd.

			\begin{itemize}
				\item \texttt{T3\_THIS\_LOCATION}
				\item \texttt{T3\_RETURN\_URL}
			\end{itemize}

		\item De bekende JavaScript functie \texttt{jumpToUrl()} is als \textbf{verouderd} gemarkeerd.
			Migratiemogelijkheden:

			\begin{itemize}
				\item gebruik \texttt{window.location.href = '...';}
				\item of gebruik een link in HTML zoals \texttt{<a href="...">link</a>}
			\end{itemize}

		\item De JavaScript functie \texttt{jumpExt()} is als \textbf{verouderd} gemarkeerd.

	\end{itemize}

\end{frame}

% ------------------------------------------------------------------------------
% Deprecation | 89215 | jQuery.clearable

\begin{frame}[fragile]
	\frametitle{Verouderde/verwijderde functies}
	\framesubtitle{Verouderd JavaScript}

	% decrease font size for code listing
	\lstset{basicstyle=\tiny\ttfamily}

	\begin{itemize}
		\item De jQuery plug-in \texttt{jquery.clearable},
			die een knop biedt om een invoerveld te legen is als \textbf{verouderd} gemarkeerd.
		\item Migratie: gebruik module \small\texttt{TYPO3/CMS/Backend/Input/Clearable}\normalsize
			en de functie \texttt{clearable()} op een gewoon HTMLInputElement.
\begin{lstlisting}
require(['TYPO3/CMS/Backend/Input/Clearable'], function() {
  const inputField = document.querySelector('#myinput');
  if (inputField !== null) {
    inputField.clearable();
  }

  const clear = Array.from(document.querySelectorAll('.t3js-clearable')).filter(inputElement => {
    return !inputElement.classList.contains('t3js-datetimepicker');
  });
  clear.forEach(clearableField => clearableField.clearable());
});
\end{lstlisting}

	\end{itemize}

\end{frame}

% ------------------------------------------------------------------------------
% Deprecation | 88839 | CLI lowlevel request handlers

\begin{frame}[fragile]
	\frametitle{Verouderde/verwijderde functies}
	\framesubtitle{CLI Commando Handler}

	\begin{itemize}
		\item CLI commando's worden afgehandeld door de \texttt{CommandApplication} klasse.
		\item Deze klasse is een omhulsel rond de
			\href{https://symfony.com/doc/current/components/console.html}{Symfony Console}.

		\item Het hiervoor gebruikte koppelvlak en de klasse \texttt{CommandRequestHandler} zijn als \textbf{verouderd} aangemerkt:

			\begin{itemize}
				\item
					\texttt{TYPO3\textbackslash
						CMS\textbackslash
						Core\textbackslash
						Console\textbackslash
						RequestHandlerInterface}
				\item
					\texttt{TYPO3\textbackslash
						CMS\textbackslash
						Core\textbackslash
						Console\textbackslash
						CommandRequestHandler}
			\end{itemize}

	\end{itemize}

\end{frame}

% ------------------------------------------------------------------------------
% Deprecation | 89127 | Cleanup RecordHistory handling

\begin{frame}[fragile]
	\frametitle{Verouderde/verwijderde functies}
	\framesubtitle{Afhandeling historie van records}

	Veranderingen in de klasse
		\smaller
			\texttt{TYPO3\textbackslash
				CMS\textbackslash
				Backend\textbackslash
				History\textbackslash
				RecordHistory}:
		\normalsize

	\begin{itemize}

		\item Zichtbaarheid van eigenschappen \texttt{changeLog} en \texttt{lastHistoryEntry}
			gewijzigd in \texttt{protected} (en publieke getters toegevoegd).
		\item Zichtbaarheid van functies \texttt{getHistoryEntry()} en \texttt{getHistoryData()}
			gewijzigd in \texttt{protected}.
		\item De volgende functies zijn aangemerkt als \textbf{verouderd}:

			\begin{itemize}\smaller
				\item \texttt{createChangeLog()}
				\item \texttt{shouldPerformRollback()}
				\item \texttt{getElementData()}
				\item \texttt{performRollback()}
				\item \texttt{createMultipleDiff()}
				\item \texttt{setLastHistoryEntry()}
			\end{itemize}\normalsize

	\end{itemize}

\end{frame}

% ------------------------------------------------------------------------------
% Deprecation | 89037 | Deprecated LocallangXmlParser

\begin{frame}[fragile]
	\frametitle{Verouderde/verwijderde functies}
	\framesubtitle{XML Taalbestanden}

	\begin{itemize}
		\item Het XLIFF-formaat wordt gebruikt voor taalbestanden sinds TYPO3 v4.6.
		\item Het gebruik van XML-taalbestanden is nu aangemerkt als \textbf{verouderd}
			en geeft een waarschuwing/fout.
		\item Hieronder valt het uitvoeren van de volgende XML-parser:\newline
			\small
				\texttt{TYPO3\textbackslash
					CMS\textbackslash
					Core\textbackslash
					Localization\textbackslash
					Parser\textbackslash
					LocallangXmlParser}
			\normalsize
	\end{itemize}

\end{frame}

% ------------------------------------------------------------------------------
% ==============================================================================
% TYPO3 Version 10.2
% ==============================================================================
% ------------------------------------------------------------------------------
% Deprecation | 89331 | FormEngine legacy functions

\begin{frame}[fragile]
	\frametitle{Verouderde/verwijderde functies}
	\framesubtitle{FormEngine}

	% decrease font size for code listing
	\lstset{basicstyle=\tiny\ttfamily}

	\begin{itemize}
		\item De volgende functies van de FormEngine zijn als \textbf{verouderd} aangemerkt:

			\begin{itemize}
				\item \texttt{setFormValueOpenBrowser()}\newline
					\smaller(gebruik \texttt{FormEngine.openPopupWindow()} hiervoor)\small

				\item \texttt{setFormValueFromBrowseWin()}\newline
					\smaller(gebruik \texttt{FormEngine.setSelectOptionFromExternalSource()} hiervoor)\small

				\item \texttt{setHiddenFromList()}\newline
					\smaller(gebruik \texttt{FormEngine.updateHiddenFieldValueFromSelect()} hiervoor)\small

				\item \texttt{setFormValueManipulate()}\newline
					\smaller(geen vervanging, dit is interne logica)\small

				\item \texttt{setFormValue\_getFObj()}\newline
					\smaller(gebruik \texttt{FormEngine.getFormElement()} hiervoor)\small

			\end{itemize}

	\end{itemize}

\end{frame}

% ------------------------------------------------------------------------------
% Deprecation | 89733 | Signal Slots in Core Extension migrated to PSR-14 events

\begin{frame}[fragile]
	\frametitle{Verouderde/verwijderde functies}
	\framesubtitle{Signal/Slot}

	% decrease font size for code listing
	\lstset{basicstyle=\tiny\ttfamily}

	\begin{itemize}
		\item De volgende Signal/Slots zijn vervangen door PSR-14 events en daarom als \textbf{verouderd} aangemerkt:
			\newline

			\begin{itemize}\tiny
				\item \texttt{TYPO3\textbackslash
					CMS\textbackslash
					Core\textbackslash
					Imaging\textbackslash
					IconFactory::buildIconForResourceSignal}
					\newline
				\item \texttt{TYPO3\textbackslash
					CMS\textbackslash
					Core\textbackslash
					Database\textbackslash
					SoftReferenceIndex::setTypoLinkPartsElement}
					\newline
				\item \texttt{TYPO3\textbackslash
					CMS\textbackslash
					Core\textbackslash
					Database\textbackslash
					ReferenceIndex::shouldExcludeTableFromReferenceIndex}
					\newline
				\item \texttt{TYPO3\textbackslash
					CMS\textbackslash
					Core\textbackslash
					Utility\textbackslash
					ExtensionManagementUtility::tcaIsBeingBuilt}
					\newline
				\item \texttt{TYPO3\textbackslash
					CMS\textbackslash
					Install\textbackslash
					Service\textbackslash
					SqlExpectedSchemaService::tablesDefinitionIsBeingBuilt}
					\newline
				\item \texttt{TYPO3\textbackslash
					CMS\textbackslash
					Core\textbackslash
					Tree\textbackslash
					TableConfiguration\textbackslash
					DatabaseTreeDataProvider::PostProcessTreeData}
					\newline
				\item \texttt{TYPO3\textbackslash
					CMS\textbackslash
					Backend\textbackslash
					Backend\textbackslash
					ToolbarItems\textbackslash
					SystemInformationToolbarItem::getSystemInformation}
					\newline
				\item \texttt{TYPO3\textbackslash
					CMS\textbackslash
					Backend\textbackslash
					Backend\textbackslash
					ToolbarItems\textbackslash
					SystemInformationToolbarItem::loadMessages}

			\end{itemize}

	\end{itemize}

\end{frame}

% ------------------------------------------------------------------------------
% Deprecation | 89631 | Use Environment API to fetch application context

\begin{frame}[fragile]
	\frametitle{Verouderde/verwijderde functies}
	\framesubtitle{Applicatiecontext}

	% decrease font size for code listing
	\lstset{basicstyle=\tiny\ttfamily}

	\begin{itemize}
		\item De functie \texttt{GeneralUtility::getApplicationContext()} is als \textbf{verouderd} aangemerkt.
		\item De volgend functie is de vervanging:\newline
		 	\texttt{TYPO3\textbackslash
				CMS\textbackslash
				Core\textbackslash
				Core\textbackslash
				Environment::getContext()}.

	\end{itemize}

\end{frame}

% ------------------------------------------------------------------------------
% ==============================================================================
% TYPO3 Version 10.3
% ==============================================================================
% ------------------------------------------------------------------------------
% Deprecation | 89463 | Deprecate switchable controller actions

\begin{frame}[fragile]
	\frametitle{Verouderde/verwijderde functies}
	\framesubtitle{Switchable Controller Actions}

	\begin{itemize}
		\item "Switchable Controller Actions" (SCA) zijn als \textbf{verouderd} aangemerkt.
		\item SCA kunnen de toegestane set van controllers en actions overschrijven via TypoScript of Flexforms.
		\item Dezelfde plug-in gebruiken voor veel verschillende functionaliteiten gaat in tegen het idee dat een plug-in een bepaald doel heeft.
		\item Plug-ins die SCA gebruiken moeten opgedeeld worden in meerdere plug-ins.
	\end{itemize}

\end{frame}

% ------------------------------------------------------------------------------
% Deprecation | 90007 | Global constants TYPO3_version and TYPO3_branch

\begin{frame}[fragile]
	\frametitle{Verouderde/verwijderde functies}
	\framesubtitle{Globale constanten}

	% decrease font size for code listing
	\lstset{basicstyle=\smaller\ttfamily}

	\begin{itemize}
		\item De volgende twee globale constanten zijn als \textbf{verouderd} aangemerkt:

			\begin{itemize}
				\item \texttt{TYPO3\_version}
				\item \texttt{TYPO3\_branch}
			\end{itemize}

		\item De volgende nieuwe PHP klasse kan hiervoor gebruikt worden:\newline
			\small
				\texttt{TYPO3\textbackslash
					CMS\textbackslash
					Core\textbackslash
					Information\textbackslash
					Typo3Version}\normalsize

	\end{itemize}

\end{frame}

% ------------------------------------------------------------------------------
% Deprecation | 89673 | Deprecate Extbases WebRequest and WebResponse

\begin{frame}[fragile]
	\frametitle{Verouderde/verwijderde functies}
	\framesubtitle{Extbase: \texttt{WebRequest}/\texttt{WebResponse}}

	\begin{itemize}
		\item De volgende twee Extbase klassen zijn als \textbf{verouderd} aangemerkt:
			\begin{itemize}
				\item \texttt{TYPO3\textbackslash
					CMS\textbackslash
					Extbase\textbackslash
					Mvc\textbackslash
					Web\textbackslash
					Request}
				\item \texttt{TYPO3\textbackslash
					CMS\textbackslash
					Extbase\textbackslash
					Mvc\textbackslash
					Web\textbackslash
					Response}
			\end{itemize}

	\end{itemize}

\end{frame}

% ------------------------------------------------------------------------------
% Deprecation | 90258 | Simplified RTE Parser API

\begin{frame}[fragile]
	\frametitle{Verouderde/verwijderde functies}
	\framesubtitle{Versimpelde RTE Parser API}

	\begin{itemize}
		\item De PHP klasse \texttt{RteHtmlParser} heeft een versimpelde API.
		\item Een gevolg is dat deze twee methodes als \textbf{verouderd} zijn aangemerkt:

			\begin{itemize}
				\item \texttt{TYPO3\textbackslash
					CMS\textbackslash
					Core\textbackslash
					Html\textbackslash
					RteHtmlParser->init()}
				\item \texttt{TYPO3\textbackslash
					CMS\textbackslash
					Core\textbackslash
					Html\textbackslash
					RteHtmlParser->RTE\_transform()}
			\end{itemize}

	\end{itemize}

\end{frame}

% ------------------------------------------------------------------------------
% Deprecation | 89139 | Console Commands configuration migrated to Symfony service tags

\begin{frame}[fragile]
	\frametitle{Verouderde/verwijderde functies}
	\framesubtitle{Configuratie van console commando's}

	\begin{itemize}
		\item Aangezien de configuratie van console commando's omgezet is naar Symfony service tags,
			is het bestand met de configuratie \texttt{Configuration/Commands.php} als
			\textbf{verouderd} aangemerkt.
		\item Gebruik de service tag voor het injecteren van afhankelijkheden \texttt{console.command} hiervoor.

	\end{itemize}

\end{frame}

% ------------------------------------------------------------------------------
% Important | 89672 | transOrigPointerField is not longer allowed to be excluded

\begin{frame}[fragile]
	\frametitle{Verouderde/verwijderde functies}
	\framesubtitle{TCA: \texttt{transOrigPointerField}}

	\begin{itemize}
		\item Het uitzonderen van deze TCA-optie zorgde voor inconsistente data in de databse onder bepaalde omstandigheden:
			\small
				\texttt{\$GLOBALS['TCA'][\$table]['ctrl']['transOrigPointerField']}
			\normalsize

		\item Daarom kan de optie niet meer uitgezonderd worden.
		\item Een migratie-assistent verwijdert de optie uit de TCA en voegt een verouderingsmelding
			toe aan de verouderingslog als code bijgewerkt moet worden.
	\end{itemize}

\end{frame}

% ------------------------------------------------------------------------------
% Deprecation | 90421 | DocumentTemplate

\begin{frame}[fragile]
	\frametitle{Verouderde/verwijderde functies}
	\framesubtitle{DocumentTemplate}

	% decrease font size for code listing
	\lstset{basicstyle=\tiny\ttfamily}

	\begin{itemize}
		\item De volgende klasse is als \textbf{verouderd} aangemerkt:

			\begin{itemize}
				\item \texttt{TYPO3\textbackslash
					CMS\textbackslash
					Backend\textbackslash
					Template\textbackslash
					DocumentTemplate}
			\end{itemize}

		\item Het werd gebruikt als basis voor de weergave van backend modules of HTML-uitvoer in de TYPO3 backend.
		\item Sinds TYPO3 v7 moet de nieuwe API via ModuleTemplate gebruikt worden hiervoor.
\begin{lstlisting}
use TYPO3\CMS\Backend\Template\ModuleTemplate;
...
$moduleTemplate = GeneralUtility::makeInstance(ModuleTemplate::class);
$content = $this->getHtmlContentFromMyModule();
$moduleTemplate->setTitle('My module');
$moduleTemplate->setContent($content);
return new HtmlResponse($moduleTemplate->renderContent());
\end{lstlisting}

	\end{itemize}

\end{frame}

% ------------------------------------------------------------------------------
% Deprecation | 90390 | Deprecate BrokenLinkRepository::getNumberOfBrokenLinks() in linkvalidator

\begin{frame}[fragile]
	\frametitle{Verouderde/verwijderde functies}
	\framesubtitle{LinkValidator}

	\begin{itemize}
		\item De volgende methode is als \textbf{verouderd} aangemerkt:
		\newline\newline
			\smaller
				\texttt{TYPO3\textbackslash
					CMS\textbackslash
					Linkvalidator\textbackslash
					Repository\textbackslash
					BrokenLinkRepository}\newline
				\texttt{->getNumberOfBrokenLinks()}\normalsize\newline

		\item Gebruik de volgende methode in dezelfde klasse hiervoor:\newline
			\small\texttt{BrokenLinkRepository::isLinkTargetBrokenLink()}\normalsize

	\end{itemize}

\end{frame}

% ------------------------------------------------------------------------------
% ==============================================================================
% TYPO3 Version 10.4
% ==============================================================================
% ------------------------------------------------------------------------------
% Deprecation | 91001 | Various methods within GeneralUtility
% Deprecation | 90956 | Alternative fetch methods and reports for GeneralUtility::getUrl()

\begin{frame}[fragile]
	\frametitle{Verouderde/verwijderde functies}
	\framesubtitle{GeneralUtility}

	\begin{itemize}
		\item De volgende functies uit GeneralUtility zijn als \textbf{verouderd} aangemerkt:
			\begin{itemize}\smaller
				\item \texttt{GeneralUtility::IPv6Hex2Bin()}
				\item \texttt{GeneralUtility::IPv6Bin2Hex()}
				\item \texttt{GeneralUtility::compressIPv6()}
				\item \texttt{GeneralUtility::milliseconds()}
				\item \texttt{GeneralUtility::linkThisUrl()}
				\item \texttt{GeneralUtility::flushDirectory()}
			\end{itemize}\normalsize
			\vspace{0.4cm}

		\item Extra argumenten, naast de URL, bij \texttt{GeneralUtility::getUrl()}
			is als \textbf{verouderd} aangemerkt.\newline
			\smaller
				(hieronder vallen: \texttt{\$includeHeader}, \texttt{\$requestHeaders}, en \texttt{\$report})
			\normalsize

	\end{itemize}

\end{frame}

% ------------------------------------------------------------------------------
% Deprecation | 90800 | GeneralUtility::isRunningOnCgiServerApi

\begin{frame}[fragile]
	\frametitle{Verouderde/verwijderde functies}
	\framesubtitle{GeneralUtility}

	\begin{itemize}
		\item De volgende functie is verwijderd uit de klasse \texttt{GeneralUtility}:
			\texttt{GeneralUtility::isRunningOnCgiServerApi()}.
		\item Deze functie is nu beschikbaar als\newline
			\texttt{Environment::isRunningOnCgiServer()}.

	\end{itemize}

\end{frame}

% ------------------------------------------------------------------------------
% Deprecation | 90964 | LanguageService functionality and internal properties

\begin{frame}[fragile]
	\frametitle{Verouderde/verwijderde functies}
	\framesubtitle{LanguageService functionaliteit}

	Wijzigingen in de LanguageService (ook bekend als \texttt{\$GLOBALS[LANG]}).
	\vspace{0.4cm}
	\begin{itemize}
		\item De zichtbaarheid van de volgende functies is veranderd:
			\begin{itemize}\smaller
				\item \texttt{LanguageService->LL\_files\_cache} (nu \texttt{protected})
				\item \texttt{LanguageService->LL\_labels\_cache} (nu \texttt{protected})
				\item \texttt{LanguageService->getLLL()} (nu \texttt{protected})
				\item \texttt{LanguageService->debugLL()} (nu \texttt{protected})
			\end{itemize}\normalsize
			\vspace{0.2cm}

		\item De volgende functie is als \textbf{verouderd} aangemerkt:
			\begin{itemize}\smaller
				\item \texttt{LanguageService->getLabelsWithPrefix()}
			\end{itemize}\normalsize
			\vspace{0.2cm}

		\item De volgende functie is als \textbf{internal} aangemerkt:
			\begin{itemize}\smaller
				\item \texttt{LanguageService->loadSingleTableDescription()}
			\end{itemize}\normalsize
			\vspace{0.2cm}

	\end{itemize}

\end{frame}

% ------------------------------------------------------------------------------
% Important | 90897 | Remove bootstrap-slider
% Important | 86343 | Replace jQuerydatatables with tablesort
% Deprecation | 90686 | Deprecate model FileMount (Extbase)

\begin{frame}[fragile]
	\frametitle{Verouderde/verwijderde functies}
	\framesubtitle{Interne bibliotheken en klassen}

	\begin{itemize}
		\item De volgende interne bibliotheken zijn verwijderd:
			\begin{itemize}
				\item "bootstrap-slider"
				\item "jQuery.datatables"
			\end{itemize}
			\vspace{0.4cm}

		\item De interne klasse \small\texttt{TYPO3\textbackslash
			CMS\textbackslash
			Extbase\textbackslash
			Domain\textbackslash
			Model\textbackslash
			FileMount}\normalsize\newline
			is als \textbf{verouderd} aangemerkt.

	\end{itemize}

	\vspace{0.6cm}
	\begin{itemize}
		\item[\ding{228}] \textit{Let op:} \textit{Auteurs van extensies horen bibliotheken die niet als publieke API zijn aangemerkt nooit te gebruiken.}
	\end{itemize}

\end{frame}

% ------------------------------------------------------------------------------
% Deprecation | 90937 | Various hooks in ContentObjectRenderer
% Deprecation | 90861 | Image-related methods within ContentObjectRenderer

\begin{frame}[fragile]
	\frametitle{Verouderde/verwijderde functies}
	\framesubtitle{ContentObjectRenderer}

	\begin{itemize}
		\item De volgende hooks hierin zijn als \textbf{verouderd} aangemerkt:\newline
			\tiny
				\texttt{\$GLOBALS['TYPO3\_CONF\_VARS']['SC\_OPTIONS']['tslib/class.tslib\_content.php']...}
			\normalsize

			\begin{itemize}\smaller
				\item \texttt{['cObjTypeAndClass']}
				\item \texttt{['cObjTypeAndClassDefault']}
				\item \texttt{['extLinkATagParamsHandler']}
				\item \texttt{['typolinkLinkHandler']}
			\end{itemize}

		\item De volgende functies zijn als \textbf{verouderd} aangemerkt:

			\begin{itemize}\smaller
				\item \texttt{cImage()}
				\item \texttt{getBorderAttr()}
				\item \texttt{getImageTagTemplate()}
				\item \texttt{getImageSourceCollection()}
				\item \texttt{linkWrap()}
				\item \texttt{getAltParam()}
			\end{itemize}

			\smaller
				(alle functies zijn verplaatst naar de ImageContentObject klasse)
			\normalsize

	\end{itemize}

\end{frame}

% ------------------------------------------------------------------------------
% Deprecation | 90856 | Deprecate Widget AutoComplete ViewHelper (Fluid)

\begin{frame}[fragile]
	\frametitle{Verouderde/verwijderde functies}
	\framesubtitle{Fluid AutoComplete ViewHelper}

	\begin{itemize}
		\item De Fluid ViewHelper \texttt{<f:widget.autocomplete>} de bijgehorende controller
			zijn als \textbf{verouderd} aangemerkt.
		\item Als de volgende gebruikt of uitgebreid worden moet er actie ondernomen worden:

			\begin{itemize}\smaller
				\item \texttt{TYPO3\textbackslash
					CMS\textbackslash
					Fluid\textbackslash
					ViewHelpers\textbackslash
					Widget\textbackslash
					AutocompleteViewHelper}
				\item \texttt{TYPO3\textbackslash
					CMS\textbackslash
					Fluid\textbackslash
					ViewHelpers\textbackslash
					Widget\textbackslash
					Controller\textbackslash
					AutocompleteController}
			\end{itemize}

	\end{itemize}

\end{frame}

% ------------------------------------------------------------------------------
% Deprecation | 90692 | Deprecate FileCollection models (Extbase)

\begin{frame}[fragile]
	\frametitle{Verouderde/verwijderde functies}
	\framesubtitle{FileCollection Models}

	\begin{itemize}
		\item De volgende FileCollection modellen zijn als \textbf{verouderd} aangemerkt:
			\vspace{0.4cm}
			\begin{itemize}\tiny
				\item \texttt{TYPO3\textbackslash
					CMS\textbackslash
					Extbase\textbackslash
					Domain\textbackslash
					Model\textbackslash
					StaticFileCollection}

				\item \texttt{TYPO3\textbackslash
					CMS\textbackslash
					Extbase\textbackslash
					Domain\textbackslash
					Model\textbackslash
					FolderBasedFileCollection}

				\item \texttt{TYPO3\textbackslash
					CMS\textbackslash
					Extbase\textbackslash
					Domain\textbackslash
					Model\textbackslash
					AbstractFileCollection}

				\item \texttt{TYPO3\textbackslash
					CMS\textbackslash
					Extbase\textbackslash
					Property\textbackslash
					TypeConverter\textbackslash
					StaticFileCollectionConverter}

				\item \texttt{TYPO3\textbackslash
					CMS\textbackslash
					Extbase\textbackslash
					Property\textbackslash
					TypeConverter\textbackslash
					FolderBasedFileCollectionConverter}

				\item \texttt{TYPO3\textbackslash
					CMS\textbackslash
					Extbase\textbackslash
					Property\textbackslash
					TypeConverter\textbackslash
					AbstractFileCollectionConverter}

			\end{itemize}

	\end{itemize}

\end{frame}

% ------------------------------------------------------------------------------
% Deprecation | 90625 | Extbase SignalSlot Dispatcher (Extbase)

\begin{frame}[fragile]
	\frametitle{Verouderde/verwijderde functies}
	\framesubtitle{Extbase SignalSlot Dispatcher}

	\begin{itemize}
		\item Tijdens de ontwikkeling van TYPO3 v10 de afgelopen maanden
			zijn alle Extbase signals omgezet naar PSR-14 Events.
		\item De Extbase "SignalSlot Dispatcher" is nu als \textbf{verouderd} aangemerkt.
		\item Extensieontwikkelaars wordt aangeraden te migreren naar PSR-14 Events en Event Listeners.
	\end{itemize}

\end{frame}

% ------------------------------------------------------------------------------
% Deprecation | 90147 | Unified File Name Validator

\begin{frame}[fragile]
	\frametitle{Verouderde/verwijderde functies}
	\framesubtitle{FileNameValidator API}

	% decrease font size for code listing
	\lstset{basicstyle=\tiny\ttfamily}

	\begin{itemize}
		\item De logica voor het valideren van de bestandsnaam van een nieuw (geüploadet) of
			hernoemd bestand is nu beschikbaar als een nieuweFileNameValidator API:\newline
			\small
				\texttt{TYPO3\textbackslash
					CMS\textbackslash
					Core\textbackslash
					Resource\textbackslash
					Security\textbackslash
					FileNameValidator}
			\normalsize

		\item Migration options:
\begin{lstlisting}
// OUD:
GeneralUtility::verifyFilenameAgainstDenyPattern($filename)
// NIEUW:
GeneralUtility::makeInstance(FileNameValidator::class)->isValid($filename)

// OUD:
FILE_DENY_PATTERN_DEFAULT
// NIEUW:
FileNameValidator::DEFAULT_FILE_DENY_PATTERN
\end{lstlisting}

	\end{itemize}

\end{frame}

% ------------------------------------------------------------------------------
% Deprecation | 88740 | Deprecate felogin pibase plugin

\begin{frame}[fragile]
	\frametitle{Verouderde/verwijderde functies}
	\framesubtitle{\texttt{EXT:felogin} Hooks}

	% decrease font size for code listing
	\lstset{basicstyle=\tiny\ttfamily}

	\begin{itemize}
		\item Alle oude hooks van \texttt{EXT:felogin} zijn uitgeschakeld en
			worden verwijderd in TYPO3 v11:

			\begin{itemize}\tiny
				\item \texttt{\$GLOBALS['TYPO3\_CONF\_VARS']['EXTCONF']['felogin']['beforeRedirect']}
				\item \texttt{\$GLOBALS['TYPO3\_CONF\_VARS']['EXTCONF']['felogin']['postProcContent']}
				\item \texttt{\$GLOBALS['TYPO3\_CONF\_VARS']['EXTCONF']['felogin']['password\_changed']}
				\item \texttt{\$GLOBALS['TYPO3\_CONF\_VARS']['EXTCONF']['felogin']['forgotPasswordMail']}
				\item \texttt{\$GLOBALS['TYPO3\_CONF\_VARS']['EXTCONF']['felogin']['login\_confirmed']}
				\item \texttt{\$GLOBALS['TYPO3\_CONF\_VARS']['EXTCONF']['felogin']['login\_error']}
				\item \texttt{\$GLOBALS['TYPO3\_CONF\_VARS']['EXTCONF']['felogin']['loginFormOnSubmitFuncs']}
				\item \texttt{\$GLOBALS['TYPO3\_CONF\_VARS']['EXTCONF']['felogin']['logout\_confirmed']}
			\end{itemize}

		\item Zie
			\href{https://docs.typo3.org/c/typo3/cms-core/master/en-us/Changelog/10.4/Deprecation-88740-DeprecateFeloginPibasePlugin.html}{change log}
			voor migratieopties.
	\end{itemize}

\end{frame}

% ------------------------------------------------------------------------------
% Miscellaneous from 10.1
% ------------------------------------------------------------------------------
% Deprecation | 88850 | ContentObjectRenderer::sendNotifyEmail
% Deprecation | 88787 | BackendUtility::editOnClick

\begin{frame}[fragile]
	\frametitle{Verouderde/verwijderde functies}
	\framesubtitle{Diversen}

	\begin{itemize}
		\item E-mailfunctionaliteit hoort niet thuisin de klasse\newline
			\small
				\texttt{TYPO3\textbackslash
					CMS\textbackslash
					Frontend\textbackslash
					ContentObject\textbackslash
					ContentObjectRenderer}.\newline
			\normalsize
			Daarom is de functie \texttt{sendNotifyEmail()} als \textbf{verouderd} aangemerkt en zal worden verwijderd in TYPO3 v11.

		\item De functie \texttt{editOnClick()} die gebruikt werd om JavaScript \texttt{onclick} doelen te maken
			is als \textbf{verouderd} aangemerkt in de volgende klasse:\newline
			\small
				\texttt{TYPO3\textbackslash
					CMS\textbackslash
					Backend\textbackslash
					Utility\textbackslash
					BackendUtility}.
			\normalsize

	\end{itemize}

\end{frame}

% ------------------------------------------------------------------------------
% Miscellaneous from 10.2
% ------------------------------------------------------------------------------
% Deprecation | 89468 | Deprecate injection of EnvironmentService in Web Request
% Deprecation | 89554 | Deprecate /TYPO3/CMS/Extbase/Mvc/Controller/AbstractController
% Deprecation | 89756 | BackendUtility::TYPO3_copyRightNotice

\begin{frame}[fragile]
	\frametitle{Verouderde/verwijderde functies}
	\framesubtitle{Diversen}

	% decrease font size for code listing
	\lstset{basicstyle=\tiny\ttfamily}

	\begin{itemize}

		\item De eigenschap \texttt{\$environmentService} van de volgende klasse is als \textbf{verouderd} aangemerkt:
			\texttt{TYPO3\textbackslash
				CMS\textbackslash
				Extbase\textbackslash
				Mvc\textbackslash
				Web\textbackslash
				Response}.\newline
			\smaller
				(injecteer indien nodig de environment service handmatig in de subklasse)
			\normalsize

		\item De volgende interne klasse is als \textbf{verouderd} aangemerkt:\newline
			\texttt{TYPO3\textbackslash
				CMS\textbackslash
				Extbase\textbackslash
				Mvc\textbackslash
				Controller\textbackslash
				AbstractController}.

		\item De volgende functie is als \textbf{verouderd} aangemerkt:\newline
			\texttt{TYPO3\
				CMS\
				Backend\
				Utility\
				BackendUtility::TYPO3\_copyRightNotice}.\newline
				\smaller
					(gebruik de PHP klasse \texttt{Typo3Copyright} en de functie \texttt{getCopyrightNotice()} als vervanging)
				\normalsize

	\end{itemize}

\end{frame}

% ------------------------------------------------------------------------------
% Deprecation | 89722 | GMENU_LAYERS related property TSFE->divSection
% Important | 89764 | Incompatible environment related dependency injection services have been removed

\begin{frame}[fragile]
	\frametitle{Verouderde/verwijderde functies}
	\framesubtitle{Diversen}

	% decrease font size for code listing
	\lstset{basicstyle=\tiny\ttfamily}

	\begin{itemize}

		\item De PHP eigenschap \texttt{TypoScriptFrontendController->divSection} is als \textbf{verouderd} aangemerkt.\newline
			\smaller
				(gebruik \texttt{\$GLOBALS['TSFE']->additionalHeaderData[]} om JavaScript aan niet-gecachete inhoud toe te voegen)
			\normalsize

		\item Als onderdeel van de Symfony 5.0 ondersteuning in TYPO3 v10.2 zijn incompatibele services voor injecteren van
			omgevingsgerelateerde afhankelijkheden verwijderd:

			\begin{itemize}
				\item \texttt{env.is\_unix}
				\item \texttt{env.is\_windows}
				\item \texttt{env.is\_cli}
				\item \texttt{env.is\_composer\_mode}
			\end{itemize}

	\end{itemize}

\end{frame}

% ------------------------------------------------------------------------------
% No Miscellaneous from 10.3 and 10.4
% ------------------------------------------------------------------------------
