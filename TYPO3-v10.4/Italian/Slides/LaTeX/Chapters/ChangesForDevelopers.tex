% ------------------------------------------------------------------------------
% TYPO3 Version 10.4 - What's New (Italian Version)
%
% @license	Creative Commons BY-NC-SA 3.0
% @link		https://typo3.org/help/documentation/whats-new/
% @language	Italian
% ------------------------------------------------------------------------------

\section{Modifiche per sviluppatori}
\begin{frame}[fragile]
	\frametitle{Modifiche per sviluppatori}

	\begin{center}\huge{Capitolo 3:}\end{center}
	\begin{center}\huge{\color{typo3darkgrey}\textbf{Modifiche per sviluppatori}}\end{center}

\end{frame}

% ------------------------------------------------------------------------------
% Breaking | 90660 | Registration of dashboard widgets changed

\begin{frame}[fragile]
	\frametitle{Modifiche per sviluppatori}
	\framesubtitle{Widget della Dashboard (1)}

	Cambiamenti tra TYPO3 v10.3 e v10.4:

	\begin{itemize}
		\item Il modo in cui sono registrati i widget di Dashboard è cambiato.
		\item Le classi astratte non vengono più utilizzate poiché i widget sono registrati
			(e configurati) nel file \texttt{Services.yaml}.
		\item Alcuni tipi di widget ora possono essere creati solo dalla configurazione.
		\item Inoltre, ora sono disponibili parole chiave valide per altezza e larghezza
			"\texttt{small}", "\texttt{medium}", o "\texttt{large}"
			(invece di valori numerici).
	\end{itemize}

\end{frame}

% ------------------------------------------------------------------------------
% Breaking | 91066 | Removed ButtonUtility
% Breaking | 91066 | Move interfaces of Dashboard

\begin{frame}[fragile]
	\frametitle{Modifiche per sviluppatori}
	\framesubtitle{Widget della Dashboard (2)}

	Cambiamenti tra TYPO3 v10.3 e v10.4:

	\begin{itemize}
		\item La classe \texttt{ButtonUtility} è stata rimosssa.
		\item Le interfacce sono state spostate e i loro spazi dei nomi probabilmente
		    devono essere aggiornati nel codice personalizzato.
			\begin{itemize}\smaller
				\item \textbf{VECCHIO:}
					\texttt{TYPO3\textbackslash
						CMS\textbackslash
						Dashboard\textbackslash
						Widgets\textbackslash
						Interfaces}
				\item \textbf{NUOVO:}
					\texttt{TYPO3\textbackslash
						CMS\textbackslash
						Dashboard\textbackslash
						Widgets}
			\end{itemize}\normalsize
	\end{itemize}

\end{frame}

% ------------------------------------------------------------------------------
% Important | 77715 | No more password trimming for third-party authentication services

\begin{frame}[fragile]
	\frametitle{Modifiche per sviluppatori}
	\framesubtitle{Servizi di autenticazione di terze parti}

	\begin{itemize}

		\item Le estensioni possono utilizzare l'API del servizio di autenticazione di TYPO3
		    per autenticare gli utenti dai servizi di autenticazione tramite "OAuth", "LDAP", "SAML2", ecc.
		\item L'oggetto TYPO3 \texttt{AbstractUserAuthentication} non taglia più le password da questi servizi di terze parti.
		\item Tuttavia, questo non è correlato al servizio di autenticazione nativo di TYPO3 che richiede ancora una password
		    senza spazi iniziali/finali.

	\end{itemize}

\end{frame}

% ------------------------------------------------------------------------------
% Feature | 89573 | Allow flexible base url for slug fields in FormEngine

\begin{frame}[fragile]
	\frametitle{Modifiche per sviluppatori}
	\framesubtitle{TCA: Prefisso URL di base}

	% decrease font size for code listing
	\lstset{basicstyle=\tiny\ttfamily}

	\begin{itemize}

		\item È ora possibile aggiungere un URL di base personalizzato per le colonne di tipo \texttt{slug} del TCA.
		\item L'URL di base viene visualizzato davanti al campo di input (prefisso).
		\item Esempio (TCA):

\vspace{-0.4cm}
\begin{lstlisting}
...
'config' => [
  'type' => 'slug',
  'appearance' => [
    'prefix' => \Vendor\MyExtension\UserFunctions\FormEngine\SlugPrefix::class . '->getPrefix'
  ]
]
...
\end{lstlisting}

	\end{itemize}

\end{frame}

% ------------------------------------------------------------------------------
% Feature | 87776 | Limit Restriction to table/s in QueryBuilder

\begin{frame}[fragile]
	\frametitle{Modifiche per sviluppatori}
	\framesubtitle{QueryBuilder}

	% decrease font size for code listing
	\lstset{basicstyle=\tiny\ttfamily}

	\begin{itemize}
		\item Ora è possibile applicare restrizioni di query per un set specifico di tabelle
		    (per essere precisi: alias di tabella).
		\item E' possibile utilizzare il seguente contenitore di restrizione:\newline
			\begingroup
				\fontsize{7}{9}
					\texttt{TYPO3\textbackslash
						CMS\textbackslash
						Core\textbackslash
						Database\textbackslash
						Query\textbackslash
						Restriction\textbackslash
						LimitToTablesRestrictionContainer}
			\endgroup

		\item Esempio:
\begin{lstlisting}
$queryBuilder = GeneralUtility::makeInstance(ConnectionPool::class)
  ->getQueryBuilderForTable('tt_content');
$queryBuilder->getRestrictions()
  ->removeByType(HiddenRestriction::class)
  ->add(
    GeneralUtility::makeInstance(LimitToTablesRestrictionContainer::class)
      ->addForTables(GeneralUtility::makeInstance(HiddenRestriction::class), ['tt'])
  );

$queryBuilder->select('tt.uid')->from('tt_content', 'tt');
\end{lstlisting}

	\end{itemize}

\end{frame}

% ------------------------------------------------------------------------------
% Feature | 91008 | Item grouping for TCA select items

\begin{frame}[fragile]
	\frametitle{Modifiche per sviluppatori}
	\framesubtitle{TCA Select Items (Raggruppamento)}

	\begin{itemize}
		\item I tipi di pagina (\texttt{pages.doktype}), i tipi di contenuto (\texttt{tt\_content.CType}),
			e i plugin (\texttt{tt\_content.list\_type}) ora hanno abilitati raggruppamenti nativi.
		\item Questo è stato precedentemente gestito applicando elementi
			"\texttt{-}\texttt{-div-}\texttt{-}".
		\item Si consiglia agli sviluppatori di rimuovere gli elementi
			"\texttt{-}\texttt{-div-}\texttt{-}"
			dalle select custom e utilizzare \texttt{itemGroups} al loro posto.
		\item Vedi
			\href{https://docs.typo3.org/c/typo3/cms-core/master/en-us/Changelog/10.4/Feature-91008-ItemGroupingForTCASelectItems.html}{feature 91008 (grouping)}
			per maggiori dettagli.
	\end{itemize}

\end{frame}

% ------------------------------------------------------------------------------
% Feature | 91008 | Item sorting for TCA select items

\begin{frame}[fragile]
	\frametitle{Modifiche per sviluppatori}
	\framesubtitle{TCA Select Items (Ordinamento)}

	\begin{itemize}
		\item La nuova opzione \texttt{sortOrders} è stata aggiunta per i campi select del TCA.
		\item Ciò consente l'ordinamento degli elementi di selezione TCA statici in base ai loro valori o etichette.
		\item Vedi
			\href{https://docs.typo3.org/c/typo3/cms-core/master/en-us/Changelog/10.4/Feature-91008-ItemSortingForTCASelectItems.html}{feature 91008 (sorting)}
			per maggiori dettagli.
	\end{itemize}

\end{frame}

% ------------------------------------------------------------------------------
% Deprecation | 90377 | Deprecate ref param types of method callUserFunction

\begin{frame}[fragile]
	\frametitle{Modifiche per sviluppatori}
	\framesubtitle{GeneralUtility}

	\begin{itemize}
		\item il terzo argomento del metodo \texttt{callUserFunction()}
			può essere un oggetto oppure \texttt{null}.
		\item Qualsiasi altro dato passato come argomento \texttt{\$ref} ora genera
			un \texttt{E\_USER\_DEPRECATED} warning.
	\end{itemize}

\end{frame}

% ------------------------------------------------------------------------------
% Important | 91079 | Various TypoScriptFrontendRenderer functionality is now internal

\begin{frame}[fragile]
	\frametitle{Modifiche per sviluppatori}
	\framesubtitle{TypoScriptFrontendController}

	\begin{itemize}
		\item Le seguenti proprietà sono contrassegnate come \textbf{internal}:
			\begin{itemize}
				\item \texttt{TypoScriptFrontendController->sPre}
				\item \texttt{TypoScriptFrontendController->pSetup}
				\item \texttt{TypoScriptFrontendController->all}
				\item \texttt{TypoScriptFrontendController->additionalJavaScript}
				\item \texttt{TypoScriptFrontendController->additionalCSS}
				\item \texttt{TypoScriptFrontendController->JSCode}
				\item \texttt{TypoScriptFrontendController->inlineJS}
				\item \texttt{TypoScriptFrontendController->indexedDocTitle}
			\end{itemize}

		\item Il seguente metodo è contrassegnato come \textbf{internal}:

			\begin{itemize}
				\item \texttt{TypoScriptFrontendController->setJS()}
			\end{itemize}

	\end{itemize}

\end{frame}

% ------------------------------------------------------------------------------
% Feature | 90613 | Add language argument to page-related LinkViewHelpers and UriViewHelpers in Fluid

\begin{frame}[fragile]
	\frametitle{Modifiche per sviluppatori}
	\framesubtitle{LinkViewHelpers e UriViewHelpers}

	% decrease font size for code listing
	\lstset{basicstyle=\tiny\ttfamily}

	\begin{itemize}
		\item Il nuovo argomento \texttt{language} è stato aggiunto ai seguenti ViewHelper:
			\begin{itemize}
				\item \texttt{<f:link.typolink>}
				\item \texttt{<f:link.page>}
				\item \texttt{<f:uri.typolink>}
				\item \texttt{<f:uri.page>}
			\end{itemize}

		\item Questo argomento si collega a una lingua specifica di una pagina.
		\item Esempio (lingua ID 3):
\begin{lstlisting}
Go to the
<f:link.page pageUid="42" language="3">french version</f:link.page>
of the "Contact Us" page.
\end{lstlisting}

	\end{itemize}

\end{frame}

% ------------------------------------------------------------------------------
% Feature | 90899 | Introduce AssetRenderer pre-rendering events

\begin{frame}[fragile]
	\frametitle{Modifiche per sviluppatori}
	\framesubtitle{LinkViewHelpers e UriViewHelpers}

	\begin{itemize}
		\item Quando viene utilizzata l'API AssetCollector, le risorse CSS
		    e JavaScript possono essere postelaborate, se necessario.
		\item I seguenti due eventi vengono attivati per questo scopo:
			\begin{itemize}\smaller
				\item \texttt{TYPO3\textbackslash
					CMS\textbackslash
					Core\textbackslash
					Page\textbackslash
					Event\textbackslash
					BeforeStylesheetsRenderingEvent}
				\item \texttt{TYPO3\textbackslash
					CMS\textbackslash
					Core\textbackslash
					Page\textbackslash
					Event\textbackslash
					BeforeJavaScriptsRenderingEvent}
			\end{itemize}

		\item Vedi il
			\href{https://docs.typo3.org/c/typo3/cms-core/master/en-us/Changelog/10.4/Feature-90899-IntroduceAssetPreRenderingEvents.html}{change log}
			per maggiori dettagli, esempi e note aggiuntive.
	\end{itemize}

\end{frame}

% ------------------------------------------------------------------------------
