% ------------------------------------------------------------------------------
% TYPO3 Version 10.4 - What's New (French Version)
%
% @license	Creative Commons BY-NC-SA 3.0
% @link		https://typo3.org/help/documentation/whats-new/
% @language	French
% ------------------------------------------------------------------------------

\section{Introduction}
\begin{frame}[fragile]
	\frametitle{Introduction}

	\begin{center}\huge{Introduction}\end{center}
	\begin{center}\huge{\color{typo3darkgrey}\textbf{Les faits}}\end{center}

\end{frame}

% ------------------------------------------------------------------------------
% TYPO3 Version 10.4 - The Facts

\begin{frame}[fragile]
	\frametitle{Introduction}
	\framesubtitle{TYPO3 Version 10.4 - Les faits}

	\begin{itemize}
		\item Date de sortie~: 21 avril 2020
		\item Type de sortie~: LTS (Support à long-terme)
	\end{itemize}

	\begin{figure}
		\includegraphics[width=0.95\linewidth]{Introduction/typo3-v10-4-banner.jpg}
	\end{figure}

\end{frame}

% ------------------------------------------------------------------------------
% TYPO3 Version 10.4 - Executive Summary

\begin{frame}[fragile]
	\frametitle{Introduction}
	\framesubtitle{En Résumé}

	\small
		TYPO3 v10.4 (aussi appelé TYPO3 v10 LTS indiquant que c'est la version de support à long-terme)
		est notre nouvelle version phare et est, sans conteste, l'un des systèmes de gestion
		de contenu open-source en PHP les plus avancés du marché à date.

		\vspace{0.2cm}

		Après la publication de cinq itérations depuis juillet 2019, nous pouvons fièrement
		affirmer que nous avec équipé TYPO3 avec les dernières bibliothèques PHP modernes et
		que nous avons introduit de nouvelles fonctionnalités fantastiques pour les entreprises.

		\vspace{0.2cm}

		Notez que ce document résume seulement les changements entre TYPO3 v10.3 et v10.4.

		\vspace{0.2cm}

		Les diaporamas «~What's New~» pour toutes les sorties TYPO3 v10.x disponibles sur
		\href{https://typo3.org/help/documentation/whats-new/}{typo3.org}.

	\normalsize

\end{frame}

% ------------------------------------------------------------------------------
% System Requirements

\begin{frame}[fragile]
	\frametitle{Introduction}
	\framesubtitle{Configuration requise}

	\begin{itemize}
		\item Version de PHP~: 7.2, 7.3 ou 7.4
		\item Configuration PHP~:

			\begin{itemize}
				\item \texttt{memory\_limit} >= 256M
				\item \texttt{max\_execution\_time} >= 240s
				\item \texttt{max\_input\_vars} >= 1500
				\item L'option de compilation \texttt{-}\texttt{-disable-ipv6}
					\underline(NE) doit \underline{PAS} être utilisée
			\end{itemize}

		\item La majorité des serveurs de base de données supportés par \textbf{Doctrine DBAL}
			fonctionnent avec TYPO3. Les moteurs testés sont par exemple~:
	\end{itemize}

	\begin{figure}
		\includegraphics[width=0.80\linewidth]{Introduction/logo-databases.png}
	\end{figure}

\end{frame}

% ------------------------------------------------------------------------------
% Development, Release, and Maintenance Timeline

\begin{frame}[fragile]
	\frametitle{Introduction}
	\framesubtitle{Chronologie des développements, sorties et maintenances}

	\textbf{TYPO3 v10}

	\begin{figure}
		\includegraphics[width=1\linewidth]{Introduction/typo3-v10-lifecycle.png}
	\end{figure}

	\textbf{Support étendu}\newline
	\smaller
		\href{https://typo3.com}{TYPO3 GmbH} propose des options de support pour TYPO3
		v10 LTS même après de 30 avril 2023, pour 3 ans supplémentaires maximum.
	\normalsize

\end{frame}

% ------------------------------------------------------------------------------
% TYPO3 v10 Roadmap

\begin{frame}[fragile]
	\frametitle{Introduction}
	\framesubtitle{Feuille de route TYPO3 v10}

	Dates de sortie et objectifs principaux~:

	\begin{itemize}

		\item v10.0 \tabto{1.1cm}23/Jui./2019\tabto{3.4cm}Ouvre la voie à de nouveaux concepts et APis
		\item v10.1 \tabto{1.1cm}01/Oct./2019\tabto{3.4cm}Améliorations routage et gestion des sites V2
		\item v10.2 \tabto{1.1cm}03/Déc./2019\tabto{3.4cm}Améliorations du moteur de rendu Fluid
		\item v10.3 \tabto{1.1cm}25/Fév./2020\tabto{3.4cm}Gèle des fonctionnalités
		\item
			\begingroup
				\color{typo3orange}
				v10.4 \tabto{1.1cm}21/Avr./2020\tabto{3.4cm}Version LTS (Long-term Support)
			\endgroup

	\end{itemize}

	\vspace{0.6cm}
	\smaller
		\url{https://typo3.org/article/typo3-v10-roadmap}\newline
		\url{https://typo3.org/article/typo3-v10-lts-safe-and-sound}
	\normalsize

\end{frame}

% ------------------------------------------------------------------------------
% Installation

\begin{frame}[fragile]
	\frametitle{Introduction}
	\framesubtitle{Installation}

	\begin{itemize}
		\item Procédure officielle \textit{classique} d'installation sous Linux/Mac OS X\newline
			(DocumentRoot considéré \texttt{/var/www/site/htdocs})~:
\begin{lstlisting}
$ cd /var/www/site
$ wget --content-disposition get.typo3.org/10.4
$ tar xzf typo3_src-10.4.0.tar.gz
$ cd htdocs
$ ln -s ../typo3_src-10.4.0 typo3_src
$ ln -s typo3_src/index.php
$ ln -s typo3_src/typo3
$ touch FIRST_INSTALL
\end{lstlisting}

		\item Liens symboliques sous Microsoft Windows~:

			\begin{itemize}
				\item Utiliser \texttt{junction} sous Windows XP/2000
				\item Utiliser \texttt{mklink} sous Windows Vista et Windows 7 ou supérieur
			\end{itemize}

	\end{itemize}
\end{frame}

% ------------------------------------------------------------------------------
% Installation using composer

\begin{frame}[fragile]
	\frametitle{Installation and Upgrade}
	\framesubtitle{Installation avec \texttt{composer}}

	\begin{itemize}
		\item Installation avec \textit{composer} sous Linux, Mac OS X et Windows 10~:
\begin{lstlisting}
$ cd /var/www/site/
$ composer create-project typo3/cms-base-distribution typo3v10 ^10.4
\end{lstlisting}

		\item Vous pouvez aussi créer votre ficher \texttt{composer.json} sur mesure
			et exécuter~:
\begin{lstlisting}
$ composer install
\end{lstlisting}

			Plus d'informations à propos de composer pour le noyau de TYPO3 et pour ses
			extensions disponibles sur~:

			\small
				\href{https://get.typo3.org/misc/composer/repository}{https://get.typo3.org/misc/composer/repository}
			\normalsize

	\end{itemize}
\end{frame}

% ------------------------------------------------------------------------------
