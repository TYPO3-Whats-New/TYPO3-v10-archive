% ------------------------------------------------------------------------------
% TYPO3 Version 10.4 - What's New (French Version)
%
% @license	Creative Commons BY-NC-SA 3.0
% @link		https://typo3.org/help/documentation/whats-new/
% @language	French
% ------------------------------------------------------------------------------

\section{Changements pour les développeurs}
\begin{frame}[fragile]
	\frametitle{Changements pour les développeurs}

	\begin{center}\huge{Chapitre 3~:}\end{center}
	\begin{center}\huge{\color{typo3darkgrey}\textbf{Changements pour les développeurs}}\end{center}

\end{frame}

% ------------------------------------------------------------------------------
% Breaking | 90660 | Registration of dashboard widgets changed

\begin{frame}[fragile]
	\frametitle{Changements pour les développeurs}
	\framesubtitle{Widgets du tableau de bord (1)}

	Changements entre TYPO3 v10.3 et v10.4~:

	\begin{itemize}
		\item La manière d'inscrire un widget a changé.
		\item Les classes abstraites ne sont plus utilisées, les widgets étant inscrits
			(et configurés) dans le fichier \texttt{Services.yaml}.
		\item Certains types de widget peuvent donc être créés uniquement par de la configuration.
		\item En plus, les mots clés valides pour les dimensions (\texttt{height} et \texttt{width}) sont
			"\texttt{small}", "\texttt{medium}", et "\texttt{large}"
			(au lieu de valeurs numériques).
	\end{itemize}

\end{frame}

% ------------------------------------------------------------------------------
% Breaking | 91066 | Removed ButtonUtility
% Breaking | 91066 | Move interfaces of Dashboard

\begin{frame}[fragile]
	\frametitle{Changements pour les développeurs}
	\framesubtitle{Widgets du tableau de bord (2)}

	Changements entre TYPO3 v10.3 et v10.4~:

	\begin{itemize}
		\item La classe \texttt{ButtonUtility} est retirée.
		\item Les interfaces ont été déplacées et leur référence dans vos fichiers
			de code doit probablement être mise à jour.
			\begin{itemize}\smaller
				\item \textbf{ANCIENNE~:}
					\texttt{TYPO3\textbackslash
						CMS\textbackslash
						Dashboard\textbackslash
						Widgets\textbackslash
						Interfaces}
				\item \textbf{NOUVELLE~:}
					\texttt{TYPO3\textbackslash
						CMS\textbackslash
						Dashboard\textbackslash
						Widgets}
			\end{itemize}\normalsize
	\end{itemize}

\end{frame}

% ------------------------------------------------------------------------------
% Important | 77715 | No more password trimming for third-party authentication services

\begin{frame}[fragile]
	\frametitle{Changements pour les développeurs}
	\framesubtitle{Services d'authentification tiers}

	\begin{itemize}

		\item Les extensions peuvent utiliser l'API du service d'authentification de TYPO3 pour
			authentifier les utilisateurs à l'aide de fournisseurs d'identité via "OAuth", "LDAP", "SAML2", etc.
		\item L'objet TYPO3 \texttt{AbstractUserAuthentication} ne retraite plus les mots de passe
			pour ces services tiers.
		\item Cependant, ça ne s'applique pas au service d'authentification natif de TYPO3
			qui requière toujours un mot de passe sans espaces en début et fin.

	\end{itemize}

\end{frame}

% ------------------------------------------------------------------------------
% Feature | 89573 | Allow flexible base url for slug fields in FormEngine

\begin{frame}[fragile]
	\frametitle{Changements pour les développeurs}
	\framesubtitle{TCA~: Préfix d'URL}

	% decrease font size for code listing
	\lstset{basicstyle=\tiny\ttfamily}

	\begin{itemize}

		\item Il est possible d'utiliser une adresse de base personnalisée pour les colonnes TCA de type \texttt{slug}.
		\item L'adresse de base est affichée avant le champ de saisie (préfixe).
		\item Exemple (TCA)~:

\vspace{-0.4cm}
\begin{lstlisting}
...
'config' => [
  'type' => 'slug',
  'appearance' => [
    'prefix' => \Vendor\MyExtension\UserFunctions\FormEngine\SlugPrefix::class . '->getPrefix'
  ]
]
...
\end{lstlisting}

	\end{itemize}

\end{frame}

% ------------------------------------------------------------------------------
% Feature | 87776 | Limit Restriction to table/s in QueryBuilder

\begin{frame}[fragile]
	\frametitle{Changements pour les développeurs}
	\framesubtitle{QueryBuilder}

	% decrease font size for code listing
	\lstset{basicstyle=\tiny\ttfamily}

	\begin{itemize}
		\item Les restrictions de requête peuvent s'appliquer à des ensembles
			de table spécifiques (plus précisément aux alias de table).
		\item Ce conteneur de restriction est alors à utiliser~:\newline
			\begingroup
				\fontsize{7}{9}
					\texttt{TYPO3\textbackslash
						CMS\textbackslash
						Core\textbackslash
						Database\textbackslash
						Query\textbackslash
						Restriction\textbackslash
						LimitToTablesRestrictionContainer}
			\endgroup

		\item Example~:
\begin{lstlisting}
$queryBuilder = GeneralUtility::makeInstance(ConnectionPool::class)
  ->getQueryBuilderForTable('tt_content');
$queryBuilder->getRestrictions()
  ->removeByType(HiddenRestriction::class)
  ->add(
    GeneralUtility::makeInstance(LimitToTablesRestrictionContainer::class)
      ->addForTables(GeneralUtility::makeInstance(HiddenRestriction::class), ['tt'])
  );

$queryBuilder->select('tt.uid')->from('tt_content', 'tt');
\end{lstlisting}

	\end{itemize}

\end{frame}

% ------------------------------------------------------------------------------
% Feature | 91008 | Item grouping for TCA select items

\begin{frame}[fragile]
	\frametitle{Changements pour les développeurs}
	\framesubtitle{Éléments de la liste déroulante TCA (Groupes)}

	\begin{itemize}
		\item Les types de pages (\texttt{pages.doktype}), de contenu (\texttt{tt\_content.CType}),
			ainsi que les plugins (\texttt{tt\_content.list\_type}) ont le regroupement natif d'activé.
		\item C'était géré par l'application d'un élément
			"\texttt{-}\texttt{-div-}\texttt{-}".
		\item Il est conseillé aux développeurs de retirer les éléments
			"\texttt{-}\texttt{-div-}\texttt{-}"
			de leurs champs liste et d'utiliser \texttt{itemGroups}.
		\item Voir
			\href{https://docs.typo3.org/c/typo3/cms-core/master/en-us/Changelog/10.4/Feature-91008-ItemGroupingForTCASelectItems.html}{fonctionnalité 91008 (regroupement) (en)}
			pour plus de détails.
	\end{itemize}

\end{frame}

% ------------------------------------------------------------------------------
% Feature | 91008 | Item sorting for TCA select items

\begin{frame}[fragile]
	\frametitle{Changements pour les développeurs}
	\framesubtitle{Éléments de la liste déroulante TCA (Tri)}

	\begin{itemize}
		\item L'option \texttt{sortOrders} pour les champs liste TCA est ajoutée.
		\item Elle permet le tri des éléments statiques du TCA par valeur ou libellé.
		\item Voir
			\href{https://docs.typo3.org/c/typo3/cms-core/master/en-us/Changelog/10.4/Feature-91008-ItemSortingForTCASelectItems.html}{fonctionnalité 91008 (tri) (en)}
			pour plus de détails
	\end{itemize}

\end{frame}

% ------------------------------------------------------------------------------
% Deprecation | 90377 | Deprecate ref param types of method callUserFunction

\begin{frame}[fragile]
	\frametitle{Changements pour les développeurs}
	\framesubtitle{GeneralUtility}

	\begin{itemize}
		\item Le troisième argument de la méthode \texttt{callUserFunction()}
			doit être un objet ou \texttt{null}.
		\item Toute autre données passée à l'argument \texttt{\$ref} génère un
			avertissement \texttt{E\_USER\_DEPRECATED}.
	\end{itemize}

\end{frame}

% ------------------------------------------------------------------------------
% Important | 91079 | Various TypoScriptFrontendRenderer functionality is now internal

\begin{frame}[fragile]
	\frametitle{Changements pour les développeurs}
	\framesubtitle{TypoScriptFrontendController}

	\begin{itemize}
		\item Ces propriétés sont marquées \textbf{interne}~:
			\begin{itemize}
				\item \texttt{TypoScriptFrontendController->sPre}
				\item \texttt{TypoScriptFrontendController->pSetup}
				\item \texttt{TypoScriptFrontendController->all}
				\item \texttt{TypoScriptFrontendController->additionalJavaScript}
				\item \texttt{TypoScriptFrontendController->additionalCSS}
				\item \texttt{TypoScriptFrontendController->JSCode}
				\item \texttt{TypoScriptFrontendController->inlineJS}
				\item \texttt{TypoScriptFrontendController->indexedDocTitle}
			\end{itemize}

		\item Cette méthode est marquée \textbf{interne}~:

			\begin{itemize}
				\item \texttt{TypoScriptFrontendController->setJS()}
			\end{itemize}

	\end{itemize}

\end{frame}

% ------------------------------------------------------------------------------
% Feature | 90613 | Add language argument to page-related LinkViewHelpers and UriViewHelpers in Fluid

\begin{frame}[fragile]
	\frametitle{Changements pour les développeurs}
	\framesubtitle{LinkViewHelpers et UriViewHelpers}

	% decrease font size for code listing
	\lstset{basicstyle=\tiny\ttfamily}

	\begin{itemize}
		\item L'argument \texttt{language} est ajouté à ces ViewHelpers~:
			\begin{itemize}
				\item \texttt{<f:link.typolink>}
				\item \texttt{<f:link.page>}
				\item \texttt{<f:uri.typolink>}
				\item \texttt{<f:uri.page>}
			\end{itemize}

		\item Cette argument cible la traduction d'une page.
		\item Exemple (ID de langue 3)~:
\begin{lstlisting}
Go to the
<f:link.page pageUid="42" language="3">french version</f:link.page>
of the "Contact Us" page.
\end{lstlisting}

	\end{itemize}

\end{frame}

% ------------------------------------------------------------------------------
% Feature | 90899 | Introduce AssetRenderer pre-rendering events

\begin{frame}[fragile]
	\frametitle{Changements pour les développeurs}
	\framesubtitle{LinkViewHelpers et UriViewHelpers}

	\begin{itemize}
		\item Lorsque l'API AssetCollector est utilisée, les ressources CSS et JavaScript
			peuvent être transformées avant l'envoi si besoin.
		\item Ces deux événements sont déclenchés dans ce but~:
			\begin{itemize}\smaller
				\item \texttt{TYPO3\textbackslash
					CMS\textbackslash
					Core\textbackslash
					Page\textbackslash
					Event\textbackslash
					BeforeStylesheetsRenderingEvent}
				\item \texttt{TYPO3\textbackslash
					CMS\textbackslash
					Core\textbackslash
					Page\textbackslash
					Event\textbackslash
					BeforeJavaScriptsRenderingEvent}
			\end{itemize}

		\item Voir le
			\href{https://docs.typo3.org/c/typo3/cms-core/master/en-us/Changelog/10.4/Feature-90899-IntroduceAssetPreRenderingEvents.html}{journal des changements (en)}
			pour plus d'informations, des exemples et des notes supplémentaires.
	\end{itemize}

\end{frame}

% ------------------------------------------------------------------------------
