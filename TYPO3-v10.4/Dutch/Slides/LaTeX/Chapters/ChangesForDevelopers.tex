% ------------------------------------------------------------------------------
% TYPO3 Version 10.4 - What's New (Dutch Version)
%
% @license	Creative Commons BY-NC-SA 3.0
% @link		https://typo3.org/help/documentation/whats-new/
% @language	Dutch
% ------------------------------------------------------------------------------

\section{Wijzigingen voor ontwikkelaars}
\begin{frame}[fragile]
	\frametitle{Wijzigingen voor ontwikkelaars}

	\begin{center}\huge{Hoofdstuk 3:}\end{center}
	\begin{center}\huge{\color{typo3darkgrey}\textbf{Wijzigingen voor ontwikkelaars}}\end{center}

\end{frame}

% ------------------------------------------------------------------------------
% Breaking | 90660 | Registration of dashboard widgets changed

\begin{frame}[fragile]
	\frametitle{Wijzigingen voor ontwikkelaars}
	\framesubtitle{Dashboard Widgets (1)}

	Wijzigingen tussen TYPO3 v10.3 en v10.4:

	\begin{itemize}
		\item De manier om Dashboard widgets te registreren is aangepast.
		\item Abstracte klassen worden niet meer gebruikt en widgets worden geregistreerd
			(en geconfigureerd) in het bestand \texttt{Services.yaml}
		\item Sommige soorten widgets kunnen nu met alleen configuratie gemaakt worden.
		\item Daarnaast zijn "\texttt{small}", "\texttt{medium}" of "\texttt{large}"
			geldige termen voor hoogte en breedte (in plaats van getallen).
	\end{itemize}

\end{frame}

% ------------------------------------------------------------------------------
% Breaking | 91066 | Removed ButtonUtility
% Breaking | 91066 | Move interfaces of Dashboard

\begin{frame}[fragile]
	\frametitle{Wijzigingen voor ontwikkelaars}
	\framesubtitle{Dashboard Widgets (2)}

	Wijzigingen tussen TYPO3 v10.3 en v10.4:

	\begin{itemize}
		\item De \texttt{ButtonUtility} klasse is verwijderd.
		\item Interfaces zijn verplaatst en hun namespaces moeten waarschijnlijk
			bijgewerkt worden in eigen code.
			\begin{itemize}\smaller
				\item \textbf{OUD:}
					\texttt{TYPO3\textbackslash
						CMS\textbackslash
						Dashboard\textbackslash
						Widgets\textbackslash
						Interfaces}
				\item \textbf{NIEUW:}
					\texttt{TYPO3\textbackslash
						CMS\textbackslash
						Dashboard\textbackslash
						Widgets}
			\end{itemize}\normalsize
	\end{itemize}

\end{frame}

% ------------------------------------------------------------------------------
% Important | 77715 | No more password trimming for third-party authentication services

\begin{frame}[fragile]
	\frametitle{Wijzigingen voor ontwikkelaars}
	\framesubtitle{Authenticatieservice van derden}

	\begin{itemize}

		\item Extensies kunnen de authenticatieservice-API van TYPO3 gebruiken om gebruikers
			te authenticeren bij identity brokers via "OAuth", "LDAP", "SAML2", enz.
		\item Het \texttt{AbstractUserAuthentication} object van TYPO3 knipt geen spaties meer af van
			wachtwoorden van deze services van derden
		\item Echter, dit geldt niet voor de eigen authenticatieservice van TYPO3 die nog steeds
			spaties voor/achter een wachtwoord verbiedt.

	\end{itemize}

\end{frame}

% ------------------------------------------------------------------------------
% Feature | 89573 | Allow flexible base url for slug fields in FormEngine

\begin{frame}[fragile]
	\frametitle{Wijzigingen voor ontwikkelaars}
	\framesubtitle{TCA: Prefix Basis URL}

	% decrease font size for code listing
	\lstset{basicstyle=\tiny\ttfamily}

	\begin{itemize}

		\item Er kan een aangepaste basis-URL toegevoegd worden voor TCA-kolommen type \texttt{slug}.
		\item De basis-URL wordt afgebeeld voor het invoerveld (prefix).
		\item Voorbeeld (TCA):

\vspace{-0.4cm}
\begin{lstlisting}
...
'config' => [
  'type' => 'slug',
  'appearance' => [
    'prefix' => \Vendor\MyExtension\UserFunctions\FormEngine\SlugPrefix::class . '->getPrefix'
  ]
]
...
\end{lstlisting}

	\end{itemize}

\end{frame}

% ------------------------------------------------------------------------------
% Feature | 87776 | Limit Restriction to table/s in QueryBuilder

\begin{frame}[fragile]
	\frametitle{Wijzigingen voor ontwikkelaars}
	\framesubtitle{QueryBuilder}

	% decrease font size for code listing
	\lstset{basicstyle=\tiny\ttfamily}

	\begin{itemize}
		\item Er kunnen beperkingen voor queries toegepast worden voor een
			specifieke groep tabellen (in feite: tabelaliassen).
		\item De volgende restrictiecontainer kan worden gebruikt:\newline
			\begingroup
				\fontsize{7}{9}
					\texttt{TYPO3\textbackslash
						CMS\textbackslash
						Core\textbackslash
						Database\textbackslash
						Query\textbackslash
						Restriction\textbackslash
						LimitToTablesRestrictionContainer}
			\endgroup

		\item Voorbeeld:
\begin{lstlisting}
$queryBuilder = GeneralUtility::makeInstance(ConnectionPool::class)
  ->getQueryBuilderForTable('tt_content');
$queryBuilder->getRestrictions()
  ->removeByType(HiddenRestriction::class)
  ->add(
    GeneralUtility::makeInstance(LimitToTablesRestrictionContainer::class)
      ->addForTables(GeneralUtility::makeInstance(HiddenRestriction::class), ['tt'])
  );

$queryBuilder->select('tt.uid')->from('tt_content', 'tt');
\end{lstlisting}

	\end{itemize}

\end{frame}

% ------------------------------------------------------------------------------
% Feature | 91008 | Item grouping for TCA select items

\begin{frame}[fragile]
	\frametitle{Wijzigingen voor ontwikkelaars}
	\framesubtitle{TCA selectie-items (groeperen)}

	\begin{itemize}
		\item Paginatypes (\texttt{pages.doktype}), inhoudstypes (\texttt{tt\_content.CType})
			en plug-ins (\texttt{tt\_content.list\_type}) hebben nu groeperen ingeschakeld.
		\item Dit werd voorheen afgehandeld door toepassen van
			"\texttt{-}\texttt{-div-}\texttt{-}" items.
		\item Het wordt aangeraden om "\texttt{-}\texttt{-div-}\texttt{-}" items
			uit eigen selectvelden te verwijderen en in plaats hiervan \texttt{itemGroups} te gebruiken.
		\item Zie
			\href{https://docs.typo3.org/c/typo3/cms-core/master/en-us/Changelog/10.4/Feature-91008-ItemGroupingForTCASelectItems.html}{feature 91008 (grouping)}
			voor meer details.
	\end{itemize}

\end{frame}

% ------------------------------------------------------------------------------
% Feature | 91008 | Item sorting for TCA select items

\begin{frame}[fragile]
	\frametitle{Wijzigingen voor ontwikkelaars}
	\framesubtitle{TCA selectie-items (sorteren)}

	\begin{itemize}
		\item Nieuwe optie \texttt{sortOrders} is toegevoegd voor TCA-gebaseerde selectievelden.
		\item Hier kunnen statische TCA selectie-items gesorteerd worden op hun waarden of labels.
		\item Zie
			\href{https://docs.typo3.org/c/typo3/cms-core/master/en-us/Changelog/10.4/Feature-91008-ItemSortingForTCASelectItems.html}{feature 91008 (sorting)}
			voor meer details.
	\end{itemize}

\end{frame}

% ------------------------------------------------------------------------------
% Deprecation | 90377 | Deprecate ref param types of method callUserFunction

\begin{frame}[fragile]
	\frametitle{Wijzigingen voor ontwikkelaars}
	\framesubtitle{GeneralUtility}

	\begin{itemize}
		\item Het derde argument van methode \texttt{callUserFunction()}
			moet een object of \texttt{null} zijn.
		\item Andere soorten data als argument \texttt{\$ref} resulteren in een
			\texttt{E\_USER\_DEPRECATED} waarschuwing.
	\end{itemize}

\end{frame}

% ------------------------------------------------------------------------------
% Important | 91079 | Various TypoScriptFrontendRenderer functionality is now internal

\begin{frame}[fragile]
	\frametitle{Wijzigingen voor ontwikkelaars}
	\framesubtitle{TypoScriptFrontendController}

	\begin{itemize}
		\item De volgende eigenschappen zijn als \textbf{internal} aangemerkt:
			\begin{itemize}
				\item \texttt{TypoScriptFrontendController->sPre}
				\item \texttt{TypoScriptFrontendController->pSetup}
				\item \texttt{TypoScriptFrontendController->all}
				\item \texttt{TypoScriptFrontendController->additionalJavaScript}
				\item \texttt{TypoScriptFrontendController->additionalCSS}
				\item \texttt{TypoScriptFrontendController->JSCode}
				\item \texttt{TypoScriptFrontendController->inlineJS}
				\item \texttt{TypoScriptFrontendController->indexedDocTitle}
			\end{itemize}

		\item De volgende methode is als \textbf{internal} aangemerkt:

			\begin{itemize}
				\item \texttt{TypoScriptFrontendController->setJS()}
			\end{itemize}

	\end{itemize}

\end{frame}

% ------------------------------------------------------------------------------
% Feature | 90613 | Add language argument to page-related LinkViewHelpers and UriViewHelpers in Fluid

\begin{frame}[fragile]
	\frametitle{Wijzigingen voor ontwikkelaars}
	\framesubtitle{LinkViewHelpers en UriViewHelpers}

	% decrease font size for code listing
	\lstset{basicstyle=\tiny\ttfamily}

	\begin{itemize}
		\item Nieuw argument \texttt{language} is toegevoegd aan de volgende ViewHelpers:
			\begin{itemize}
				\item \texttt{<f:link.typolink>}
				\item \texttt{<f:link.page>}
				\item \texttt{<f:uri.typolink>}
				\item \texttt{<f:uri.page>}
			\end{itemize}

		\item Dit argument linkt naar een bepaalde taal van de pagina.
		\item Voorbeeld (language ID 3):
\begin{lstlisting}
Ga naar de
<f:link.page pageUid="42" language="3">Franse versie</f:link.page>
van de "Contact" pagina.
\end{lstlisting}

	\end{itemize}

\end{frame}

% ------------------------------------------------------------------------------
% Feature | 90899 | Introduce AssetRenderer pre-rendering events

\begin{frame}[fragile]
	\frametitle{Wijzigingen voor ontwikkelaars}
	\framesubtitle{LinkViewHelpers en UriViewHelpers}

	\begin{itemize}
		\item Als de AssetCollector API wordt gebruikt kunnen CSS en JavaScript
			achteraf bewerkt worden indien nodig.
		\item De volgende twee Events worden aangestuurd hiervoor:
			\begin{itemize}\smaller
				\item \texttt{TYPO3\textbackslash
					CMS\textbackslash
					Core\textbackslash
					Page\textbackslash
					Event\textbackslash
					BeforeStylesheetsRenderingEvent}
				\item \texttt{TYPO3\textbackslash
					CMS\textbackslash
					Core\textbackslash
					Page\textbackslash
					Event\textbackslash
					BeforeJavaScriptsRenderingEvent}
			\end{itemize}

		\item Zie
			\href{https://docs.typo3.org/c/typo3/cms-core/master/en-us/Changelog/10.4/Feature-90899-IntroduceAssetPreRenderingEvents.html}{change log}
			voor meer details, voorbeelden en extra opmerkingen.
	\end{itemize}

\end{frame}

% ------------------------------------------------------------------------------
