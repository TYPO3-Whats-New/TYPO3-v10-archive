% ------------------------------------------------------------------------------
% TYPO3 Version 10.4 - What's New (Serbian Version)
%
% @license	Creative Commons BY-NC-SA 3.0
% @link		https://typo3.org/help/documentation/whats-new/
% @language	Serbian
% ------------------------------------------------------------------------------

\section{Izmene za programere}
\begin{frame}[fragile]
	\frametitle{Izmene za programere}

	\begin{center}\huge{Poglavlje 3:}\end{center}
	\begin{center}\huge{\color{typo3darkgrey}\textbf{Izmene za programere}}\end{center}

\end{frame}

% ------------------------------------------------------------------------------
% Breaking | 90660 | Registration of dashboard widgets changed

\begin{frame}[fragile]
	\frametitle{Izmene za programere}
	\framesubtitle{Dashboard vidžeti (1)}

	Izmene izmedju TYPO3 v10.3 i v10.4:

	\begin{itemize}
		\item Način registrovanja Dashboard vidžeta je promenjen.
		\item Abstraktne klase se više ne koriste jer se vidžeti registruju
			(i konfigurišu) u fajlu \texttt{Services.yaml}.
		\item Neki tipovi vidžeta se mogu napraviti samo pomoću konfiguracije.
		\item Takodje, validne ključne reči za visinu i širinu su
			"\texttt{small}", "\texttt{medium}", ili "\texttt{large}"
			(umesto brojčanih vrednosti).
	\end{itemize}

\end{frame}

% ------------------------------------------------------------------------------
% Breaking | 91066 | Removed ButtonUtility
% Breaking | 91066 | Move interfaces of Dashboard

\begin{frame}[fragile]
	\frametitle{Izmene za programere}
	\framesubtitle{Dashboard vidžeti (2)}

	Izmene izmedju TYPO3 v10.3 i v10.4:

	\begin{itemize}
		\item Uklonjena je klasa \texttt{ButtonUtility}.
		\item Interfejsi su premešteni i njihov namespace treba da ažurirate u
			u Vašem prilagodjenom kodu.
			\begin{itemize}\smaller
				\item \textbf{STARO:}
					\texttt{TYPO3\textbackslash
						CMS\textbackslash
						Dashboard\textbackslash
						Widgets\textbackslash
						Interfaces}
				\item \textbf{NOVO:}
					\texttt{TYPO3\textbackslash
						CMS\textbackslash
						Dashboard\textbackslash
						Widgets}
			\end{itemize}\normalsize
	\end{itemize}

\end{frame}

% ------------------------------------------------------------------------------
% Important | 77715 | No more password trimming for third-party authentication services

\begin{frame}[fragile]
	\frametitle{Izmene za programere}
	\framesubtitle{Eksterni servisi za autentifikaciju}

	\begin{itemize}

		\item Proširenja mogu da koriste TYPO3 Authentication Service API
			da autentifikuju korisnike putem "OAuth", "LDAP", "SAML2" i tako dalje...
		\item TYPO3 \texttt{AbstractUserAuthentication} objekt više ne trimuje lozinke sa eksternih servisa.
		\item Medjutim, ovo nije povezano sa nativnim servisom za autentifikaciju u okviru TYPO3
			koji i dalje zahteva lozinke bez početnih/završnih praznih prostora.

	\end{itemize}

\end{frame}

% ------------------------------------------------------------------------------
% Feature | 89573 | Allow flexible base url for slug fields in FormEngine

\begin{frame}[fragile]
	\frametitle{Izmene za programere}
	\framesubtitle{TCA: Base URL prefix}

	% decrease font size for code listing
	\lstset{basicstyle=\tiny\ttfamily}

	\begin{itemize}

		\item Sad je moguće dodati prilagodjen base URL za TCA kolone tipa \texttt{slug}.
		\item Base URL je prikazan pre polja (prefix).
		\item Primer (TCA):

\vspace{-0.4cm}
\begin{lstlisting}
...
'config' => [
  'type' => 'slug',
  'appearance' => [
    'prefix' => \Vendor\MyExtension\UserFunctions\FormEngine\SlugPrefix::class . '->getPrefix'
  ]
]
...
\end{lstlisting}

	\end{itemize}

\end{frame}

% ------------------------------------------------------------------------------
% Feature | 87776 | Limit Restriction to table/s in QueryBuilder

\begin{frame}[fragile]
	\frametitle{Izmene za programere}
	\framesubtitle{QueryBuilder}

	% decrease font size for code listing
	\lstset{basicstyle=\tiny\ttfamily}

	\begin{itemize}
		\item Od sada je moguće postaviti ograničenja na upite za odredjene grupe tabela
			(preciznije: na alijase tabela).
		\item Sledeći kontejner ograničenja se može koristiti u ovu svrhu:\newline
			\begingroup
				\fontsize{7}{9}
					\texttt{TYPO3\textbackslash
						CMS\textbackslash
						Core\textbackslash
						Database\textbackslash
						Query\textbackslash
						Restriction\textbackslash
						LimitToTablesRestrictionContainer}
			\endgroup

		\item Primer:
\begin{lstlisting}
$queryBuilder = GeneralUtility::makeInstance(ConnectionPool::class)
  ->getQueryBuilderForTable('tt_content');
$queryBuilder->getRestrictions()
  ->removeByType(HiddenRestriction::class)
  ->add(
    GeneralUtility::makeInstance(LimitToTablesRestrictionContainer::class)
      ->addForTables(GeneralUtility::makeInstance(HiddenRestriction::class), ['tt'])
  );

$queryBuilder->select('tt.uid')->from('tt_content', 'tt');
\end{lstlisting}

	\end{itemize}

\end{frame}

% ------------------------------------------------------------------------------
% Feature | 91008 | Item grouping for TCA select items

\begin{frame}[fragile]
	\frametitle{Izmene za programere}
	\framesubtitle{TCA selekt stavke (grupisanje)}

	\begin{itemize}
		\item Tipovi stranica (\texttt{pages.doktype}), tipovi sadržaja (\texttt{tt\_content.CType}),
			i plugina (\texttt{tt\_content.list\_type}) sada imaju opciju grupisanja.
		\item Za ovo se ranije koristila stavka
			"\texttt{-}\texttt{-div-}\texttt{-}".
		\item Programeri se savetuju da uklone stavke
			"\texttt{-}\texttt{-div-}\texttt{-}"
			iz prilagodjenih selekta i umesto njih koriste \texttt{itemGroups}.
		\item Pogledajte
			\href{https://docs.typo3.org/c/typo3/cms-core/master/en-us/Changelog/10.4/Feature-91008-ItemGroupingForTCASelectItems.html}{feature 91008 (grouping)}
			za više detalja.
	\end{itemize}

\end{frame}

% ------------------------------------------------------------------------------
% Feature | 91008 | Item sorting for TCA select items

\begin{frame}[fragile]
	\frametitle{Izmene za programere}
	\framesubtitle{TCA selekt stavke (sortiranje)}

	\begin{itemize}
		\item Dodata je nova opcija \texttt{sortOrders} za TCA selekt polja.
		\item Ovo omogućava sortiranje stavki po vrednosti ili labeli.
		\item Pogledajte
			\href{https://docs.typo3.org/c/typo3/cms-core/master/en-us/Changelog/10.4/Feature-91008-ItemSortingForTCASelectItems.html}{feature 91008 (sorting)}
			za više detalja.
	\end{itemize}

\end{frame}

% ------------------------------------------------------------------------------
% Deprecation | 90377 | Deprecate ref param types of method callUserFunction

\begin{frame}[fragile]
	\frametitle{Izmene za programere}
	\framesubtitle{GeneralUtility}

	\begin{itemize}
		\item Treći argument metode \texttt{callUserFunction()}
			treba da bude objekat ili \texttt{null}.
		\item Bilo koji drugi tip podatka prosledjen kao argument \texttt{\$ref} generiše sledeće upozorenje
			\texttt{E\_USER\_DEPRECATED}.
	\end{itemize}

\end{frame}

% ------------------------------------------------------------------------------
% Important | 91079 | Various TypoScriptFrontendRenderer functionality is now internal

\begin{frame}[fragile]
	\frametitle{Izmene za programere}
	\framesubtitle{TypoScriptFrontendController}

	\begin{itemize}
		\item Sledeće osobine su sada označene kao \textbf{unutrašnje}:
			\begin{itemize}
				\item \texttt{TypoScriptFrontendController->sPre}
				\item \texttt{TypoScriptFrontendController->pSetup}
				\item \texttt{TypoScriptFrontendController->all}
				\item \texttt{TypoScriptFrontendController->additionalJavaScript}
				\item \texttt{TypoScriptFrontendController->additionalCSS}
				\item \texttt{TypoScriptFrontendController->JSCode}
				\item \texttt{TypoScriptFrontendController->inlineJS}
				\item \texttt{TypoScriptFrontendController->indexedDocTitle}
			\end{itemize}

		\item Sledeći metod je sada označen kao \textbf{unutrašnji}:

			\begin{itemize}
				\item \texttt{TypoScriptFrontendController->setJS()}
			\end{itemize}

	\end{itemize}

\end{frame}

% ------------------------------------------------------------------------------
% Feature | 90613 | Add language argument to page-related LinkViewHelpers and UriViewHelpers in Fluid

\begin{frame}[fragile]
	\frametitle{Izmene za programere}
	\framesubtitle{LinkViewHelpers i UriViewHelpers}

	% decrease font size for code listing
	\lstset{basicstyle=\tiny\ttfamily}

	\begin{itemize}
		\item Novi argument \texttt{language} je dodat sledećim ViewHelper-ima:
			\begin{itemize}
				\item \texttt{<f:link.typolink>}
				\item \texttt{<f:link.page>}
				\item \texttt{<f:uri.typolink>}
				\item \texttt{<f:uri.page>}
			\end{itemize}

		\item Ovaj argument upućuje na specifičan jezik stranice.
		\item Primer (jezik sa ID 3):
\begin{lstlisting}
Idi na
<f:link.page pageUid="42" language="3">francusku verziju</f:link.page>
"Kontakt" stranice.
\end{lstlisting}

	\end{itemize}

\end{frame}

% ------------------------------------------------------------------------------
% Feature | 90899 | Introduce AssetRenderer pre-rendering events

\begin{frame}[fragile]
	\frametitle{Izmene za programere}
	\framesubtitle{AssetCollector API}

	\begin{itemize}
		\item Kada se koristi AssetCollector API, CSS i JavaScript aseti
			 mogu, ukoliko je potrebno, da se procesuiraju kasnije.
		\item Sledeći dogadjaji se okidaju za ove potrebe:
			\begin{itemize}\smaller
				\item \texttt{TYPO3\textbackslash
					CMS\textbackslash
					Core\textbackslash
					Page\textbackslash
					Event\textbackslash
					BeforeStylesheetsRenderingEvent}
				\item \texttt{TYPO3\textbackslash
					CMS\textbackslash
					Core\textbackslash
					Page\textbackslash
					Event\textbackslash
					BeforeJavaScriptsRenderingEvent}
			\end{itemize}

		\item Pogledajte
			\href{https://docs.typo3.org/c/typo3/cms-core/master/en-us/Changelog/10.4/Feature-90899-IntroduceAssetPreRenderingEvents.html}{change log}
			za više detalja, primera i dodatnih objašnjenja.
	\end{itemize}

\end{frame}

% ------------------------------------------------------------------------------
