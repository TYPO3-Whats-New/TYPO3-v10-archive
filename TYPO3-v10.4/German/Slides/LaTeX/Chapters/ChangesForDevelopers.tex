% ------------------------------------------------------------------------------
% TYPO3 Version 10.4 - What's New (German Version)
%
% @license	Creative Commons BY-NC-SA 3.0
% @link		https://typo3.org/help/documentation/whats-new/
% @language	German
% ------------------------------------------------------------------------------

\section{Änderungen für Entwickler}
\begin{frame}[fragile]
	\frametitle{Änderungen für Entwickler}

	\begin{center}\huge{Kapitel 3:}\end{center}
	\begin{center}\huge{\color{typo3darkgrey}\textbf{Änderungen für Entwickler}}\end{center}

\end{frame}

% ------------------------------------------------------------------------------
% Breaking | 90660 | Registration of dashboard widgets changed

\begin{frame}[fragile]
	\frametitle{Änderungen für Entwickler}
	\framesubtitle{Dashboard-Widgets (1)}

	Änderungen zwischen TYPO3 v10.3 und v10.4:

	\begin{itemize}
		\item Die Art und Weise, wie Dashboard-Widgets registriert werden, hat sich geändert.
		\item Abstrakte Klassen werden nicht mehr verwendet, da Widgets in der Datei
			\texttt{Services.yaml} registriert (und konfiguriert) werden.
		\item Einige Widget-Typen können jetzt nur durch Konfiguration erstellt werden.
		\item Zusätzlich sind die gültigen Schlüsselwörter für Höhe und Breite jetzt
			"\texttt{small}", "\texttt{medium}", oder "\texttt{large}"
			(anstelle von numerischen Werten).
	\end{itemize}

\end{frame}

% ------------------------------------------------------------------------------
% Breaking | 91066 | Removed ButtonUtility
% Breaking | 91066 | Move interfaces of Dashboard

\begin{frame}[fragile]
	\frametitle{Änderungen für Entwickler}
	\framesubtitle{Dashboard-Widgets (2)}

	Änderungen zwischen TYPO3 v10.3 und v10.4:

	\begin{itemize}
		\item Die Klasse \texttt{ButtonUtility} wurde entfernt.
		\item Schnittstellen wurden verschoben und ihre Namespaces müssen wahrscheinlich
			in Ihrem benutzerdefinierten Code aktualisiert werden.
			\begin{itemize}\smaller
				\item \textbf{ALT:}
					\texttt{TYPO3\textbackslash
						CMS\textbackslash
						Dashboard\textbackslash
						Widgets\textbackslash
						Interfaces}
				\item \textbf{NEU:}
					\texttt{TYPO3\textbackslash
						CMS\textbackslash
						Dashboard\textbackslash
						Widgets}
			\end{itemize}\normalsize
	\end{itemize}

\end{frame}

% ------------------------------------------------------------------------------
% Important | 77715 | No more password trimming for third-party authentication services

\begin{frame}[fragile]
	\frametitle{Änderungen für Entwickler}
	\framesubtitle{Third-party Authentication Services}

	\begin{itemize}

		\item Erweiterungen können die Authentifizierungsdienst-API von TYPO3 verwenden, um Benutzer gegenüber
			Identitätsvermittlern über "OAuth", "LDAP", "SAML2", usw. zu authentifizieren.
		\item Das \texttt{AbstractUserAuthentication}-Objekt von TYPO3 beschneidet die Passwörter
			dieser third-party services nicht mehr.
		\item Dies hat jedoch nichts mit dem nativen Authentifizierungsdienst von TYPO3 zu tun,
			der immer noch ein Passwort ohne führende/nachlaufende Leerzeichen erfordert.

	\end{itemize}

\end{frame}

% ------------------------------------------------------------------------------
% Feature | 89573 | Allow flexible base url for slug fields in FormEngine

\begin{frame}[fragile]
	\frametitle{Änderungen für Entwickler}
	\framesubtitle{TCA: Basis-URL-Präfix}

	% decrease font size for code listing
	\lstset{basicstyle=\tiny\ttfamily}

	\begin{itemize}

		\item Es ist jetzt möglich, eine benutzerdefinierte Basis-URL für TCA-Spalten vom Typ \texttt{slug} hinzuzufügen.
		\item Die Basis-URL wird vor dem Eingabefeld angezeigt (Präfix).
		\item Beispiel (TCA):

\vspace{-0.4cm}
\begin{lstlisting}
...
'config' => [
  'type' => 'slug',
  'appearance' => [
    'prefix' => \Vendor\MyExtension\UserFunctions\FormEngine\SlugPrefix::class . '->getPrefix'
  ]
]
...
\end{lstlisting}

	\end{itemize}

\end{frame}

% ------------------------------------------------------------------------------
% Feature | 87776 | Limit Restriction to table/s in QueryBuilder

\begin{frame}[fragile]
	\frametitle{Änderungen für Entwickler}
	\framesubtitle{QueryBuilder}

	% decrease font size for code listing
	\lstset{basicstyle=\tiny\ttfamily}

	\begin{itemize}
		\item Es ist jetzt möglich, Abfragebeschränkungen
			für einen bestimmten Satz von Tabellen (genauer gesagt: Tabellen-Aliase) anzuwenden.
		\item Der folgende Restriktionsbehälter kann verwendet werden:\newline
			\begingroup
				\fontsize{7}{9}
					\texttt{TYPO3\textbackslash
						CMS\textbackslash
						Core\textbackslash
						Database\textbackslash
						Query\textbackslash
						Restriction\textbackslash
						LimitToTablesRestrictionContainer}
			\endgroup

		\item Beispiel:
\begin{lstlisting}
$queryBuilder = GeneralUtility::makeInstance(ConnectionPool::class)
  ->getQueryBuilderForTable('tt_content');
$queryBuilder->getRestrictions()
  ->removeByType(HiddenRestriction::class)
  ->add(
    GeneralUtility::makeInstance(LimitToTablesRestrictionContainer::class)
      ->addForTables(GeneralUtility::makeInstance(HiddenRestriction::class), ['tt'])
  );

$queryBuilder->select('tt.uid')->from('tt_content', 'tt');
\end{lstlisting}

	\end{itemize}

\end{frame}

% ------------------------------------------------------------------------------
% Feature | 91008 | Item grouping for TCA select items

\begin{frame}[fragile]
	\frametitle{Änderungen für Entwickler}
	\framesubtitle{TCA Select Items (Gruppierung)}

	\begin{itemize}
		\item Seitentypen (\texttt{pages.doktype}), Inhaltstypen (\texttt{tt\_content.CType}),
			und Plugins (\texttt{tt\_content.list\_type}) haben jetzt die native Gruppierung aktiviert.
		\item Zuvor wurde dies durch die Anwendung von
			"\texttt{-}\texttt{-div-}\texttt{-}"-Punkten erledigt.
		\item Entwicklern wird empfohlen, "\texttt{-}\texttt{-div-}\texttt{-}"
			-Elemente aus benutzerdefinierten Auswahlmöglichkeiten
			zu entfernen und stattdessen \texttt{itemGroups} zu verwenden.
		\item Siehe
			\href{https://docs.typo3.org/c/typo3/cms-core/master/en-us/Changelog/10.4/Feature-91008-ItemGroupingForTCASelectItems.html}{Feature 91008 (Gruppierung)}
			für weitere Details.
	\end{itemize}

\end{frame}

% ------------------------------------------------------------------------------
% Feature | 91008 | Item sorting for TCA select items

\begin{frame}[fragile]
	\frametitle{Änderungen für Entwickler}
	\framesubtitle{TCA Select Items (Sortierung)}

	\begin{itemize}
		\item Es wurde eine neue Option \texttt{sortOrders} für TCA-basierte Auswahlfelder hinzugefügt.
		\item Dies ermöglicht die Sortierung von statischen TCA-Auswahlfeldern nach ihren Werten oder Labels.
		\item Siehe
			\href{https://docs.typo3.org/c/typo3/cms-core/master/en-us/Changelog/10.4/Feature-91008-ItemSortingForTCASelectItems.html}{Feature 91008 (Sortierung)}
			für weitere Details.
	\end{itemize}

\end{frame}

% ------------------------------------------------------------------------------
% Deprecation | 90377 | Deprecate ref param types of method callUserFunction

\begin{frame}[fragile]
	\frametitle{Änderungen für Entwickler}
	\framesubtitle{GeneralUtility}

	\begin{itemize}
		\item Das dritte Argument der Methode \texttt{callUserFunction()}
			sollte ein Objekt oder \texttt{null} sein.
		\item Alle anderen Daten, die als Argument \texttt{\$ref} übergeben werden,
			erzeugen jetzt eine Warnung \texttt{E\_USER\_DEPRECATED}.
	\end{itemize}

\end{frame}

% ------------------------------------------------------------------------------
% Important | 91079 | Various TypoScriptFrontendRenderer functionality is now internal

\begin{frame}[fragile]
	\frametitle{Änderungen für Entwickler}
	\framesubtitle{TypoScriptFrontendController}

	\begin{itemize}
		\item Die folgenden Eigenschaften sind jetzt als \textbf{intern} gekennzeichnet:
			\begin{itemize}
				\item \texttt{TypoScriptFrontendController->sPre}
				\item \texttt{TypoScriptFrontendController->pSetup}
				\item \texttt{TypoScriptFrontendController->all}
				\item \texttt{TypoScriptFrontendController->additionalJavaScript}
				\item \texttt{TypoScriptFrontendController->additionalCSS}
				\item \texttt{TypoScriptFrontendController->JSCode}
				\item \texttt{TypoScriptFrontendController->inlineJS}
				\item \texttt{TypoScriptFrontendController->indexedDocTitle}
			\end{itemize}

		\item Die folgende Methode ist jetzt \textbf{intern} markiert:

			\begin{itemize}
				\item \texttt{TypoScriptFrontendController->setJS()}
			\end{itemize}

	\end{itemize}

\end{frame}

% ------------------------------------------------------------------------------
% Feature | 90613 | Add language argument to page-related LinkViewHelpers and UriViewHelpers in Fluid

\begin{frame}[fragile]
	\frametitle{Änderungen für Entwickler}
	\framesubtitle{LinkViewHelpers und UriViewHelpers}

	% decrease font size for code listing
	\lstset{basicstyle=\tiny\ttfamily}

	\begin{itemize}
		\item Das neue Argument \texttt{language} wurde den folgenden ViewHelpers hinzugefügt:
			\begin{itemize}
				\item \texttt{<f:link.typolink>}
				\item \texttt{<f:link.page>}
				\item \texttt{<f:uri.typolink>}
				\item \texttt{<f:uri.page>}
			\end{itemize}

		\item Dieses Argument verweist auf eine bestimmte Sprache einer Seite.
		\item Beispiel (language ID 3):
\begin{lstlisting}
Go to the
<f:link.page pageUid="42" language="3">french version</f:link.page>
of the "Contact Us" page.
\end{lstlisting}

	\end{itemize}

\end{frame}

% ------------------------------------------------------------------------------
% Feature | 90899 | Introduce AssetRenderer pre-rendering events

\begin{frame}[fragile]
	\frametitle{Änderungen für Entwickler}
	\framesubtitle{LinkViewHelpers und UriViewHelpers}

	\begin{itemize}
		\item Wenn die AssetCollector-API verwendet wird, können CSS- und JavaScript-Assets
			bei Bedarf nachbearbeitet werden.
		\item Zu diesem Zweck werden die folgenden zwei Events durchgeführt:
			\begin{itemize}\smaller
				\item \texttt{TYPO3\textbackslash
					CMS\textbackslash
					Core\textbackslash
					Page\textbackslash
					Event\textbackslash
					BeforeStylesheetsRenderingEvent}
				\item \texttt{TYPO3\textbackslash
					CMS\textbackslash
					Core\textbackslash
					Page\textbackslash
					Event\textbackslash
					BeforeJavaScriptsRenderingEvent}
			\end{itemize}

		\item Siehe
			\href{https://docs.typo3.org/c/typo3/cms-core/master/en-us/Changelog/10.4/Feature-90899-IntroduceAssetPreRenderingEvents.html}{change log}
			für weitere Details, Beispiele und zusätzliche Hinweise.
	\end{itemize}

\end{frame}

% ------------------------------------------------------------------------------
