% ------------------------------------------------------------------------------
% TYPO3 Version 10.0 - What's New (English Version)
%
% @author	Michael Schams <schams.net>
% @license	Creative Commons BY-NC-SA 3.0
% @link		http://typo3.org/download/release-notes/whats-new/
% @language	English
% ------------------------------------------------------------------------------

\section{Modifiche per integratori e sviluppatori}
\begin{frame}[fragile]
	\frametitle{Modifiche per integratori e sviluppatori}

	\begin{center}\huge{Capitolo 2:}\end{center}
	\begin{center}\huge{\color{typo3darkgrey}\textbf{Modifiche per integratori e sviluppatori}}\end{center}

\end{frame}

% ------------------------------------------------------------------------------
% Breaking | 88376 | Removed obsolete pageNotFound_handling settings

\begin{frame}[fragile]
	\frametitle{Modifiche per integratori e sviluppatori}
	\framesubtitle{Gestione pagina non trovata}

	\begin{itemize}

		\item Le seguenti impostazioni globali di TYPO3 sono state rimosse:

			\begin{itemize}
				\item {\fontsize{7}{8}\selectfont\texttt{\$GLOBALS['TYPO3\_CONF\_VARS']['FE']['pageNotFound\_handling']}}
				\item {\fontsize{7}{8}\selectfont\texttt{\$GLOBALS['TYPO3\_CONF\_VARS']['FE']['pageNotFound\_handling\_statheader']}}
				\item {\fontsize{7}{8}\selectfont\texttt{\$GLOBALS['TYPO3\_CONF\_VARS']['FE']['pageNotFound\_handling\_accessdeniedheader']}}
				\item {\fontsize{7}{8}\selectfont\texttt{\$GLOBALS['TYPO3\_CONF\_VARS']['FE']['pageUnavailable\_handling']}}
				\item {\fontsize{7}{8}\selectfont\texttt{\$GLOBALS['TYPO3\_CONF\_VARS']['FE']['pageUnavailable\_handling\_statheader']}}
			\end{itemize}

			\begin{itemize}\smaller
				\item[\ding{228}] La gestione di sito introdotta in TYPO3 v9 ha sostituito queste impostazioni.
			\end{itemize}\normalsize

	\end{itemize}

\end{frame}

% ------------------------------------------------------------------------------
% Important | 87980 | Page Is Being Generated Message Has Been Removed

\begin{frame}[fragile]
	\frametitle{Modifiche per integratori e sviluppatori}
	\framesubtitle{Gestione pagina non trovata}

	\begin{itemize}

		\item Il messaggio "\textit{Page is being generated}" e la corrispondente risposta temporanea
			HTTP 503 sono stati rimossi.
	\end{itemize}

	\begin{figure}
		\includegraphics[width=0.70\linewidth]{ChangesForIntegratorsAndDevelopers/87980-PageIsBeingGenerated.png}
	\end{figure}

	\begin{itemize}
		\item Invece di fermare l'esecuzione in attesa del contenuto della pagina finale, le richieste simultanee
		rimangono in attesa del rendering del contenuto della pagina reale.
	\end{itemize}

\end{frame}

% ------------------------------------------------------------------------------
% Breaking | 86862 | Default Layout of ext:fluid_styled_content does not use spaceless viewHelper anymore

\begin{frame}[fragile]
	\frametitle{Deprecated/Removed Functions}
	\framesubtitle{Fluid}

	\begin{itemize}
		\item La rimozione di spazi bianchi nel layout di default di \texttt{EXT:fluid\_styled\_content}
			è stata causa di problemi saltuari; è stata quindi rimossa.

	\end{itemize}

\end{frame}

% ------------------------------------------------------------------------------
% Feature | 80420 | Allow multiple recipients in email finisher

\begin{frame}[fragile]
	\frametitle{Modifiche per integratori e sviluppatori}
	\framesubtitle{Form Framework: Destinatari multipli (1)}

	\begin{itemize}
		\item Le email inviate tramite \textit{EmailFinisher} possono ora avere destinatari multipli.

		\item Sono state introdotte le seguenti nuovi opzioni:

			\begin{itemize}
				\item \texttt{recipients} (To)
				\item \texttt{replyToRecipients} (Reply-To)
				\item \texttt{carbonCopyRecipients} (CC)
				\item \texttt{blindCarbonCopyRecipients} (BCC)
			\end{itemize}

	\end{itemize}

\end{frame}

% ------------------------------------------------------------------------------
% Feature | 80420 | Allow multiple recipients in email finisher

\begin{frame}[fragile]
	\frametitle{Deprecated/Removed Functions}
	\framesubtitle{Form Framework: Destinatari multipli (2)}

	% decrease font size for code listing
	\lstset{basicstyle=\tiny\ttfamily}

	\begin{itemize}
		\item Questo cambiamento richiede una migrazione manuale delle opzioni singole nelle opzioni di elenco.

		\smaller\textbf{Vecchia} configurazione del Finisher:\normalsize

\begin{lstlisting}
finishers:
  -
    identifier: EmailToReceiver
    options:
      recipientAddress: user@example.com
      recipientName: 'Firstname Lastname'
\end{lstlisting}

		\smaller\textbf{Nuova} configurazione del Finisher:\normalsize

\begin{lstlisting}
finishers:
  -
    identifier: EmailToReceiver
    options:
      recipients:
        user@example.com: 'Firstname Lastname'
\end{lstlisting}

		\item Vedi il \href{https://docs.typo3.org/c/typo3/cms-core/10.0/en-us/Changelog/master/Deprecation-80420-EmailFinisherSingleAddressOptions.html}{log dei cambiamenti}
			per maggiori esempi di migrazione.

	\end{itemize}

\end{frame}

% ------------------------------------------------------------------------------
% Deprecation | 87200 | EmailFinisher format option
% Deprecation | 87200 | EmailFinisher FORMAT_* constants

\begin{frame}[fragile]
	\frametitle{Deprecated/Removed Functions}
	\framesubtitle{Form Framework: Plaintext/HTML (1)}

	\begin{itemize}
		\item Le email inviate da \textit{EmailFinisher} possono ora contenere testo in chiaro e/o in formato HTML.

		\item Allo stesso tempo l'opzione \texttt{format} è stata segnata come deprecata e sarà rimossa in TYPO3 v11.

		\item I valori esistenti verranno migrati automaticamente:

			\begin{itemize}\smaller
				\item \texttt{format:html} \tabto{3cm}\textrightarrow\hspace{0.1cm}\texttt{addHtmlPart:\textbf{true}}
				\item \texttt{format:plaintext} \tabto{3cm}\textrightarrow\hspace{0.1cm}\texttt{addHtmlPart:\textbf{false}}
				\item a missing "\texttt{format}" \tabto{3cm}\textrightarrow\hspace{0.1cm}\texttt{addHtmlPart:\textbf{true}}
			\end{itemize}\normalsize

		\item Gli sviluppatori dovrebbero essere consapevoli che le due costanti seguenti sono state segnate come deprecate:

			\begin{itemize}\smaller
				\item \texttt{EmailFinisher::FORMAT\_PLAINTEXT}
				\item \texttt{EmailFinisher::FORMAT\_HTML}
			\end{itemize}\normalsize

	\end{itemize}

\end{frame}

% ------------------------------------------------------------------------------
% Feature | 84262 | Update EXT:felogin to Extbase (long-term goal, not completed yet)
% Breaking | 88706 | Streamline felogin locallang keys
% Breaking | 88129 | Renamed felogin flexform fields

\begin{frame}[fragile]
	\frametitle{Deprecated/Removed Functions}
	\framesubtitle{Frontend Login: Extbase}

	\begin{itemize}
		\item Il login dell'utente di frontend (\texttt{EXT:felogin}) è stato convertito in Extbase e Fluid.

		\item Questo è un obiettivo a lungo termine, attualmente in corso di lavorazione.\newline
			Vedi \href{https://forge.typo3.org/issues/84262}{forge \#84262} per maggiori dettagli.

		\item Le seguenti modifiche sono state implementate in TYPO3 v10.0:

		\begin{itemize}
			\item[\ding{202}] Il prefisso "\texttt{ll\_}" è stato rimosso dalle chiavi locallang.

				\begin{itemize}
					\item[\ding{228}] Aggiorna il tuo TypoScript se hai sovrascritto le etichette in lingua rimuovendo il prefisso "\texttt{ll\_}" dalle tue chiavi.
				\end{itemize}

			\item[\ding{203}] La struttura del FlexForm esistente è stata rielaborata.

				\begin{itemize}
					\item[\ding{228}] Esegui la procedura guidata di aggiornamento per migrare i valori del FlexForm.
				\end{itemize}

		\end{itemize}

	\end{itemize}

\end{frame}

% ------------------------------------------------------------------------------
% Breaking | 88583 | Database field sys_language.static_lang_isocode removed
% Deprecation | 88567 | $GLOBALS['LOCAL_LANG']

\begin{frame}[fragile]
	\frametitle{Deprecated/Removed Functions}
	\framesubtitle{Lingue}

	\begin{itemize}
		\item Codici ISO:

			\begin{itemize}
				\item Il campo del database inutilizzato \texttt{static\_lang\_isocode} è stato rimosso.
				\item L'estensione \texttt{EXT:static\_info\_tables} può essere installata per reimplementare la funzionalità se necessario.
				\item Si consiglia agli sviluppatori di recuperare tutti i metadati di una lingua utilizzando il Configuratore di Sito e le API SiteLanguage API.
			\end{itemize}

		\item File di lingua:

			\begin{itemize}
				\item L'uso dell'array globale \texttt{\$GLOBALS[LOCAL\_LANG]} è stato deprecato.
				\item Il secondo e terzo argomento di \texttt{LanguageService->includeLLFile()} sono stati deprecati.
			\end{itemize}

	\end{itemize}

\end{frame}

% ------------------------------------------------------------------------------
% Feature | 21638 | Introduced IP locking for IPv6
% Breaking | 21638 | AbstractUserAuthentication::lockIP property removed

\begin{frame}[fragile]
	\frametitle{Changes for Integrators}
	\framesubtitle{Varie}

	% decrease font size for code listing
	\lstset{basicstyle=\tiny\ttfamily}

	\begin{itemize}

		\item  La funzionalità di blocco IP è stata estesa anche a IPv6
			(frontend e backend).

\begin{lstlisting}
$GLOBALS['TYPO3_CONF_VARS']['FE']['lockIPv6'] = 2;
$GLOBALS['TYPO3_CONF_VARS']['BE']['lockIPv6'] = 2;
\end{lstlisting}

		\item La proprietà pubblica \texttt{lockIP} nella seguente classe PHP è stata rimossa:\newline
			\small
				\texttt{\textbackslash
					TYPO3\textbackslash
					CMS\textbackslash
					Core\textbackslash
					Authentication\textbackslash
					AbstractUserAuthentication}.
			\normalsize

		\item Opzioni di migrazione:

			\begin{itemize}\smaller
				\item[\ding{228}] Imposta \texttt{lockIP} e \texttt{lockIPv6} in \texttt{\$GLOBALS['TYPO3\_CONF\_VARS'][...]}.
				\item[\ding{228}] Usa le nuovi API IP-Locker:
					\texttt{\textbackslash
						TYPO3\textbackslash
						CMS\textbackslash
						Core\textbackslash
						Authentication\textbackslash
						IpLocker}.
			\end{itemize}\normalsize


	\end{itemize}

\end{frame}

% ------------------------------------------------------------------------------
