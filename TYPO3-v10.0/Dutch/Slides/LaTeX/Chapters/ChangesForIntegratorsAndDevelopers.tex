% ------------------------------------------------------------------------------
% TYPO3 Version 10.0 - What's New (English Version)
%
% @author	Michael Schams <schams.net>
% @license	Creative Commons BY-NC-SA 3.0
% @link		http://typo3.org/download/release-notes/whats-new/
% @language	Dutch
% ------------------------------------------------------------------------------

\section{Wijzigingen voor Integrators en Ontwikkelaars}
\begin{frame}[fragile]
	\frametitle{Wijzigingen voor Integrators en Ontwikkelaars}

	\begin{center}\huge{Hoofdstuk 2:}\end{center}
	\begin{center}\huge{\color{typo3darkgrey}\textbf{Wijzigingen voor Integrators en Ontwikkelaars}}\end{center}

\end{frame}

% ------------------------------------------------------------------------------
% Breaking | 88376 | Removed obsolete pageNotFound_handling settings

\begin{frame}[fragile]
	\frametitle{Wijzigingen voor Integrators en Ontwikkelaars}
	\framesubtitle{Afhandeling pagina niet gevonden}

	\begin{itemize}

		\item De volgende globale TYPO3 instellingen zijn verwijderd:

			\begin{itemize}
				\item {\fontsize{7}{8}\selectfont\texttt{\$GLOBALS['TYPO3\_CONF\_VARS']['FE']['pageNotFound\_handling']}}
				\item {\fontsize{7}{8}\selectfont\texttt{\$GLOBALS['TYPO3\_CONF\_VARS']['FE']['pageNotFound\_handling\_statheader']}}
				\item {\fontsize{7}{8}\selectfont\texttt{\$GLOBALS['TYPO3\_CONF\_VARS']['FE']['pageNotFound\_handling\_accessdeniedheader']}}
				\item {\fontsize{7}{8}\selectfont\texttt{\$GLOBALS['TYPO3\_CONF\_VARS']['FE']['pageUnavailable\_handling']}}
				\item {\fontsize{7}{8}\selectfont\texttt{\$GLOBALS['TYPO3\_CONF\_VARS']['FE']['pageUnavailable\_handling\_statheader']}}
			\end{itemize}

			\begin{itemize}\smaller
				\item[\ding{228}] De site-afhandeling die in TYPO3 v9 is ge\"{\i}ntroduceerd vervangt deze opties.
			\end{itemize}\normalsize

	\end{itemize}

\end{frame}

% ------------------------------------------------------------------------------
% Important | 87980 | Page Is Being Generated Message Has Been Removed

\begin{frame}[fragile]
	\frametitle{Wijzigingen voor Integrators en Ontwikkelaars}
	\framesubtitle{Afhandeling pagina niet gevonden}

	\begin{itemize}

		\item De melding "\textit{Page is being generated}" en het bijbehorende tijdelijke
			HTTP 503 antwoord zijn verwijderd.
	\end{itemize}

	\begin{figure}
		\includegraphics[width=0.70\linewidth]{ChangesForIntegratorsAndDevelopers/87980-PageIsBeingGenerated.png}
	\end{figure}

	\begin{itemize}
		\item In plaats van het uitstellen van het werk en te wachten op de uiteindelijk pagina-inhoud wachten
			gelijktijdige requests nu gewoon tot de echte pagina-inhoud is gemaakt.
	\end{itemize}

\end{frame}

% ------------------------------------------------------------------------------
% Breaking | 86862 | Default Layout of ext:fluid_styled_content does not use spaceless viewHelper anymore

\begin{frame}[fragile]
	\frametitle{Verouderde/verwijderde functies}
	\framesubtitle{Fluid}

	\begin{itemize}
		\item Het verwijderen van witruimte in het standaard lay-outbestand van \texttt{EXT:fluid\_styled\_content}
			veroorzaakte af en toe problemen en is verwijderd.

	\end{itemize}

\end{frame}

% ------------------------------------------------------------------------------
% Feature | 80420 | Allow multiple recipients in email finisher

\begin{frame}[fragile]
	\frametitle{Verouderde/verwijderde functies}
	\framesubtitle{Formulier-raamwerk: Meerdere ontvangers (1)}

	\begin{itemize}
		\item Mails kunnen door de \textit{EmailFinisher} verstuurd worden aan meerdere ontvangers.

		\item De volgende nieuwe opties zijn ge\"{\i}ntroduceerd:

			\begin{itemize}
				\item \texttt{recipients} (To)
				\item \texttt{replyToRecipients} (Reply-To)
				\item \texttt{carbonCopyRecipients} (CC)
				\item \texttt{blindCarbonCopyRecipients} (BCC)
			\end{itemize}

	\end{itemize}

\end{frame}

% ------------------------------------------------------------------------------
% Feature | 80420 | Allow multiple recipients in email finisher

\begin{frame}[fragile]
	\frametitle{Verouderde/verwijderde functies}
	\framesubtitle{Formulier-raamwerk: Meerdere ontvangers (2)}

	% decrease font size for code listing
	\lstset{basicstyle=\tiny\ttfamily}

	\begin{itemize}
		\item Deze wijziging vergt het handmatig aanpassen van enkele waarden naar lijsten.

		\smaller\textbf{Oude} Finisher configuratie:\normalsize

\begin{lstlisting}
finishers:
  -
    identifier: EmailToReceiver
    options:
      recipientAddress: user@example.com
      recipientName: 'Voornaam Achternaam'
\end{lstlisting}

		\smaller\textbf{Nieuwe} Finisher configuratie:\normalsize

\begin{lstlisting}
finishers:
  -
    identifier: EmailToReceiver
    options:
      recipients:
        user@example.com: 'Voornaam Achternaam'
\end{lstlisting}

		\item Zie \href{https://docs.typo3.org/c/typo3/cms-core/10.0/en-us/Changelog/master/Deprecation-80420-EmailFinisherSingleAddressOptions.html}{change log}
			voor meer migratievoorbeelden.

	\end{itemize}

\end{frame}

% ------------------------------------------------------------------------------
% Deprecation | 87200 | EmailFinisher format option
% Deprecation | 87200 | EmailFinisher FORMAT_* constants

\begin{frame}[fragile]
	\frametitle{Verouderde/verwijderde functies}
	\framesubtitle{Formulier-raamwerk: Platte tekst/HTML (1)}

	\begin{itemize}
		\item Mails verstuurd door de \textit{EmailFinisher} kunnen zowel een HTML en/of tekstdeel hebben.

		\item Tegelijkertijd is de optie \texttt{format} aangemerkt als verouderd en zal verwijderd worden in TYPO3 v11.

		\item Bestaande waarden zullen automatisch gemigreerd worden:

			\begin{itemize}\smaller
				\item \texttt{format:html} \tabto{3cm}\textrightarrow\hspace{0.1cm}\texttt{addHtmlPart:\textbf{true}}
				\item \texttt{format:plaintext} \tabto{3cm}\textrightarrow\hspace{0.1cm}\texttt{addHtmlPart:\textbf{false}}
				\item een ontbrekende "\texttt{format}" \tabto{3cm}\textrightarrow\hspace{0.1cm}\texttt{addHtmlPart:\textbf{true}}
			\end{itemize}\normalsize

		\item Ontwikkelaars moeten letten op de volgende twee constanten die als verouderd gemarkeerd zijn:

			\begin{itemize}\smaller
				\item \texttt{EmailFinisher::FORMAT\_PLAINTEXT}
				\item \texttt{EmailFinisher::FORMAT\_HTML}
			\end{itemize}\normalsize

	\end{itemize}

\end{frame}

% ------------------------------------------------------------------------------
% Feature | 84262 | Update EXT:felogin to Extbase (long-term goal, not completed yet)
% Breaking | 88706 | Streamline felogin locallang keys
% Breaking | 88129 | Renamed felogin flexform fields

\begin{frame}[fragile]
	\frametitle{Verouderde/verwijderde functies}
	\framesubtitle{Frontend-aanmelding: Extbase}

	\begin{itemize}
		\item De aanmelding voor frontend-gebruikers (\texttt{EXT:felogin}) zal omgezet worden naar Extbase/Fluid.

		\item Dit is een langeretermijndoel, momenteel werk in uitvoering.\newline
			Zie \href{https://forge.typo3.org/issues/84262}{forge \#84262} voor meer details.

		\item De volgende wijzigingen zijn ge\"{\i}mplementeerd in TYPO3 v10.0:

		\begin{itemize}
			\item[\ding{202}] Prefix "\texttt{ll\_}" is verwijderd van locallang-sleutels.

				\begin{itemize}
					\item[\ding{228}] Werk TypoScript bij als er taallabels zijn overschreven en verwijder de prefix "\texttt{ll\_}".
				\end{itemize}

			\item[\ding{203}] Bestaande FlexForm structuren zijn bijgewerkt.

				\begin{itemize}
					\item[\ding{228}] Voer de Upgrade-assistent uit om de FLexForm-waarden te migreren.
				\end{itemize}

		\end{itemize}

	\end{itemize}

\end{frame}

% ------------------------------------------------------------------------------
% Breaking | 88583 | Database field sys_language.static_lang_isocode removed
% Deprecation | 88567 | $GLOBALS['LOCAL_LANG']

\begin{frame}[fragile]
	\frametitle{Verouderde/verwijderde functies}
	\framesubtitle{Talen}

	\begin{itemize}
		\item ISO Codes:

			\begin{itemize}
				\item Het ongebruikte database-veld \texttt{static\_lang\_isocode} is verwijderd.
				\item \texttt{EXT:static\_info\_tables} kan worden ge\"{\i}nstalleerd om de functionaliteit te herintroduceren.
				\item Ontwikkelaars wordt aangeraden alle metadata van een taal uit de Site Configuration en de SiteLanguage API te halen.
			\end{itemize}

		\item Taalbestanden:

			\begin{itemize}
				\item Gebruik van de globale array \texttt{\$GLOBALS[LOCAL\_LANG]} is als verouderd aangemerkt.
				\item Het 2e en 3e argument van \texttt{LanguageService->includeLLFile()} is als verouderd aangemerkt.
			\end{itemize}

	\end{itemize}

\end{frame}

% ------------------------------------------------------------------------------
% Feature | 21638 | Introduced IP locking for IPv6
% Breaking | 21638 | AbstractUserAuthentication::lockIP property removed

\begin{frame}[fragile]
	\frametitle{Wijzigingen voor Integrators}
	\framesubtitle{Diversen}

	% decrease font size for code listing
	\lstset{basicstyle=\tiny\ttfamily}

	\begin{itemize}

		\item De functie voor het vastzetten op IP is uitgebreid om ook IPv6 te ondersteunen
			(frontend en backend).

\begin{lstlisting}
$GLOBALS['TYPO3_CONF_VARS']['FE']['lockIPv6'] = 2;
$GLOBALS['TYPO3_CONF_VARS']['BE']['lockIPv6'] = 2;
\end{lstlisting}

		\item De publieke eigenschap \texttt{lockIP} in de volgende PHP-klasse is verwijderd:\newline
			\small
				\texttt{\textbackslash
					TYPO3\textbackslash
					CMS\textbackslash
					Core\textbackslash
					Authentication\textbackslash
					AbstractUserAuthentication}.
			\normalsize

		\item Migratiemogelijkheden:

			\begin{itemize}\smaller
				\item[\ding{228}] Stel \texttt{lockIP} en \texttt{lockIPv6} in binnen \texttt{\$GLOBALS['TYPO3\_CONF\_VARS'][...]}.
				\item[\ding{228}] Gebruik de nieuwe IP-Locker API:
					\texttt{\textbackslash
						TYPO3\textbackslash
						CMS\textbackslash
						Core\textbackslash
						Authentication\textbackslash
						IpLocker}.
			\end{itemize}\normalsize


	\end{itemize}

\end{frame}

% ------------------------------------------------------------------------------
