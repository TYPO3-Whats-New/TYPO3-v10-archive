% ------------------------------------------------------------------------------
% TYPO3 Version 10.0 - What's New (English Version)
%
% @author	Michael Schams <schams.net>
% @license	Creative Commons BY-NC-SA 3.0
% @link		http://typo3.org/download/release-notes/whats-new/
% @language	English
% ------------------------------------------------------------------------------

\section{Änderungen für Integratoren und Entwickler}
\begin{frame}[fragile]
	\frametitle{Änderungen für Integratoren und Entwickler}

	\begin{center}\huge{Kapitel 2:}\end{center}
	\begin{center}\huge{\color{typo3darkgrey}\textbf{Änderungen für Integratoren und Entwickler}}\end{center}

\end{frame}

% ------------------------------------------------------------------------------
% Breaking | 88376 | Removed obsolete pageNotFound_handling settings

\begin{frame}[fragile]
	\frametitle{Änderungen für Integratoren und Entwickler}
	\framesubtitle{Handhabung Seite nicht gefunden}

	\begin{itemize}

		\item Die folgenden globalen TYPO3-Einstellungen wurden entfernt:

			\begin{itemize}
				\item {\fontsize{7}{8}\selectfont\texttt{\$GLOBALS['TYPO3\_CONF\_VARS']['FE']['pageNotFound\_handling']}}
				\item {\fontsize{7}{8}\selectfont\texttt{\$GLOBALS['TYPO3\_CONF\_VARS']['FE']['pageNotFound\_handling\_statheader']}}
				\item {\fontsize{7}{8}\selectfont\texttt{\$GLOBALS['TYPO3\_CONF\_VARS']['FE']['pageNotFound\_handling\_accessdeniedheader']}}
				\item {\fontsize{7}{8}\selectfont\texttt{\$GLOBALS['TYPO3\_CONF\_VARS']['FE']['pageUnavailable\_handling']}}
				\item {\fontsize{7}{8}\selectfont\texttt{\$GLOBALS['TYPO3\_CONF\_VARS']['FE']['pageUnavailable\_handling\_statheader']}}
			\end{itemize}

			\begin{itemize}\smaller
				\item[\ding{228}] Das in TYPO3 v9 eingeführte Site Handling ersetzt diese Einstellungen.
			\end{itemize}\normalsize

	\end{itemize}

\end{frame}

% ------------------------------------------------------------------------------
% Important | 87980 | Page Is Being Generated Message Has Been Removed

\begin{frame}[fragile]
	\frametitle{Änderungen für Integratoren und Entwickler}
	\framesubtitle{Handhabung Seite nicht gefunden}

	\begin{itemize}

		\item Die "\textit{Seite wird generiert}" Nachricht und die entsprechende 
			HTTP 503 Antwort wurden entfernt.
	\end{itemize}

	\begin{figure}
		\includegraphics[width=0.70\linewidth]{ChangesForIntegratorsAndDevelopers/87980-PageIsBeingGenerated.png}
	\end{figure}

	\begin{itemize}
		\item Statt die Arbeit auszulagern, um auf den endgültigen Seiteninhalt zu warten, warten gleichzeitige
			Anforderungen nun darauf, dass der eigentliche Seiteninhalt gerendert wird.
	\end{itemize}

\end{frame}

% ------------------------------------------------------------------------------
% Breaking | 86862 | Default Layout of ext:fluid_styled_content does not use spaceless viewHelper anymore

\begin{frame}[fragile]
	\frametitle{Veraltete/Entfernte Funktionen}
	\framesubtitle{Fluid}

	\begin{itemize}
		\item Das Entfernen von Leerzeichen in der \texttt{EXT:fluid\_styled\_content} Datei hat 
			gelegentliche Probleme verursacht und wurde entfernt.

	\end{itemize}

\end{frame}

% ------------------------------------------------------------------------------
% Feature | 80420 | Allow multiple recipients in email finisher

\begin{frame}[fragile]
	\frametitle{Veraltete/Entfernte Funktionen}
	\framesubtitle{Formular-Framework: Mehrere Empfänger (1)}

	\begin{itemize}
		\item Die vom \textit{EmailFinisher} gesendeten Mails können nun mehrere Empfänger haben.

		\item Die folgenden neuen Optionen wurden eingeführt:

			\begin{itemize}
				\item \texttt{recipients} (To)
				\item \texttt{replyToRecipients} (Reply-To)
				\item \texttt{carbonCopyRecipients} (CC)
				\item \texttt{blindCarbonCopyRecipients} (BCC)
			\end{itemize}

	\end{itemize}

\end{frame}

% ------------------------------------------------------------------------------
% Feature | 80420 | Allow multiple recipients in email finisher

\begin{frame}[fragile]
	\frametitle{Veraltete/Entfernte Funktionen}
	\framesubtitle{Formular-Framework: Mehrere Empfänger (2)}

	% decrease font size for code listing
	\lstset{basicstyle=\tiny\ttfamily}

	\begin{itemize}
		\item Diese Änderung erfordert eine manuelle Migration von Einzelwertopionen zu ihren Listenwertnachfolgern.

		\smaller\ Die alte Finisher Konfiguration:\normalsize

\begin{lstlisting}
finishers:
  -
    identifier: EmailToReceiver
    options:
      recipientAddress: user@example.com
      recipientName: 'Firstname Lastname'
\end{lstlisting}

		\smaller\ Die neue Finisher Konfiguration:\normalsize

\begin{lstlisting}
finishers:
  -
    identifier: EmailToReceiver
    options:
      recipients:
        user@example.com: 'Firstname Lastname'
\end{lstlisting}

		\item Siehe weitere Migrationsbeispiele im \href{https://docs.typo3.org/c/typo3/cms-core/10.0/en-us/Changelog/master/Deprecation-80420-EmailFinisherSingleAddressOptions.html}{Change Log}.

	\end{itemize}

\end{frame}

% ------------------------------------------------------------------------------
% Deprecation | 87200 | EmailFinisher format option
% Deprecation | 87200 | EmailFinisher FORMAT_* constants

\begin{frame}[fragile]
	\frametitle{Veraltete/Entfernte Funktionen}
	\framesubtitle{Formular-Framework: Klartext/HTML (1)}

	\begin{itemize}
		\item Die vom \textit{EmailFinisher} gesendeten Mails können nun sowohl Klartext als auch HTML enthalten.

		\item Die Option \texttt{format} wurde gleichzeitig als veraltet markiert und wird in TYPO3 v11 entfernt werden.

		\item Bestehende Werte werden automatisch angepasst:

			\begin{itemize}\smaller
				\item \texttt{format:html} \tabto{3cm}\textrightarrow\hspace{0.1cm}\texttt{addHtmlPart:\textbf{true}}
				\item \texttt{format:plaintext} \tabto{3cm}\textrightarrow\hspace{0.1cm}\texttt{addHtmlPart:\textbf{false}}
				\item a missing "\texttt{format}" \tabto{3cm}\textrightarrow\hspace{0.1cm}\texttt{addHtmlPart:\textbf{true}}
			\end{itemize}\normalsize

		\item Die folgenden beiden Konstanten wurden als veraltet markiert:

			\begin{itemize}\smaller
				\item \texttt{EmailFinisher::FORMAT\_PLAINTEXT}
				\item \texttt{EmailFinisher::FORMAT\_HTML}
			\end{itemize}\normalsize

	\end{itemize}

\end{frame}

% ------------------------------------------------------------------------------
% Feature | 84262 | Update EXT:felogin to Extbase (long-term goal, not completed yet)
% Breaking | 88706 | Streamline felogin locallang keys
% Breaking | 88129 | Renamed felogin flexform fields

\begin{frame}[fragile]
	\frametitle{Veraltete/Entfernte Funktionen}
	\framesubtitle{Frontend Login: Extbase}

	\begin{itemize}
		\item Die Benutzeranmeldung im Frontend (\texttt{EXT:felogin}) wird in Extbase und Fluid konvertiert.

		\item Dies ist ein langfristiges Ziel, das aktuell noch in Arbeit ist.\newline
			Siehe \href{https://forge.typo3.org/issues/84262}{forge \#84262} für weitere Details.

		\item Die folgenden Änderungen wurden in TYPO3 v10.0 implementiert:

		\begin{itemize}
			\item[\ding{202}] Das Prefix "\texttt{ll\_}" wurde von den locallang Schlüsseln entfernt.

				\begin{itemize}
					\item[\ding{228}] Wenn  Sprachbezeichnungen überschrieben werden, bitte das TypoScript aktualisieren, und das Präfix "\texttt{ll\_}" von den Schlüsseln entfernen.
				\end{itemize}

			\item[\ding{203}] Die bestehende FlexForm-Struktur wurde überarbeitet.

				\begin{itemize}
					\item[\ding{228}] Bitte den Upgrade-Wizard ausführen, um die FlexForm-Werte zu migrieren.
				\end{itemize}

		\end{itemize}

	\end{itemize}

\end{frame}

% ------------------------------------------------------------------------------
% Breaking | 88583 | Database field sys_language.static_lang_isocode removed
% Deprecation | 88567 | $GLOBALS['LOCAL_LANG']

\begin{frame}[fragile]
	\frametitle{Veraltete/Entfernte Funktionen}
	\framesubtitle{Sprachen}

	\begin{itemize}
		\item ISO Codes:

			\begin{itemize}
				\item Das nicht verwendete Datenbankfeld \texttt{static\_lang\_isocode} wurde entfernt.
				\item \texttt{EXT:static\_info\_tables} kann installiert werden, um die Funktionalität bei Bedarf erneut zu implementieren.
				\item Entwicklern wird empfohlen, alle Metadaten für eine Sprache mithilfe der Site-Konfiguration und der SiteLanguage-API abzurufen.
			\end{itemize}

		\item Sprachdateien:

			\begin{itemize}
				\item Die Verwendung des globalen Arrays \texttt{\$GLOBALS[LOCAL\_LANG]} wurde als veraltet markiert.
				\item Das zweite und dritte Argument von \texttt{LanguageService->includeLLFile()} wurde als veraltet markiert.
			\end{itemize}

	\end{itemize}

\end{frame}

% ------------------------------------------------------------------------------
% Feature | 21638 | Introduced IP locking for IPv6
% Breaking | 21638 | AbstractUserAuthentication::lockIP property removed

\begin{frame}[fragile]
	\frametitle{Änderungen für Integratoren}
	\framesubtitle{Sonsitges}

	% decrease font size for code listing
	\lstset{basicstyle=\tiny\ttfamily}

	\begin{itemize}

		\item  Die IP-Sperrfunktion wurde erweitert und unterstützt nun auch IPv6
			(Frontend und Backend).

\begin{lstlisting}
$GLOBALS['TYPO3_CONF_VARS']['FE']['lockIPv6'] = 2;
$GLOBALS['TYPO3_CONF_VARS']['BE']['lockIPv6'] = 2;
\end{lstlisting}

		\item Die public property \texttt{lockIP} wurde aus der folgenden PHP-Klasse entfernt:\newline
			\small
				\texttt{\textbackslash
					TYPO3\textbackslash
					CMS\textbackslash
					Core\textbackslash
					Authentication\textbackslash
					AbstractUserAuthentication}.
			\normalsize

		\item Migrationsoptionen:

			\begin{itemize}\smaller
				\item[\ding{228}] Set \texttt{lockIP} and \texttt{lockIPv6} in \texttt{\$GLOBALS['TYPO3\_CONF\_VARS'][...]}.
				\item[\ding{228}] Use the new IP-Locker API:
					\texttt{\textbackslash
						TYPO3\textbackslash
						CMS\textbackslash
						Core\textbackslash
						Authentication\textbackslash
						IpLocker}.
			\end{itemize}\normalsize


	\end{itemize}

\end{frame}

% ------------------------------------------------------------------------------
