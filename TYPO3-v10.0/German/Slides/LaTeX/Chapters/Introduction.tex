% ------------------------------------------------------------------------------
% TYPO3 Version 10.0 - What's New (English Version)
%
% @author	Michael Schams <schams.net>
% @license	Creative Commons BY-NC-SA 3.0
% @link		http://typo3.org/download/release-notes/whats-new/
% @language	English
% ------------------------------------------------------------------------------

\section{Einführung}
\begin{frame}[fragile]
	\frametitle{Einführung}

	\begin{center}\huge{Einführung}\end{center}
	\begin{center}\huge{\color{typo3darkgrey}\textbf{Fakten}}\end{center}

\end{frame}

% ------------------------------------------------------------------------------
% TYPO3 Version 10.0 - The Facts

\begin{frame}[fragile]
	\frametitle{Einführung}
	\framesubtitle{TYPO3 Version 10.0 - Fakten}

	\begin{itemize}
		\item Veröffentlichungsdatum: 23. Juli 2019
		\item Releasetyp: Sprint Release
		\item Entwicklungszeit: cca. 6 Monate
	\end{itemize}

	\begin{figure}
		\includegraphics[width=0.95\linewidth]{Introduction/typo3-v10-0-banner.png}
	\end{figure}

\end{frame}

% ------------------------------------------------------------------------------
% TYPO3 Version 10.0 - Executive Summary

\begin{frame}[fragile]
	\frametitle{Einführung}
	\framesubtitle{Zusammenfassung}

	\small
		TYPO3 v10.0 ist das erste Sprint Release auf dem Weg zur LTS
		(langfristige Unterstützung) im Jahr 2020.

		\vspace{0.2cm}

		Da der Hauptfokus von Version 10.0 auf Bereinigungsaufgaben liegt, ist es nicht
		verwunderlich, dass in dieser Version eine große Anzahl wichtiger Änderungen vorgenommen wurden.

		\vspace{0.2cm}

		Mit diesem Ansatz können wir in einem frühen Stadium der Entwicklung neue Bibliotheken, moderne 
		Konzepte und optimierte APIs einführen um sicherzustellen, dass TYPO3 eines der besten 
		Enterprise Content Management-Systeme auf dem Markt bleibt.

		\vspace{0.2cm}

		Eine Reihe spannender 
		\href{https://typo3.org/community/teams/typo3-development/initiatives/}{Initiativen}
		wurde ins Leben gerufen, um in einem bestimmten Bereich von TYPO3 langfristige Verbesserungen zu 
		erzielen.
	\normalsize

\end{frame}

% ------------------------------------------------------------------------------
% System Requirements

\begin{frame}[fragile]
	\frametitle{Einführung}
	\framesubtitle{Systemvoraussetzungen}

	\begin{itemize}
		\item PHP Version 7.2 oder 7.3
		\item PHP Einstellungen:

			\begin{itemize}
				\item \texttt{memory\_limit} >= 256M
				\item \texttt{max\_execution\_time} >= 240s
				\item \texttt{max\_input\_vars} >= 1500
				\item Die Option \texttt{-}\texttt{-disable-ipv6} darf \underline{nicht} genutzt werden
			\end{itemize}

		\item Die meisten von \textbf{Doctrine DBAL} unterstützten Datenbankserver arbeiten auch mit TYPO3.
			Getestete DB-Engines sind zum Beispiel:
	\end{itemize}

	\begin{figure}
		\includegraphics[width=0.80\linewidth]{Introduction/logo-databases.png}
	\end{figure}

\end{frame}

% ------------------------------------------------------------------------------
% Development, Release and Maintenance Timeline

\begin{frame}[fragile]
	\frametitle{Einführung}
	\framesubtitle{Zeitplan für Entwicklung, Veröffentlichung und Instandhaltung}

	\textbf{TYPO3 v10}

	\begin{figure}
		\includegraphics[width=1\linewidth]{Introduction/typo3-v10-lifecycle.png}
	\end{figure}

	\textbf{Erweiterter Support}\newline
	\smaller
		Die \href{https://typo3.com}{TYPO3 GmbH} bietet weitere Supportmöglichkeiten für 
		TYPO3 v10 LTS auch nach dem 30. April 2023 für bis zu zwei weitere
		Jahre.
	\normalsize

\end{frame}

% ------------------------------------------------------------------------------
% TYPO3 v10 Roadmap

\begin{frame}[fragile]
	\frametitle{Einführung}
	\framesubtitle{TYPO3 v10 Roadmap}

	Voraussichtliche Veröffentlichungen und deren Hauptfokus:

	\begin{itemize}

		\item
			\begingroup
				\color{typo3orange}
				v10.0 \tabto{1.1cm}23/July/2019\tabto{3.4cm}Pave the way for exciting new concepts and APIs
			\endgroup
		\item v10.1 \tabto{1.1cm}01/Oct/2019\tabto{3.4cm}Routing Improvements and Site Handling v2
		\item v10.2 \tabto{1.1cm}03/Dec/2019\tabto{3.4cm}Fluid/Rendering Engine Improvements
		\item v10.3 \tabto{1.1cm}04/Feb/2020\tabto{3.4cm}Feature Freeze
		\item v10.4 \tabto{1.1cm}07/Apr/2020\tabto{3.4cm}LTS Release 

	\end{itemize}

	\smaller
		\url{https://typo3.org/article/typo3-v10-roadmap/}\newline
		\url{https://typo3.org/article/typo3-v10-safe-and-sound/}
	\normalsize

\end{frame}

% ------------------------------------------------------------------------------
% Installation

\begin{frame}[fragile]
	\frametitle{Einführung}
	\framesubtitle{Installation}

	\begin{itemize}
		\item Empfohlene \textit{klassische} Installationsschritte unter Linux/Mac OS X\newline
			(DocumentRoot ist beispielsweise \texttt{/var/www/site/htdocs}):
		\begin{lstlisting}
$ cd /var/www/site
$ wget --content-disposition get.typo3.org/10.0
$ tar xzf typo3_src-10.0.0.tar.gz
$ cd htdocs
$ ln -s ../typo3_src-10.0.0 typo3_src
$ ln -s typo3_src/index.php
$ ln -s typo3_src/typo3
$ touch FIRST_INSTALL
		\end{lstlisting}

		\item Symbolische Links unter Microsoft Windows:

			\begin{itemize}
				\item Unter Windows XP/2000 kann \texttt{junction} benutzt werden.
				\item Unter Windows Vista und Windows 7 oder höher kann \texttt{mklink} benutzt werden.
			\end{itemize}

	\end{itemize}
\end{frame}

% ------------------------------------------------------------------------------
% Installation using composer

\begin{frame}[fragile]
	\frametitle{Installation und Upgrade}
	\framesubtitle{Installation mit\texttt{composer}}

	\begin{itemize}
		\item Installation mit \textit{composer} unter Linux, Mac OS X und Windows 10:

			\begin{lstlisting}
$ cd /var/www/site/
$ composer create-project typo3/cms-base-distribution typo3v10 ^10
			\end{lstlisting}

		\item Alternativ eine benutzerdefinierte \texttt{composer.json} Datei erstellen und ausführen:

			\begin{lstlisting}
$ composer install
			\end{lstlisting}

			Weitere \texttt{composer.json} Beispielsdateien können unter:\newline
				\href{https://composer.typo3.org}{https://composer.typo3.org} heruntergeladen werden

	\end{itemize}
\end{frame}

% ------------------------------------------------------------------------------
