% ------------------------------------------------------------------------------
% TYPO3 Version 10.0 - What's New (French Version)
%
% @author	Michael Schams <schams.net>
% @translator	Paul Blondiaux
% @reviewer	Pierrick Caillon
% @license	Creative Commons BY-NC-SA 3.0
% @link		http://typo3.org/download/release-notes/whats-new/
% @language	French
% ------------------------------------------------------------------------------

\section{Changements pour les intégrateurs et les développeurs}
\begin{frame}[fragile]
	\frametitle{Changements pour les intégrateurs et les développeurs}

	\begin{center}\huge{Chapitre 2~:}\end{center}
	\begin{center}\huge{\color{typo3darkgrey}\textbf{Changements pour les intégrateurs et les développeurs}}\end{center}

\end{frame}

% ------------------------------------------------------------------------------
% Breaking | 88376 | Removed obsolete pageNotFound_handling settings

\begin{frame}[fragile]
	\frametitle{Changements pour les intégrateurs et les développeurs}
	\framesubtitle{Gestion de Page non trouvée}

	\begin{itemize}

		\item Les paramètres globaux suivants sont retirés~:

			\begin{itemize}
				\item {\fontsize{7}{8}\selectfont\texttt{\$GLOBALS['TYPO3\_CONF\_VARS']['FE']['pageNotFound\_handling']}}
				\item {\fontsize{7}{8}\selectfont\texttt{\$GLOBALS['TYPO3\_CONF\_VARS']['FE']['pageNotFound\_handling\_statheader']}}
				\item {\fontsize{7}{8}\selectfont\texttt{\$GLOBALS['TYPO3\_CONF\_VARS']['FE']['pageNotFound\_handling\_accessdeniedheader']}}
				\item {\fontsize{7}{8}\selectfont\texttt{\$GLOBALS['TYPO3\_CONF\_VARS']['FE']['pageUnavailable\_handling']}}
				\item {\fontsize{7}{8}\selectfont\texttt{\$GLOBALS['TYPO3\_CONF\_VARS']['FE']['pageUnavailable\_handling\_statheader']}}
			\end{itemize}

			\begin{itemize}\smaller
				\item[\ding{228}] La gestion de site mise en place en TYPO3 v9 remplace ceux-ci.
			\end{itemize}\normalsize

	\end{itemize}

\end{frame}

% ------------------------------------------------------------------------------
% Important | 87980 | Page Is Being Generated Message Has Been Removed

\begin{frame}[fragile]
	\frametitle{Changements pour les intégrateurs et les développeurs}
	\framesubtitle{Gestion de Page non trouvée}

	\begin{itemize}

		\item Le message «~\textit{Page is being generated}~» et la réponse temporaire
			HTTP 503 correspondante ont été retirés.
	\end{itemize}

	\begin{figure}
		\includegraphics[width=0.70\linewidth]{ChangesForIntegratorsAndDevelopers/87980-PageIsBeingGenerated.png}
	\end{figure}

	\begin{itemize}
		\item Au lieu de se décharger de l'attente pour le contenu final de la page,
			les requêtes simultanées attendent le rendu du contenu réel de la page.
	\end{itemize}

\end{frame}

% ------------------------------------------------------------------------------
% Breaking | 86862 | Default Layout of ext:fluid_styled_content does not use spaceless viewHelper anymore

\begin{frame}[fragile]
	\frametitle{Changements pour les intégrateurs et les développeurs}
	\framesubtitle{Fluid}

	\begin{itemize}
		\item Le retrait des espaces dans la disposition par défaut de l'extension \texttt{EXT:fluid\_styled\_content}
			causant occasionnellement des problèmes est retiré.

	\end{itemize}

\end{frame}

% ------------------------------------------------------------------------------
% Feature | 80420 | Allow multiple recipients in email finisher

\begin{frame}[fragile]
	\frametitle{Changements pour les intégrateurs et les développeurs}
	\framesubtitle{Framework de formulaire~: Destinataires multiples (1)}

	\begin{itemize}
		\item Les mails envoyés au travers de \textit{EmailFinisher} peuvent avoir de multiples destinataires.

		\item Les options introduites sont~:

			\begin{itemize}
				\item \texttt{recipients} (To)
				\item \texttt{replyToRecipients} (Reply-To)
				\item \texttt{carbonCopyRecipients} (CC)
				\item \texttt{blindCarbonCopyRecipients} (BCC)
			\end{itemize}

	\end{itemize}

\end{frame}

% ------------------------------------------------------------------------------
% Feature | 80420 | Allow multiple recipients in email finisher

\begin{frame}[fragile]
	\frametitle{Changements pour les intégrateurs et les développeurs}
	\framesubtitle{Framework de formulaire~: Destinataires multiples (2)}

	% decrease font size for code listing
	\lstset{basicstyle=\tiny\ttfamily}

	\begin{itemize}
		\item Ce changement nécessite la migration manuelle des options valeur simple à leur nouvelle forme~:

		\smaller\textbf{Ancienne} configuration \textit{Finisher}~:\normalsize

\begin{lstlisting}
finishers:
  -
    identifier: EmailToReceiver
    options:
      recipientAddress: utilisateur@example.com
      recipientName: 'Prenom Nom'
\end{lstlisting}

		\smaller\textbf{Nouvelle} configuration \textit{Finisher}~:\normalsize

\begin{lstlisting}
finishers:
  -
    identifier: EmailToReceiver
    options:
      recipients:
        utilisateur@example.com: 'Prenom Nom'
\end{lstlisting}

		\item Voir le \href{https://docs.typo3.org/c/typo3/cms-core/10.0/en-us/Changelog/master/Deprecation-80420-EmailFinisherSingleAddressOptions.html}{journal des changements}
			pour plus d'exemples de migration.

	\end{itemize}

\end{frame}

% ------------------------------------------------------------------------------
% Deprecation | 87200 | EmailFinisher format option
% Deprecation | 87200 | EmailFinisher FORMAT_* constants

\begin{frame}[fragile]
	\frametitle{Changements pour les intégrateurs et les développeurs}
	\framesubtitle{Framework de formulaire~: Texte brut/HTML}

	\begin{itemize}
		\item Les emails envoyés par \textit{EmailFinisher} peuvent être envoyés en texte brut, HTML ou les deux.

		\item Dans le même temps, l'option \texttt{format} est marquée dépréciée et sera retirée en TYPO3 v11.

		\item Les valeurs existantes seront automatiquement migrées~:

			\begin{itemize}\smaller
				\item \texttt{format:html} \tabto{3cm}\textrightarrow\hspace{0.1cm}\texttt{addHtmlPart:\textbf{true}}
				\item \texttt{format:plaintext} \tabto{3cm}\textrightarrow\hspace{0.1cm}\texttt{addHtmlPart:\textbf{false}}
				\item sans «~\texttt{format}~» \tabto{3cm}\textrightarrow\hspace{0.1cm}\texttt{addHtmlPart:\textbf{true}}
			\end{itemize}\normalsize

		\item Ces deux constantes sont marquées dépréciées~:

			\begin{itemize}\smaller
				\item \texttt{EmailFinisher::FORMAT\_PLAINTEXT}
				\item \texttt{EmailFinisher::FORMAT\_HTML}
			\end{itemize}\normalsize

	\end{itemize}

\end{frame}

% ------------------------------------------------------------------------------
% Feature | 84262 | Update EXT:felogin to Extbase (long-term goal, not completed yet)
% Breaking | 88706 | Streamline felogin locallang keys
% Breaking | 88129 | Renamed felogin flexform fields

\begin{frame}[fragile]
	\frametitle{Changements pour les intégrateurs et les développeurs}
	\framesubtitle{Connexion Frontend~: Extbase}

	\begin{itemize}
		\item Le formulaire de connexion frontend (\texttt{EXT:felogin}) sera convertit vers Extbase et Fluid.

		\item Il s'agit d'un objectif sur le long terme, actuellement en cours.\newline
			Voir \href{https://forge.typo3.org/issues/84262}{forge \#84262} pour plus de détails.

		\item Les changements suivants sont implémentés dans TYPO3 v10.0~:

		\begin{itemize}
			\item[\ding{202}] Le préfixe «~\texttt{ll\_}~» est retiré des clés locallang.

				\begin{itemize}
					\item[\ding{228}] Mettez à jour votre TypoScript si vous avez surchargé les libellés de langue et effacez le préfixe «~\texttt{ll\_}~» de vos clés.
				\end{itemize}

			\item[\ding{203}] La structure des FlexForms est remaniée.

				\begin{itemize}
					\item[\ding{228}] Exécutez l'assistant de mise à jour pour migrer les valeurs des FlexForm.
				\end{itemize}

		\end{itemize}

	\end{itemize}

\end{frame}

% ------------------------------------------------------------------------------
% Breaking | 88583 | Database field sys_language.static_lang_isocode removed
% Deprecation | 88567 | $GLOBALS['LOCAL_LANG']

\begin{frame}[fragile]
	\frametitle{Changements pour les intégrateurs et les développeurs}
	\framesubtitle{Langues}

	\begin{itemize}
		\item Codes ISO~:

			\begin{itemize}
				\item Le champ inutilisé de base de données \texttt{static\_lang\_isocode} est retiré.
				\item \texttt{EXT:static\_info\_tables} peut être installée pour le rétablir si requis.
				\item Il est conseillé aux développeurs de récupérer toutes les métadonnées d'une langue
					en utilisation la configuration de site et l'API SiteLanguage.
			\end{itemize}

		\item Fichiers de langue~:

			\begin{itemize}
				\item Le tableau global \texttt{\$GLOBALS[LOCAL\_LANG]} est déprécié.
				\item Les 2ième et 3ième arguments de \texttt{LanguageService->includeLLFile()} sont dépréciés.
			\end{itemize}

	\end{itemize}

\end{frame}

% ------------------------------------------------------------------------------
% Feature | 21638 | Introduced IP locking for IPv6
% Breaking | 21638 | AbstractUserAuthentication::lockIP property removed

\begin{frame}[fragile]
	\frametitle{Changements pour les intégrateurs et les développeurs}
	\framesubtitle{Divers}

	% decrease font size for code listing
	\lstset{basicstyle=\tiny\ttfamily}

	\begin{itemize}

		\item La fonctionnalité de verrouillage d'IP a été étendue pour supporter IPv6
			(frontend et backend).

\begin{lstlisting}
$GLOBALS['TYPO3_CONF_VARS']['FE']['lockIPv6'] = 2;
$GLOBALS['TYPO3_CONF_VARS']['BE']['lockIPv6'] = 2;
\end{lstlisting}

		\item La propriété publique \texttt{lockIP} dans la classe PHP suivante est retirée~:\newline
			\small
				\texttt{\textbackslash
					TYPO3\textbackslash
					CMS\textbackslash
					Core\textbackslash
					Authentication\textbackslash
					AbstractUserAuthentication}.
			\normalsize

		\item Options de migration~:

			\begin{itemize}\smaller
				\item[\ding{228}] Paramétrer \texttt{lockIP} et \texttt{lockIPv6} dans \texttt{\$GLOBALS['TYPO3\_CONF\_VARS'][...]}.
				\item[\ding{228}] Utiliser la nouvelle API IP-Locker~:
					\texttt{\textbackslash
						TYPO3\textbackslash
						CMS\textbackslash
						Core\textbackslash
						Authentication\textbackslash
						IpLocker}.
			\end{itemize}\normalsize


	\end{itemize}

\end{frame}

% ------------------------------------------------------------------------------
