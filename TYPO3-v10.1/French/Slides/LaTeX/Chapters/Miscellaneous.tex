% ------------------------------------------------------------------------------
% TYPO3 Version 10.1 - What's New (French Version)
%
% @license	Creative Commons BY-NC-SA 3.0
% @link		http://typo3.org/download/release-notes/whats-new/
% @language	French
% ------------------------------------------------------------------------------

\section{Divers}
\begin{frame}[fragile]
	\frametitle{Divers}

	\begin{center}\huge{Chapitre 5~:}\end{center}
	\begin{center}\huge{\color{typo3darkgrey}\textbf{Divers}}\end{center}

\end{frame}

% ------------------------------------------------------------------------------
% Feature | 78488 | Add rel=”noopener noreferrer” to external links

\begin{frame}[fragile]
	\frametitle{Divers}
	\framesubtitle{Sécurité renforcée}

	\begin{itemize}
		\item Les liens externes générés par TypoLink, ou les liens utilisant la cible \texttt{\_blank},
			incluent l'attribut \texttt{rel="noopener noreferrer"}.
		\item Cet ajout vise à renforcer la sécurité des sites TYPO3~:

			\begin{itemize}
				\item «~\textbf{noopener}~» informe le navigateur d'ouvrir le lien dans un nouveau contexte sans
					donner l'accès au document d'où il a été ouvert.
				\item «~\textbf{noreferrer}~» empêche le navigateur, lors de la navigation vers une autre page,
					d'envoyer l'adresse de la page d'origine, ou toute autre valeur, en référant dans l'en-tête
					HTTP \texttt{Referer:}.
			\end{itemize}

	\end{itemize}

\end{frame}

% ------------------------------------------------------------------------------
% Feature | 88742 | Import YAML files relative to the current YAML file

\begin{frame}[fragile]
	\frametitle{Divers}
	\framesubtitle{Inclusion de fichiers YAML}

    % decrease font size for code listing
	\lstset{basicstyle=\tiny\ttfamily}

	\begin{itemize}
		\item Les fichiers YAML peuvent déjà inclure d'autres fichiers à l'aide de la syntaxe suivante~:

\begin{lstlisting}
imports:
  - { resource: "EXT:my_extension/Configuration/FooBar/Example.yaml" }

another:
  option: true
\end{lstlisting}

		\item Elle est étendue pour l'inclusion des ressources à l'aide d'un chemin relatif au fichier actuel~:

\begin{lstlisting}
imports:
  - { resource: "subfolder/AnotherExample.yaml" }
  - { resource: "../path/to/configuration/AnotherExample.yaml" }

another:
  option: true
\end{lstlisting}

	\end{itemize}

\end{frame}

% ------------------------------------------------------------------------------
