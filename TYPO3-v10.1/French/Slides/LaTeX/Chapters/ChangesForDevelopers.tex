% ------------------------------------------------------------------------------
% TYPO3 Version 10.1 - What's New (French Version)
%
% @license	Creative Commons BY-NC-SA 3.0
% @link		http://typo3.org/download/release-notes/whats-new/
% @language	French
% ------------------------------------------------------------------------------

\section{Changements pour les développeurs}
\begin{frame}[fragile]
	\frametitle{Changements pour les développeurs}

	\begin{center}\huge{Chapitre 3~:}\end{center}
	\begin{center}\huge{\color{typo3darkgrey}\textbf{Changements pour les développeurs}}\end{center}

\end{frame}

% ------------------------------------------------------------------------------
% Feature | 89054 | Provide core cache frontends via dependency injection

\begin{frame}[fragile]
	\frametitle{Changements pour les développeurs}
	\framesubtitle{Injection du cache par dépendance (1)}

	% decrease font size for code listing
	\lstset{basicstyle=\tiny\ttfamily}

	\begin{itemize}
		\item Les développeurs d'extension sont encouragés à injecter les caches directement plutôt
			que d'utiliser CacheManager.
		\item Les changements requis sont à effectuer comme suit.

		\item \textbf{Précédemment~:}

\begin{lstlisting}
class MyClass
{
  /**
   * @var TYPO3\CMS\Core\Cache\Frontend\FrontendInterface
   */
  private $cache;

  public function __construct()
  {
      $cacheManager = GeneralUtility::makeInstance(CacheManager::class);
      $this->cache = $cacheManager->getCache('my_cache');
  }
}
\end{lstlisting}

	\end{itemize}

\end{frame}

% ------------------------------------------------------------------------------
% Feature | 89054 | Provide core cache frontends via dependency injection

\begin{frame}[fragile]
	\frametitle{Changements pour les développeurs}
	\framesubtitle{Injection du cache par dépendance (2)}

	% decrease font size for code listing
	\lstset{basicstyle=\tiny\ttfamily}

	\begin{itemize}
		\item Depuis \textbf{TYPO3 v10.1}, la classe doit être comme ci-dessous~:

\begin{lstlisting}
class MyClass
{
  /**
   * @var TYPO3\CMS\Core\Cache\Frontend\FrontendInterface
   */
  private $cache;

  public function __construct(FrontendInterface $cache)
  {
    $this->cache = $cache;
  }
}
\end{lstlisting}

	\end{itemize}

\end{frame}

% ------------------------------------------------------------------------------
% Feature | 89054 | Provide core cache frontends via dependency injection

\begin{frame}[fragile]
	\frametitle{Changements pour les développeurs}
	\framesubtitle{Injection du cache par dépendance (3)}

	% decrease font size for code listing
	\lstset{basicstyle=\tiny\ttfamily}

	\begin{itemize}
		\item ... et la configuration de service suivante est requise~:

\begin{lstlisting}
services:
  cache.my_cache:
    class: TYPO3\CMS\Core\Cache\Frontend\FrontendInterface
    factory: ['@TYPO3\CMS\Core\Cache\CacheManager', 'getCache']
    arguments: ['my_cache']

  MyClass:
    arguments:
      $cache: '@cache.my_cache'
\end{lstlisting}

	\end{itemize}

\end{frame}

% ------------------------------------------------------------------------------
% Feature | 89066 | Add PHP API for Notifications in backend
%
%\begin{frame}[fragile]
%	\frametitle{Changements pour les développeurs}
%	\framesubtitle{Backend Notifications}
%
%	% decrease font size for code listing
%	\lstset{basicstyle=\smaller\ttfamily}
%
%	\begin{itemize}
%		\item A new PHP API provides a simple way to create JavaScript backend notifications.
%		\item For example:
%
%\begin{lstlisting}
%GeneralUtility::makeInstance(NotificationService::class)
%   ->notice('Notice', 'notice');
%\end{lstlisting}
%
%	\end{itemize}
%
%	\begin{figure}
%		\includegraphics[width=0.90\linewidth]{ChangesForDevelopers/89066-NotificationApi.png}
%	\end{figure}
%
%\end{frame}

% ------------------------------------------------------------------------------
% Feature | 89061 | Introduce Notification Actions

\begin{frame}[fragile]
	\frametitle{Changements pour les développeurs}
	\framesubtitle{Actions dans les notifications}

	\begin{itemize}
		\item En backend, les notifications JavaScript supportent l'usage d'actions (boutons).
	\end{itemize}

	\begin{figure}
		\includegraphics[width=0.90\linewidth]{ChangesForDevelopers/89061-NotificationActionsAndButtons.png}
	\end{figure}

\end{frame}

% ------------------------------------------------------------------------------
% Feature | 89244 | Broadcast Channels and Messaging

\begin{frame}[fragile]
	\frametitle{Changements pour les développeurs}
	\framesubtitle{Canaux de diffusion et messages (1)}

	% decrease font size for code listing
	\lstset{basicstyle=\tiny\ttfamily}

	\begin{itemize}
		\item Il est possible de transmettre et recevoir des messages en JavaScript.
	\end{itemize}

	\vspace{-0.2cm}
	\begingroup
		\color{red}
			\begin{center}
				L'API est considérée \textbf{interne} pour le moment\newline
				et peut changer jusqu'à qu'elle soit déclarée «~stable~».
			\end{center}
	\endgroup

	\begin{itemize}
		\item Exemple pour \textbf{transmettre} un message~:

\begin{lstlisting}
require(['TYPO3/CMS/Backend/BroadcastService'], function (BroadcastService) {
  const payload = {
    componentName: 'my_extension',
    eventName: 'my_event',
    foo: 'bar'
  };
  BroadcastService.post(payload);
});
\end{lstlisting}

	\end{itemize}

\end{frame}

% ------------------------------------------------------------------------------
% Feature | 89244 | Broadcast Channels and Messaging

\begin{frame}[fragile]
	\frametitle{Changements pour les développeurs}
	\framesubtitle{Canaux de diffusion et messages (2)}

	% decrease font size for code listing
	\lstset{basicstyle=\tiny\ttfamily}

	\begin{itemize}
		\item Exemple pour \textbf{recevoir} le message~:

\begin{lstlisting}
define([], function() {
  document.addEventListener('typo3:my_component:my_event', (e) => eventHandler(e.detail));
  function eventHandler(detail) {
    // output contains key 'foo' as the payload
    console.log(detail);
  }
});
\end{lstlisting}

		\item Voir \href{https://developer.mozilla.org/en-US/docs/Web/API/Broadcast_Channel_API}{developer.mozilla.org (en)}
			pour plus d'informations.

	\end{itemize}

\end{frame}

% ------------------------------------------------------------------------------
% Feature | 89018 | Provide implementation for PSR-17 HTTP Message Factories

\begin{frame}[fragile]
	\frametitle{Changements pour les développeurs}
	\framesubtitle{Fabriques de messages HTTP PSR-17}

	\begin{itemize}
		\item L'implémentation des fabriques de messages HTTP
			\href{https://www.php-fig.org/psr/psr-17/}{PSR-17} est ajoutée.
		\item Les interfaces de fabrique de messages HTTP devraient être
			utilisées comme dépendance des gestionnaires de requête ou services
			créant des objet message PSR-7.
		\item PSR-17 consiste en six interfaces de fabrique~:

			\begin{itemize}\smaller
				\item \texttt{\textbackslash
					Psr\textbackslash
					Http\textbackslash
					Message\textbackslash
					RequestFactoryInterface}
				\item \texttt{\textbackslash
					Psr\textbackslash
					Http\textbackslash
					Message\textbackslash
					ResponseFactoryInterface}
				\item \texttt{\textbackslash
					Psr\textbackslash
					Http\textbackslash
					Message\textbackslash
					ServerRequestFactoryInterface}
				\item \texttt{\textbackslash
					Psr\textbackslash
					Http\textbackslash
					Message\textbackslash
					StreamFactoryInterface}
				\item \texttt{\textbackslash
					Psr\textbackslash
					Http\textbackslash
					Message\textbackslash
					UploadedFileFactoryInterface}
				\item \texttt{\textbackslash
					Psr\textbackslash
					Http\textbackslash
					Message\textbackslash
					UriFactoryInterface}

			\end{itemize}\normalsize

		\item Voir la
			\href{https://docs.typo3.org/c/typo3/cms-core/master/en-us/Changelog/10.1/Feature-89018-ProvideImplementationForPSR-17HTTPMessageFactories.html}{documentation (en)}
			pour des exemples de code.

	\end{itemize}

\end{frame}

% ------------------------------------------------------------------------------
% Feature | 89216 | PSR-18 HTTP Client Implementation

\begin{frame}[fragile]
	\frametitle{Changements pour les développeurs}
	\framesubtitle{Client HTTP PSR-18}

	\begin{itemize}
		\item L'implémentation du client HTTP
			\href{https://www.php-fig.org/psr/psr-18/}{PSR-18} est ajoutée.
		\item Elle permet de générer des requêtes basées sur des objets message
			PSR-7 sans dépendre d'une implémentation de client HTTP spécifique.
		\item Ça ne remplace pas l'actuelle encapsulation de \href{http://guzzlephp.org/}{Guzzle},
			mais fournie une alternative plus générique.
		\item PSR-18 consiste en une interface de client et trois interfaces d'exception~:

			\begin{itemize}\smaller
				\item \texttt{\textbackslash
					Psr\textbackslash
					Http\textbackslash
					Client\textbackslash
					ClientInterface}
				\item \texttt{\textbackslash
					Psr\textbackslash
					Http\textbackslash
					Client\textbackslash
					ClientExceptionInterface}
				\item \texttt{\textbackslash
					Psr\textbackslash
					Http\textbackslash
					Client\textbackslash
					NetworkExceptionInterface}
				\item \texttt{\textbackslash
					Psr\textbackslash
					Http\textbackslash
					Client\textbackslash
					RequestExceptionInterface}
			\end{itemize}\normalsize

		\item Voir la
			\href{https://docs.typo3.org/c/typo3/cms-core/master/en-us/Changelog/10.1/Feature-89216-PSR-18HTTPClientImplementation.html}{documentation (en)}
			pour un exemple de code.

	\end{itemize}

\end{frame}

% ------------------------------------------------------------------------------
% Feature | 88871 | Handle middleware handler in RequestFactory correctly

\begin{frame}[fragile]
	\frametitle{Changements pour les développeurs}
	\framesubtitle{Gestionnaire Middleware de RequestFactory}

	% decrease font size for code listing
	\lstset{basicstyle=\tiny\ttfamily}

	\begin{itemize}
		\item Il est possible de définir la liste des gestionnaires middleware.
		\item RequestFactory construire une pile de gestionnaires basé sur\newline
			\small
				\texttt{\$GLOBALS['TYPO3\_CONF\_VARS']['HTTP']['handler']}
			\normalsize
			et l'injecte dans le client.
		\item Par exemple~:

\begin{lstlisting}
use \TYPO3\CMS\Core\Utility\GeneralUtility;
use \Vendor\MyExtension\Middleware\Guzzle\CustomMiddleware;
use \Vendor\MyExtension\Middleware\Guzzle\SecondCustomMiddleware;

# Add custom middleware to default Guzzle handler stack
$GLOBALS['TYPO3_CONF_VARS']['HTTP']['handler'][] =
  (GeneralUtility::makeInstance(CustomMiddleware::class))->handler();
$GLOBALS['TYPO3_CONF_VARS']['HTTP']['handler'][] =
  (GeneralUtility::makeInstance(SecondCustomMiddleware::class))->handler();
\end{lstlisting}

	\end{itemize}

\end{frame}

% ------------------------------------------------------------------------------
% Feature | 88602 | Allow registering additional file processors

\begin{frame}[fragile]
	\frametitle{Changements pour les développeurs}
	\framesubtitle{Processeurs de fichier personnalisés}

	% decrease font size for code listing
	\lstset{basicstyle=\tiny\ttfamily}

	\begin{itemize}
		\item Les développeurs peuvent inscrire leur propre processeurs de fichiers.
		\item Ajouter le code suivant au fichier \texttt{ext\_localconf.php}~:

\begin{lstlisting}
$GLOBALS['TYPO3_CONF_VARS']['SYS']['fal']['processors']['ExampleImageProcessor'] = [
  'className' => \Vendor\MyExtension\Resource\Processing\ExampleImageProcessor::class,
  'before' => 'LocalImageProcessor',
];
\end{lstlisting}

		\item Cas d'usage typiques~:

			\begin{itemize}
				\item Ajouter un filigrane aux images
				\item Compresser les fichiers envoyés dans une archive ZIP
				\item Enregistrer les versions modifiées des images
				\item etc.
			\end{itemize}

	\end{itemize}

\end{frame}

% ------------------------------------------------------------------------------
% Feature | 88995 | Calling registerPlugin with vendor name

\begin{frame}[fragile]
	\frametitle{Changements pour les développeurs}
	\framesubtitle{Extbase et Fluid}

	% decrease font size for code listing
	\lstset{basicstyle=\smaller\ttfamily}

	\begin{itemize}
		\item Omettez le nom de fournisseur lors de l'inscription d'un plugin avec\newline
			\smaller
				\texttt{\textbackslash
					TYPO3\textbackslash
					CMS\textbackslash
					Extbase\textbackslash
					Utility\textbackslash
					ExtensionUtility::registerPlugin()}
			\normalsize

		\item Par exemple, utiliser «~\texttt{Form}~» au lieu de «~\texttt{TYPO3.CMS.Form}~»\newline
			\small(premier paramètre)\normalsize

\begin{lstlisting}
\TYPO3\CMS\Extbase\Utility\ExtensionUtility::registerPlugin(
  'Form',
  'Formframework',
  'Form',
  'content-form',
);
\end{lstlisting}

	\end{itemize}

\end{frame}

% ------------------------------------------------------------------------------
% Important | 89001 | TSFE->createHashBase
% Feature | 89150 | Add events before and after rollback of record history entries

\begin{frame}[fragile]
	\frametitle{Changements pour les développeurs}
	\framesubtitle{Divers (1)}

	\begin{itemize}
		\item Les paramètres de base utilisés pour calculer les signatures de cache sont
			modifiés dans la classe~:\newline
			\small
				\texttt{TYPO3\textbackslash
					CMS\textbackslash
					Frontend\textbackslash
					Controller\textbackslash
					TypoScriptFrontendController}
			\normalsize

			\begin{itemize}
				\item \texttt{gr\_list} est remplacé par \texttt{groupIds}.
				\item \texttt{cHash} est remplacé par \texttt{dynamicArguments}.
				\item \texttt{domainStartPage} est remplacé par \texttt{site} (identifiant du site).
			\end{itemize}

		\item Deux événements sont déclenchés lors d'un retour arrière d'un enregistrements~:

			\begin{itemize}\smaller
				\item \texttt{TYPO3\textbackslash
					CMS\textbackslash
					Backend\textbackslash
					History\textbackslash
					Event\textbackslash
					BeforeHistoryRollbackStartEvent}
				\item \texttt{TYPO3\textbackslash
					CMS\textbackslash
					Backend\textbackslash
					History\textbackslash
					Event\textbackslash
					AfterHistoryRollbackFinishedEvent}
			\end{itemize}\normalsize

	\end{itemize}

\end{frame}

% ------------------------------------------------------------------------------
% Feature | 88805 | Add type to TYPO3/CMS/Core/Database/Query/QueryBuilder::set()

\begin{frame}[fragile]
	\frametitle{Changements pour les développeurs}
	\framesubtitle{Divers (2)}

	\begin{itemize}
		\item La méthode \texttt{set()} du générateur de requêtes accepte un 4ième argument
			pour spécifier le type du paramètre nommé~:\newline
			\small
				\texttt{TYPO3\textbackslash
					CMS\textbackslash
					Core\textbackslash
					Database\textbackslash
					Query\textbackslash
					QueryBuilder::set()}
			\normalsize\newline
			\vspace{0.2cm}
			(par défaut, c'est \texttt{\textbackslash PDO::PARAM\_STR})

	\end{itemize}

\end{frame}

% ------------------------------------------------------------------------------
