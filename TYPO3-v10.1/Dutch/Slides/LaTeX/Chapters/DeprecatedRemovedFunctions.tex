% ------------------------------------------------------------------------------
% TYPO3 Version 10.1 - What's New (Dutch Version)
%
% @license	Creative Commons BY-NC-SA 3.0
% @link		http://typo3.org/download/release-notes/whats-new/
% @language	Dutch
% ------------------------------------------------------------------------------

\section{Verouderde/verwijderde functies}
\begin{frame}[fragile]
	\frametitle{Verouderde/verwijderde functies}

	\begin{center}\huge{Hoofdstuk 4:}\end{center}
	\begin{center}\huge{\color{typo3darkgrey}\textbf{Verouderde/verwijderde functies}}\end{center}

\end{frame}

% ------------------------------------------------------------------------------
% Deprecation | 88854 | T3_THIS_LOCATION
% Deprecation | 88862 | T3_RETURN_URL
% Deprecation | 89033 | jumpToUrl
% Deprecation | 88854 | jumpExt() of RecordListController

\begin{frame}[fragile]
	\frametitle{Verouderde/verwijderde functies}
	\framesubtitle{Verouderd in JavaScript (1)}

	\begin{itemize}
		\item Twee globale JavaScript-variabelen zjin als verouderd gemarkeerd.

			\begin{itemize}
				\item \texttt{T3\_THIS\_LOCATION}
				\item \texttt{T3\_RETURN\_URL}
			\end{itemize}

		\item De bekende JavaScript functie \texttt{jumpToUrl()} is als \textbf{verouderd} gemarkeerd.
			Migratiemogelijkheden:

			\begin{itemize}
				\item gebruik \texttt{window.location.href = '...';}
				\item of gebruik een link in HTML zoals \texttt{<a href="...">link</a>}
			\end{itemize}

		\item De JavaScript functie \texttt{jumpExt()} is als \textbf{verouderd} gemarkeerd.

	\end{itemize}

\end{frame}


% ------------------------------------------------------------------------------
% Deprecation | 89215 | jQuery.clearable

\begin{frame}[fragile]
	\frametitle{Verouderde/verwijderde functies}
	\framesubtitle{Verouderd in JavaScript (2)}

	% decrease font size for code listing
	\lstset{basicstyle=\tiny\ttfamily}

	\begin{itemize}
		\item De jQuery plug-in \texttt{jquery.clearable},
			die een knop biedt om een invoerveld te legen is als \textbf{verouderd} gemarkeerd.
		\item Migratie: gebruik module \small\texttt{TYPO3/CMS/Backend/Input/Clearable}\normalsize
			en de functie \texttt{clearable()} op een gewoon HTMLInputElement.

\begin{lstlisting}
require(['TYPO3/CMS/Backend/Input/Clearable'], function() {
  const inputField = document.querySelector('#myinput');
  if (inputField !== null) {
    inputField.clearable();
  }

  const clear = Array.from(document.querySelectorAll('.t3js-clearable')).filter(inputElement => {
    return !inputElement.classList.contains('t3js-datetimepicker');
  });
  clear.forEach(clearableField => clearableField.clearable());
});
\end{lstlisting}

	\end{itemize}

\end{frame}

% ------------------------------------------------------------------------------
% Deprecation | 88839 | CLI lowlevel request handlers

\begin{frame}[fragile]
	\frametitle{Verouderde/verwijderde functies}
	\framesubtitle{CLI Commando Handler}

	\begin{itemize}
		\item CLI commando's worden afgehandeld door de \texttt{CommandApplication} klasse.
		\item Deze klasse is een omhulsel rond de
			\href{https://symfony.com/doc/current/components/console.html}{Symfony Console}.

		\item Het hiervoor gebruikte koppelvlak en de klasse \texttt{CommandRequestHandler} zijn als \textbf{verouderd} aangemerkt:

			\begin{itemize}
				\item
					\texttt{TYPO3\textbackslash
						CMS\textbackslash
						Core\textbackslash
						Console\textbackslash
						RequestHandlerInterface}
				\item
					\texttt{TYPO3\textbackslash
						CMS\textbackslash
						Core\textbackslash
						Console\textbackslash
						CommandRequestHandler}
			\end{itemize}

	\end{itemize}

\end{frame}

% ------------------------------------------------------------------------------
% Deprecation | 88850 | ContentObjectRenderer::sendNotifyEmail
% Deprecation | 88787 | BackendUtility::editOnClick

\begin{frame}[fragile]
	\frametitle{Verouderde/verwijderde functies}
	\framesubtitle{Diversen}

	\begin{itemize}
		\item E-mailfunctionaliteit hoort niet thuisin de klasse\newline
			\small
				\texttt{TYPO3\textbackslash
					CMS\textbackslash
					Frontend\textbackslash
					ContentObject\textbackslash
					ContentObjectRenderer}.\newline
			\normalsize
			Daarom is de functie \texttt{sendNotifyEmail()} als \textbf{verouderd} aangemerkt en zal worden verwijderd in TYPO3 v11.

		\item De functie \texttt{editOnClick()} die gebruikt werd om JavaScript \texttt{onclick} doelen te maken
			is als \textbf{verouderd} aangemerkt in de volgende klasse:\newline
			\small
				\texttt{TYPO3\textbackslash
					CMS\textbackslash
					Backend\textbackslash
					Utility\textbackslash
					BackendUtility}.
			\normalsize

	\end{itemize}

\end{frame}

% ------------------------------------------------------------------------------
% Deprecation | 89127 | Cleanup RecordHistory handling

\begin{frame}[fragile]
	\frametitle{Verouderde/verwijderde functies}
	\framesubtitle{Afhandeling historie van records}

	Veranderingen in de klasse
		\smaller
			\texttt{TYPO3\textbackslash
				CMS\textbackslash
				Backend\textbackslash
				History\textbackslash
				RecordHistory}:
		\normalsize

	\begin{itemize}

		\item Zichtbaarheid van eigenschappen \texttt{changeLog} en \texttt{lastHistoryEntry}
			gewijzigd in \texttt{protected} (en publieke getters toegevoegd).
		\item Zichtbaarheid van functies \texttt{getHistoryEntry()} en \texttt{getHistoryData()}
			gewijzigd in \texttt{protected}.
		\item De volgende functies zijn aangemerkt als \textbf{verouderd}:

			\begin{itemize}\smaller
				\item \texttt{createChangeLog()}
				\item \texttt{shouldPerformRollback()}
				\item \texttt{getElementData()}
				\item \texttt{performRollback()}
				\item \texttt{createMultipleDiff()}
				\item \texttt{setLastHistoryEntry()}
			\end{itemize}\normalsize

	\end{itemize}

\end{frame}

% ------------------------------------------------------------------------------
% Deprecation | 89037 | Deprecated LocallangXmlParser

\begin{frame}[fragile]
	\frametitle{Verouderde/verwijderde functies}
	\framesubtitle{XML Taalbestanden}

	\begin{itemize}
		\item Het XLIFF-formaat wordt gebruikt voor taalbestanden sinds TYPO3 v4.6.
		\item Het gebruik van XML-taalbestanden is nu aangemerkt als \textbf{verouderd}
			en geeft een waarschuwing/fout.
		\item Hieronder valt het uitvoeren van de volgende XML-parser:\newline
			\small
				\texttt{TYPO3\textbackslash
					CMS\textbackslash
					Core\textbackslash
					Localization\textbackslash
					Parser\textbackslash
					LocallangXmlParser}
			\normalsize
	\end{itemize}

\end{frame}

% ------------------------------------------------------------------------------
