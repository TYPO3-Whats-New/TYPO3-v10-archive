% ------------------------------------------------------------------------------
% TYPO3 Version 10.1 - What's New (Dutch Version)
%
% @license	Creative Commons BY-NC-SA 3.0
% @link		http://typo3.org/download/release-notes/whats-new/
% @language	Dutch
% ------------------------------------------------------------------------------

\section{Diversen}
\begin{frame}[fragile]
	\frametitle{Diversen}

	\begin{center}\huge{Hoofdstuk 5:}\end{center}
	\begin{center}\huge{\color{typo3darkgrey}\textbf{Diversen}}\end{center}

\end{frame}

% ------------------------------------------------------------------------------
% Feature | 78488 | Add rel=”noopener noreferrer” to external links

\begin{frame}[fragile]
	\frametitle{Diversen}
	\framesubtitle{Verbeterde beveiliging}

	\begin{itemize}
		\item Externe links gemaakt door TypoLink, of links met \texttt{\_blank},
			krijgen nu het attribuut \texttt{rel="noopener noreferrer"}.
		\item Dit zorgt voor meer veiligheid voor de TYPO3 site:

			\begin{itemize}
				\item "\textbf{noopener}" vertelt de browser om de link te openen zonder
					de nieuwe context toegang te geven tot het voorgaande document.
				\item "\textbf{noreferrer}" voorkomt dat de browser het adres van de vorige pagina
					doorgeeft aan de volgende via de
					\texttt{Referer:} HTTP header.
			\end{itemize}

	\end{itemize}

\end{frame}

% ------------------------------------------------------------------------------
% Feature | 88742 | Import YAML files relative to the current YAML file

\begin{frame}[fragile]
	\frametitle{Diversen}
	\framesubtitle{Invoegen van YAML bestanden}

    % decrease font size for code listing
	\lstset{basicstyle=\tiny\ttfamily}

	\begin{itemize}
		\item YAML-bestanden kunnen al ingevoegd worden door andere YAML-bestanden met de volgende syntax:

\begin{lstlisting}
imports:
  - { resource: "EXT:my_extension/Configuration/FooBar/Example.yaml" }

another:
  option: true
\end{lstlisting}

		\item Dit is uitgebreid zodat bestanden relatief tot het huidige YAML-bestand geïmporteerd kunnen worden:

\begin{lstlisting}
imports:
  - { resource: "subfolder/AnotherExample.yaml" }
  - { resource: "../pad/naar/configuratie/AnotherExample.yaml" }

another:
  option: true
\end{lstlisting}

	\end{itemize}

\end{frame}

% ------------------------------------------------------------------------------
