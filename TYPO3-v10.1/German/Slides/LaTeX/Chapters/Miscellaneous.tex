% ------------------------------------------------------------------------------
% TYPO3 Version 10.1 - What's New (German Version)
%
% @license	Creative Commons BY-NC-SA 3.0
% @link		http://typo3.org/download/release-notes/whats-new/
% @language	German
% ------------------------------------------------------------------------------

\section{Sonstiges}
\begin{frame}[fragile]
	\frametitle{Sonstiges}

	\begin{center}\huge{Kapitel 5:}\end{center}
	\begin{center}\huge{\color{typo3darkgrey}\textbf{Sonstiges}}\end{center}

\end{frame}

% ------------------------------------------------------------------------------
% Feature | 78488 | Add rel=”noopener noreferrer” to external links

\begin{frame}[fragile]
	\frametitle{Sonstiges}
	\framesubtitle{Erhöhung der Sicherheit}

	\begin{itemize}
		\item Externe Links, die von TypoLink generiert wurden, oder Links die \texttt{\_blank} verwenden,
			zeigen jetzt das Attribut \texttt{rel="noopener noreferrer"}.
		\item Damit soll die Sicherheit der TYPO3-Website erhöht werden:

			\begin{itemize}
				\item Der "\textbf{noopener}" weist den Browser an, den Link zu öffnen, ohne dem neuen
					Broswerkontext Zugriff auf das Dokument zu gewähren, das ihn geöffnet hat.
				\item Der "\textbf{noreferrer}" verhindert, dass der Browser beim Navigieren zu einer
					anderen Seite die Seitenadresse oder einen anderen Wert über die Funktion
					\texttt{Referer:} HTTP Header sendet.
			\end{itemize}

	\end{itemize}

\end{frame}

% ------------------------------------------------------------------------------
% Feature | 88742 | Import YAML files relative to the current YAML file

\begin{frame}[fragile]
	\frametitle{Sonstiges}
	\framesubtitle{YAML-Dateieinbindung}

    % decrease font size for code listing
	\lstset{basicstyle=\tiny\ttfamily}

	\begin{itemize}
		\item YAML-Dateien können bereits von anderen YAML-Dateien mit der folgenden Syntax eingebunden werden:

\begin{lstlisting}
imports:
  - { resource: "EXT:my_extension/Configuration/FooBar/Example.yaml" }

another:
  option: true
\end{lstlisting}

		\item Dies wurde erweitert, um Ressourcen relativ zur aktuellen YAML-Datei zu importieren:

\begin{lstlisting}
imports:
  - { resource: "subfolder/AnotherExample.yaml" }
  - { resource: "../path/to/configuration/AnotherExample.yaml" }

another:
  option: true
\end{lstlisting}

	\end{itemize}

\end{frame}

% ------------------------------------------------------------------------------
