% ------------------------------------------------------------------------------
% TYPO3 Version 10.1 - What's New (German Version)
%
% @license	Creative Commons BY-NC-SA 3.0
% @link		http://typo3.org/download/release-notes/whats-new/
% @language	German
% ------------------------------------------------------------------------------

\section{Veraltete/Entfernte Funktionen}
\begin{frame}[fragile]
	\frametitle{Veraltete/Entfernte Funktionen}

	\begin{center}\huge{Kapitel 4:}\end{center}
	\begin{center}\huge{\color{typo3darkgrey}\textbf{Veraltete/Entfernte Funktionen}}\end{center}

\end{frame}

% ------------------------------------------------------------------------------
% Deprecation | 88854 | T3_THIS_LOCATION
% Deprecation | 88862 | T3_RETURN_URL
% Deprecation | 89033 | jumpToUrl
% Deprecation | 88854 | jumpExt() of RecordListController

\begin{frame}[fragile]
	\frametitle{Veraltete/Entfernte Funktionen}
	\framesubtitle{Veraltetes JavaScript (1)}

	\begin{itemize}
		\item Zwei globale JavaScript-Variablen wurden als \textbf{veraltet} markiert:

			\begin{itemize}
				\item \texttt{T3\_THIS\_LOCATION}
				\item \texttt{T3\_RETURN\_URL}
			\end{itemize}

		\item Die bekannte JavaScript-Funktion \texttt{jumpToUrl()} wurde als \textbf{veraltet} markiert.
			Migrationsoptionen:

			\begin{itemize}
				\item verwenden Sie \texttt{window.location.href = '...';}
				\item oder benutzen Sie einen Link in HTML wie zum Beispiel \texttt{<a href="...">link</a>}
			\end{itemize}

		\item Die JavaScript-Funktion \texttt{jumpExt()} wurde als \textbf{veraltet} markiert.

	\end{itemize}

\end{frame}


% ------------------------------------------------------------------------------
% Deprecation | 89215 | jQuery.clearable

\begin{frame}[fragile]
	\frametitle{Veraltete/Entfernte Funktionen}
	\framesubtitle{Veraltetes JavaScript (2)}

	% decrease font size for code listing
	\lstset{basicstyle=\tiny\ttfamily}

	\begin{itemize}
		\item Das jQuery Plugin \texttt{jquery.clearable}, das eine Schaltflächte zum Löschen eines
			 Eingabefeldes bereitgestellt wurde, wurde als \textbf{veraltet} markiert.
		\item Migration: verwenden Sie den Modul \small\texttt{TYPO3/CMS/Backend/Input/Clearable}\normalsize
			und die Methode \texttt{clearable()} auf einem nativen HTMLInputElement.

\begin{lstlisting}
require(['TYPO3/CMS/Backend/Input/Clearable'], function() {
  const inputField = document.querySelector('#myinput');
  if (inputField !== null) {
    inputField.clearable();
  }

  const clear = Array.from(document.querySelectorAll('.t3js-clearable')).filter(inputElement => {
    return !inputElement.classList.contains('t3js-datetimepicker');
  });
  clear.forEach(clearableField => clearableField.clearable());
});
\end{lstlisting}

	\end{itemize}

\end{frame}

% ------------------------------------------------------------------------------
% Deprecation | 88839 | CLI lowlevel request handlers

\begin{frame}[fragile]
	\frametitle{Veraltete/Entfernte Funktionen}
	\framesubtitle{CLI Command Handler}

	\begin{itemize}
		\item CLI-Befehle werden mit der Klasse \texttt{CommandApplication} verarbeitet.
		\item Diese Klasse ist ein Wrapper um die
			\href{https://symfony.com/doc/current/components/console.html}{Symfony Konsole}.

		\item Die zuvor verwendete Schnittstelle und die Klasse \texttt{CommandRequestHandler} wurden als \textbf{veraltet} markiert:

			\begin{itemize}
				\item
					\texttt{TYPO3\textbackslash
						CMS\textbackslash
						Core\textbackslash
						Console\textbackslash
						RequestHandlerInterface}
				\item
					\texttt{TYPO3\textbackslash
						CMS\textbackslash
						Core\textbackslash
						Console\textbackslash
						CommandRequestHandler}
			\end{itemize}

	\end{itemize}

\end{frame}

% ------------------------------------------------------------------------------
% Deprecation | 88850 | ContentObjectRenderer::sendNotifyEmail
% Deprecation | 88787 | BackendUtility::editOnClick

\begin{frame}[fragile]
	\frametitle{Veraltete/Entfernte Funktionen}
	\framesubtitle{Sonstiges}

	\begin{itemize}
		\item Die Mail-Functionalität sollte nicht in der Klasse\newline
			\small
				\texttt{TYPO3\textbackslash
					CMS\textbackslash
					Frontend\textbackslash
					ContentObject\textbackslash
					ContentObjectRenderer} verwendet werden.\newline
			\normalsize
			Daher wurde die Methode \texttt{sendNotifyEmail()} als \textbf{veraltet} markiert und wird in der TYPO3 v11 entfernt werden.

		\item Die Methode \texttt{editOnClick()}, die zum Generieren von JavaScript \texttt{onclick}-Zielen
			verwendet wurde, wurde in der folgenden Klasse als \textbf{veraltet} markiert:\newline
			\small
				\texttt{TYPO3\textbackslash
					CMS\textbackslash
					Backend\textbackslash
					Utility\textbackslash
					BackendUtility}.
			\normalsize

	\end{itemize}

\end{frame}

% ------------------------------------------------------------------------------
% Deprecation | 89127 | Cleanup RecordHistory handling

\begin{frame}[fragile]
	\frametitle{Veraltete/Entfernte Funktionen}
	\framesubtitle{RecordHistory Handling}

	Änderungen wurden an folgender Klasse vorgenommen:
		\smaller
			\texttt{TYPO3\textbackslash
				CMS\textbackslash
				Backend\textbackslash
				History\textbackslash
				RecordHistory}:
		\normalsize

	\begin{itemize}

		\item Die Sichtbarkeit der Eigenschaften \texttt{changeLog} und \texttt{lastHistoryEntry}
			wurden auf \texttt{protected} geändert (und die Funktion public Getter wurde auch hinzugefügt).
		\item Die Sichtbarkeit der Methoden \texttt{getHistoryEntry()} und \texttt{getHistoryData()}
			wurde auf \texttt{protected} geändert.
		\item Die folgenden Methoden wurden als \textbf{veraltet} makiert:

			\begin{itemize}\smaller
				\item \texttt{createChangeLog()}
				\item \texttt{shouldPerformRollback()}
				\item \texttt{getElementData()}
				\item \texttt{performRollback()}
				\item \texttt{createMultipleDiff()}
				\item \texttt{setLastHistoryEntry()}
			\end{itemize}\normalsize

	\end{itemize}

\end{frame}

% ------------------------------------------------------------------------------
% Deprecation | 89037 | Deprecated LocallangXmlParser

\begin{frame}[fragile]
	\frametitle{Veraltete/Entfernte Funktionen}
	\framesubtitle{XML-Sprachdateien}

	\begin{itemize}
		\item Das XLIFF-Format wird für Sprachdateien seit TYPO3 v4.6 verwendet.
		\item Die Verwendung von XML-Sprachdateien wird nun als \textbf{veraltet}
			markiert und löst eine Warnung/einen Fehler aus.
		\item Dazu gehört auch die Ausführung des folgenden XML-Parameters:\newline
			\small
				\texttt{TYPO3\textbackslash
					CMS\textbackslash
					Core\textbackslash
					Localization\textbackslash
					Parser\textbackslash
					LocallangXmlParser}
			\normalsize
	\end{itemize}

\end{frame}

% ------------------------------------------------------------------------------
