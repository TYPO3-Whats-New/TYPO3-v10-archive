% ------------------------------------------------------------------------------
% TYPO3 Version 10.1 - What's New (Serbian Version)
%
% @license	Creative Commons BY-NC-SA 3.0
% @link		http://typo3.org/download/release-notes/whats-new/
% @language	Serbian
% ------------------------------------------------------------------------------

\section{Zastarele/izbačene funkcije}
\begin{frame}[fragile]
	\frametitle{Zastarele/izbačene funkcije}

	\begin{center}\huge{Poglavlje 4:}\end{center}
	\begin{center}\huge{\color{typo3darkgrey}\textbf{Zastarele/izbačene funkcije}}\end{center}

\end{frame}

% ------------------------------------------------------------------------------
% Deprecation | 88854 | T3_THIS_LOCATION
% Deprecation | 88862 | T3_RETURN_URL
% Deprecation | 89033 | jumpToUrl
% Deprecation | 88854 | jumpExt() of RecordListController

\begin{frame}[fragile]
	\frametitle{Zastarele/izbačene funkcije}
	\framesubtitle{JavaScript zastarelosti (1)}

	\begin{itemize}
		\item Dve globalne JavaScript promenljive su označene kao \textbf{zastarele}:

			\begin{itemize}
				\item \texttt{T3\_THIS\_LOCATION}
				\item \texttt{T3\_RETURN\_URL}
			\end{itemize}

		\item Dobro poznata JavaScript funkcija \texttt{jumpToUrl()} je označena kao \textbf{zastarela}.
			Opcije za migraciju:

			\begin{itemize}
				\item koristite \texttt{window.location.href = '...';}
				\item ili koristite link u HTML-u \texttt{<a href="...">link</a>}
			\end{itemize}

		\item JavaScript funkcija \texttt{jumpExt()} je označena kao \textbf{zastarela}.

	\end{itemize}

\end{frame}


% ------------------------------------------------------------------------------
% Deprecation | 89215 | jQuery.clearable

\begin{frame}[fragile]
	\frametitle{Zastarele/izbačene funkcije}
	\framesubtitle{JavaScript zastarelosti (2)}

	% decrease font size for code listing
	\lstset{basicstyle=\tiny\ttfamily}

	\begin{itemize}
		\item jQuery plagin \texttt{jquery.clearable}, koji omogućava dugme da se očisti
			polje unosa, je označen kao \textbf{zastareo}.
		\item Migracija: koristite modul \small\texttt{TYPO3/CMS/Backend/Input/Clearable}\normalsize 
			i metod \texttt{clearable()} na HTMLInputElement.

\begin{lstlisting}
require(['TYPO3/CMS/Backend/Input/Clearable'], function() {
  const inputField = document.querySelector('#myinput');
  if (inputField !== null) {
    inputField.clearable();
  }

  const clear = Array.from(document.querySelectorAll('.t3js-clearable')).filter(inputElement => {
    return !inputElement.classList.contains('t3js-datetimepicker');
  });
  clear.forEach(clearableField => clearableField.clearable());
});
\end{lstlisting}

	\end{itemize}

\end{frame}

% ------------------------------------------------------------------------------
% Deprecation | 88839 | CLI lowlevel request handlers

\begin{frame}[fragile]
	\frametitle{Zastarele/izbačene funkcije}
	\framesubtitle{Upravljač CLI komandama}

	\begin{itemize}
		\item CLI komande se upravljaju korišćenjem \texttt{CommandApplication} klase.
		\item Ova klasa je omotač oko
			\href{https://symfony.com/doc/current/components/console.html}{Symfony Console}.

		\item Prethodno korišćeni interfejs i klasa \texttt{CommandRequestHandler} su označeni kao \textbf{zastareli}:

			\begin{itemize}
				\item
					\texttt{TYPO3\textbackslash
						CMS\textbackslash
						Core\textbackslash
						Console\textbackslash
						RequestHandlerInterface}
				\item
					\texttt{TYPO3\textbackslash
						CMS\textbackslash
						Core\textbackslash
						Console\textbackslash
						CommandRequestHandler}
			\end{itemize}

	\end{itemize}

\end{frame}

% ------------------------------------------------------------------------------
% Deprecation | 88850 | ContentObjectRenderer::sendNotifyEmail
% Deprecation | 88787 | BackendUtility::editOnClick

\begin{frame}[fragile]
	\frametitle{Zastarele/izbačene funkcije}
	\framesubtitle{Razno}

	\begin{itemize}
		\item Funkcionalnosti za email ne smeju se uključivati u klasi\newline
			\small
				\texttt{TYPO3\textbackslash
					CMS\textbackslash
					Frontend\textbackslash
					ContentObject\textbackslash
					ContentObjectRenderer}.\newline
			\normalsize
			Zato, metod \texttt{sendNotifyEmail()} je označen kao \textbf{zastareo} i biće uklonjen u TYPO3 v11.

		\item Metod \texttt{editOnClick()} korišćen da se generiše JavaScript \texttt{onclick}
			je označen kao \textbf{zastareo} u sledećoj klasi:\newline
			\small
				\texttt{TYPO3\textbackslash
					CMS\textbackslash
					Backend\textbackslash
					Utility\textbackslash
					BackendUtility}.
			\normalsize

	\end{itemize}

\end{frame}

% ------------------------------------------------------------------------------
% Deprecation | 89127 | Cleanup RecordHistory handling

\begin{frame}[fragile]
	\frametitle{Zastarele/izbačene funkcije}
	\framesubtitle{Upravljanje istorijom rekorda (RecordHistory)}

	Izmene u klasi
		\smaller
			\texttt{TYPO3\textbackslash
				CMS\textbackslash
				Backend\textbackslash
				History\textbackslash
				RecordHistory}:
		\normalsize

	\begin{itemize}

		\item Vidljivost osobina \texttt{changeLog} i \texttt{lastHistoryEntry}
			promenjena je na \texttt{protected} (i javna geter metoda je dodata).
		\item Vidljivost metoda \texttt{getHistoryEntry()} i \texttt{getHistoryData()}
			promenjena je na \texttt{protected}.
		\item Sledeći metodi su označeni kao \textbf{zastareli}:

			\begin{itemize}\smaller
				\item \texttt{createChangeLog()}
				\item \texttt{shouldPerformRollback()}
				\item \texttt{getElementData()}
				\item \texttt{performRollback()}
				\item \texttt{createMultipleDiff()}
				\item \texttt{setLastHistoryEntry()}
			\end{itemize}\normalsize

	\end{itemize}

\end{frame}

% ------------------------------------------------------------------------------
% Deprecation | 89037 | Deprecated LocallangXmlParser

\begin{frame}[fragile]
	\frametitle{Zastarele/izbačene funkcije}
	\framesubtitle{XML jezički fajlovi}

	\begin{itemize}
		\item XLIFF je korišćen za jezičke fajlove od TYPO3 v4.6.
		\item Korišćenje XML jezičkih fajlova je sada označeno kao \textbf{zastarelo}
			 i izaziva upozorenje/grešku.
		\item Ovo uključuje izvršavanje sledećeg XML parsera:\newline
			\small
				\texttt{TYPO3\textbackslash
					CMS\textbackslash
					Core\textbackslash
					Localization\textbackslash
					Parser\textbackslash
					LocallangXmlParser}
			\normalsize
	\end{itemize}

\end{frame}

% ------------------------------------------------------------------------------
