% ------------------------------------------------------------------------------
% TYPO3 Version 10.1 - What's New (Serbian Version)
%
% @license	Creative Commons BY-NC-SA 3.0
% @link		http://typo3.org/download/release-notes/whats-new/
% @language	Serbian
% ------------------------------------------------------------------------------

\section{Razno}
\begin{frame}[fragile]
	\frametitle{Razno}

	\begin{center}\huge{Poglavlje 5:}\end{center}
	\begin{center}\huge{\color{typo3darkgrey}\textbf{Razno}}\end{center}

\end{frame}

% ------------------------------------------------------------------------------
% Feature | 78488 | Add rel=”noopener noreferrer” to external links

\begin{frame}[fragile]
	\frametitle{Razno}
	\framesubtitle{Pojačana sigurnost}

	\begin{itemize}
		\item Spoljni linkovi koji se generišu putem TypoLink-a, ili linkovi koji koriste \texttt{\_blank},
			sada prikazuju atribut \texttt{rel="noopener noreferrer"}.
		\item Ovo ima za cilj da ojača sigurnost TYPO3 sajta:

			\begin{itemize}
				\item "\textbf{noopener}" upućuje pretraživač da otvori link bez generisanja
				 	novog konteksta pristupa sa dokumentom koji ga je otvorio.
				\item "\textbf{noreferrer}" sprečava pretraživač, kada prelazi na novu stranicu,
					da pošalje adresu stranice, ili bilo koju drugu vrednost, kao referentnu u
					\texttt{Referer:} HTTP header.
			\end{itemize}

	\end{itemize}

\end{frame}

% ------------------------------------------------------------------------------
% Feature | 88742 | Import YAML files relative to the current YAML file

\begin{frame}[fragile]
	\frametitle{Razno}
	\framesubtitle{Uključivanje YAML fajlova}

    % decrease font size for code listing
	\lstset{basicstyle=\tiny\ttfamily}

	\begin{itemize}
		\item YAML fajlovi se već mogu uključivati od strane drugih YAML fajlova korišćenjem sledeće sintakse:

\begin{lstlisting}
imports:
  - { resource: "EXT:my_extension/Configuration/FooBar/Example.yaml" }

another:
  option: true
\end{lstlisting}

		\item Ovo je prošireno da mogu da se uvezu i resursi relativni u odnosu na trenutni YAML fajl:

\begin{lstlisting}
imports:
  - { resource: "subfolder/AnotherExample.yaml" }
  - { resource: "../path/to/configuration/AnotherExample.yaml" }

another:
  option: true
\end{lstlisting}

	\end{itemize}

\end{frame}

% ------------------------------------------------------------------------------
