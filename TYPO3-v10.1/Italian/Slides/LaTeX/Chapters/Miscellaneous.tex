% ------------------------------------------------------------------------------
% TYPO3 Version 10.1 - What's New (Italian Version)
%
% @license	Creative Commons BY-NC-SA 3.0
% @link		http://typo3.org/download/release-notes/whats-new/
% @language	Italian
% ------------------------------------------------------------------------------

\section{Miscellaneous}
\begin{frame}[fragile]
	\frametitle{Varie}

	\begin{center}\huge{Capitolo 5:}\end{center}
	\begin{center}\huge{\color{typo3darkgrey}\textbf{Varie}}\end{center}

\end{frame}

% ------------------------------------------------------------------------------
% Feature | 78488 | Add rel=”noopener noreferrer” to external links

\begin{frame}[fragile]
	\frametitle{Varie}
	\framesubtitle{Strengthen Security}

	\begin{itemize}
		\item I link esterni generati da TypoLink, o i link che utilizzano \texttt{\_blank},
			mostrano ora l'attributo \texttt{rel="noopener noreferrer"}.
		\item Questo punta a rafforzare la sicurezza del sito TYPO3:

			\begin{itemize}
				\item "\textbf{noopener}" indica al browser di aprire il link senza concedere
					al nuovo contesto di navigazione l'accesso al documento che lo ha aperto.
				\item "\textbf{noreferrer}" impedisce al browser, durante la navigazione verso un'altra pagina,
					di inviare l'indirizzo della pagina o  qualsiasi altro valore, come referrer tramite il
					\texttt{Referer:} HTTP.
			\end{itemize}

	\end{itemize}

\end{frame}

% ------------------------------------------------------------------------------
% Feature | 88742 | Import YAML files relative to the current YAML file

\begin{frame}[fragile]
	\frametitle{Varie}
	\framesubtitle{Inclusione file YAML}

    % decrease font size for code listing
	\lstset{basicstyle=\tiny\ttfamily}

	\begin{itemize}
		\item I file YAML possono essere inclusi da altri file YAML utilizzando la sintassi seguente:

\begin{lstlisting}
imports:
  - { resource: "EXT:my_extension/Configuration/FooBar/Example.yaml" }

another:
  option: true
\end{lstlisting}

		\item Questo è stato esteso per importare risorse relative al file YAML corrente:

\begin{lstlisting}
imports:
  - { resource: "subfolder/AnotherExample.yaml" }
  - { resource: "../path/to/configuration/AnotherExample.yaml" }

another:
  option: true
\end{lstlisting}

	\end{itemize}

\end{frame}

% ------------------------------------------------------------------------------
