% ------------------------------------------------------------------------------
% TYPO3 Version 10.1 - What's New (Italian Version)
%
% @license	Creative Commons BY-NC-SA 3.0
% @link		http://typo3.org/download/release-notes/whats-new/
% @language	Italian
% ------------------------------------------------------------------------------

\section{Funzioni deprecate/rimosse}
\begin{frame}[fragile]
	\frametitle{Funzioni deprecate/rimosse}

	\begin{center}\huge{Capitolo 4:}\end{center}
	\begin{center}\huge{\color{typo3darkgrey}\textbf{Funzioni deprecate/rimosse}}\end{center}

\end{frame}

% ------------------------------------------------------------------------------
% Deprecation | 88854 | T3_THIS_LOCATION
% Deprecation | 88862 | T3_RETURN_URL
% Deprecation | 89033 | jumpToUrl
% Deprecation | 88854 | jumpExt() of RecordListController

\begin{frame}[fragile]
	\frametitle{Funzioni deprecate/rimosse}
	\framesubtitle{Deprecazioni JavaScript (1)}

	\begin{itemize}
		\item Due variabili globali JavaScript sono state segnate come \textbf{deprecate}:

			\begin{itemize}
				\item \texttt{T3\_THIS\_LOCATION}
				\item \texttt{T3\_RETURN\_URL}
			\end{itemize}

		\item La nota funzione JavaScript \texttt{jumpToUrl()} è stata segnata come \textbf{deprecata}.
			Possibilità di migrazione:

			\begin{itemize}
				\item usa \texttt{window.location.href = '...';}
				\item oppure usa un link nell'HTML come \texttt{<a href="...">link</a>}
			\end{itemize}

		\item La funzione JavaScript \texttt{jumpExt()} è stata segnata come \textbf{deprecata}.

	\end{itemize}

\end{frame}


% ------------------------------------------------------------------------------
% Deprecation | 89215 | jQuery.clearable

\begin{frame}[fragile]
	\frametitle{Funzioni deprecate/rimosse}
	\framesubtitle{Deprecazioni JavaScript (2)}

	% decrease font size for code listing
	\lstset{basicstyle=\tiny\ttfamily}

	\begin{itemize}
		\item Il plugin jQuery \texttt{jquery.clearable},
			che fornisce un bottone per cancellare un campo di input, è stato segnato come \textbf{deprecato}.
		\item Migrazione: usa il modulo \small\texttt{TYPO3/CMS/Backend/Input/Clearable}\normalsize e
			il metodo \texttt{clearable()} su un HTMLInputElement nativo.

\begin{lstlisting}
require(['TYPO3/CMS/Backend/Input/Clearable'], function() {
  const inputField = document.querySelector('#myinput');
  if (inputField !== null) {
    inputField.clearable();
  }

  const clear = Array.from(document.querySelectorAll('.t3js-clearable')).filter(inputElement => {
    return !inputElement.classList.contains('t3js-datetimepicker');
  });
  clear.forEach(clearableField => clearableField.clearable());
});
\end{lstlisting}

	\end{itemize}

\end{frame}

% ------------------------------------------------------------------------------
% Deprecation | 88839 | CLI lowlevel request handlers

\begin{frame}[fragile]
	\frametitle{Funzioni deprecate/rimosse}
	\framesubtitle{Gestore di comandi CLI}

	\begin{itemize}
		\item I comandi CLI vengono gestiti utilizzando la classe \texttt{CommandApplication}.
		\item Questa classe è un wrapper attorno alla
			\href{https://symfony.com/doc/current/components/console.html}{Symfony Console}.

		\item La precedente interfaccia e la classe \texttt{CommandRequestHandler} sono state segnate come \textbf{deprecate}:

			\begin{itemize}
				\item
					\texttt{TYPO3\textbackslash
						CMS\textbackslash
						Core\textbackslash
						Console\textbackslash
						RequestHandlerInterface}
				\item
					\texttt{TYPO3\textbackslash
						CMS\textbackslash
						Core\textbackslash
						Console\textbackslash
						CommandRequestHandler}
			\end{itemize}

	\end{itemize}

\end{frame}

% ------------------------------------------------------------------------------
% Deprecation | 88850 | ContentObjectRenderer::sendNotifyEmail
% Deprecation | 88787 | BackendUtility::editOnClick

\begin{frame}[fragile]
	\frametitle{Funzioni deprecate/rimosse}
	\framesubtitle{Varie}

	\begin{itemize}
		\item La funzionalità delle email non deve essere inclusa nella classe\newline
			\small
				\texttt{TYPO3\textbackslash
					CMS\textbackslash
					Frontend\textbackslash
					ContentObject\textbackslash
					ContentObjectRenderer}.\newline
			\normalsize
			Pertanto, il metodo \texttt{sendNotifyEmail()} è stato segnato come \textbf{deprecato} e sarà rimosso in TYPO3 v11.

		\item Il metodo \texttt{editOnClick()} usato per generare target Javascript \texttt{onclick}
			è stato segnato come \textbf{deprecato} nella seguente classe:\newline
			\small
				\texttt{TYPO3\textbackslash
					CMS\textbackslash
					Backend\textbackslash
					Utility\textbackslash
					BackendUtility}.
			\normalsize

	\end{itemize}

\end{frame}

% ------------------------------------------------------------------------------
% Deprecation | 89127 | Cleanup RecordHistory handling

\begin{frame}[fragile]
	\frametitle{Funzioni deprecate/rimosse}
	\framesubtitle{Gestione della RecordHistory}

	Modifiche fatte alla classe
		\smaller
			\texttt{TYPO3\textbackslash
				CMS\textbackslash
				Backend\textbackslash
				History\textbackslash
				RecordHistory}:
		\normalsize

	\begin{itemize}

		\item Visibilità delle proprietà \texttt{changeLog} e \texttt{lastHistoryEntry}
			modificate in \texttt{protected} (e aggiunta una funzione pubblica getter).
		\item Visibilità dei metodi \texttt{getHistoryEntry()} e \texttt{getHistoryData()}
			modificate in \texttt{protected}.
		\item I seguenti metodi sono stati segnati come \textbf{deprecati}:

			\begin{itemize}\smaller
				\item \texttt{createChangeLog()}
				\item \texttt{shouldPerformRollback()}
				\item \texttt{getElementData()}
				\item \texttt{performRollback()}
				\item \texttt{createMultipleDiff()}
				\item \texttt{setLastHistoryEntry()}
			\end{itemize}\normalsize

	\end{itemize}

\end{frame}

% ------------------------------------------------------------------------------
% Deprecation | 89037 | Deprecated LocallangXmlParser

\begin{frame}[fragile]
	\frametitle{Funzioni deprecate/rimosse}
	\framesubtitle{File XML per le lingue}

	\begin{itemize}
		\item Il formato XLIFF è utilizzato per i file delle lingue a partire da TYPO3 v4.6.
		\item L'utilizzo dei file XML per le lingue sono segnati come \textbf{deprecati}
			e generano un warning/error.
		\item Questo comporta l'esecuzione del seguente parser XML:\newline
			\small
				\texttt{TYPO3\textbackslash
					CMS\textbackslash
					Core\textbackslash
					Localization\textbackslash
					Parser\textbackslash
					LocallangXmlParser}
			\normalsize
	\end{itemize}

\end{frame}

% ------------------------------------------------------------------------------
