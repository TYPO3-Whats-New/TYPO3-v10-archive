% ------------------------------------------------------------------------------
% TYPO3 Version 10.3 - What's New (French Version)
%
% @license	Creative Commons BY-NC-SA 3.0
% @link		https://typo3.org/help/documentation/whats-new/
% @language	French
% ------------------------------------------------------------------------------

\section{Fonctions dépréciées et retirées}
\begin{frame}[fragile]
	\frametitle{Fonctions dépréciées et retirées}

	\begin{center}\huge{Chapitre 4~:}\end{center}
	\begin{center}\huge{\color{typo3darkgrey}\textbf{Fonctions dépréciées et retirées}}\end{center}

\end{frame}

% ------------------------------------------------------------------------------
% Deprecation | 89463 | Deprecate switchable controller actions

\begin{frame}[fragile]
	\frametitle{Fonctions dépréciées et retirées}
	\framesubtitle{Switchable Controller Actions}

	\begin{itemize}
		\item Les «~Switchable Controller Actions~» (SCA) sont marquées \textbf{dépréciées}.
		\item Les SCA sont utilisés pour surcharger l'ensemble des contrôleurs et actions à l'aide
			du TypoScript ou des Flexforms.
		\item Utiliser le même plugin comme point d'entrée de nombreuses fonctionnalités contredit
			le concept d'une plugin pour un objectif précis.
		\item Les plugins qui utilisent SCA devraient être scindés en différents plugins.
	\end{itemize}

\end{frame}

% ------------------------------------------------------------------------------
% Deprecation | 90007 | Global constants TYPO3_version and TYPO3_branch

\begin{frame}[fragile]
	\frametitle{Fonctions dépréciées et retirées}
	\framesubtitle{Constantes globales}

	% decrease font size for code listing
	\lstset{basicstyle=\smaller\ttfamily}

	\begin{itemize}
		\item Ces deux constantes globales sont marquées \textbf{dépréciées}~:

			\begin{itemize}
				\item \texttt{TYPO3\_version}
				\item \texttt{TYPO3\_branch}
			\end{itemize}

		\item Cette classe PHP doit être utilisée à la place~:\newline
			\small
				\texttt{TYPO3\textbackslash
					CMS\textbackslash
					Core\textbackslash
					Information\textbackslash
					Typo3Version}\normalsize

	\end{itemize}

\end{frame}

% ------------------------------------------------------------------------------
% Deprecation | 89673 | Deprecate Extbases WebRequest and WebResponse

\begin{frame}[fragile]
	\frametitle{Fonctions dépréciées et retirées}
	\framesubtitle{Extbase~: \texttt{WebRequest}/\texttt{WebResponse}}

	\begin{itemize}
		\item Ces deux classes Extbase sont marquées \textbf{dépréciées}~:
			\begin{itemize}
				\item \texttt{TYPO3\textbackslash
					CMS\textbackslash
					Extbase\textbackslash
					Mvc\textbackslash
					Web\textbackslash
					Request}
				\item \texttt{TYPO3\textbackslash
					CMS\textbackslash
					Extbase\textbackslash
					Mvc\textbackslash
					Web\textbackslash
					Response}
			\end{itemize}

	\end{itemize}

\end{frame}

% ------------------------------------------------------------------------------
% Deprecation | 90258 | Simplified RTE Parser API

\begin{frame}[fragile]
	\frametitle{Fonctions dépréciées et retirées}
	\framesubtitle{API de l'analyseur RTE simplifiée}

	\begin{itemize}
		\item La classe PHP \texttt{RteHtmlParser} fournie une API simplifiée.
		\item En conséquence, ces deux méthodes sont marquées \textbf{dépréciées}~:

			\begin{itemize}
				\item \texttt{TYPO3\textbackslash
					CMS\textbackslash
					Core\textbackslash
					Html\textbackslash
					RteHtmlParser->init()}
				\item \texttt{TYPO3\textbackslash
					CMS\textbackslash
					Core\textbackslash
					Html\textbackslash
					RteHtmlParser->RTE\_transform()}
			\end{itemize}

	\end{itemize}

\end{frame}

% ------------------------------------------------------------------------------
% Deprecation | 89139 | Console Commands configuration migrated to Symfony service tags

\begin{frame}[fragile]
	\frametitle{Fonctions dépréciées et retirées}
	\framesubtitle{Configuration des commandes console}

	\begin{itemize}
		\item Comme les configurations des commandes de console sont migrées en annotation de service Symfony,
			le fichier de configuration des commandes de console \texttt{Configuration/Commands.php}
			est marqué \textbf{déprécié}.
		\item Utilisez l'annotation d'injection de dépendance de service Symfony \texttt{console.command}.

	\end{itemize}

\end{frame}

% ------------------------------------------------------------------------------
% Important | 89672 | transOrigPointerField is not longer allowed to be excluded

\begin{frame}[fragile]
	\frametitle{Fonctions dépréciées et retirées}
	\framesubtitle{TCA~: \texttt{transOrigPointerField}}

	\begin{itemize}
		\item Lorsque le champ indiqué dans l'option TCA suivante pouvait être exclus,
			les données enregistrées en base pouvant devenir incohérentes~:
			\small
				\texttt{\$GLOBALS['TCA'][\$table]['ctrl']['transOrigPointerField']}
			\normalsize

		\item En conséquence, il n'est plus possible d'exclure le champ.
		\item Un assistant de migration retire l'option d'exclusion du TCA et ajoute
			un message de dépréciation lorsque du code doit être mis à jour.
	\end{itemize}

\end{frame}

% ------------------------------------------------------------------------------
% Deprecation | 90421 | DocumentTemplate

\begin{frame}[fragile]
	\frametitle{Fonctions dépréciées et retirées}
	\framesubtitle{DocumentTemplate}

	% decrease font size for code listing
	\lstset{basicstyle=\tiny\ttfamily}

	\begin{itemize}
		\item La classe suivante est marquée \textbf{dépréciée}~:

			\begin{itemize}
				\item \texttt{TYPO3\textbackslash
					CMS\textbackslash
					Backend\textbackslash
					Template\textbackslash
					DocumentTemplate}
			\end{itemize}

		\item Elle était utilisée comme base pour le rendu des modules backend ou pour des rendus HTML du backend.
		\item Depuis TYPO3 v7, l'API de ModuleTemplate doit être utilisée.

\vspace{-0.4cm}
\begin{lstlisting}
use TYPO3\CMS\Backend\Template\ModuleTemplate;
...
$moduleTemplate = GeneralUtility::makeInstance(ModuleTemplate::class);
$content = $this->getHtmlContentFromMyModule();
$moduleTemplate->setTitle('My module');
$moduleTemplate->setContent($content);
return new HtmlResponse($moduleTemplate->renderContent());
\end{lstlisting}

	\end{itemize}

\end{frame}

% ------------------------------------------------------------------------------
% Deprecation | 85613 | Introduce simple way to register category fields
%
%\begin{frame}[fragile]
%	\frametitle{Fonctions dépréciées et retirées}
%	\framesubtitle{CategoryRegistry API}
%
%	\begin{itemize}
%		\item La classe suivante est marquée \textbf{dépréciée}~:
%
%			\begin{itemize}
%				\item \texttt{TYPO3\textbackslash
%					CMS\textbackslash
%					Core\textbackslash
%					Category\textbackslash
%					CategoryRegistry}
%			\end{itemize}
%
%		\item La méthode suivante est marquée \textbf{dépréciée}~:
%
%			\begin{itemize}
%				\item \texttt{ExtgensionManagementUtility::makeCategorizable()}
%			\end{itemize}
%
%		\item Les champs de catégorie peuvent être inscrits facilement en configuration TCA.
%
%	\end{itemize}
%
%\end{frame}
%
% ------------------------------------------------------------------------------
% Deprecation | 90390 | Deprecate BrokenLinkRepository::getNumberOfBrokenLinks() in linkvalidator

\begin{frame}[fragile]
	\frametitle{Fonctions dépréciées et retirées}
	\framesubtitle{Validateur de liens}

	\begin{itemize}
		\item La méthode suivante est marquée \textbf{dépréciée}~:
		\newline\newline
			\smaller
				\texttt{TYPO3\textbackslash
					CMS\textbackslash
					Linkvalidator\textbackslash
					Repository\textbackslash
					BrokenLinkRepository}\newline
				\texttt{->getNumberOfBrokenLinks()}\normalsize\newline

		\item Utilisez cette méthode à la place~:\newline
			\small\texttt{BrokenLinkRepository::isLinkTargetBrokenLink()}\normalsize

	\end{itemize}

\end{frame}


% ------------------------------------------------------------------------------
