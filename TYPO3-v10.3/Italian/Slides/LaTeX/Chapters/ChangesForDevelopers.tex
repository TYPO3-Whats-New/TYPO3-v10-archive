% ------------------------------------------------------------------------------
% TYPO3 Version 10.3 - What's New (Italian Version)
%
% @license	Creative Commons BY-NC-SA 3.0
% @link		https://typo3.org/help/documentation/whats-new/
% @language	Italian
% ------------------------------------------------------------------------------

\section{Modifiche per sviluppatori}
\begin{frame}[fragile]
	\frametitle{Modifiche per sviluppatori}

	\begin{center}\huge{Capitolo 3:}\end{center}
	\begin{center}\huge{\color{typo3darkgrey}\textbf{Modifiche per sviluppatori}}\end{center}

\end{frame}

% ------------------------------------------------------------------------------
% Feature | 90333 | Dashboard

\begin{frame}[fragile]
	\frametitle{Modifiche per sviluppatori}
	\framesubtitle{Dashboard (1)}

	% decrease font size for code listing
	\lstset{basicstyle=\smaller\ttfamily}

	\begin{itemize}
		\item Gli sviluppatori possono creare widget personalizzati per la dashboard estendendo uno dei seguenti widget \textit{abstract}:

			\begin{itemize}
				\item \texttt{AbstractWidget}\newline
					\small
						Un abstract di base che può essere utilizzato come inizio di semplici widget.
					\normalsize
				\item \texttt{AbstractRssWidget}\newline
					\small
						Un abstract per visualizzare un widget che mostra un feed RSS.
					\normalsize
				\item \texttt{AbstractListWidget}\newline
					\small
						Un abstract per creare un widget che mostra un elenco di elementi
					\normalsize
				\item \texttt{AbstractCtaButtonWidget}\newline
					\small
						Un abstract per mostrare un widget che mostra un pulsante di "invito all'azione"
					\normalsize
			\end{itemize}

	\end{itemize}

\end{frame}

% ------------------------------------------------------------------------------
% Feature | 90333 | Dashboard

\begin{frame}[fragile]
	\frametitle{Modifiche per sviluppatori}
	\framesubtitle{Dashboard (2)}

	% decrease font size for code listing
	\lstset{basicstyle=\tiny\ttfamily}

	\begin{itemize}
		\item Registra i tuoi widget nei seguenti file della tua estensione:\newline
			\texttt{EXT:my\_extension/Configuration/Services.yaml}

		\item Opzione 1: identificatore del widget come attributo

\vspace{-0.4cm}
\begin{lstlisting}
Vendor\MyExtension\Widgets\MyFirstWidget:
  tags:
    - name: dashboard.widget
      identifier: widget-identifier-1
      widgetGroups: 'general'
\end{lstlisting}

		\item Opzione 2: un nome del servizio personalizzato consente a più identificatori di widget di condividere una classe

\vspace{-0.4cm}
\begin{lstlisting}
widget.identifier:
  class: Vendor\MyExtension\Widgets\MySecondWidget
  tags:
    - name: dashboard.widget
      identifier: widget-identifier-2
      widgetGroups: 'general, typo3'
\end{lstlisting}

	\end{itemize}

\end{frame}

% ------------------------------------------------------------------------------
% Feature | 90333 | Dashboard

\begin{frame}[fragile]
	\frametitle{Modifiche per sviluppatori}
	\framesubtitle{Dashboard (3)}

	% decrease font size for code listing
	\lstset{basicstyle=\tiny\ttfamily}

	\begin{itemize}
		\item Ogni widget è collegato ad uno o più gruppi di widget.
		\item Questi gruppi sono mostrati in modale quando si aggiunge un nuovo widget alla dashboard.
		\item Gli sviluppatori possono configurare gruppi di widget personalizzati creando un file\newline
			\smaller
				\texttt{EXT:my\_extension/Configuration/Backend/DashboardWidgetGroups.php}
			\normalsize

\vspace{-0.4cm}
\begin{lstlisting}
return [
  'widgetGroup-exampleGroup' => [
    'title' => 'LLL:EXT:my_extension/Resources/Private/Language/locallang.xlf:widget_group_name',
  ],
];
\end{lstlisting}

	\end{itemize}

\end{frame}

% ------------------------------------------------------------------------------
% Feature | 89870 | New PSR-14 Events for Extbase-related signals

\begin{frame}[fragile]
	\frametitle{Modifiche per sviluppatori}
	\framesubtitle{Extbase e Fluid}

	% decrease font size for code listing
	\lstset{basicstyle=\tiny\ttfamily}

	\begin{itemize}
		\item I seguenti eventi basati su PSR-14 sono stati introdotti per signals correlati ad Extbase:

\vspace{-0.4cm}
\begin{lstlisting}
TYPO3\CMS\Extbase\Event\Mvc\AfterRequestDispatchedEvent
TYPO3\CMS\Extbase\Event\Mvc\BeforeActionCallEvent
TYPO3\CMS\Extbase\Event\Persistence\AfterObjectThawedEvent
TYPO3\CMS\Extbase\Event\Persistence\ModifyQueryBeforeFetchingObjectDataEvent
TYPO3\CMS\Extbase\Event\Persistence\ModifyResultAfterFetchingObjectDataEvent
TYPO3\CMS\Extbase\Event\Persistence\EntityAddedToPersistenceEvent
TYPO3\CMS\Extbase\Event\Persistence\EntityFinalizedAfterPersistenceEvent
TYPO3\CMS\Extbase\Event\Persistence\EntityUpdatedInPersistenceEvent
TYPO3\CMS\Extbase\Event\Persistence\EntityRemovedFromPersistenceEvent
TYPO3\CMS\Extbase\Event\Persistence\EntityPersistedEvent
\end{lstlisting}

		\item I signals esistenti sono stati sostituiti e non dovrebbero essere più utilizzati.

	\end{itemize}

\end{frame}

% ------------------------------------------------------------------------------
% Feature | 89644 | Add optional argument fields to editRecord ViewHelpers

\begin{frame}[fragile]
	\frametitle{Modifiche per sviluppatori}
	\framesubtitle{ViewHelper \texttt{editRecord}}

	% decrease font size for code listing
	\lstset{basicstyle=\tiny\ttfamily}

	\begin{itemize}
		\item Un \texttt{campo} argomento opzionale è stato aggiunto ai Viewhelper
			\texttt{uri.editRecord} e \texttt{link.editRecord}.
		\item Se impostato, FormEngine crea un form per modificare solo i campi del database specificato.
		\item L'esempio seguente crea un collegamento per modificare il campo \texttt{tt\_content.bodytext}
			 del record con UID 42.

\begin{lstlisting}
<be:link.editRecord uid="42" table="tt_content" fields="bodytext" returnUrl="foo/bar">
  Modifica record
</be:link.editRecord>
\end{lstlisting}

	\end{itemize}

\end{frame}

% ------------------------------------------------------------------------------
% Feature | xxxxx | Introduce AssetCollector

\begin{frame}[fragile]
	\frametitle{Modifiche per sviluppatori}
	\framesubtitle{AssetCollector}

	\begin{itemize}
		\item Sono state implementate le fasi iniziali di integrazione di un AssetCollector.
		\item Il concetto permette agli sviluppatori di aggiungere più volte un codice CSS/JS personalizzato (inline o esterno),
			ma TYPO3 lo serve solo una volta.
		\item A questo riguardo, sono stati aggiunti due nuovi ViewHelper di Fluid:
			\begin{itemize}
				\item \texttt{<f:css>}
				\item \texttt{<f:script>}
			\end{itemize}
		\item A lungo termine, AssetCollector mira a sostituire le varie opzioni TypoScript
			esistenti che sono piuttosto confuse.
	\end{itemize}

\end{frame}

% ------------------------------------------------------------------------------
% Feature | 86614 | Add PSR-14 event to control hreflang tags to be rendered

\begin{frame}[fragile]
	\frametitle{Modifiche per sviluppatori}
	\framesubtitle{Modificato \texttt{hreflang}-tag}

	% decrease font size for code listing
	\lstset{basicstyle=\smaller\ttfamily}

	\begin{itemize}
		\item Ora è possibile modificare i tag \texttt{hreflang} prima di essere renderizzati.
		\item Gli sviluppatori possono fare questo registrando un listener di eventi per il seguente evento:\newline
			\smaller
				\texttt{TYPO3\textbackslash
					CMS\textbackslash
					Frontend\textbackslash
					Event\textbackslash
					ModifyHrefLangTagsEvent}
			\normalsize
	\end{itemize}

\end{frame}

% ------------------------------------------------------------------------------
% Feature | 88818 | Introduce events to modify CKEditor configuration

\begin{frame}[fragile]
	\frametitle{Modifiche per sviluppatori}
	\framesubtitle{Modificare la configurazione di CKEditor}

	% decrease font size for code listing
	\lstset{basicstyle=\tiny\ttfamily}

	\begin{itemize}
		\item Sono stati introdotti i seguenti eventi basati su PSR-14 che consentono di modificare la configurazionedi CKEditor:

\vspace{-0.4cm}
\begin{lstlisting}
TYPO3\CMS\RteCKEditor\Form\Element\Event\AfterGetExternalPluginsEvent
TYPO3\CMS\RteCKEditor\Form\Element\Event\BeforeGetExternalPluginsEvent
TYPO3\CMS\RteCKEditor\Form\Element\Event\AfterPrepareConfigurationForEditorEvent
TYPO3\CMS\RteCKEditor\Form\Element\Event\BeforePrepareConfigurationForEditorEvent
\end{lstlisting}

		\item Vedi il
			\href{https://docs.typo3.org/c/typo3/cms-core/master/en-us/Changelog/10.3/Feature-88818-IntroduceEventsToModifyCKEditorConfiguration.html}{change log}
			per un esempio.
	\end{itemize}

\end{frame}

% ------------------------------------------------------------------------------
% Feature | 90265 | Show dispatched Events in Admin Panel

\begin{frame}[fragile]
	\frametitle{Modifiche per sviluppatori}
	\framesubtitle{Eventi PSR-14 nel pannello di amministrazione}

	\begin{itemize}
		\item Il pannello di amministrazione mostra tutti gli eventi PSR-14 che sono stati inviati nella richiesta corrente.
	\end{itemize}

	\begin{figure}
		\includegraphics[width=0.85\linewidth]{ChangesForDevelopers/90265-ShowDispatchedEventsInAdminPanel.png}
	\end{figure}

\end{frame}

% ------------------------------------------------------------------------------
% Feature | 89738 | API for AJAX Requests

\begin{frame}[fragile]
	\frametitle{Modifiche per sviluppatori}
	\framesubtitle{API per richieste AJAX}

	% decrease font size for code listing
	\lstset{basicstyle=\tiny\ttfamily}

	\begin{itemize}
		\item L' \textbf{API Fetch} è stata introdotta per eseguire richieste AJAX e
		    per rendere TYPO3 meno dipendente da jQuery.
		\item L'API fornisce una definizione generica di oggetti Richiesta e Risposta
			(e altre cose relative alle richieste di rete).
		\item Supportato da tutti i browser moderni, vedi le
			\href{https://developer.mozilla.org/en-US/docs/Web/API/Fetch_API}{tabelle di compatibilità}.
		\item Il core TYPO3 utilizza già la nuova API in Install Tool, FormEngine, e
			nei menù contestuali.
		\item Vedi le
			\href{https://docs.typo3.org/c/typo3/cms-core/master/en-us/Changelog/10.3/Feature-89738-ApiForAjaxRequests.html}{change log}
			per alcuni esempi su come usare le API Fetch.

	\end{itemize}

\end{frame}

% ------------------------------------------------------------------------------
% Feature | 89650 | Allow line breaks in TCA descriptions

\begin{frame}[fragile]
	\frametitle{Modifiche per sviluppatori}
	\framesubtitle{Campo descrizione del TCA}

	\begin{itemize}
		\item Il campo della descrizione del TCA ora può contenere delle interruzioni di riga
		    per rendere più leggibili i testi lunghi.
	\end{itemize}

\end{frame}

% ------------------------------------------------------------------------------
% Important | 90020 | Legacy BasicFileUtility and ExtendedFileUtility classes marked as internal

\begin{frame}[fragile]
	\frametitle{Modifiche per sviluppatori}
	\framesubtitle{Classi \texttt{BasicFileUtility} e \texttt{ExtendedFileUtility}}

	\begin{itemize}
		\item Le seguenti due classi sono state segnate come \textbf{internal}
			e non dovrebbero essere più utilizzate:

			\begin{itemize}\small
				\item \texttt{TYPO3\textbackslash
					CMS\textbackslash
					Core\textbackslash
					Utility\textbackslash
					File\textbackslash
					BasicFileUtility}
				\item \texttt{TYPO3\textbackslash
					CMS\textbackslash
					Core\textbackslash
					Utility\textbackslash
					File\textbackslash
					ExtendedFileUtility}
			\end{itemize}

		\item Gli sviluppatori di estensioni dovrebbero usare invece le classi \texttt{ResourceStorage}
			e \texttt{ResourceFactory} per la gestione delle risorse.

	\end{itemize}

\end{frame}

% ------------------------------------------------------------------------------
% Feature | 89139 | Add dependency injection support for console commands

\begin{frame}[fragile]
	\frametitle{Modifiche per sviluppatori}
	\framesubtitle{Comandi da console: Symfony DI Support}

	\begin{itemize}
		\item Ora le dipendenze dei comandi possono essere iniettate tramite il costruttore o altre tecniche di iniezione.
		\item Aggiungi il tag \texttt{console.command} alle classi dei comandi.
		\item Utilizzare l'attributo di tag \texttt{command} per specificare il nome del comando.
		\item L'attributo opzionale di tag \texttt{schedulable} può essere impostato a \texttt{false}
			per escludere il comando dallo scheduler di TYPO3.

		\item Vedi il
			\href{https://docs.typo3.org/c/typo3/cms-core/master/en-us/Changelog/10.3/Feature-89139-AddDependencyInjectionSupportForConsoleCommands.html}{change log}
			per un esempio.
	\end{itemize}

\end{frame}

% ------------------------------------------------------------------------------
% Feature | 90168 | Introduce Modal Actions

\begin{frame}[fragile]
	\frametitle{Modifiche per sviluppatori}
	\framesubtitle{Bottoni di azioni in modale}

	% decrease font size for code listing
	\lstset{basicstyle=\tiny\ttfamily}

	\begin{itemize}
		\item I popup modali supportano ora i bottoni di azione.
		\item In alternativa all'esistente opzione \texttt{trigger}, è possibile utilizzare
			la nuova opzione \texttt{action}.
		\item Per esempio:

\vspace{-0.4cm}
\begin{lstlisting}
Modal.confirm('Header', 'Some content', Severity.error, [
  {
    text: 'Based on trigger()',
    trigger: function () {
      console.log('Vintage!');
    }
  },
  {
    text: 'Based on action()',
    action: new DeferredAction(() => {
      return new AjaxRequest('/any/endpoint').post({});
    })
  }
]);
\end{lstlisting}

	\end{itemize}

\end{frame}

% ------------------------------------------------------------------------------
% Feature | 90471 | JavaScript Event API

\begin{frame}[fragile]
	\frametitle{Modifiche per sviluppatori}
	\framesubtitle{API di eventi JavaScript}

	\begin{itemize}
		\item Una nuova API di eventi consente agli sviluppatori JavaScript di avere un'interfaccia di ascolto stabile degli eventi.
		\item L'API si occupa delle insidie più comuni come la delega degli eventi e la non separazione degli eventi.
		\item Ogni \textit{strategia di evento} offre due modi per associare un listener ad un evento.
		\item L'API dell'evento offre diverse strategie per gestire i listener di eventi.
		\item Vedi il
			\href{https://docs.typo3.org/c/typo3/cms-core/master/en-us/Changelog/10.3/Feature-90471-JavaScriptEventAPI.html}{change log}
			per esempi e maggiori dettagli.
	\end{itemize}

\end{frame}

% ------------------------------------------------------------------------------
