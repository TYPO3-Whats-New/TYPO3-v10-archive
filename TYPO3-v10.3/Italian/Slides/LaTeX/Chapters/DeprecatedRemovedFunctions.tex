% ------------------------------------------------------------------------------
% TYPO3 Version 10.3 - What's New (Italian Version)
%
% @license	Creative Commons BY-NC-SA 3.0
% @link		https://typo3.org/help/documentation/whats-new/
% @language	Italian
% ------------------------------------------------------------------------------

\section{Funzioni deprecate/rimosse}
\begin{frame}[fragile]
	\frametitle{Funzioni deprecate/rimosse}

	\begin{center}\huge{Capitolo 4:}\end{center}
	\begin{center}\huge{\color{typo3darkgrey}\textbf{Funzioni deprecate/rimosse}}\end{center}

\end{frame}

% ------------------------------------------------------------------------------
% Deprecation | 89463 | Deprecate switchable controller actions

\begin{frame}[fragile]
	\frametitle{Funzioni deprecate/rimosse}
	\framesubtitle{Switchable Controller Actions}

	\begin{itemize}
		\item "Switchable Controller Actions" (SCA) è stato contrassegnato come \textbf{deprecato}.
		\item SCA veniva utilizzato per sovrascrivere il set consentito di controller e azioni mediante TypoScript o Flexforms.
		\item L'uso dello stesso plug-in come punto di ingresso per molte funzionalità diverse contraddice l'idea di un plug-in con uno scopo specifico.
		\item I plugin che utilizzano SCA devono essere suddivisi in più plugin diversi.
	\end{itemize}

\end{frame}

% ------------------------------------------------------------------------------
% Deprecation | 90007 | Global constants TYPO3_version and TYPO3_branch

\begin{frame}[fragile]
	\frametitle{Funzioni deprecate/rimosse}
	\framesubtitle{Costanti globali}

	% decrease font size for code listing
	\lstset{basicstyle=\smaller\ttfamily}

	\begin{itemize}
		\item Le seguenti due costanti globali sono state contrassegnate come \textbf{deprecate}:

			\begin{itemize}
				\item \texttt{TYPO3\_version}
				\item \texttt{TYPO3\_branch}
			\end{itemize}

		\item Al suo posto va usata la seguente nuova classe PHP:\newline
			\small
				\texttt{TYPO3\textbackslash
					CMS\textbackslash
					Core\textbackslash
					Information\textbackslash
					Typo3Version}\normalsize

	\end{itemize}

\end{frame}

% ------------------------------------------------------------------------------
% Deprecation | 89673 | Deprecate Extbases WebRequest and WebResponse

\begin{frame}[fragile]
	\frametitle{Funzioni deprecate/rimosse}
	\framesubtitle{Extbase: \texttt{WebRequest}/\texttt{WebResponse}}

	\begin{itemize}
		\item Le seguenti due classi Extbase sono state contrassegnate come \textbf{deprecate}:
			\begin{itemize}
				\item \texttt{TYPO3\textbackslash
					CMS\textbackslash
					Extbase\textbackslash
					Mvc\textbackslash
					Web\textbackslash
					Request}
				\item \texttt{TYPO3\textbackslash
					CMS\textbackslash
					Extbase\textbackslash
					Mvc\textbackslash
					Web\textbackslash
					Response}
			\end{itemize}

	\end{itemize}

\end{frame}

% ------------------------------------------------------------------------------
% Deprecation | 90258 | Simplified RTE Parser API

\begin{frame}[fragile]
	\frametitle{Funzioni deprecate/rimosse}
	\framesubtitle{Simplified RTE Parser API}

	\begin{itemize}
		\item La classe PHP \texttt{RteHtmlParser} presenta ora un'API semplificata.
		\item Di conseguenza, i seguenti due metodi sono stati contrassegnati come \textbf{deprecati}:

			\begin{itemize}
				\item \texttt{TYPO3\textbackslash
					CMS\textbackslash
					Core\textbackslash
					Html\textbackslash
					RteHtmlParser->init()}
				\item \texttt{TYPO3\textbackslash
					CMS\textbackslash
					Core\textbackslash
					Html\textbackslash
					RteHtmlParser->RTE\_transform()}
			\end{itemize}

	\end{itemize}

\end{frame}

% ------------------------------------------------------------------------------
% Deprecation | 89139 | Console Commands configuration migrated to Symfony service tags

\begin{frame}[fragile]
	\frametitle{Funzioni deprecate/rimosse}
	\framesubtitle{Configurazione comandi della console}

	\begin{itemize}
		\item Poiché la configurazione dei comandi della console è stata migrata nei tag di servizio di Symfony,
			il file di configurazione dei comandi della console  \texttt{Configuration/Commands.php}
			è stato contrassegnato come \textbf{deprecato}.
		\item Utilizzare invece il tag del servizio di iniezione delle dipendenze \texttt{console.command}.

	\end{itemize}

\end{frame}

% ------------------------------------------------------------------------------
% Important | 89672 | transOrigPointerField is not longer allowed to be excluded

\begin{frame}[fragile]
	\frametitle{Funzioni deprecate/rimosse}
	\framesubtitle{TCA: \texttt{transOrigPointerField}}

	\begin{itemize}
		\item L'esclusione della seguente opzione TCA ha portato a dati incoerenti archiviati nel database in determinate circostanze:
			\small
				\texttt{\$GLOBALS['TCA'][\$table]['ctrl']['transOrigPointerField']}
			\normalsize

		\item Pertanto, questa opzione non può più essere esclusa.
		\item Una procedura guidata di migrazione rimuove l'opzione dal TCA e aggiunge un messaggio di deprecazione al log
		    delle deprecazioni nel caso in cui il codice debba essere aggiornato.
	\end{itemize}

\end{frame}

% ------------------------------------------------------------------------------
% Deprecation | 90421 | DocumentTemplate

\begin{frame}[fragile]
	\frametitle{Funzioni deprecate/rimosse}
	\framesubtitle{DocumentTemplate}

	% decrease font size for code listing
	\lstset{basicstyle=\tiny\ttfamily}

	\begin{itemize}
		\item Le seguenti classi sono state contrassegnate come \textbf{deprecate}:

			\begin{itemize}
				\item \texttt{TYPO3\textbackslash
					CMS\textbackslash
					Backend\textbackslash
					Template\textbackslash
					DocumentTemplate}
			\end{itemize}

		\item È stato utilizzato come base per il rendering di moduli backend o output basato su HTML nel backend TYPO3.
		\item A partire da TYPO3 v7, è invece necessario utilizzare la nuova API tramite ModuleTemplate.

\vspace{-0.4cm}
\begin{lstlisting}
use TYPO3\CMS\Backend\Template\ModuleTemplate;
...
$moduleTemplate = GeneralUtility::makeInstance(ModuleTemplate::class);
$content = $this->getHtmlContentFromMyModule();
$moduleTemplate->setTitle('My module');
$moduleTemplate->setContent($content);
return new HtmlResponse($moduleTemplate->renderContent());
\end{lstlisting}

	\end{itemize}

\end{frame}

% ------------------------------------------------------------------------------
% Deprecation | 85613 | Introduce simple way to register category fields
%
%\begin{frame}[fragile]
%	\frametitle{Deprecated/Removed Functions}
%	\framesubtitle{CategoryRegistry API}
%
%	\begin{itemize}
%		\item The following class has been marked as \textbf{deprecated}:
%
%			\begin{itemize}
%				\item \texttt{TYPO3\textbackslash
%					CMS\textbackslash
%					Core\textbackslash
%					Category\textbackslash
%					CategoryRegistry}
%			\end{itemize}
%
%		\item The following method has been marked as \textbf{deprecated}:
%
%			\begin{itemize}
%				\item \texttt{ExtgensionManagementUtility::makeCategorizable()}
%			\end{itemize}
%
%		\item Category fields can easily by registered by adding TCA configuration.
%
%	\end{itemize}
%
%\end{frame}
%
% ------------------------------------------------------------------------------
% Deprecation | 90390 | Deprecate BrokenLinkRepository::getNumberOfBrokenLinks() in linkvalidator

\begin{frame}[fragile]
	\frametitle{Funzioni deprecate/rimosse}
	\framesubtitle{LinkValidator}

	\begin{itemize}
		\item Il metodo seguente è stato segnato come \textbf{deprecato}:
		\newline\newline
			\smaller
				\texttt{TYPO3\textbackslash
					CMS\textbackslash
					Linkvalidator\textbackslash
					Repository\textbackslash
					BrokenLinkRepository}\newline
				\texttt{->getNumberOfBrokenLinks()}\normalsize\newline

		\item Utilizzare al suo posto il seguente metodo nella stessa classe:\newline
			\small\texttt{BrokenLinkRepository::isLinkTargetBrokenLink()}\normalsize

	\end{itemize}

\end{frame}


% ------------------------------------------------------------------------------
