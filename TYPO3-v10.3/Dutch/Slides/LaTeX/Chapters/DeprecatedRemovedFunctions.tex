% ------------------------------------------------------------------------------
% TYPO3 Version 10.3 - What's New (Dutch Version)
%
% @license	Creative Commons BY-NC-SA 3.0
% @link		https://typo3.org/help/documentation/whats-new/
% @language	Dutch
% ------------------------------------------------------------------------------

\section{Verouderde/verwijderde functies}
\begin{frame}[fragile]
	\frametitle{Verouderde/verwijderde functies}

	\begin{center}\huge{Hoofdstuk 4:}\end{center}
	\begin{center}\huge{\color{typo3darkgrey}\textbf{Verouderde/verwijderde functies}}\end{center}

\end{frame}

% ------------------------------------------------------------------------------
% Deprecation | 89463 | Deprecate switchable controller actions

\begin{frame}[fragile]
	\frametitle{Verouderde/verwijderde functies}
	\framesubtitle{Switchable Controller Actions}

	\begin{itemize}
		\item "Switchable Controller Actions" (SCA) zijn als \textbf{verouderd} aangemerkt.
		\item SCA kunnen de toegestane set van controllers en actions overschrijven via TypoScript of Flexforms.
		\item Dezelfde plug-in gebruiken voor veel verschillende functionaliteiten gaat in tegen het idee dat een plug-in een bepaald doel heeft.
		\item Plug-ins die SCA gebruiken moeten opgedeeld worden in meerdere plug-ins.
	\end{itemize}

\end{frame}

% ------------------------------------------------------------------------------
% Deprecation | 90007 | Global constants TYPO3_version and TYPO3_branch

\begin{frame}[fragile]
	\frametitle{Verouderde/verwijderde functies}
	\framesubtitle{Globale constanten}

	% decrease font size for code listing
	\lstset{basicstyle=\smaller\ttfamily}

	\begin{itemize}
		\item De volgende twee globale constanten zijn als \textbf{verouderd} aangemerkt:

			\begin{itemize}
				\item \texttt{TYPO3\_version}
				\item \texttt{TYPO3\_branch}
			\end{itemize}

		\item De volgende nieuwe PHP klasse kan hiervoor gebruikt worden:\newline
			\small
				\texttt{TYPO3\textbackslash
					CMS\textbackslash
					Core\textbackslash
					Information\textbackslash
					Typo3Version}\normalsize

	\end{itemize}

\end{frame}

% ------------------------------------------------------------------------------
% Deprecation | 89673 | Deprecate Extbases WebRequest and WebResponse

\begin{frame}[fragile]
	\frametitle{Verouderde/verwijderde functies}
	\framesubtitle{Extbase: \texttt{WebRequest}/\texttt{WebResponse}}

	\begin{itemize}
		\item De volgende twee Extbase klassen zijn als \textbf{verouderd} aangemerkt:
			\begin{itemize}
				\item \texttt{TYPO3\textbackslash
					CMS\textbackslash
					Extbase\textbackslash
					Mvc\textbackslash
					Web\textbackslash
					Request}
				\item \texttt{TYPO3\textbackslash
					CMS\textbackslash
					Extbase\textbackslash
					Mvc\textbackslash
					Web\textbackslash
					Response}
			\end{itemize}

	\end{itemize}

\end{frame}

% ------------------------------------------------------------------------------
% Deprecation | 90258 | Simplified RTE Parser API

\begin{frame}[fragile]
	\frametitle{Verouderde/verwijderde functies}
	\framesubtitle{Versimpelde RTE Parser API}

	\begin{itemize}
		\item De PHP klasse \texttt{RteHtmlParser} heeft een versimpelde API.
		\item Een gevolg is dat deze twee methodes als \textbf{verouderd} zijn aangemerkt:

			\begin{itemize}
				\item \texttt{TYPO3\textbackslash
					CMS\textbackslash
					Core\textbackslash
					Html\textbackslash
					RteHtmlParser->init()}
				\item \texttt{TYPO3\textbackslash
					CMS\textbackslash
					Core\textbackslash
					Html\textbackslash
					RteHtmlParser->RTE\_transform()}
			\end{itemize}

	\end{itemize}

\end{frame}

% ------------------------------------------------------------------------------
% Deprecation | 89139 | Console Commands configuration migrated to Symfony service tags

\begin{frame}[fragile]
	\frametitle{Verouderde/verwijderde functies}
	\framesubtitle{Configuratie van console commando's}

	\begin{itemize}
		\item Aangezien de configuratie van console commando's omgezet is naar Symfony service tags,
			is het bestand met de configuratie \texttt{Configuration/Commands.php} als
			\textbf{verouderd} aangemerkt.
		\item Gebruik de service tag voor het injecteren van afhankelijkheden \texttt{console.command} hiervoor.

	\end{itemize}

\end{frame}

% ------------------------------------------------------------------------------
% Important | 89672 | transOrigPointerField is not longer allowed to be excluded

\begin{frame}[fragile]
	\frametitle{Verouderde/verwijderde functies}
	\framesubtitle{TCA: \texttt{transOrigPointerField}}

	\begin{itemize}
		\item Het uitzonderen van deze TCA-optie zorgde voor inconsistente data in de databse onder bepaalde omstandigheden:
			\small
				\texttt{\$GLOBALS['TCA'][\$table]['ctrl']['transOrigPointerField']}
			\normalsize

		\item Daarom kan de optie niet meer uitgezonderd worden.
		\item Een migratie-assistent verwijdert de optie uit de TCA en voegt een verouderingsmelding
			toe aan de verouderingslog als code bijgewerkt moet worden.
	\end{itemize}

\end{frame}

% ------------------------------------------------------------------------------
% Deprecation | 90421 | DocumentTemplate

\begin{frame}[fragile]
	\frametitle{Verouderde/verwijderde functies}
	\framesubtitle{DocumentTemplate}

	% decrease font size for code listing
	\lstset{basicstyle=\tiny\ttfamily}

	\begin{itemize}
		\item De volgende klasse is als \textbf{verouderd} aangemerkt:

			\begin{itemize}
				\item \texttt{TYPO3\textbackslash
					CMS\textbackslash
					Backend\textbackslash
					Template\textbackslash
					DocumentTemplate}
			\end{itemize}

		\item Het werd gebruikt als basis voor de weergave van backend modules of HTML-uitvoer in de TYPO3 backend.
		\item Sinds TYPO3 v7 moet de nieuwe API via ModuleTemplate gebruikt worden hiervoor.

\vspace{-0.4cm}
\begin{lstlisting}
use TYPO3\CMS\Backend\Template\ModuleTemplate;
...
$moduleTemplate = GeneralUtility::makeInstance(ModuleTemplate::class);
$content = $this->getHtmlContentFromMyModule();
$moduleTemplate->setTitle('My module');
$moduleTemplate->setContent($content);
return new HtmlResponse($moduleTemplate->renderContent());
\end{lstlisting}

	\end{itemize}

\end{frame}

% ------------------------------------------------------------------------------
% Deprecation | 85613 | Introduce simple way to register category fields
%
%\begin{frame}[fragile]
%	\frametitle{Verouderde/verwijderde functies}
%	\framesubtitle{CategoryRegistry API}
%
%	\begin{itemize}
%		\item The following class has been marked as \textbf{deprecated}:
%
%			\begin{itemize}
%				\item \texttt{TYPO3\textbackslash
%					CMS\textbackslash
%					Core\textbackslash
%					Category\textbackslash
%					CategoryRegistry}
%			\end{itemize}
%
%		\item The following method has been marked as \textbf{deprecated}:
%
%			\begin{itemize}
%				\item \texttt{ExtgensionManagementUtility::makeCategorizable()}
%			\end{itemize}
%
%		\item Category fields can easily by registered by adding TCA configuration.
%
%	\end{itemize}
%
%\end{frame}
%
% ------------------------------------------------------------------------------
% Deprecation | 90390 | Deprecate BrokenLinkRepository::getNumberOfBrokenLinks() in linkvalidator

\begin{frame}[fragile]
	\frametitle{Verouderde/verwijderde functies}
	\framesubtitle{LinkValidator}

	\begin{itemize}
		\item De volgende mehtode is als \textbf{verouderd} aangemerkt:
		\newline\newline
			\smaller
				\texttt{TYPO3\textbackslash
					CMS\textbackslash
					Linkvalidator\textbackslash
					Repository\textbackslash
					BrokenLinkRepository}\newline
				\texttt{->getNumberOfBrokenLinks()}\normalsize\newline

		\item Gebruik de volgende methode in dezelfde klasse hiervoor:\newline
			\small\texttt{BrokenLinkRepository::isLinkTargetBrokenLink()}\normalsize

	\end{itemize}

\end{frame}


% ------------------------------------------------------------------------------
