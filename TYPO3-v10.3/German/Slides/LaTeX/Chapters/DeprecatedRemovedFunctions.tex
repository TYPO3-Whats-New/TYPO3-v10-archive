% ------------------------------------------------------------------------------
% TYPO3 Version 10.3 - What's New (German Version)
%
% @license	Creative Commons BY-NC-SA 3.0
% @link		https://typo3.org/help/documentation/whats-new/
% @language	German
% ------------------------------------------------------------------------------

\section{Veraltete/entfernte Funktionen}
\begin{frame}[fragile]
	\frametitle{Veraltete/entfernte Funktionen}

	\begin{center}\huge{Kapitel 4:}\end{center}
	\begin{center}\huge{\color{typo3darkgrey}\textbf{Veraltete/entfernte Funktionen}}\end{center}

\end{frame}

% ------------------------------------------------------------------------------
% Deprecation | 89463 | Deprecate switchable controller actions

\begin{frame}[fragile]
	\frametitle{Veraltete/entfernte Funktionen}
	\framesubtitle{Umschaltbare Controller-Aktionen}

	\begin{itemize}
		\item "Switchable Controller Actions" (SCA) wurden als \textbf{veraltet} markiert.
		\item SCA werden verwendet um den zulässigen Satz von Controllern und Aktionen mit TypoScript oder Flexforms zu überschreiben.
		\item Die Verwendung desselben Plugins als Einstiegspunkt für verschiedene Funktionalitäten widerspricht der Idee eines Plugins, das einem bestimmten Zweck dient.
		\item Plugins, die SCA verwenden, sollten in mehrere verschiedene Plugins aufgeteilt werden.
	\end{itemize}

\end{frame}

% ------------------------------------------------------------------------------
% Deprecation | 90007 | Global constants TYPO3_version and TYPO3_branch

\begin{frame}[fragile]
	\frametitle{Veraltete/entfernte Funktionen}
	\framesubtitle{Globale Konstanten}

	% decrease font size for code listing
	\lstset{basicstyle=\smaller\ttfamily}

	\begin{itemize}
		\item Die folgenden beiden globalen Konstanten wurden als \textbf{veraltet} markiert:

			\begin{itemize}
				\item \texttt{TYPO3\_version}
				\item \texttt{TYPO3\_branch}
			\end{itemize}

		\item Folgende neue PHP-Klasse sollte stattdessen verwendet werden:\newline
			\small
				\texttt{TYPO3\textbackslash
					CMS\textbackslash
					Core\textbackslash
					Information\textbackslash
					Typo3Version}\normalsize

	\end{itemize}

\end{frame}

% ------------------------------------------------------------------------------
% Deprecation | 89673 | Deprecate Extbases WebRequest and WebResponse

\begin{frame}[fragile]
	\frametitle{Veraltete/entfernte Funktionen}
	\framesubtitle{Extbase: \texttt{WebRequest}/\texttt{WebResponse}}

	\begin{itemize}
		\item Die folgenden beiden Extbase-Klassen wurden als \textbf{veraltet} markiert:
			\begin{itemize}
				\item \texttt{TYPO3\textbackslash
					CMS\textbackslash
					Extbase\textbackslash
					Mvc\textbackslash
					Web\textbackslash
					Request}
				\item \texttt{TYPO3\textbackslash
					CMS\textbackslash
					Extbase\textbackslash
					Mvc\textbackslash
					Web\textbackslash
					Response}
			\end{itemize}

	\end{itemize}

\end{frame}

% ------------------------------------------------------------------------------
% Deprecation | 90258 | Simplified RTE Parser API

\begin{frame}[fragile]
	\frametitle{Veraltete/entfernte Funktionen}
	\framesubtitle{Vereinfachte RTE-Parser-API}

	\begin{itemize}
		\item Die PHP-Klasse \texttt{RteHtmlParser} verfügt jetzt über eine vereinfachte API.
		\item Als Folge davon wurden die folgenden beiden Methoden als \textbf{veraltet} markiert:

			\begin{itemize}
				\item \texttt{TYPO3\textbackslash
					CMS\textbackslash
					Core\textbackslash
					Html\textbackslash
					RteHtmlParser->init()}
				\item \texttt{TYPO3\textbackslash
					CMS\textbackslash
					Core\textbackslash
					Html\textbackslash
					RteHtmlParser->RTE\_transform()}
			\end{itemize}

	\end{itemize}

\end{frame}

% ------------------------------------------------------------------------------
% Deprecation | 89139 | Console Commands configuration migrated to Symfony service tags

\begin{frame}[fragile]
	\frametitle{Veraltete/entfernte Funktionen}
	\framesubtitle{Konfiguration der Konsolenbefehle}

	\begin{itemize}
		\item Da die Konfigurationsdatei für Konsolenbefehle in Symfony-Service-Tags migriert wurde,
			wurde die Datei \texttt{Configuration/Commands.php}
			als \textbf{veraltet} markiert.
		\item Verwenden Sie stattdessen den Dependency Injection Service Tag \texttt{console.command}.

	\end{itemize}

\end{frame}

% ------------------------------------------------------------------------------
% Important | 89672 | transOrigPointerField is not longer allowed to be excluded

\begin{frame}[fragile]
	\frametitle{Veraltete/entfernte Funktionen}
	\framesubtitle{TCA: \texttt{transOrigPointerField}}

	\begin{itemize}
		\item Das Ausschließen des Feldes, auf das die folgende TCA-Option hinweist, führte unter bestimmten Umständen zu inkonsistenten Daten, die in der Datenbank gespeichert sind:
			\small
				\texttt{\$GLOBALS['TCA'][\$table]['ctrl']['transOrigPointerField']}
			\normalsize

		\item Daher kann der Zielbereich nicht mehr ausgeschlossen werden.
		\item Ein Migrationsassistent entfernt die Option aus dem TCA und fügt dem Verwerfungsprotokoll
			eine Verwerfungsmeldung hinzu, falls der Code aktualisiert werden muss.
	\end{itemize}

\end{frame}

% ------------------------------------------------------------------------------
% Deprecation | 90421 | DocumentTemplate

\begin{frame}[fragile]
	\frametitle{Veraltete/entfernte Funktionen}
	\framesubtitle{DocumentTemplate}

	% decrease font size for code listing
	\lstset{basicstyle=\tiny\ttfamily}

	\begin{itemize}
		\item Die folgende Klasse wurde als \textbf{veraltet} markiert:

			\begin{itemize}
				\item \texttt{TYPO3\textbackslash
					CMS\textbackslash
					Backend\textbackslash
					Template\textbackslash
					DocumentTemplate}
			\end{itemize}

		\item Dies wurde als Grundlage für die Darstellung von Backend-Modulen oder HTML-basierten Ausgaben im TYPO3-Backend verwendet.
		\item Seit TYPO3 v7 sollte stattdessen die neue API über ModuleTemplate verwendet werden.

\vspace{-0.4cm}
\begin{lstlisting}
use TYPO3\CMS\Backend\Template\ModuleTemplate;
...
$moduleTemplate = GeneralUtility::makeInstance(ModuleTemplate::class);
$content = $this->getHtmlContentFromMyModule();
$moduleTemplate->setTitle('My module');
$moduleTemplate->setContent($content);
return new HtmlResponse($moduleTemplate->renderContent());
\end{lstlisting}

	\end{itemize}

\end{frame}

% ------------------------------------------------------------------------------
% Deprecation | 85613 | Introduce simple way to register category fields
%
%\begin{frame}[fragile]
%	\frametitle{Deprecated/Removed Functions}
%	\framesubtitle{CategoryRegistry API}
%
%	\begin{itemize}
%		\item The following class has been marked as \textbf{deprecated}:
%
%			\begin{itemize}
%				\item \texttt{TYPO3\textbackslash
%					CMS\textbackslash
%					Core\textbackslash
%					Category\textbackslash
%					CategoryRegistry}
%			\end{itemize}
%
%		\item The following method has been marked as \textbf{deprecated}:
%
%			\begin{itemize}
%				\item \texttt{ExtgensionManagementUtility::makeCategorizable()}
%			\end{itemize}
%
%		\item Category fields can easily by registered by adding TCA configuration.
%
%	\end{itemize}
%
%\end{frame}
%
% ------------------------------------------------------------------------------
% Deprecation | 90390 | Deprecate BrokenLinkRepository::getNumberOfBrokenLinks() in linkvalidator

\begin{frame}[fragile]
	\frametitle{Veraltete/entfernte Funktionen}
	\framesubtitle{LinkValidator}

	\begin{itemize}
		\item Folgende Methode wurde als \textbf{veraltet} markiert:
		\newline\newline
			\smaller
				\texttt{TYPO3\textbackslash
					CMS\textbackslash
					Linkvalidator\textbackslash
					Repository\textbackslash
					BrokenLinkRepository}\newline
				\texttt{->getNumberOfBrokenLinks()}\normalsize\newline

		\item Verwenden Sie stattdessen folgende Methode in der gleichen Klasse:\newline
			\small\texttt{BrokenLinkRepository::isLinkTargetBrokenLink()}\normalsize

	\end{itemize}

\end{frame}


% ------------------------------------------------------------------------------
