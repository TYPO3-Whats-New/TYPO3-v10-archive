% ------------------------------------------------------------------------------
% TYPO3 Version 10.3 - What's New (Serbian Version)
%
% @license	Creative Commons BY-NC-SA 3.0
% @link		https://typo3.org/help/documentation/whats-new/
% @language	Serbian
% ------------------------------------------------------------------------------

\section{Izmene za programere}
\begin{frame}[fragile]
	\frametitle{Izmene za programere}

	\begin{center}\huge{Poglavlje 3:}\end{center}
	\begin{center}\huge{\color{typo3darkgrey}\textbf{Izmene za programere}}\end{center}

\end{frame}

% ------------------------------------------------------------------------------
% Feature | 90333 | Dashboard

\begin{frame}[fragile]
	\frametitle{Izmene za programere}
	\framesubtitle{Dashboard (1)}

	% decrease font size for code listing
	\lstset{basicstyle=\smaller\ttfamily}

	\begin{itemize}
		\item Programeri mogu da naprave prilagodjene vidžete za dashboard proširenjem jednog od sledećih \textit{abstraktnih} vidžeta:

			\begin{itemize}
				\item \texttt{AbstractWidget}\newline
					\small
						Osnovna abstraktna verzija koja se može koristit za početak prostijih vidžeta.
					\normalsize
				\item \texttt{AbstractRssWidget}\newline
					\small
						Abstraktna verzija za kreiranje vidžeta za prikazivanje RSS feed-a.
					\normalsize
				\item \texttt{AbstractListWidget}\newline
					\small
						Abstraktna verzija za kreiranje vidžeta za prikazivanje liste stavki.
					\normalsize
				\item \texttt{AbstractCtaButtonWidget}\newline
					\small
						Abstraktna verzija za kreiranje vidžeta za prikazivanje "call-to-action" dugmeta.
					\normalsize
			\end{itemize}

	\end{itemize}

\end{frame}

% ------------------------------------------------------------------------------
% Feature | 90333 | Dashboard

\begin{frame}[fragile]
	\frametitle{Izmene za programere}
	\framesubtitle{Dashboard (2)}

	% decrease font size for code listing
	\lstset{basicstyle=\tiny\ttfamily}

	\begin{itemize}
		\item Registrujte Vaše vidžete u sledećem fajlu Vašeg proširenja:\newline
			\texttt{EXT:my\_extension/Configuration/Services.yaml}

		\item Opcija 1: identifikator vidžeta kao atribut

\vspace{-0.4cm}
\begin{lstlisting}
Vendor\MyExtension\Widgets\MyFirstWidget:
  tags:
    - name: dashboard.widget
      identifier: widget-identifier-1
      widgetGroups: 'general'
\end{lstlisting}

		\item Opcija 2: prilagodjeno ime servisa omogućuje da više identifikatora vidžeta koriste istu klasu

\vspace{-0.4cm}
\begin{lstlisting}
widget.identifier:
  class: Vendor\MyExtension\Widgets\MySecondWidget
  tags:
    - name: dashboard.widget
      identifier: widget-identifier-2
      widgetGroups: 'general, typo3'
\end{lstlisting}

	\end{itemize}

\end{frame}

% ------------------------------------------------------------------------------
% Feature | 90333 | Dashboard

\begin{frame}[fragile]
	\frametitle{Izmene za programere}
	\framesubtitle{Dashboard (3)}

	% decrease font size for code listing
	\lstset{basicstyle=\tiny\ttfamily}

	\begin{itemize}
		\item Svaki vidžet je vezan za jednu ili više vidžet grupa.
		\item Ove grupe se vide u modalu prilikom dodavanja novog vidžeta na dashboard.
		\item Programeri mogu da konfigurišu prilagodjene grupe vidžeta kreiranjem fajla\newline
			\smaller
				\texttt{EXT:my\_extension/Configuration/Backend/DashboardWidgetGroups.php}
			\normalsize

\vspace{-0.4cm}
\begin{lstlisting}
return [
  'widgetGroup-exampleGroup' => [
    'title' => 'LLL:EXT:my_extension/Resources/Private/Language/locallang.xlf:widget_group_name',
  ],
];
\end{lstlisting}

	\end{itemize}

\end{frame}

% ------------------------------------------------------------------------------
% Feature | 89870 | New PSR-14 Events for Extbase-related signals

\begin{frame}[fragile]
	\frametitle{Izmene za programere}
	\framesubtitle{Extbase i Fluid}

	% decrease font size for code listing
	\lstset{basicstyle=\tiny\ttfamily}

	\begin{itemize}
		\item Sledeći PSR-14 dogadjaji su dodati za signale vezane za Extbase:

\vspace{-0.4cm}
\begin{lstlisting}
TYPO3\CMS\Extbase\Event\Mvc\AfterRequestDispatchedEvent
TYPO3\CMS\Extbase\Event\Mvc\BeforeActionCallEvent
TYPO3\CMS\Extbase\Event\Persistence\AfterObjectThawedEvent
TYPO3\CMS\Extbase\Event\Persistence\ModifyQueryBeforeFetchingObjectDataEvent
TYPO3\CMS\Extbase\Event\Persistence\ModifyResultAfterFetchingObjectDataEvent
TYPO3\CMS\Extbase\Event\Persistence\EntityAddedToPersistenceEvent
TYPO3\CMS\Extbase\Event\Persistence\EntityFinalizedAfterPersistenceEvent
TYPO3\CMS\Extbase\Event\Persistence\EntityUpdatedInPersistenceEvent
TYPO3\CMS\Extbase\Event\Persistence\EntityRemovedFromPersistenceEvent
TYPO3\CMS\Extbase\Event\Persistence\EntityPersistedEvent
\end{lstlisting}

		\item Postojeći signali su zamenjeni i ne trebju se više koristiti.

	\end{itemize}

\end{frame}

% ------------------------------------------------------------------------------
% Feature | 89644 | Add optional argument fields to editRecord ViewHelpers

\begin{frame}[fragile]
	\frametitle{Izmene za programere}
	\framesubtitle{ViewHelper \texttt{editRecord}}

	% decrease font size for code listing
	\lstset{basicstyle=\tiny\ttfamily}

	\begin{itemize}
		\item Opcioni argument \texttt{fields} je dodat u
			\texttt{uri.editRecord} i \texttt{link.editRecord} ViewHelper-e.
		\item Ukoliko je postavljen, FormEngine kreira formu koja može da menja samo dato polje/polja u bazi podataka.
		\item Sledeći primer kreira link za menjanje \texttt{tt\_content.bodytext} polja od rekorda sa UID 42.

\begin{lstlisting}
<be:link.editRecord uid="42" table="tt_content" fields="bodytext" returnUrl="foo/bar">
  Edit record
</be:link.editRecord>
\end{lstlisting}

	\end{itemize}

\end{frame}

% ------------------------------------------------------------------------------
% Feature | xxxxx | Introduce AssetCollector

\begin{frame}[fragile]
	\frametitle{Izmene za programere}
	\framesubtitle{AssetCollector}

	\begin{itemize}
		\item Početni koraci integracije AssetCollector-a su implementirani.
		\item Ovaj koncept omogućuje programerima da dodaju prilagodjen CSS/JS kod (inline ili external)
			više puta, ali ga TYPO3 ispisuje samo jednom.
		\item U ove svrhe, dodata su dva nova Fluid ViewHelper-a:
			\begin{itemize}
				\item \texttt{<f:css>}
				\item \texttt{<f:script>}
			\end{itemize}
		\item Dugoročno, AssetCollector  treba ra zameni razne TypoScript opcije koje su sada konfuzne.
	\end{itemize}

\end{frame}

% ------------------------------------------------------------------------------
% Feature | 86614 | Add PSR-14 event to control hreflang tags to be rendered

\begin{frame}[fragile]
	\frametitle{Izmene za programere}
	\framesubtitle{Izmena \texttt{hreflang} taga}

	% decrease font size for code listing
	\lstset{basicstyle=\smaller\ttfamily}

	\begin{itemize}
		\item Od sada je moguće izmeniti \texttt{hreflang} tagove pre njihovog renderanja.
		\item Programeri ovo mogu da postignu registrovanjem slušaoca dogadjaja za sledeći dogadjaj:\newline
			\smaller
				\texttt{TYPO3\textbackslash
					CMS\textbackslash
					Frontend\textbackslash
					Event\textbackslash
					ModifyHrefLangTagsEvent}
			\normalsize
	\end{itemize}

\end{frame}

% ------------------------------------------------------------------------------
% Feature | 88818 | Introduce events to modify CKEditor configuration

\begin{frame}[fragile]
	\frametitle{Izmene za programere}
	\framesubtitle{Menjanje konfiguracije CKEditor-a}

	% decrease font size for code listing
	\lstset{basicstyle=\tiny\ttfamily}

	\begin{itemize}
		\item Sledeći dogadjaji zasnovani na PSR-14 su dodati da omoguće izmenu konfiguracije CKEditor-a:

\vspace{-0.4cm}
\begin{lstlisting}
TYPO3\CMS\RteCKEditor\Form\Element\Event\AfterGetExternalPluginsEvent
TYPO3\CMS\RteCKEditor\Form\Element\Event\BeforeGetExternalPluginsEvent
TYPO3\CMS\RteCKEditor\Form\Element\Event\AfterPrepareConfigurationForEditorEvent
TYPO3\CMS\RteCKEditor\Form\Element\Event\BeforePrepareConfigurationForEditorEvent
\end{lstlisting}

		\item Pogledajte
			\href{https://docs.typo3.org/c/typo3/cms-core/master/en-us/Changelog/10.3/Feature-88818-IntroduceEventsToModifyCKEditorConfiguration.html}{change log}
			za primer.
	\end{itemize}

\end{frame}

% ------------------------------------------------------------------------------
% Feature | 90265 | Show dispatched Events in Admin Panel

\begin{frame}[fragile]
	\frametitle{Izmene za programere}
	\framesubtitle{PSR-14 dogadjaji u Admin PANELU}

	\begin{itemize}
		\item Admin Panel prikazuje sve PSR-14 dogjadjaje koji su trigerovani u trenutnom zahtevu.
	\end{itemize}

	\begin{figure}
		\includegraphics[width=0.85\linewidth]{ChangesForDevelopers/90265-ShowDispatchedEventsInAdminPanel.png}
	\end{figure}

\end{frame}

% ------------------------------------------------------------------------------
% Feature | 89738 | API for AJAX Requests

\begin{frame}[fragile]
	\frametitle{Izmene za programere}
	\framesubtitle{API za AJAX zahteve}

	% decrease font size for code listing
	\lstset{basicstyle=\tiny\ttfamily}

	\begin{itemize}
		\item \textbf{Fetch API} je implementiran za AJAX zahteve, sa svrhom da smanji zavisnost TYPO3 od jQuery.
		\item API nudi generičke definicije objekata zahteva i odgovora
			(kao i ostale stvari koje su uključene u mrežne zahteve).
		\item Podržano je od strane svih modernih pretraživača, za više detalja pogledajte
			\href{https://developer.mozilla.org/en-US/docs/Web/API/Fetch_API}{grafik kompatibilnosti}.
		\item TYPO3 jezgro već sada koristi novi API u Install Tool, FormEngine i kontekstualnim menijima.
		\item Pogledajte
			\href{https://docs.typo3.org/c/typo3/cms-core/master/en-us/Changelog/10.3/Feature-89738-ApiForAjaxRequests.html}{change log}
			za primere kako da koristite Fetch API.

	\end{itemize}

\end{frame}

% ------------------------------------------------------------------------------
% Feature | 89650 | Allow line breaks in TCA descriptions

\begin{frame}[fragile]
	\frametitle{Izmene za programere}
	\framesubtitle{TCA polja opisa}

	\begin{itemize}
		\item Polje opisa u TCA od sada može da ima prelomljene linije teksta,
			ovo za rezultat ima bolju čitljivost dugačkih tekstova.
	\end{itemize}

\end{frame}

% ------------------------------------------------------------------------------
% Important | 90020 | Legacy BasicFileUtility and ExtendedFileUtility classes marked as internal

\begin{frame}[fragile]
	\frametitle{Izmene za programere}
	\framesubtitle{Klase \texttt{BasicFileUtility} i \texttt{ExtendedFileUtility}}

	\begin{itemize}
		\item Sledeće dve klase su označene kao \textbf{unutrašnje}
			i više se ne trebaju koristiti:

			\begin{itemize}\small
				\item \texttt{TYPO3\textbackslash
					CMS\textbackslash
					Core\textbackslash
					Utility\textbackslash
					File\textbackslash
					BasicFileUtility}
				\item \texttt{TYPO3\textbackslash
					CMS\textbackslash
					Core\textbackslash
					Utility\textbackslash
					File\textbackslash
					ExtendedFileUtility}
			\end{itemize}

		\item Programeri umesto njih treba da koriste \texttt{ResourceStorage}
			i \texttt{ResourceFactory}.

	\end{itemize}

\end{frame}

% ------------------------------------------------------------------------------
% Feature | 89139 | Add dependency injection support for console commands

\begin{frame}[fragile]
	\frametitle{Izmene za programere}
	\framesubtitle{Komande u konzoli: podrška za Symfony DI}

	\begin{itemize}
		\item Zavisnosti u komandama se sada mogu dodavati preko konstruktora ili nekom drugom tehnikom dodavanja.
		\item Dodajte tag \texttt{console.command} komand klasama.
		\item Koristite atribut taga \texttt{command} da postavite ime komande.
		\item Opcioni atribut taga \texttt{schedulable} se može postaviti na \texttt{false}
			da bi se komanda ignorisala u TYPO3 scheduler-u.

		\item Pogledajte
			\href{https://docs.typo3.org/c/typo3/cms-core/master/en-us/Changelog/10.3/Feature-89139-AddDependencyInjectionSupportForConsoleCommands.html}{change log}
			za primer.
	\end{itemize}

\end{frame}

% ------------------------------------------------------------------------------
% Feature | 90168 | Introduce Modal Actions

\begin{frame}[fragile]
	\frametitle{Izmene za programere}
	\framesubtitle{Akciona dugmad u modalima}

	% decrease font size for code listing
	\lstset{basicstyle=\tiny\ttfamily}

	\begin{itemize}
		\item Modali sada podržavaju akcionu dugmad.
		\item Kao alternativa na postojeći \texttt{trigger}, može se koristiti \texttt{action}.
		\item Na primer:

\vspace{-0.4cm}
\begin{lstlisting}
Modal.confirm('Header', 'Some content', Severity.error, [
  {
    text: 'Based on trigger()',
    trigger: function () {
      console.log('Vintage!');
    }
  },
  {
    text: 'Based on action()',
    action: new DeferredAction(() => {
      return new AjaxRequest('/any/endpoint').post({});
    })
  }
]);
\end{lstlisting}

	\end{itemize}

\end{frame}

% ------------------------------------------------------------------------------
% Feature | 90471 | JavaScript Event API

\begin{frame}[fragile]
	\frametitle{Izmene za programere}
	\framesubtitle{JavaScript API za dogadjaje}

	\begin{itemize}
		\item Novi JavaScript API za dogadjaje omogućuje programerima da imaju stabilan interfejs za slušanje dogadjaja.
		\item API se brine za česte probleme kao sto je delegiranje dogadjaja i čisto odvezivanje dogadjaja.
		\item Svaka \textit{strategija dogadjaja} nudi dva načina da se slušalac veže za dogadjaj.
		\item API za dogadjaje nudi više strategija da se upravlja slušaocima dogadjaja.
		\item Pogledajte
			\href{https://docs.typo3.org/c/typo3/cms-core/master/en-us/Changelog/10.3/Feature-90471-JavaScriptEventAPI.html}{change log}
			za primere i više detalja.
	\end{itemize}

\end{frame}

% ------------------------------------------------------------------------------
