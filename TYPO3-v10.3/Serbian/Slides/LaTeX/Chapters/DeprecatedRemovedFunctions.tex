% ------------------------------------------------------------------------------
% TYPO3 Version 10.3 - What's New (Serbian Version)
%
% @license	Creative Commons BY-NC-SA 3.0
% @link		https://typo3.org/help/documentation/whats-new/
% @language	Serbian
% ------------------------------------------------------------------------------

\section{Zastarele/izbačene funkcije}
\begin{frame}[fragile]
	\frametitle{Zastarele/izbačene funkcije}

	\begin{center}\huge{Poglavlje 4:}\end{center}
	\begin{center}\huge{\color{typo3darkgrey}\textbf{Zastarele/izbačene funkcije}}\end{center}

\end{frame}

% ------------------------------------------------------------------------------
% Deprecation | 89463 | Deprecate switchable controller actions

\begin{frame}[fragile]
	\frametitle{Zastarele/izbačene funkcije}
	\framesubtitle{Switchable Controller Actions}

	\begin{itemize}
		\item "Switchable Controller Actions" (SCA) su označene kao \textbf{zastarele}.
		\item SCA su korišćene da se prepiše dozvoljen set kontrolera i akcija korišćenjem TypoScript-a ili Flexform-i.
		\item Korišćenje jednog plagina kao ulazne tačke za više funkcionalnosti je u kontradikciji sa idejom da svaki plagin služi samo za jednu svrhu.
		\item Plaginove koji koriste SCA treba podeliti u više plaginova.
	\end{itemize}

\end{frame}

% ------------------------------------------------------------------------------
% Deprecation | 90007 | Global constants TYPO3_version and TYPO3_branch

\begin{frame}[fragile]
	\frametitle{Zastarele/izbačene funkcije}
	\framesubtitle{Globalne konstante}

	% decrease font size for code listing
	\lstset{basicstyle=\smaller\ttfamily}

	\begin{itemize}
		\item Sledeće dve globalne konstante su označene kao \textbf{zastarele}:

			\begin{itemize}
				\item \texttt{TYPO3\_version}
				\item \texttt{TYPO3\_branch}
			\end{itemize}

		\item Umesto njih treba koristiti sledeću PHP klasu:\newline
			\small
				\texttt{TYPO3\textbackslash
					CMS\textbackslash
					Core\textbackslash
					Information\textbackslash
					Typo3Version}\normalsize

	\end{itemize}

\end{frame}

% ------------------------------------------------------------------------------
% Deprecation | 89673 | Deprecate Extbases WebRequest and WebResponse

\begin{frame}[fragile]
	\frametitle{Zastarele/izbačene funkcije}
	\framesubtitle{Extbase: \texttt{WebRequest}/\texttt{WebResponse}}

	\begin{itemize}
		\item Sledeće dve Extbase klase su označene kao \textbf{zastarele}:
			\begin{itemize}
				\item \texttt{TYPO3\textbackslash
					CMS\textbackslash
					Extbase\textbackslash
					Mvc\textbackslash
					Web\textbackslash
					Request}
				\item \texttt{TYPO3\textbackslash
					CMS\textbackslash
					Extbase\textbackslash
					Mvc\textbackslash
					Web\textbackslash
					Response}
			\end{itemize}

	\end{itemize}

\end{frame}

% ------------------------------------------------------------------------------
% Deprecation | 90258 | Simplified RTE Parser API

\begin{frame}[fragile]
	\frametitle{Zastarele/izbačene funkcije}
	\framesubtitle{Uprošćen API za parsiranje RTE}

	\begin{itemize}
		\item PHP klasa \texttt{RteHtmlParser} sada poseduje jednostavniji API.
		\item Kao posledica ovoga, sledeća dva metoda su označena kao \textbf{zastarela}:

			\begin{itemize}
				\item \texttt{TYPO3\textbackslash
					CMS\textbackslash
					Core\textbackslash
					Html\textbackslash
					RteHtmlParser->init()}
				\item \texttt{TYPO3\textbackslash
					CMS\textbackslash
					Core\textbackslash
					Html\textbackslash
					RteHtmlParser->RTE\_transform()}
			\end{itemize}

	\end{itemize}

\end{frame}

% ------------------------------------------------------------------------------
% Deprecation | 89139 | Console Commands configuration migrated to Symfony service tags

\begin{frame}[fragile]
	\frametitle{Zastarele/izbačene funkcije}
	\framesubtitle{Konfiguracija konzolnih komandi}

	\begin{itemize}
		\item Pošto je konfiguracija konzolnih komandi implementirana kroz servisne tagove iz Symfony-ja,
			fajl \texttt{Configuration/Commands.php} za konfiguraciju konzolnih komandi je označen kao
			\textbf{zastareo}.
		\item Kao zamenu koristiti DI servisni tag \texttt{console.command}.

	\end{itemize}

\end{frame}

% ------------------------------------------------------------------------------
% Important | 89672 | transOrigPointerField is not longer allowed to be excluded

\begin{frame}[fragile]
	\frametitle{Zastarele/izbačene funkcije}
	\framesubtitle{TCA: \texttt{transOrigPointerField}}

	\begin{itemize}
		\item Izostavljanje polja sa sledećim TCA opcijama dovodi do nekonzistentnih podataka
			snimljenih u bazu podataka pod odredjenim okolnostima:
			\small
				\texttt{\$GLOBALS['TCA'][\$table]['ctrl']['transOrigPointerField']}
			\normalsize

		\item Stoga, polje sa ovom opcijom ne može se više izostavljati.
		\item Čarobnjak za migraciju uklanja opciju za izostavljanje iz TCA i dodaje poruku o zastarelosti
			u deprecation log u slučaju da kod treba da se izmeni.
	\end{itemize}

\end{frame}

% ------------------------------------------------------------------------------
% Deprecation | 90421 | DocumentTemplate

\begin{frame}[fragile]
	\frametitle{Zastarele/izbačene funkcije}
	\framesubtitle{DocumentTemplate}

	% decrease font size for code listing
	\lstset{basicstyle=\tiny\ttfamily}

	\begin{itemize}
		\item Sledeća klasa je označena kao \textbf{zastarela}:

			\begin{itemize}
				\item \texttt{TYPO3\textbackslash
					CMS\textbackslash
					Backend\textbackslash
					Template\textbackslash
					DocumentTemplate}
			\end{itemize}

		\item Koristila se za renderanje modula administratorskog interfejsa
			ili ispisivanje sadržaja zasnovanog na HTML-u u TYPO3 administratorskom interfejsu.
		\item Od TYPO3 v7 se koristi novi ModuleTemplate API.

\vspace{-0.4cm}
\begin{lstlisting}
use TYPO3\CMS\Backend\Template\ModuleTemplate;
...
$moduleTemplate = GeneralUtility::makeInstance(ModuleTemplate::class);
$content = $this->getHtmlContentFromMyModule();
$moduleTemplate->setTitle('My module');
$moduleTemplate->setContent($content);
return new HtmlResponse($moduleTemplate->renderContent());
\end{lstlisting}

	\end{itemize}

\end{frame}

% ------------------------------------------------------------------------------
% Deprecation | 85613 | Introduce simple way to register category fields
%
%\begin{frame}[fragile]
%	\frametitle{Zastarele/izbačene funkcije}
%	\framesubtitle{CategoryRegistry API}
%
%	\begin{itemize}
%		\item The following class has been marked as \textbf{deprecated}:
%
%			\begin{itemize}
%				\item \texttt{TYPO3\textbackslash
%					CMS\textbackslash
%					Core\textbackslash
%					Category\textbackslash
%					CategoryRegistry}
%			\end{itemize}
%
%		\item The following method has been marked as \textbf{deprecated}:
%
%			\begin{itemize}
%				\item \texttt{ExtgensionManagementUtility::makeCategorizable()}
%			\end{itemize}
%
%		\item Category fields can easily by registered by adding TCA configuration.
%
%	\end{itemize}
%
%\end{frame}
%
% ------------------------------------------------------------------------------
% Deprecation | 90390 | Deprecate BrokenLinkRepository::getNumberOfBrokenLinks() in linkvalidator

\begin{frame}[fragile]
	\frametitle{Zastarele/izbačene funkcije}
	\framesubtitle{LinkValidator}

	\begin{itemize}
		\item Sledeći metod je označen kao \textbf{zastareo}:
		\newline\newline
			\smaller
				\texttt{TYPO3\textbackslash
					CMS\textbackslash
					Linkvalidator\textbackslash
					Repository\textbackslash
					BrokenLinkRepository}\newline
				\texttt{->getNumberOfBrokenLinks()}\normalsize\newline

		\item Kao zamenu koristiti sledeći metod u istoj klasi:\newline
			\small\texttt{BrokenLinkRepository::isLinkTargetBrokenLink()}\normalsize

	\end{itemize}

\end{frame}


% ------------------------------------------------------------------------------
