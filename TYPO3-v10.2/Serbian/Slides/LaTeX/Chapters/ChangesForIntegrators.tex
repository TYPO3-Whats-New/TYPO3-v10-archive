% ------------------------------------------------------------------------------
% TYPO3 Version 10.2 - What's New (Serbian Version)
%
% @license	Creative Commons BY-NC-SA 3.0
% @link		http://typo3.org/download/release-notes/whats-new/
% @language	Serbian
% ------------------------------------------------------------------------------

\section{Izmene za integratore}
\begin{frame}[fragile]
	\frametitle{Izmene za integratore}

	\begin{center}\huge{Poglavlje 2:}\end{center}
	\begin{center}\huge{\color{typo3darkgrey}\textbf{Izmene za integratore}}\end{center}

\end{frame}

% ------------------------------------------------------------------------------
% Feature | 85592 | Add site title configuration to sites module

\begin{frame}[fragile]
	\frametitle{Izmene za integratore}
	\framesubtitle{Konfiguracija sajta (1)}

	\begin{itemize}

		\item Naslov sajta se sada može podesiti u
			\textbf{SITE CONFIGURATION} $\rightarrow$ \textbf{Sites}.
		\item Ovo omogućava razlicite naslove sajta za različite jezike.
		\item Ovo polje u rekordu šablona je označeno kao \textbf{zastarelo}.
		\item Polje \texttt{sys\_template.sitetitle} (baza podataka i TCA) će biti uklonjeno u TYPO3 v11.
		\item Naslov sajta se koristi za naslov stranice, a u budućnosti i za integraciju sa
		  \texttt{schema.org}.
	\end{itemize}

\end{frame}

% ------------------------------------------------------------------------------
% Feature | 89398 | Support for environment variables in imports in site configurations

\begin{frame}[fragile]
	\frametitle{Izmene za integratore}
	\framesubtitle{Konfiguracija sajta (2)}

	% decrease font size for code listing
	\lstset{basicstyle=\tiny\ttfamily}

	\begin{itemize}

		\item Sada se mogu koristiti promenljive okruženja u importima YAML fajlova konfiguracije sajta:
\begin{lstlisting}
imports:
  -
    resource: 'Env_%env("foo")%.yaml'
\end{lstlisting}

	\end{itemize}

\end{frame}

% ------------------------------------------------------------------------------
% Feature | 88102 | Frontend Login Form Via Fluid And Extbase

\begin{frame}[fragile]
	\frametitle{Izmene za integratore}
	\framesubtitle{Logovanje na korisnički interfejs (1)}

	\begin{itemize}

		\item TYPO3 v10.2 poseduje Extbase verziju funkcionalnosti za logovanje na korisnički interfejs.
		\item Ovo rešenje ima nekoliko prednosti:

			\begin{itemize}
				\item Lakša izmena šablona.
				\item Slanje mejlova za resetovanje lozinke na bazi HTML-a.
				\item Podešavanjem i menjanjem validatora je moguće da se podese ograničenja za lozinke.
			\end{itemize}

		\item Novi Extbase plugin je odmah dostupan za nove instalacije.
		\item Postojeće TYPO3 instance će i dalje koristiti stare šablone.
		\item Integratori mogu da menjaju "stari" i "novi" plugin.

	\end{itemize}

\end{frame}

% ------------------------------------------------------------------------------
% Feature | 88110 | Felogin extbase password recovery

\begin{frame}[fragile]
	\frametitle{Izmene za integratore}
	\framesubtitle{Logovanje na korisnički interfejs (2)}

	\begin{itemize}

		\item Forma za resetovanje lozinke je dodata kao deo Extbase plugin-a.
		\item Korisnik može da zahteva promenu lozinke i dobiće email sa linkom ka formi.
		\item Podrazumevana pravila validacije lozinke su:

			\begin{itemize}
				\item \texttt{NotEmptyValidator} - lozinka ne može biti prazna.
				\item \texttt{StringLengthValidator} - lozinka mora imati minimalnu dužinu.
			\end{itemize}

	\end{itemize}

\end{frame}

% ------------------------------------------------------------------------------
% Feature | 88110 | Felogin extbase password recovery

\begin{frame}[fragile]
	\frametitle{Izmene za integratore}
	\framesubtitle{Logovanje na korisnički interfejs (3)}

	% decrease font size for code listing
	\lstset{basicstyle=\tiny\ttfamily}

	\begin{itemize}
		\item Ova pravila validacije se mogu podešavati.
		\item Na primer:
\begin{lstlisting}
plugin.tx_felogin_login {
  settings {
    passwordValidators {
      10 = TYPO3\CMS\Extbase\Validation\Validator\AlphanumericValidator
      20 {
        className = TYPO3\CMS\Extbase\Validation\Validator\StringLengthValidator
        options {
          minimum = 12
          maximum = 32
        }
      }
      30 = \Vendor\MyExtension\Validation\Validator\MyCustomPasswordPolicyValidator
    }
  }
}
\end{lstlisting}

	\end{itemize}

\end{frame}

% ------------------------------------------------------------------------------
% Feature | 89526 | FeatureFlag: betaTranslationServer

\begin{frame}[fragile]
	\frametitle{Izmene za integratore}
	\framesubtitle{Platforma za upravljanje prevodima}

	\begin{itemize}

		\item \href{https://crowdin.com/}{crowdin} ima za cilj da zameni postojeće rešenje
			za upravljanje prevodima \href{https://translation.typo3.org/}{Pootle}.

		\item Mogućnost izbora ove funkcionalnosti je dodata u TYPO3 v10.2 i ona koristi crowdin.com
			kao izvor za prevode ako je omogućena.

		\item Napomena: ovo je \textbf{beta verzija}.

		\item Više možete pročitati na
			\href{https://typo3.org/community/teams/typo3-development/initiatives/localization-with-crowdin/}{initiative}.

	\end{itemize}

	\begin{figure}
		\includegraphics[width=0.50\linewidth]{ChangesForIntegrators/crowdin-logo.png}
	\end{figure}

\end{frame}

% ------------------------------------------------------------------------------
% Feature | 89171 | Added possibility to have multiple sitemaps

\begin{frame}[fragile]
	\frametitle{Izmene za integratore}
	\framesubtitle{Višestruke mape sajtova}

	% decrease font size for code listing
	\lstset{basicstyle=\tiny\ttfamily}

	\begin{itemize}

		\item Sada se može konfigurisati više mapa sajtova.
		\item Sintaksa:
\begin{lstlisting}
plugin.tx_seo {
  config {
    <sitemapType> {
      sitemaps {
        <unique key> {
          provider = TYPO3\CMS\Seo\XmlSitemap\RecordsXmlSitemapDataProvider
          config {
            ...
          }
        }
      }
    }
  }
}
\end{lstlisting}

	\end{itemize}

\end{frame}

% ------------------------------------------------------------------------------
% Feature | 86759 | Support nomodule attribute for JavaScript includes

\begin{frame}[fragile]
	\frametitle{Izmene za integratore}
	\framesubtitle{HTML5 atribut \texttt{nomodule}}

	% decrease font size for code listing
	\lstset{basicstyle=\tiny\ttfamily}

	\begin{itemize}
		\item HTML5 atribut \texttt{nomodule} je podržan kada se uključuju JavaScript fajlovi putem TypoScript-a.
\begin{lstlisting}
page.includeJSFooter.file = path/to/classic-file.js
page.includeJSFooter.file.nomodule = 1
\end{lstlisting}

		\item Ovaj atribut sprečava izvršenje skripte u pretraživačima koji podržavaju modularne skripte.

		\item Više o ovom standardu možete pročitati u
			\href{https://html.spec.whatwg.org/multipage/scripting.html#attr-script-nomodule}{specifikaciji},
			a ovde o konceptu \href{https://hacks.mozilla.org/2015/08/es6-in-depth-modules/}{modula}.

	\end{itemize}

% <script type="module" src="path/to/file.js"></script>
% <script nomodule src="path/to/file/classic-file.js"></script>

\end{frame}

% ------------------------------------------------------------------------------
% Feature | 87798 | Provide a way to sort form lists in ext:form

\begin{frame}[fragile]
	\frametitle{Izmene za integratore}
	\framesubtitle{Sortiranje formi}

	% decrease font size for code listing
	\lstset{basicstyle=\tiny\ttfamily}

	\begin{itemize}
		\item Forme se mogu sortirati u rastućem ili opadajućem redosledu.
		\item Dodata su dva nova podešavanja: \texttt{sortByKeys} i \texttt{sortAscending}.
		\item Forme su u startu sortirane po imenu i UID-u (u rastućem redosledu).
		\item Da se promeni redosled, neophodna je sledeća konfiguracija u YAML konfiguracionom fajlu:
\begin{lstlisting}
TYPO3:
  CMS:
    Form:
      persistenceManager:
        sortByKeys: ['name', 'fileUid']
        sortAscending: true
\end{lstlisting}

	\end{itemize}

\end{frame}

% ------------------------------------------------------------------------------
% Feature | 86918 | Add additional configuration for external link types in Linkvalidator

\begin{frame}[fragile]
	\frametitle{Izmene za integratore}
	\framesubtitle{Validator linkova (1)}

	% decrease font size for code listing
	\lstset{basicstyle=\tiny\ttfamily}

	\begin{itemize}
		\item Validator linkova sada podržava dodatna podešavanja za konfiguraciju eksternih linkova.
		\item Vrednosti za \texttt{httpAgentUrl} i \texttt{httpAgentEmail} se moraju dostaviti.
		\item Podešavanja \texttt{headers}, \texttt{method} i \texttt{range} su napredna podešavanja.
\begin{lstlisting}
mod.linkvalidator {
  linktypesConfig {
    external {
      httpAgentName = ...
      httpAgentUrl = ...
      httpAgentEmail = ...
      headers {
      }
      method = HEAD
      range = 0-4048
    }
  }
}
\end{lstlisting}

	\end{itemize}

\end{frame}

% ------------------------------------------------------------------------------
% Feature | 84990 | Add event for checking external links in RTE

\begin{frame}[fragile]
	\frametitle{Izmene za integratore}
	\framesubtitle{Validator linkova (2)}

	\begin{itemize}
		\item Validator linkova označava \textbf{eksterne} linkove koji ne rade i u RTE.
		\item Ova funkcionalnost je bila dostupna samo za interne linkove.
		\item Preporuka je da se Validator linkova pokreće redovno putem Scheduler-a da bi tražio linkove koji ne rade.
	\end{itemize}

\end{frame}

% ------------------------------------------------------------------------------
