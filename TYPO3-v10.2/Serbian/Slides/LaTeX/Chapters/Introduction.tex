% ------------------------------------------------------------------------------
% TYPO3 Version 10.2 - What's New (Serbian Version)
%
% @license	Creative Commons BY-NC-SA 3.0
% @link		https://typo3.org/help/documentation/whats-new/
% @language	Serbian
% ------------------------------------------------------------------------------

\section{Uvod}
\begin{frame}[fragile]
	\frametitle{Uvod}

	\begin{center}\huge{Uvod}\end{center}
	\begin{center}\huge{\color{typo3darkgrey}\textbf{Činjenice}}\end{center}

\end{frame}

% ------------------------------------------------------------------------------
% TYPO3 Version 10.2 - The Facts

\begin{frame}[fragile]
	\frametitle{Uvod}
	\framesubtitle{TYPO3 Verzija 10.2 - Činjenice}

	\begin{itemize}
		\item Datum objavljivanja: 03. decembar 2019.
		\item Tip objavljivanja: Brza objava (Sprint Release)
	\end{itemize}

	\begin{figure}
		\includegraphics[width=0.95\linewidth]{Introduction/typo3-v10-2-banner.png}
	\end{figure}

\end{frame}

% ------------------------------------------------------------------------------
% TYPO3 Version 10.2 - Executive Summary

\begin{frame}[fragile]
	\frametitle{Uvod}
	\framesubtitle{Rezime}

	\small
		TYPO3 verzija 10.2 je treća brza objava na putu ka LTS-verziji (verzija sa
		dugoročnom podrškom) u 2020. Ovo je takodje poslednja brza objava u ovoj godini.

		\vspace{0.2cm}

		Tokom TYPO3 Initiative Week (T3INIT19) je razvijeno dosta funkcionalnosti
		i TYPO3 v10.2 već sadrži neke od ovih koncepta.

		\vspace{0.2cm}

		Ova objava pravi put ka njmodernijem okruženju. Pored toga što TYPO3 v10.2 podržava
		Symfony verziju 5.0, takodje, ovo je prva TYPO3 objava koja podržava
		PHP verziju 7.4. Takodje je poslednja objava pre zamrzavanja funkcionalnosti
		u februaru 2020.

	\normalsize

\end{frame}

% ------------------------------------------------------------------------------
% System Requirements

\begin{frame}[fragile]
	\frametitle{Uvod}
	\framesubtitle{Sistemski zahtevi}

	\begin{itemize}
		\item PHP verzija 7.2, 7.3 ili 7.4
		\item PHP podešavanja:

			\begin{itemize}
				\item \texttt{memory\_limit} >= 256M
				\item \texttt{max\_execution\_time} >= 240s
				\item \texttt{max\_input\_vars} >= 1500
				\item opcija \texttt{-}\texttt{-disable-ipv6} \underline{ne sme} se koristiti
			\end{itemize}

		\item Većina DB servera koji rade sa \textbf{Doctrine DBAL} rade takodje i sa TYPO3.
			Testirani DB serveri su:
	\end{itemize}

	\begin{figure}
		\includegraphics[width=0.80\linewidth]{Introduction/logo-databases.png}
	\end{figure}

\end{frame}

% ------------------------------------------------------------------------------
% Development, Release and Maintenance Timeline

\begin{frame}[fragile]
	\frametitle{Uvod}
	\framesubtitle{Razvoj, objavljivanje i vreme održavanja}

	\textbf{TYPO3 v10}

	\begin{figure}
		\includegraphics[width=1\linewidth]{Introduction/typo3-v10-lifecycle.png}
	\end{figure}

	\textbf{Produženo vreme podrške}\newline
	\smaller
		\href{https://typo3.com}{TYPO3 GmbH} nudi dodatne opcije za podršku za
		TYPO3 v10 LTS čak i posle 30. aprila 2023. za dodatne dve godine.
	\normalsize

\end{frame}

% ------------------------------------------------------------------------------
% TYPO3 v10 Roadmap

\begin{frame}[fragile]
	\frametitle{Uvod}
	\framesubtitle{TYPO3 v10 plan}

	Predvidjeni datumi objavljivanja i njihov osnovni fokus:

	\begin{itemize}

		\item v10.0 \tabto{1.1cm}23. jul 2019.\tabto{3.4cm}Otvaranje puta za uzbudljive nove koncepte i API-je
		\item v10.1 \tabto{1.1cm}01. okt. 2019.\tabto{3.4cm}Unapredjenje routing-a i upravljanje sajtom v2
		\item
			\begingroup
				\color{typo3orange}
				v10.2 \tabto{1.1cm}03. dec. 2019.\tabto{3.4cm}Fluid/Rendering Engine unapredjenja
			\endgroup
		\item v10.3 \tabto{1.1cm}04. feb. 2020.\tabto{3.4cm}Zamrzavanje funkcionalnosti
		\item v10.4 \tabto{1.1cm}07. apr. 2020.\tabto{3.4cm}LTS objava (objava sa dugoro£nom podrškom)

	\end{itemize}

	\smaller
		\url{https://typo3.org/article/typo3-v10-roadmap/}\newline
		\url{https://typo3.org/article/typo3-v10-safe-and-sound/}
	\normalsize

\end{frame}

% ------------------------------------------------------------------------------
% Installation

\begin{frame}[fragile]
	\frametitle{Uvod}
	\framesubtitle{Instalacija}

	\begin{itemize}
		\item Zvanična \textit{klasična} procedura za instalaciju na Linux/Mac OS X
			(DocumentRoot na primer \texttt{/var/www/site/htdocs}):
\begin{lstlisting}
$ cd /var/www/site
$ wget --content-disposition get.typo3.org/10.2
$ tar xzf typo3_src-10.2.0.tar.gz
$ cd htdocs
$ ln -s ../typo3_src-10.2.0 typo3_src
$ ln -s typo3_src/index.php
$ ln -s typo3_src/typo3
$ touch FIRST_INSTALL
\end{lstlisting}

		\item Simbolički linkovi (Symbolic links) na Microsoft Windows:

			\begin{itemize}
				\item Koristiti \texttt{junction} za Windows XP/2000
				\item Koristiti \texttt{mklink} za Windows Vista i Windows 7 i novije
			\end{itemize}

	\end{itemize}
\end{frame}

% ------------------------------------------------------------------------------
% Installation using composer

\begin{frame}[fragile]
	\frametitle{Uvod}
	\framesubtitle{Instalacija korišćenjem \texttt{composer-a}}

	\begin{itemize}
		\item Instalacija korišćenjem \textit{composer-a} na Linux/Mac OS X i Windows 10:
\begin{lstlisting}
$ cd /var/www/site/
$ composer create-project typo3/cms-base-distribution typo3v10 ^10.2
\end{lstlisting}

		\item Alternativno, napravite Vaš \texttt{composer.json} fajl i pokrenite:
\begin{lstlisting}
$ composer install
\end{lstlisting}

			Više detalja i primer \texttt{composer.json} fajla možete skinuti sa:
			\smaller
				\href{https://composer.typo3.org}{https://composer.typo3.org}
			\normalsize

	\end{itemize}
\end{frame}

% ------------------------------------------------------------------------------
