% ------------------------------------------------------------------------------
% TYPO3 Version 10.2 - What's New (Serbian Version)
%
% @license	Creative Commons BY-NC-SA 3.0
% @link		http://typo3.org/download/release-notes/whats-new/
% @language	Serbian
% ------------------------------------------------------------------------------

\section{Izmene za programere}
\begin{frame}[fragile]
	\frametitle{Izmene za programere}

	\begin{center}\huge{Poglavlje 3:}\end{center}
	\begin{center}\huge{\color{typo3darkgrey}\textbf{Izmene za programere}}\end{center}

\end{frame}

% ------------------------------------------------------------------------------
% Feature | 88950 | Add storeSession argument to Widget ViewHelpers

\begin{frame}[fragile]
	\frametitle{Izmene za programere}
	\framesubtitle{Widget ViewHelper-i}

	% decrease font size for code listing
	\lstset{basicstyle=\smaller\ttfamily}

	\begin{itemize}
		\item Widget ViewHelper-i u odredjenim situacijama postavljaju session cookie na korisničkom interfejsu.
		\item Pošto ovo nije poželjno u svim situacijama (na primer zbog GDPR), sada može da se kontroliše.
		\item Boolean promenljiva \texttt{storeSession} je dodata da omogući programerima da omoguće/onemoguće
		 	ovu funkcionalnost.
\begin{lstlisting}
<f:widget.autocomplete
  for="name"
  objects="{posts}"
  searchProperty="author"
  storeSession="false" />
\end{lstlisting}

	\end{itemize}

\end{frame}

% ------------------------------------------------------------------------------
% Feature | 89577 | New PSR-14 based events for File Abstraction Layer
% Deprecation | 89577 | FAL SignalSlot handling migrated to PSR-14 events

\begin{frame}[fragile]
	\frametitle{Izmene za programere}
	\framesubtitle{PSR-14 dogadjaji u FAL}

	% decrease font size for code listing
	\lstset{basicstyle=\tiny\ttfamily}

	\begin{itemize}
		\item U File Abstraction Layer (FAL) dodato je oko 40 novih dogadjaja zasnovanih na
			\href{https://www.php-fig.org/psr/psr-14/}{PSR-14}.
		\item Oni zamenjuju postojeće Extbase signale/slotove.
		\item Korišćenje signala i dalje radi (bez poruke o zastarelosti!).
			Medjutim, signali iz FAL-a će najverovatnije biti uklonjeni u TYPO3 v11.
		\item Autori proširenja se savetuju da migriraju kod i koriste dogadjaje.
		\item Pogledajte novie PHP klase da saznate više o PSR-14.
	\end{itemize}

\end{frame}


% ------------------------------------------------------------------------------
% Deprecation | 89733 | Signal Slots in Core Extension migrated to PSR-14 events

\begin{frame}[fragile]
	\frametitle{Izmene za programere}
	\framesubtitle{PSR-14 dogadjaji u jezgru TYPO3}

	\begin{itemize}
		\item Veliki broj novih dogadjaja zasnovanih na PSR-14 zamenuje signale/slotove u jezgru TYPO3:
			\newline

			\begin{itemize}\tiny
				\item \texttt{TYPO3\textbackslash
					CMS\textbackslash
					Core\textbackslash
					Imaging\textbackslash
					Event\textbackslash
					ModifyIconForResourcePropertiesEvent}
					\newline
				\item \texttt{TYPO3\textbackslash
					CMS\textbackslash
					Core\textbackslash
					DataHandling\textbackslash
					Event\textbackslash
					IsTableExcludedFromReferenceIndexEvent}
					\newline
				\item \texttt{TYPO3\textbackslash
					CMS\textbackslash
					Core\textbackslash
					DataHandling\textbackslash
					Event\textbackslash
					AppendLinkHandlerElementsEvent}
					\newline
				\item \texttt{TYPO3\textbackslash
					CMS\textbackslash
					Core\textbackslash
					Configuration\textbackslash
					Event\textbackslash
					AfterTcaCompilationEvent}
					\newline
				\item \texttt{TYPO3\textbackslash
					CMS\textbackslash
					Core\textbackslash
					Database\textbackslash
					Event\textbackslash
					AlterTableDefinitionStatementsEvent}
					\newline
				\item \texttt{TYPO3\textbackslash
					CMS\textbackslash
					Core\textbackslash
					Tree\textbackslash
					Event\textbackslash
					ModifyTreeDataEvent}
					\newline
				\item \texttt{TYPO3\textbackslash
					CMS\textbackslash
					Backend\textbackslash
					Backend\textbackslash
					Event\textbackslash
					SystemInformationToolbarCollectorEvent}
					\newline
			\end{itemize}

	\end{itemize}

\end{frame}

% ------------------------------------------------------------------------------
% Deprecation | 89718 | Legacy PageTSconfig parsing lowlevel API

\begin{frame}[fragile]
	\frametitle{Izmene za programere}
	\framesubtitle{Prepared Statements}

	% decrease font size for code listing
	\lstset{basicstyle=\tiny\ttfamily}

	\begin{itemize}
		\item Dodate su dve nove PHP klase za čitanje i parsiranje PageTSconfig:
			\begin{itemize}\smaller
				\item \texttt{TYPO3\textbackslash
					CMS\textbackslash
					Core\textbackslash
					Configuration\textbackslash
					Loader\textbackslash
					PageTsConfigLoader}
				\item \texttt{TYPO3\textbackslash
					CMS\textbackslash
					Core\textbackslash
					Configuration\textbackslash
					Parser\textbackslash
					PageTsConfigParser}
			\end{itemize}

		\item Na primer:
\begin{lstlisting}
// Fetch all available PageTS of a page/rootline:
$loader = GeneralUtility::makeInstance(PageTsConfigLoader::class);
$tsConfigString = $loader->load($rootLine);

// Parse the string and apply conditions:
$parser = GeneralUtility::makeInstance(
  PageTsConfigParser::class, $typoScriptParser, $hashCache
);

$pagesTSconfig = $parser->parse($tsConfigString, $conditionMatcher);
\end{lstlisting}

	\end{itemize}

\end{frame}

% ------------------------------------------------------------------------------
% Important | 87518 | Use prepared statements for pdo_mysql per default

\begin{frame}[fragile]
	\frametitle{Izmene za programere}
	\framesubtitle{Prepared Statements}

	% decrease font size for code listing
	\lstset{basicstyle=\tiny\ttfamily}

	\begin{itemize}
		\item Po podrazumevanim podešavanjima \texttt{pdo\_mysql} drajver koristi prepared statements.
		\item U TYPO3 < v10.2 korišćene su \textit{emulirane prepared statements}.
			Ovo znači da su sve vrednosti upita koje su vraćene bile tekstualnog tipa.
		\item Ovo ponašanje je promenjeno i prepared statements sada vraćaju nativne tipove podataka.
		\item Na primer: vrednosti za kolonu koja je definisana kao celobrojna se vraćaju u PHP kao \texttt{int}.
		\item Ova funkcionalnost može da se isključi u podešavanjima konekcije sa bazom podataka
			u opciji \texttt{PDO::ATTR\_EMULATE\_PREPARES}.

	\end{itemize}

\end{frame}

% ------------------------------------------------------------------------------
% Feature | 86967 | Allow fetching uid of a LazyLoadingProxy without loading the object first

\begin{frame}[fragile]
	\frametitle{Izmene za programere}
	\framesubtitle{Lazy Loading Proxy}

	% decrease font size for code listing
	\lstset{basicstyle=\tiny\ttfamily}

	\begin{itemize}
		\item Metod \texttt{getUid()} je dodat klasi\newline
			\texttt{TYPO3\textbackslash
				CMS\textbackslash
				Extbase\textbackslash
				Persistence\textbackslash
				Generic\textbackslash
				LazyLoadingProxy}.
		\item Ovo omogućava programerima da dodju do UID-a objekta koji je iza proksija,
			bez potrebe da se objekat dovlači iz baze podataka.

	\end{itemize}

\end{frame}

% ------------------------------------------------------------------------------
% Feature | 87380 | Introduce SiteLanguageAwareInterface to denote site language awareness

\begin{frame}[fragile]
	\frametitle{Izmene za programere}
	\framesubtitle{Označavanje jezika sajta}

	% decrease font size for code listing
	\lstset{basicstyle=\tiny\ttfamily}

% SiteLanguageAwareInterface with the methods setSiteLanguage(Entity\SiteLanguage $siteLanguage) and getSiteLanguage() has been introduced.
% The interface can be used to denote a class as aware of the site language.

	\begin{itemize}
		\item \texttt{SiteLanguageAwareInterface} je dodat.
		\item Interfejs se može koristiti da označi da je klasa svesna jezika sajta.
		\item Aspekti rutiranja, koji uzimaju u obzir jezik sajta,
			sada koriste i \texttt{SiteLanguageAwareInterface}
			uz \texttt{SiteLanguageAwareTrait}.
	\end{itemize}

\end{frame}

% ------------------------------------------------------------------------------
% Important | 89645 | Removed systemLog options

\begin{frame}[fragile]
	\frametitle{Izmene za programere}
	\framesubtitle{System Log API}

	% decrease font size for code listing
	\lstset{basicstyle=\tiny\ttfamily}

	\begin{itemize}
		\item Sledeće opcije su uklonjene iz podrazumevane konfiguracije TYPO3:

			\begin{itemize}\smaller
				\item \texttt{\$GLOBALS['TYPO3\_CONF\_VARS']['SYS']['systemLog']}
				\item \texttt{\$GLOBALS['TYPO3\_CONF\_VARS']['SYS']['systemLogLevel']}
			\end{itemize}\normalsize

		\item Autori proširenja se savetuju da koriste Logging API i uklone systemLog opcije.
	\end{itemize}

\end{frame}

% ------------------------------------------------------------------------------
% Feature | 89603 | Introduce native pagination for lists

\begin{frame}[fragile]
	\frametitle{Izmene za programere}
	\framesubtitle{Podrazumevana paginacija lista}

	% decrease font size for code listing
	\lstset{basicstyle=\tiny\ttfamily}

	\begin{itemize}
		\item Dodata je podrazumevana podrška za paginaciju lista kao što su nizovi i Extbase QueryResults.
		\item \texttt{PaginatorInterface} definiše osnovni skup metoda.
		\item \texttt{AbstractPaginator} klasa sadrži osnovnu logiku za paginaciju.
		\item Ovo omogućava programerima da naprave paginatore po svojoj potrebi.
\begin{lstlisting}
use TYPO3\CMS\Core\Pagination\ArrayPaginator;

$items = ['apple', 'banana', 'strawberry', 'raspberry', 'ananas'];
$currentPageNumber = 3;
$itemsPerPage = 2;

$paginator = new ArrayPaginator($itemsToBePaginated, $currentPageNumber, $itemsPerPage);
$paginator->getNumberOfPages(); // returns 3
$paginator->getCurrentPageNumber(); // returns 3
$paginator->getKeyOfFirstPaginatedItem(); // returns 5
$paginator->getKeyOfLastPaginatedItem(); // returns 5
\end{lstlisting}

	\end{itemize}

\end{frame}

% ------------------------------------------------------------------------------
% Deprecation | 89579 | ServiceChains require an array for excluded Service keys

\begin{frame}[fragile]
	\frametitle{Izmene za programere}
	\framesubtitle{API za servise}

	\begin{itemize}
		\item Argument \texttt{\$excludeServiceKeys} se koristi da se preskoči odredjeni servis kada se koristi lanac servisa.
		\item U TYPO3 v10.2 argument je promenjen iz liste razdvojene zarezom u niz.
		\item Ova izmena utiče na API za servise kod sledećih komponenti:

			\begin{itemize}
				\item \texttt{GeneralUtility::makeInstanceService()}
				\item \texttt{ExtensionManagementUtility::findService()}
			\end{itemize}

		\item Prosledjivanje liste razdvojene zarezom i dalje radi, ali je označeno kao \textbf{zastarelo}.

	\end{itemize}

\end{frame}

% ------------------------------------------------------------------------------
