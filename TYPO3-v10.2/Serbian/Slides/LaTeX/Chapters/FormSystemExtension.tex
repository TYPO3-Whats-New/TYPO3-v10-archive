% ------------------------------------------------------------------------------
% TYPO3 Version 10.2 - What's New (Serbian Version)
%
% @license	Creative Commons BY-NC-SA 3.0
% @link		http://typo3.org/download/release-notes/whats-new/
% @language	Serbian
% ------------------------------------------------------------------------------

\section{Sistemsko proširenje "Form"}
\begin{frame}[fragile]
	\frametitle{Sistemsko proširenje "Form"}

	\begin{center}\huge{Poglavlje 4:}\end{center}
	\begin{center}\huge{\color{typo3darkgrey}\textbf{Sistemsko proširenje "Form"}}\end{center}

\end{frame}

% ------------------------------------------------------------------------------
% System Extension "Form" - Summary

\begin{frame}[fragile]
	\frametitle{Sistemsko proširenje "Form"}
	\framesubtitle{Rezime}

	\small
		Sistemsko proširenje "Form" je doživelo nekoliko izmena.
		Ove izmene utiču na editore, integratore kao i na programere.

		\vspace{0.2cm}

		Neke od izmena su zasnovane na konceptima sa TYPO3 Initiative Week (T3INIT19).

	\normalsize

\end{frame}

% ------------------------------------------------------------------------------
% Important | 84221 | Restructuring of form setup

\begin{frame}[fragile]
	\frametitle{Sistemsko proširenje "Form"}
	\framesubtitle{Postavke Form-e}

	\begin{itemize}
		\item Do sada su se koristila tri fajla: \texttt{BaseSetup.yaml}, \texttt{FormEditorSetup.yaml}, i \texttt{FormEngineSetup.yaml}.
		\item Ovo je izmenjeno i sada se koristi samo jedan fajl: \texttt{FormSetup.yaml}.
		\item Ovaj fajl sadrži osnovna podešavanja uključujući importe konfiguracija za validatore, elemente forme i finišere.
		\item Svi ranije korišćeni miksini i nasledjivanja su sredjeni što za rezultat ima
			lako razumevanje cele konfiguracije.
	\end{itemize}

\end{frame}

% ------------------------------------------------------------------------------
% Feature | 84203 | Unify form setup YAML loading

\begin{frame}[fragile]
	\frametitle{Sistemsko proširenje "Form"}
	\framesubtitle{YAML Files}

	\begin{itemize}
		\item YAML fajlovi sada koriste YAML file loader iz jezgra TYPO3.
		\item Ovo omogućava funkcionalnosti kao sto su:

			\begin{itemize}
				\item Importovanje drugih YAML fajlova korišćenjem komande \texttt{imports}.
				\item Zamena za \texttt{\%placeholders\%}.
			\end{itemize}

	\end{itemize}

\end{frame}

% ------------------------------------------------------------------------------
% Feature | 79445 | Add Multistep Wizard

\begin{frame}[fragile]
	\frametitle{Sistemsko proširenje "Form"}
	\framesubtitle{Wizard za formu sa više koraka}

	\begin{itemize}
		\item Dodat je novi JavaScript modul \texttt{MultiStepWizard} koji dodaje sledeće
			funkcionalnosti:

			\begin{itemize}
				\item Navigacija na prethodni korak.
				\item Opisne labele koraka kao što su "Start" ili "Finish", umesto numeričkih indikatora "Korak x od y".
				\item Optimizovana struktura konfiguracije.
			\end{itemize}

		\item Pogledajte \href{https://docs.typo3.org/c/typo3/cms-core/master/en-us/Changelog/10.2/Feature-79445-AddMultistepWizard.html}{ChangeLog}
			za primere JavaScript koda.

		\item Ove nove funkcionalnosti značajno unapredjuju korisničko iskustvo:
			korisnici administratorskog interfejsa će primetiti unapredjen wizard za kreiranje forme.

	\end{itemize}

\end{frame}

% ------------------------------------------------------------------------------
% Feature | 89747 | Custom tables with record browser in forms

\begin{frame}[fragile]
	\frametitle{Sistemsko proširenje "Form"}
	\framesubtitle{Pretraživač rekorda}

	% decrease font size for code listing
	\lstset{basicstyle=\tiny\ttfamily}

	\begin{itemize}
		\item Pretraživač rekorda sada može da se podesi da koristi prilagodjene tabele:
\begin{lstlisting}
TYPO3:
  CMS:
    Form:
      prototypes:
        standard:
          formElementsDefinition:
            MyCustomElement:
              formEditor:
                editors:
                  # ...
                  300:
                    identifier: myRecord
                    # ...
                    browsableType: tx_myext_mytable
                    propertyPath: properties.myRecordUid
                    # ...
\end{lstlisting}

	\end{itemize}

\end{frame}

% ------------------------------------------------------------------------------
% Feature | 89746 | Custom icon for record browser button in forms

\begin{frame}[fragile]
	\frametitle{Sistemsko proširenje "Form"}
	\framesubtitle{Pretraživač rekorda}

	% decrease font size for code listing
	\lstset{basicstyle=\tiny\ttfamily}

	\begin{itemize}
		\item Ikonica dugmeta pretraživača rekorda je sada podesiva:
\begin{lstlisting}
TYPO3:
  CMS:
    Form:
      prototypes:
        standard:
          formElementsDefinition:
            MyCustomElement:
              formEditor:
                editors:
                  # ...
                  300:
                    identifier: contentElement
                    # ...
                    browsableType: tt_content
                    iconIdentifier: mimetypes-x-content-text
                    propertyPath: properties.contentElementUid
                    # ...
\end{lstlisting}

	\end{itemize}

\end{frame}

% ------------------------------------------------------------------------------
% Feature | 84713 | Access single values in form templates

\begin{frame}[fragile]
	\frametitle{Sistemsko proširenje "Form"}
	\framesubtitle{Pretraživač rekorda}

	% decrease font size for code listing
	\lstset{basicstyle=\tiny\ttfamily}

	\begin{itemize}
		\item Novi \textit{RenderFormValue-ViewHelper} omogućava integratorima/programerima pristup vrednostima u šablonima:
\begin{lstlisting}
<p>
  The following message was just sent by
  <formvh:renderFormValue renderable="{page.rootForm.elements.name}" as="formValue">
    {formValue.processedValue}
  </formvh:renderFormValue>:
</p>

<blockquote>
  <formvh:renderFormValue renderable="{page.rootForm.elements.message}" as="formValue">
    {formValue.processedValue}
  </formvh:renderFormValue>
</blockquote>
\end{lstlisting}

	\end{itemize}

\end{frame}

% ------------------------------------------------------------------------------
% Feature | 82706 | Render fieldset labels in form templates

\begin{frame}[fragile]
	\frametitle{Sistemsko proširenje "Form"}
	\framesubtitle{Lablele skupa polja}

	\begin{itemize}
		\item Element sekcije \texttt{Fieldset} je dostupan u šablonima.
		\item Po podrazumevanim podešavanima ovo utiče na element forme \textbf{SummaryPage} kao i \textbf{EmailToReceiver} i \textbf{EmailToSender} finišere.
		\item Tipičan primer primene:\newline
			\small
				Forma sa adresom za dostavu i za račun. Obe sekcije mogu da imaju polje sa istim imenom, na primer \texttt{street}.
				Sada ova dva polja mogu da se razlikuju korišćenjem lagbele skupa polja.
			\normalsize

	\end{itemize}

\end{frame}

% ------------------------------------------------------------------------------
% Deprecation | 88238 | Allowed MIME types of FileUpload and ImageUpload

\begin{frame}[fragile]
	\frametitle{Sistemsko proširenje "Form"}
	\framesubtitle{File Uploads}

	\begin{itemize}
		\item Predefinisani \texttt{allowedMimeTypes} sledećih elemenata forme, su označeni kao \textbf{zastareli}:

			\begin{itemize}
				\item \texttt{FileUpload}
				\item \texttt{ImageUpload}
			\end{itemize}

		\item Od sada svi validni MIME tipovi moraju biti eksplicitno izlistani u definiciji forme\newline
			\smaller
				(predefinisani MIME tipovi će biti uklonjeni u TYPO3 v11)
			\normalsize

		\item Integratori već sada mogu da uključe ovo ponašanje u TYPO3 v10.

	\end{itemize}

\end{frame}

% ------------------------------------------------------------------------------
% Deprecation | 89742 | Form mixins

\begin{frame}[fragile]
	\frametitle{Sistemsko proširenje "Form"}
	\framesubtitle{Mixini forme}

	\begin{itemize}
		\item Mixini su označeni kao \textbf{zastareli} i više se ne treba da se koristite.
		\item Ovo utiče na sva nasledjivanja iz \texttt{TYPO3.CMS.Form.mixins.*}.
		\item Opcije za migraciju:

			\begin{itemize}
				\item Ugradite neophodne delove iz \texttt{TYPO3.CMS.Form.mixins.*}, ili
				\item ih migrirajte u prilagodjene miksine.
			\end{itemize}

	\end{itemize}

\end{frame}

% ------------------------------------------------------------------------------
