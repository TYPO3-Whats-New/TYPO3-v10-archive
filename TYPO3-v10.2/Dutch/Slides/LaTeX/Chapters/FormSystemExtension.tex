% ------------------------------------------------------------------------------
% TYPO3 Version 10.2 - What's New (Dutch Version)
%
% @license	Creative Commons BY-NC-SA 3.0
% @link		http://typo3.org/download/release-notes/whats-new/
% @language	Dutch
% ------------------------------------------------------------------------------

\section{Systeemextensie "Form"}
\begin{frame}[fragile]
	\frametitle{Systeemextensie "Form"}

	\begin{center}\huge{Chapter 4:}\end{center}
	\begin{center}\huge{\color{typo3darkgrey}\textbf{Systeemextensie "Form"}}\end{center}

\end{frame}

% ------------------------------------------------------------------------------
% System Extension "Form" - Summary

\begin{frame}[fragile]
	\frametitle{Systeemextensie "Form"}
	\framesubtitle{Samenvatting}

	\small
		Diverse wijzigingen zijn aangebracht in de systeemextensie \textbf{"Form"}.
		Deze wijzigingen treffen redacteuren, integrators en ontwikkelaars.

		\vspace{0.2cm}

		Sommige wijzigingen zijn gebaseerd op concepten uit de TYPO3 Initiative Week (T3INIT19).

	\normalsize

\end{frame}

% ------------------------------------------------------------------------------
% Important | 84221 | Restructuring of form setup

\begin{frame}[fragile]
	\frametitle{Systeemextensie "Form"}
	\framesubtitle{Formulierdefinitie}

	\begin{itemize}
		\item Voorheen werden drie bestanden gebruikt: \texttt{BaseSetup.yaml}, \texttt{FormEditorSetup.yaml}, en \texttt{FormEngineSetup.yaml}.
		\item Deze zijn samengevoegd in een enkel bestand: \texttt{FormSetup.yaml}.
		\item Dit bestand bevat de basisopzet en imports van configuratie voor validators, formulierelementen en finishers.
		\item Alle voorheen gebruikte overervingen en mixins zijn opgelost waardoor het eenvoudiger wordt om de configuratie te begrijpen.
	\end{itemize}

\end{frame}

% ------------------------------------------------------------------------------
% Feature | 84203 | Unify form setup YAML loading

\begin{frame}[fragile]
	\frametitle{Systeemextensie "Form"}
	\framesubtitle{YAML-bestanden}

	\begin{itemize}
		\item YAML-bestanden gebruiken nu de YAML-lader van de TYPO3 core.
		\item Dit biedt functies als:

			\begin{itemize}
				\item Andere YAML-bestanden importeren via \texttt{imports} aanwijzing.
				\item Vervanging van \texttt{\%placeholders\%}.
			\end{itemize}

	\end{itemize}

\end{frame}

% ------------------------------------------------------------------------------
% Feature | 79445 | Add Multistep Wizard

\begin{frame}[fragile]
	\frametitle{Systeemextensie "Form"}
	\framesubtitle{Meerstapsassistent}

	\begin{itemize}
		\item Er is een nieuwe JavaScript module \texttt{MultiStepWizard},
			met de volgende functies:

			\begin{itemize}
				\item Navigatie naar vorige stappen.
				\item Stappen ondersteunen beschrijvende labels zoals "Begin" of "Einde", in plaats van nummers zoals "Stap x van y".
				\item Geoptimaliseerde configuratiestructuur.
			\end{itemize}

		\item Zie \href{https://docs.typo3.org/c/typo3/cms-core/master/en-us/Changelog/10.2/Feature-79445-AddMultistepWizard.html}{Overzicht wijzigingen}
			voor voorbeelden van JavaScript code.

		\item Deze functies verbeteren de gebruikerservaring enorm: backend-gebruikers hebben een verbeterde assistent voor het maken van formulieren.

	\end{itemize}

\end{frame}

% ------------------------------------------------------------------------------
% Feature | 89747 | Custom tables with record browser in forms

\begin{frame}[fragile]
	\frametitle{Systeemextensie "Form"}
	\framesubtitle{Bladeren door records}

	% decrease font size for code listing
	\lstset{basicstyle=\tiny\ttfamily}

	\begin{itemize}
		\item Het bladeren door records kan geconfigureerd worden om gebruikerstabellen te gebruiken:
\begin{lstlisting}
TYPO3:
  CMS:
    Form:
      prototypes:
        standard:
          formElementsDefinition:
            MyCustomElement:
              formEditor:
                editors:
                  # ...
                  300:
                    identifier: myRecord
                    # ...
                    browsableType: tx_myext_mytable
                    propertyPath: properties.myRecordUid
                    # ...
\end{lstlisting}

	\end{itemize}

\end{frame}

% ------------------------------------------------------------------------------
% Feature | 89746 | Custom icon for record browser button in forms

\begin{frame}[fragile]
	\frametitle{Systeemextensie "Form"}
	\framesubtitle{Bladeren door records}

	% decrease font size for code listing
	\lstset{basicstyle=\tiny\ttfamily}

	\begin{itemize}
		\item Het icoon op de knop voor het bladeren door records is instelbaar:
\begin{lstlisting}
TYPO3:
  CMS:
    Form:
      prototypes:
        standard:
          formElementsDefinition:
            MyCustomElement:
              formEditor:
                editors:
                  # ...
                  300:
                    identifier: contentElement
                    # ...
                    browsableType: tt_content
                    iconIdentifier: mimetypes-x-content-text
                    propertyPath: properties.contentElementUid
                    # ...
\end{lstlisting}

	\end{itemize}

\end{frame}

% ------------------------------------------------------------------------------
% Feature | 84713 | Access single values in form templates

\begin{frame}[fragile]
	\frametitle{Systeemextensie "Form"}
	\framesubtitle{Bladeren door records}

	% decrease font size for code listing
	\lstset{basicstyle=\tiny\ttfamily}

	\begin{itemize}
		\item Een nieuwe \textit{RenderFormValue-ViewHelper} zorgt voor toegang tot losse formulierwaardes in sjablonen voor integrators/developers:
\begin{lstlisting}
<p>
  Het volgende bericht was verstuurd door
  <formvh:renderFormValue renderable="{page.rootForm.elements.name}" as="formValue">
    {formValue.processedValue}
  </formvh:renderFormValue>:
</p>

<blockquote>
  <formvh:renderFormValue renderable="{page.rootForm.elements.message}" as="formValue">
    {formValue.processedValue}
  </formvh:renderFormValue>
</blockquote>
\end{lstlisting}

	\end{itemize}

\end{frame}

% ------------------------------------------------------------------------------
% Feature | 82706 | Render fieldset labels in form templates

\begin{frame}[fragile]
	\frametitle{Systeemextensie "Form"}
	\framesubtitle{Labels set velden}

	\begin{itemize}
		\item Het sectie-element \texttt{Fieldset} is nu toegankelijk in sjablonen.
		\item Standaard heeft dit invloed op zowel het \textbf{SummaryPage} formulierelement als de \textbf{EmailToReceiver} en \textbf{EmailToSender} finishers.
		\item Voorbeeld toepassing:\newline
			\small
				Een formulier met verzend- en factuuradres. Beide secties kunnen een veld hebben met dezelfde naam, bijv. \texttt{straat}.
				Er kan nu onderscheid gemaakt worden tussen de velden door de labels van de Fieldset.
			\normalsize

	\end{itemize}

\end{frame}

% ------------------------------------------------------------------------------
% Deprecation | 88238 | Allowed MIME types of FileUpload and ImageUpload

\begin{frame}[fragile]
	\frametitle{Systeemextensie "Form"}
	\framesubtitle{Bestanden uploaden}

	\begin{itemize}
		\item Voorgedefinieerde \texttt{allowedMimeTypes} van de volgende formulierelementen zijn als \textbf{verouderd} aangemerkt:

			\begin{itemize}
				\item \texttt{FileUpload}
				\item \texttt{ImageUpload}
			\end{itemize}

		\item Alle geldige MIME-types moeten nu expliciet opgesomd worden in de formulierdefinitie\newline
			\smaller
				(voorgedefinieerde MIME-types worden verwijderd in TYPO3 v11)
			\normalsize

		\item Integrators kunnen het nieuwe gedrag in TYPO3 v10 inschakelen via een functieschakelaar.

	\end{itemize}

\end{frame}

% ------------------------------------------------------------------------------
% Deprecation | 89742 | Form mixins

\begin{frame}[fragile]
	\frametitle{Systeemextensie "Form"}
	\framesubtitle{Formulier-mixins}

	\begin{itemize}
		\item Mixins zijn als \textbf{verouderd} aangemerkt en zouden niet meer gebruikt moeten worden.
		\item Dit geldt voor alle overervingen van \texttt{TYPO3.CMS.Form.mixins.*}.
		\item Opties voor migratie:

			\begin{itemize}
				\item Voeg de essenti\"ele delen van \texttt{TYPO3.CMS.Form.mixins.*} in, of
				\item verhuis ze naar maatwerk-mixins.
			\end{itemize}

	\end{itemize}

\end{frame}

% ------------------------------------------------------------------------------
