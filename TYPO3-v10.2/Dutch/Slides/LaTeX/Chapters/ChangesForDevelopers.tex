% ------------------------------------------------------------------------------
% TYPO3 Version 10.2 - What's New (Dutch Version)
%
% @license	Creative Commons BY-NC-SA 3.0
% @link		http://typo3.org/download/release-notes/whats-new/
% @language	Dutch
% ------------------------------------------------------------------------------

\section{Wijzigingen voor Ontwikkelaars}
\begin{frame}[fragile]
	\frametitle{Wijzigingen voor Ontwikkelaars}

	\begin{center}\huge{Hoofdstuk 3:}\end{center}
	\begin{center}\huge{\color{typo3darkgrey}\textbf{Wijzigingen voor Ontwikkelaars}}\end{center}

\end{frame}

% ------------------------------------------------------------------------------
% Feature | 88950 | Add storeSession argument to Widget ViewHelpers

\begin{frame}[fragile]
	\frametitle{Wijzigingen voor Ontwikkelaars}
	\framesubtitle{Widget ViewHelpers}

	% decrease font size for code listing
	\lstset{basicstyle=\smaller\ttfamily}

	\begin{itemize}
		\item Widget ViewHelpers plaatsen in sommige situaties een sessie-cookie in de frontend.
		\item Omdat dit niet altijd gewenst is (bijv. de AVG), kan dit nu worden be\"{\i}nvloed.
		\item Een instelling \texttt{storeSession} is ge\"{\i}ntroduceerd voor ontwikkelaars om deze functie te kunnen in- of uitschakelen.
\begin{lstlisting}
<f:widget.autocomplete
  for="name"
  objects="{posts}"
  searchProperty="author"
  storeSession="false" />
\end{lstlisting}

	\end{itemize}

\end{frame}

% ------------------------------------------------------------------------------
% Feature | 89577 | New PSR-14 based events for File Abstraction Layer
% Deprecation | 89577 | FAL SignalSlot handling migrated to PSR-14 events

\begin{frame}[fragile]
	\frametitle{Wijzigingen voor Ontwikkelaars}
	\framesubtitle{PSR-14 Events in FAL}

	% decrease font size for code listing
	\lstset{basicstyle=\tiny\ttfamily}

	\begin{itemize}
		\item Ongeveer 40 nieuwe op
			\href{https://www.php-fig.org/psr/psr-14/}{PSR-14}
			gebaseerde Events zijn toegevoegd aan de bestandsabstractielaag (FAL).
		\item Ze vervangen bestaande Extbase Signal/Slots.
		\item De Signals kunnen gebruikt blijven (zonder een melding dat het verouderd is!).
			De Signals in FAL worden waarschijnlijk verwijderd in TYPO3 v11.
		\item Extensiebouwers wordt geadviseerd de code te migreren en Events te gebruiken.
		\item Bekijk de nieuwe PHP klassen om meer te leren over PSR-14.
	\end{itemize}

\end{frame}


% ------------------------------------------------------------------------------
% Deprecation | 89733 | Signal Slots in Core Extension migrated to PSR-14 events

\begin{frame}[fragile]
	\frametitle{Wijzigingen voor Ontwikkelaars}
	\framesubtitle{PSR-14 Events in de TYPO3 Core}

	\begin{itemize}
		\item Een aantal nieuwe PSR-14 Events vervangen Signal/Slots in de TYPO3 core:
			\newline

			\begin{itemize}\tiny
				\item \texttt{TYPO3\textbackslash
					CMS\textbackslash
					Core\textbackslash
					Imaging\textbackslash
					Event\textbackslash
					ModifyIconForResourcePropertiesEvent}
					\newline
				\item \texttt{TYPO3\textbackslash
					CMS\textbackslash
					Core\textbackslash
					DataHandling\textbackslash
					Event\textbackslash
					IsTableExcludedFromReferenceIndexEvent}
					\newline
				\item \texttt{TYPO3\textbackslash
					CMS\textbackslash
					Core\textbackslash
					DataHandling\textbackslash
					Event\textbackslash
					AppendLinkHandlerElementsEvent}
					\newline
				\item \texttt{TYPO3\textbackslash
					CMS\textbackslash
					Core\textbackslash
					Configuration\textbackslash
					Event\textbackslash
					AfterTcaCompilationEvent}
					\newline
				\item \texttt{TYPO3\textbackslash
					CMS\textbackslash
					Core\textbackslash
					Database\textbackslash
					Event\textbackslash
					AlterTableDefinitionStatementsEvent}
					\newline
				\item \texttt{TYPO3\textbackslash
					CMS\textbackslash
					Core\textbackslash
					Tree\textbackslash
					Event\textbackslash
					ModifyTreeDataEvent}
					\newline
				\item \texttt{TYPO3\textbackslash
					CMS\textbackslash
					Backend\textbackslash
					Backend\textbackslash
					Event\textbackslash
					SystemInformationToolbarCollectorEvent}
					\newline
			\end{itemize}

	\end{itemize}

\end{frame}

% ------------------------------------------------------------------------------
% Deprecation | 89718 | Legacy PageTSconfig parsing lowlevel API

\begin{frame}[fragile]
	\frametitle{Wijzigingen voor Ontwikkelaars}
	\framesubtitle{Page TSconfig}

	% decrease font size for code listing
	\lstset{basicstyle=\tiny\ttfamily}

	\begin{itemize}
		\item Twee nieuwe PHP-klassen zijn toegevoegd om Page TSconfig te laden en te parsen:
			\begin{itemize}\smaller
				\item \texttt{TYPO3\textbackslash
					CMS\textbackslash
					Core\textbackslash
					Configuration\textbackslash
					Loader\textbackslash
					PageTsConfigLoader}
				\item \texttt{TYPO3\textbackslash
					CMS\textbackslash
					Core\textbackslash
					Configuration\textbackslash
					Parser\textbackslash
					PageTsConfigParser}
			\end{itemize}

		\item Voorbeeld:
\begin{lstlisting}
// Fetch all available PageTS of a page/rootline:
$loader = GeneralUtility::makeInstance(PageTsConfigLoader::class);
$tsConfigString = $loader->load($rootLine);

// Parse the string and apply conditions:
$parser = GeneralUtility::makeInstance(
  PageTsConfigParser::class, $typoScriptParser, $hashCache
);

$pagesTSconfig = $parser->parse($tsConfigString, $conditionMatcher);
\end{lstlisting}

	\end{itemize}

\end{frame}

% ------------------------------------------------------------------------------
% Important | 87518 | Use prepared statements for pdo_mysql per default

\begin{frame}[fragile]
	\frametitle{Wijzigingen voor Ontwikkelaars}
	\framesubtitle{Prepared Statements}

	% decrease font size for code listing
	\lstset{basicstyle=\tiny\ttfamily}

	\begin{itemize}
		\item Het \texttt{pdo\_mysql} stuurprogramma gebruikt nu standaard prepared statements.
		\item In TYPO3 < v10.2, worden \textit{nagebootste prepared statements} gebruikt.
			Dit betekent dat alle teruggegeven waarden van een query teksten zijn.
		\item Dit gedrag is gewijzigd en prepared statements worden gebruikt
			die de eigenlijke gegevenstypes teruggeven.
		\item Voorbeeld: waardes van een veld dat gedefinieerd is als geheel getal worden in PHP
			teruggegeven als \texttt{int}.
		\item Deze functie kan uitgeschakeld worden met de optie
			\texttt{PDO::ATTR\_EMULATE\_PREPARES} in de database-connectie.

	\end{itemize}

\end{frame}

% ------------------------------------------------------------------------------
% Feature | 86967 | Allow fetching uid of a LazyLoadingProxy without loading the object first

\begin{frame}[fragile]
	\frametitle{Wijzigingen voor Ontwikkelaars}
	\framesubtitle{Langzaam ladende Proxy}

	% decrease font size for code listing
	\lstset{basicstyle=\tiny\ttfamily}

	\begin{itemize}
		\item De functie \texttt{getUid()} is toegevoegd aan de klasse\newline
			\texttt{TYPO3\textbackslash
				CMS\textbackslash
				Extbase\textbackslash
				Persistence\textbackslash
				Generic\textbackslash
				LazyLoadingProxy}.
		\item Hiermee kan de UID van een object opgevraagd worden zonder het object op te halen uit de database.

	\end{itemize}

\end{frame}

% ------------------------------------------------------------------------------
% Feature | 87380 | Introduce SiteLanguageAwareInterface to denote site language awareness

\begin{frame}[fragile]
	\frametitle{Wijzigingen voor Ontwikkelaars}
	\framesubtitle{Aanduiding ondersteuning voor taal van site}

	% decrease font size for code listing
	\lstset{basicstyle=\tiny\ttfamily}

% A SiteLanguageAwareInterface with the methods setSiteLanguage(Entity\SiteLanguage $siteLanguage) and getSiteLanguage() has been introduced.
% The interface can be used to denote a class as aware of the site language.

	\begin{itemize}
		\item Een \texttt{SiteLanguageAwareInterface} is toegevoegd..
		\item De interface kan gebruikt worden om aan te duiden dat een klasse de taal
			van een site ondersteunt.
		\item Routering-aspecten die rekening houden met de taal van een site
			gebruiken nu de \texttt{SiteLanguageAwareInterface} naast
			de \texttt{SiteLanguageAwareTrait}.
	\end{itemize}

\end{frame}

% ------------------------------------------------------------------------------
% Important | 89645 | Removed systemLog options

\begin{frame}[fragile]
	\frametitle{Wijzigingen voor Ontwikkelaars}
	\framesubtitle{API voor systeemlog}

	% decrease font size for code listing
	\lstset{basicstyle=\tiny\ttfamily}

	\begin{itemize}
		\item De volgende opties zijn verwijderd uit de standaardconfiguratie van TYPO3:

			\begin{itemize}\smaller
				\item \texttt{\$GLOBALS['TYPO3\_CONF\_VARS']['SYS']['systemLog']}
				\item \texttt{\$GLOBALS['TYPO3\_CONF\_VARS']['SYS']['systemLogLevel']}
			\end{itemize}\normalsize

		\item Makers van extensies wordt aangeraden gebruik te maken van de log-API en de systeemlog-opties te verwijderen.
	\end{itemize}

\end{frame}

% ------------------------------------------------------------------------------
% Feature | 89603 | Introduce native pagination for lists

\begin{frame}[fragile]
	\frametitle{Wijzigingen voor Ontwikkelaars}
	\framesubtitle{Ingebouwde paginering van lijsten}

	% decrease font size for code listing
	\lstset{basicstyle=\tiny\ttfamily}

	\begin{itemize}
		\item Ondersteuning voor paginering van lijsten zoals array's of QueryResults van Extbase is nu ingebouwd.
		\item De \texttt{PaginatorInterface} definieert een aantal basisfuncties.
		\item De \texttt{AbstractPaginator} klasse bevat de belangrijkste logica van paginering.
		\item Hiermee kunnen allerlei soorten paginering gebouwd worden.
\begin{lstlisting}
use TYPO3\CMS\Core\Pagination\ArrayPaginator;

$items = ['appel', 'banaan', 'aardbei', 'framboos', 'ananas'];
$currentPageNumber = 3;
$itemsPerPage = 2;

$paginator = new ArrayPaginator($itemsToBePaginated, $currentPageNumber, $itemsPerPage);
$paginator->getNumberOfPages(); // resultaat: 3
$paginator->getCurrentPageNumber(); // resultaat: 3
$paginator->getKeyOfFirstPaginatedItem(); // resultaat: 5
$paginator->getKeyOfLastPaginatedItem(); // resultaat: 5
\end{lstlisting}

	\end{itemize}

\end{frame}

% ------------------------------------------------------------------------------
% Deprecation | 89579 | ServiceChains require an array for excluded Service keys

\begin{frame}[fragile]
	\frametitle{Wijzigingen voor Ontwikkelaars}
	\framesubtitle{Service API}

	\begin{itemize}
		\item Parameter \texttt{\$excludeServiceKeys} wordt gebruikt voor het overslaan van bepaalde services
			als deze geschakeld zijn.
		\item De parameter is gewijzigd van kommagescheiden lijst naar array in TYPO3 v10.2.
		\item Deze wijziging treft de Service API in de volgende componenten:

			\begin{itemize}
				\item \texttt{GeneralUtility::makeInstanceService()}
				\item \texttt{ExtensionManagementUtility::findService()}
			\end{itemize}

		\item Het doorgeven van een kommagescheiden lijst werkt maar is aangemerkt als \textbf{verouderd}.

	\end{itemize}

\end{frame}

% ------------------------------------------------------------------------------
