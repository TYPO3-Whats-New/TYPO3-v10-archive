% ------------------------------------------------------------------------------
% TYPO3 Version 10.2 - What's New (Dutch Version)
%
% @license	Creative Commons BY-NC-SA 3.0
% @link		http://typo3.org/download/release-notes/whats-new/
% @language	Dutch
% ------------------------------------------------------------------------------

\section{Wijzigingen voor Integrators}
\begin{frame}[fragile]
	\frametitle{Wijzigingen voor Integrators}

	\begin{center}\huge{Hoofdstuk 2:}\end{center}
	\begin{center}\huge{\color{typo3darkgrey}\textbf{Wijzigingen voor Integrators}}\end{center}

\end{frame}

% ------------------------------------------------------------------------------
% Feature | 85592 | Add site title configuration to sites module

\begin{frame}[fragile]
	\frametitle{Wijzigingen voor Integrators}
	\framesubtitle{Site configuratie (1)}

	\begin{itemize}

		\item De titel van de site kan nu geconfigureerd worden in
			\textbf{SITE CONFIGURATIE} $\rightarrow$ \textbf{Sites}.
		\item Dit geeft integrators de mogelijkheid een site-titel per taal in te stellen.
		\item Het veld in de sjabloonrecord is overbodig en is als \textbf{verouderd} gemarkeerd.
		\item Het veld \texttt{sys\_template.sitetitle} (database en TCA) zal worden verwijderd in TYPO3 v11.
		\item De titel van de site wordt gebruikt voor de paginatitel en ook voor toekomstige
			\texttt{schema.org} integraties.
	\end{itemize}

\end{frame}

% ------------------------------------------------------------------------------
% Feature | 89398 | Support for environment variables in imports in site configurations

\begin{frame}[fragile]
	\frametitle{Wijzigingen voor Integrators}
	\framesubtitle{Site configuratie (2)}

	% decrease font size for code listing
	\lstset{basicstyle=\tiny\ttfamily}

	\begin{itemize}

		\item Het is nu mogelijk om omgevingsvariabelen te gebruiken in imports van siteconfiguratie YAML bestanden:
\begin{lstlisting}
imports:
  -
    resource: 'Env_%env("foo")%.yaml'
\end{lstlisting}

	\end{itemize}

\end{frame}

% ------------------------------------------------------------------------------
% Feature | 88102 | Frontend Login Form Via Fluid And Extbase

\begin{frame}[fragile]
	\frametitle{Wijzigingen voor Integrators}
	\framesubtitle{Aanmelding Frontend (1)}

	\begin{itemize}

		\item TYPO3 v10.2 bevat nu een Extbase-versie van de functionaliteit voor het aanmelden in de frontend.
		\item Deze oplossing heeft een aantal voordelen:

			\begin{itemize}
				\item Eenvoudiger sjablonen aanpassen.
				\item Verstuur wachtwoord vergeten e-mails gebaseerd op HTML.
				\item Instellen en aanpassen van validators om wachtwoordrestricties te forceren.
			\end{itemize}

		\item De nieuwe Extbase plugin is standaard beschikbaar voor nieuwe installaties.
		\item Bestaande installaties blijven gebruik maken van de oude sjablonen.
		\item Integrators kunnen wisselen tussen de "oude" en de "nieuwe" plug-in met een functieschakelaar.

	\end{itemize}

\end{frame}

% ------------------------------------------------------------------------------
% Feature | 88110 | Felogin extbase password recovery

\begin{frame}[fragile]
	\frametitle{Wijzigingen voor Integrators}
	\framesubtitle{Aanmelding Frontend (2)}

	\begin{itemize}

		\item Aan de Extbase plug-in is een wachtwoord-vergeten-formulier toegevoegd.
		\item Gebruikers kunnen aangeven het wachtwoord opnieuw te willen instellen. Ze krijgen vervolgens een e-mail met een link naar het formulier.
		\item Standaard wachtwoordvalidatieregels:

			\begin{itemize}
				\item \texttt{NotEmptyValidator} - wachtwoorden mogen niet leeg zijn.
				\item \texttt{StringLengthValidator} - wachtwoorden moeten een minimale lengte hebben.
			\end{itemize}

	\end{itemize}

\end{frame}

% ------------------------------------------------------------------------------
% Feature | 88110 | Felogin extbase password recovery

\begin{frame}[fragile]
	\frametitle{Wijzigingen voor Integrators}
	\framesubtitle{Aanmelding Frontend (3)}

	% decrease font size for code listing
	\lstset{basicstyle=\tiny\ttfamily}

	\begin{itemize}
		\item De validatieregels kunnen worden aangepast.
		\item Bijvoorbeeld:
\begin{lstlisting}
plugin.tx_felogin_login {
  settings {
    passwordValidators {
      10 = TYPO3\CMS\Extbase\Validation\Validator\AlphanumericValidator
      20 {
        className = TYPO3\CMS\Extbase\Validation\Validator\StringLengthValidator
        options {
          minimum = 12
          maximum = 32
        }
      }
      30 = \Vendor\MyExtension\Validation\Validator\MyCustomPasswordPolicyValidator
    }
  }
}
\end{lstlisting}

	\end{itemize}

\end{frame}

% ------------------------------------------------------------------------------
% Feature | 89526 | FeatureFlag: betaTranslationServer

\begin{frame}[fragile]
	\frametitle{Wijzigingen voor Integrators}
	\framesubtitle{Beheerplatform vertalingen}

	\begin{itemize}

		\item \href{https://crowdin.com/}{crowdin} richt zich erop om de huidige
			\href{https://translation.typo3.org/}{Pootle}
			oplossing te vervangen als een beheerplatform voor vertalingen.

		\item Een functieschakelaar is toegevoegd in TYPO3 v10.2 die ervoor zorgt dat crowdin.com
			wordt gebruikt als de bron van vertalingen.

		\item Let op: deze functie is nog in \textbf{beta fase}.

		\item Lees meer over het
			\href{https://typo3.org/community/teams/typo3-development/initiatives/localization-with-crowdin/}{initiatief}.

	\end{itemize}

	\begin{figure}
		\includegraphics[width=0.50\linewidth]{ChangesForIntegrators/crowdin-logo.png}
	\end{figure}

\end{frame}

% ------------------------------------------------------------------------------
% Feature | 89171 | Added possibility to have multiple sitemaps

\begin{frame}[fragile]
	\frametitle{Wijzigingen voor Integrators}
	\framesubtitle{Meerdere sitemaps}

	% decrease font size for code listing
	\lstset{basicstyle=\tiny\ttfamily}

	\begin{itemize}

		\item Het is nu mogelijk om meerdere sitemaps te configureren.
		\item Voorbeeld:
\begin{lstlisting}
plugin.tx_seo {
  config {
    <sitemapType> {
      sitemaps {
        <unique key> {
          provider = TYPO3\CMS\Seo\XmlSitemap\RecordsXmlSitemapDataProvider
          config {
            ...
          }
        }
      }
    }
  }
}
\end{lstlisting}

	\end{itemize}

\end{frame}

% ------------------------------------------------------------------------------
% Feature | 86759 | Support nomodule attribute for JavaScript includes

\begin{frame}[fragile]
	\frametitle{Wijzigingen voor Integrators}
	\framesubtitle{HTML5 attribuut \texttt{nomodule}}

	% decrease font size for code listing
	\lstset{basicstyle=\tiny\ttfamily}

	\begin{itemize}
		\item Het HTML5 attribuut \texttt{nomodule} wordt nu ondersteund bij het inladen van JavaScript-bestanden via TypoScript.
\begin{lstlisting}
page.includeJSFooter.file = path/to/classic-file.js
page.includeJSFooter.file.nomodule = 1
\end{lstlisting}

		\item Dit attribuut voorkomt dat het script wordt uitgevoerd wanneer de browser modulescripts ondersteunt.

		\item Lees meer over deze standaard in de
			\href{https://html.spec.whatwg.org/multipage/scripting.html#attr-script-nomodule}{specificatie}
			en over het concept van
			\href{https://hacks.mozilla.org/2015/08/es6-in-depth-modules/}{modules}.

	\end{itemize}

% <script type="module" src="path/to/file.js"></script>
% <script nomodule src="path/to/file/classic-file.js"></script>

\end{frame}

% ------------------------------------------------------------------------------
% Feature | 87798 | Provide a way to sort form lists in ext:form

\begin{frame}[fragile]
	\frametitle{Wijzigingen voor Integrators}
	\framesubtitle{Sorteren van formulieren}

	% decrease font size for code listing
	\lstset{basicstyle=\tiny\ttfamily}

	\begin{itemize}
		\item Formulieren kunnen nu zowel oplopend als aflopend worden gesorteerd.
		\item Twee nieuwe instellingen zijn ge\"{\i}ntroduceerd: \texttt{sortByKeys} en \texttt{sortAscending}.
		\item Formulieren worden in de basis gesorteerd op naam en bestands-UID (oplopend).
		\item Om de sortering te wijzigen moet de volgende configuratie worden toegevoegd in de YAML configuratiebestanden:
\begin{lstlisting}
TYPO3:
  CMS:
    Form:
      persistenceManager:
        sortByKeys: ['name', 'fileUid']
        sortAscending: true
\end{lstlisting}

	\end{itemize}

\end{frame}

% ------------------------------------------------------------------------------
% Feature | 86918 | Add additional configuration for external link types in Linkvalidator

\begin{frame}[fragile]
	\frametitle{Wijzigingen voor Integrators}
	\framesubtitle{Link Validator (1)}

	% decrease font size for code listing
	\lstset{basicstyle=\tiny\ttfamily}

	\begin{itemize}
		\item De Link Validator ondersteunt nu extra configuratie voor externe links.
		\item Waarden voor \texttt{httpAgentUrl} en \texttt{httpAgentEmail} moeten worden opgegeven.
		\item De instellingen \texttt{headers}, \texttt{method} en \texttt{range} zijn geavanceerde instellingen.
\begin{lstlisting}
mod.linkvalidator {
  linktypesConfig {
    external {
      httpAgentName = ...
      httpAgentUrl = ...
      httpAgentEmail = ...
      headers {
      }
      method = HEAD
      range = 0-4048
    }
  }
}
\end{lstlisting}

	\end{itemize}

\end{frame}

% ------------------------------------------------------------------------------
% Feature | 84990 | Add event for checking external links in RTE

\begin{frame}[fragile]
	\frametitle{Wijzigingen voor Integrators}
	\framesubtitle{Link Validator (2)}

	\begin{itemize}
		\item De Link Validator markeert nu ook kapotte \textbf{externe} links in de RTE.
		\item Deze functie was alleen beschikbaar voor interne links.
		\item Het wordt aangeraden om de Link Validator in de Taakplanner uit te voeren om regelmatig te controleren op kapotte links.
	\end{itemize}

\end{frame}

% ------------------------------------------------------------------------------
