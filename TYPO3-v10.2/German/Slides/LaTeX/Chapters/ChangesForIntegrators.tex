% ------------------------------------------------------------------------------
% TYPO3 Version 10.2 - What's New (German Version)
%
% @license	Creative Commons BY-NC-SA 3.0
% @link		http://typo3.org/download/release-notes/whats-new/
% @language	German
% ------------------------------------------------------------------------------

\section{Änderungen für Integratoren}
\begin{frame}[fragile]
	\frametitle{Änderungen für Integratoren}

	\begin{center}\huge{Kapitel 2:}\end{center}
	\begin{center}\huge{\color{typo3darkgrey}\textbf{Änderungen für Integratoren}}\end{center}

\end{frame}

% ------------------------------------------------------------------------------
% Feature | 85592 | Add site title configuration to sites module

\begin{frame}[fragile]
	\frametitle{Änderungen für Integratoren}
	\framesubtitle{Site-Konfiguration (1)}

	\begin{itemize}

		\item Der Seitentitel kann nun in der
			\textbf{SITE CONFIGURATION} $\rightarrow$ \textbf{Sites} konfiguriert werden.
		\item Dadurch können Integratoren verschiedene Seitentitel pro Sprache angeben.
		\item Das Feld im Vorlagendatensatz ist obsolet und wurde als \textbf{veraltet} gekennzeichnet.
		\item Das Feld \texttt{sys\_template.sitetitle} (Datenbank und TCA) wird in TYPO3 v11 entfernt werden.
		\item Der Titel der Website wird sowohl für den Seitentitel als auch für 
			zukünftige \texttt{schema.org} Integrationen verwendet.
	\end{itemize}

\end{frame}

% ------------------------------------------------------------------------------
% Feature | 89398 | Support for environment variables in imports in site configurations

\begin{frame}[fragile]
	\frametitle{Änderungen für Integratoren}
	\framesubtitle{Site-Konfiguration (2)}

	% decrease font size for code listing
	\lstset{basicstyle=\tiny\ttfamily}

	\begin{itemize}

		\item Es ist nun möglich, Umgebungsvariablen beim Import von YAML-Dateien der Site-Konfiguration zu verwenden:
\begin{lstlisting}
imports:
  -
    resource: 'Env_%env("foo")%.yaml'
\end{lstlisting}

	\end{itemize}

\end{frame}

% ------------------------------------------------------------------------------
% Feature | 88102 | Frontend Login Form Via Fluid And Extbase

\begin{frame}[fragile]
	\frametitle{Änderungen für Integratoren}
	\framesubtitle{Frontend Login (1)}
	\begin{itemize}
		\item TYPO3 v10.2 enthält jetzt eine Extbase-Version der Frontend-Login-Funktionalität.
		\item Diese Lösung hat folgende Vorteile:

			\begin{itemize}
				\item Die Vorlagen können einfacher geändert werden.
				\item HTML-basierte Passwort-Wiederherstellungs-E-Mails können versendet werden.
				\item Die Anpassung und Änderung von Validatoren um Passwortbeschränkungen zu erzwingen ist nun möglich.
			\end{itemize}
		\item Das neue Extbase Plug-in ist ab sofort für Neuinstallationen verfügbar.
		\item Existierende TYPO3-Instanzen werden weiterhin die alten Vorlagen verwenden.
		\item Integratoren können mit Hilfe eines Funktionsumschalters zwischen dem "alten" und dem "neuen" Plug-in umschalten. 

	\end{itemize}

\end{frame}

% ------------------------------------------------------------------------------
% Feature | 88110 | Felogin extbase password recovery

\begin{frame}[fragile]
	\frametitle{Änderungen für Integratoren}
	\framesubtitle{Frontend Login (2)}

	\begin{itemize}

		\item Ein Formular zur Wiederherstellung des Passworts wurde als Teil des Extbase Plug-ins hinzugefügt.
		\item Benutzer können eine Passwortänderung beantragen und erhalten eine E-Mail mit einem Link, der sie zum Formular weiterleitet.
		\item Standardregeln für die Passwortbestätigung:

			\begin{itemize}
				\item \texttt{NotEmptyValidator} - Passwörter dürfen nicht leer sein.
				\item \texttt{StringLengthValidator} - Passwörter müssen eine Mindestlänge haben.
			\end{itemize}

	\end{itemize}

\end{frame}

% ------------------------------------------------------------------------------
% Feature | 88110 | Felogin extbase password recovery

\begin{frame}[fragile]
	\frametitle{Änderungen für Integratoren}
	\framesubtitle{Frontend Login (3)}

	% decrease font size for code listing
	\lstset{basicstyle=\tiny\ttfamily}

	\begin{itemize}
		\item Diese Validierungsregeln können angepasst werden.
		\item Zum Beispiel:
\begin{lstlisting}
plugin.tx_felogin_login {
  settings {
    passwordValidators {
      10 = TYPO3\CMS\Extbase\Validation\Validator\AlphanumericValidator
      20 {
        className = TYPO3\CMS\Extbase\Validation\Validator\StringLengthValidator
        options {
          minimum = 12
          maximum = 32
        }
      }
      30 = \Vendor\MyExtension\Validation\Validator\MyCustomPasswordPolicyValidator
    }
  }
}
\end{lstlisting}

	\end{itemize}

\end{frame}

% ------------------------------------------------------------------------------
% Feature | 89526 | FeatureFlag: betaTranslationServer

\begin{frame}[fragile]
	\frametitle{Änderungen für Integratoren}
	\framesubtitle{Lokalisierungs-Management Plattform}

	\begin{itemize}

		\item \href{https://crowdin.com/}{Crowdin} beabsichtigt, die bestehende
			\href{https://translation.typo3.org/}{Pootle}-Lösung 
			 als Lokalisierungs-/Übersetzungs Management-Plattform zu ersetzen.

		\item In TYPO3 v10.2 wurde ein Funktionsumschalter hinzugefügt, der crowdin.com
			als Quelle für Übersetzungen verwendet (wenn dies aktiviert ist).

		\item Bitte beachten Sie: dies ist im \textbf{Beta-Status}.

		\item Lesen Sie hier mehr über die
			\href{https://typo3.org/community/teams/typo3-development/initiatives/localization-with-crowdin/}{Initiative}.

	\end{itemize}

	\begin{figure}
		\includegraphics[width=0.50\linewidth]{ChangesForIntegrators/crowdin-logo.png}
	\end{figure}

\end{frame}

% ------------------------------------------------------------------------------
% Feature | 89171 | Added possibility to have multiple sitemaps

\begin{frame}[fragile]
	\frametitle{Änderungen für Integratoren}
	\framesubtitle{Mehrere Sitemaps}

	% decrease font size for code listing
	\lstset{basicstyle=\tiny\ttfamily}

	\begin{itemize}

		\item Es ist nun möglich, mehrere Sitemaps zu konfigurieren.
		\item Die Syntax:
\begin{lstlisting}
plugin.tx_seo {
  config {
    <sitemapType> {
      sitemaps {
        <unique key> {
          provider = TYPO3\CMS\Seo\XmlSitemap\RecordsXmlSitemapDataProvider
          config {
            ...
          }
        }
      }
    }
  }
}
\end{lstlisting}

	\end{itemize}

\end{frame}

% ------------------------------------------------------------------------------
% Feature | 86759 | Support nomodule attribute for JavaScript includes

\begin{frame}[fragile]
	\frametitle{Änderungen für Integratoren}
	\framesubtitle{Das HTML5-Attribut \texttt{nomodule}}

	% decrease font size for code listing
	\lstset{basicstyle=\tiny\ttfamily}

	\begin{itemize}
		\item Das HTML5-Attribute \texttt{nomodule} wird nun unterstützt, wenn JavaScript-Dateien in TypoScript eingebunden werden.
\begin{lstlisting}
page.includeJSFooter.file = path/to/classic-file.js
page.includeJSFooter.file.nomodule = 1
\end{lstlisting}

		\item Dieses Attribut verhindert, dass Skripte, in Browsern die Modulskripte unterstützen, ausgeführt werden.

		\item Lesen Sie mehr über die Vorgabe in der
			\href{https://html.spec.whatwg.org/multipage/scripting.html#attr-script-nomodule}{Spezifikation}
			und über das Konzept
			\href{https://hacks.mozilla.org/2015/08/es6-in-depth-modules/}{Module}.

	\end{itemize}

% <script type="module" src="path/to/file.js"></script>
% <script nomodule src="path/to/file/classic-file.js"></script>

\end{frame}

% ------------------------------------------------------------------------------
% Feature | 87798 | Provide a way to sort form lists in ext:form

\begin{frame}[fragile]
	\frametitle{Änderungen für Integratoren}
	\framesubtitle{Formulare Sortieren}

	% decrease font size for code listing
	\lstset{basicstyle=\tiny\ttfamily}

	\begin{itemize}
		\item Formulare können jetzt entweder aufsteigend oder absteigend sortiert werden.
		\item Zwei neue Einstellungen wurden eingefügt: \texttt{sortByKeys} and \texttt{sortAscending}.
		\item Formulare werden zunächst nach ihrem Namen und ihrer Datei-UID (aufsteigend) sortiert.
		\item Um die Sortierung zu ändern, muss folgende Konfiguration in der YAML-Konfigurationsdatei hinzugefügt werden:
\begin{lstlisting}
TYPO3:
  CMS:
    Form:
      persistenceManager:
        sortByKeys: ['name', 'fileUid']
        sortAscending: true
\end{lstlisting}

	\end{itemize}

\end{frame}

% ------------------------------------------------------------------------------
% Feature | 86918 | Add additional configuration for external link types in Linkvalidator

\begin{frame}[fragile]
	\frametitle{Änderungen für Integratoren}
	\framesubtitle{Link-Validator (1)}

	% decrease font size for code listing
	\lstset{basicstyle=\tiny\ttfamily}

	\begin{itemize}
		\item Der Link-Validator unterstützt nun eine zusätzliche Konfiguration für externe Links.
		\item Die Werte für \texttt{httpAgentUrl} und \texttt{httpAgentEmail} sollten angegeben werden.
		\item Erweiterte Einstellungen sind zum Beispiel \texttt{headers}, \texttt{method} und \texttt{range}.
\begin{lstlisting}
mod.linkvalidator {
  linktypesConfig {
    external {
      httpAgentName = ...
      httpAgentUrl = ...
      httpAgentEmail = ...
      headers {
      }
      method = HEAD
      range = 0-4048
    }
  }
}
\end{lstlisting}

	\end{itemize}

\end{frame}

% ------------------------------------------------------------------------------
% Feature | 84990 | Add event for checking external links in RTE

\begin{frame}[fragile]
	\frametitle{Änderungen für Integratoren}
	\framesubtitle{Link-Validator (2)}

	\begin{itemize}
		\item Link-Validator markiert nun auch defekte \textbf{externe} Links im RTE.
		\item Diese Funktion war nur für interne Links verfügbar.
		\item Es ist empfohlen, den Link-Validator als Scheduler-Task auszuführen, um regelmäßig nach kaputten Links zu suchen.
	\end{itemize}

\end{frame}

% ------------------------------------------------------------------------------
