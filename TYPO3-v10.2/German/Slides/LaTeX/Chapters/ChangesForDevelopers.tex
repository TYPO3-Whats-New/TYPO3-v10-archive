% ------------------------------------------------------------------------------
% TYPO3 Version 10.2 - What's New (German Version)
%
% @license	Creative Commons BY-NC-SA 3.0
% @link		http://typo3.org/download/release-notes/whats-new/
% @language	German
% ------------------------------------------------------------------------------

\section{Änderungen für Entwickler}
\begin{frame}[fragile]
	\frametitle{Änderungen für Entwickler}

	\begin{center}\huge{Kapitel 3:}\end{center}
	\begin{center}\huge{\color{typo3darkgrey}\textbf{Änderungen für Entwickler}}\end{center}

\end{frame}

% ------------------------------------------------------------------------------
% Feature | 88950 | Add storeSession argument to Widget ViewHelpers

\begin{frame}[fragile]
	\frametitle{Änderungen für Entwickler}
	\framesubtitle{Widget ViewHelper}

	% decrease font size for code listing
	\lstset{basicstyle=\smaller\ttfamily}

	\begin{itemize}
		\item Widget ViewHelper setzten unter bestimmten Umständen ein Session-Cookie im Frontend.
		\item Da dies nicht immer erwünscht ist (z.B. auf Grund von DSGVO), kann dies nun kontrolliert werden.
		\item Es wurde einw Boolean \texttt{storeSession} eingeführt, mit der Entwickler diese Funktion aktivieren/deaktivieren können.
\begin{lstlisting}
<f:widget.autocomplete
  for="name"
  objects="{posts}"
  searchProperty="author"
  storeSession="false" />
\end{lstlisting}

	\end{itemize}

\end{frame}

% ------------------------------------------------------------------------------
% Feature | 89577 | New PSR-14 based events for File Abstraction Layer
% Deprecation | 89577 | FAL SignalSlot handling migrated to PSR-14 events

\begin{frame}[fragile]
	\frametitle{Änderungen für Entwickler}
	\framesubtitle{PSR-14 Events in FAL}

	% decrease font size for code listing
	\lstset{basicstyle=\tiny\ttfamily}

	\begin{itemize}
		\item Ungefähr 40 neue
			\href{https://www.php-fig.org/psr/psr-14/}{PSR-14}
			basierte Events wurden in den File Abstraction Layer (FAL) eingefügt.
		\item Sie ersetzen bestehende Extbase Signal/Slots.
		\item Die Verwendung der Signale funktioniert weiterhin (ohne eine Veraltet-Meldung zu erzeugen!). Allerdings werden die Signale im FAL in TYPO3 v11 wahrscheinlich entfernt werden.
		\item Autoren von Erweiterungen wird empfohlen, ihren Code zu migrieren und Events zu verwenden.
		\item Sehen Sie sich die neuen PHP-Klassen, um mehr über PSR-14 zu erfahren.
	\end{itemize}

\end{frame}


% ------------------------------------------------------------------------------
% Deprecation | 89733 | Signal Slots in Core Extension migrated to PSR-14 events

\begin{frame}[fragile]
	\frametitle{Änderungen für Entwickler}
	\framesubtitle{PSR-14 Events im TYPO3-Kern}

	\begin{itemize}
		\item Eine Reihe neuer PSR-14 Events ersetzen Signal/Slots im TYPO3-Kern:
			\newline

			\begin{itemize}\tiny
				\item \texttt{TYPO3\textbackslash
					CMS\textbackslash
					Core\textbackslash
					Imaging\textbackslash
					Event\textbackslash
					ModifyIconForResourcePropertiesEvent}
					\newline
				\item \texttt{TYPO3\textbackslash
					CMS\textbackslash
					Core\textbackslash
					DataHandling\textbackslash
					Event\textbackslash
					IsTableExcludedFromReferenceIndexEvent}
					\newline
				\item \texttt{TYPO3\textbackslash
					CMS\textbackslash
					Core\textbackslash
					DataHandling\textbackslash
					Event\textbackslash
					AppendLinkHandlerElementsEvent}
					\newline
				\item \texttt{TYPO3\textbackslash
					CMS\textbackslash
					Core\textbackslash
					Configuration\textbackslash
					Event\textbackslash
					AfterTcaCompilationEvent}
					\newline
				\item \texttt{TYPO3\textbackslash
					CMS\textbackslash
					Core\textbackslash
					Database\textbackslash
					Event\textbackslash
					AlterTableDefinitionStatementsEvent}
					\newline
				\item \texttt{TYPO3\textbackslash
					CMS\textbackslash
					Core\textbackslash
					Tree\textbackslash
					Event\textbackslash
					ModifyTreeDataEvent}
					\newline
				\item \texttt{TYPO3\textbackslash
					CMS\textbackslash
					Backend\textbackslash
					Backend\textbackslash
					Event\textbackslash
					SystemInformationToolbarCollectorEvent}
					\newline
			\end{itemize}

	\end{itemize}

\end{frame}

% ------------------------------------------------------------------------------
% Deprecation | 89718 | Legacy PageTSconfig parsing lowlevel API

\begin{frame}[fragile]
	\frametitle{Änderungen für Entwickler}
	\framesubtitle{Page TSconfig}

	% decrease font size for code listing
	\lstset{basicstyle=\tiny\ttfamily}

	\begin{itemize}
		\item Es wurden zwei neue PHP-Klassen zum Laden und Parsen von PageTSconfig eingeführt:
			\begin{itemize}\smaller
				\item \texttt{TYPO3\textbackslash
					CMS\textbackslash
					Core\textbackslash
					Configuration\textbackslash
					Loader\textbackslash
					PageTsConfigLoader}
				\item \texttt{TYPO3\textbackslash
					CMS\textbackslash
					Core\textbackslash
					Configuration\textbackslash
					Parser\textbackslash
					PageTsConfigParser}
			\end{itemize}

		\item Zum Beispiel:
\begin{lstlisting}
// Fetch all available PageTS of a page/rootline:
$loader = GeneralUtility::makeInstance(PageTsConfigLoader::class);
$tsConfigString = $loader->load($rootLine);

// Parse the string and apply conditions:
$parser = GeneralUtility::makeInstance(
  PageTsConfigParser::class, $typoScriptParser, $hashCache
);

$pagesTSconfig = $parser->parse($tsConfigString, $conditionMatcher);
\end{lstlisting}

	\end{itemize}

\end{frame}

% ------------------------------------------------------------------------------
% Important | 87518 | Use prepared statements for pdo_mysql per default

\begin{frame}[fragile]
	\frametitle{Änderungen für Entwickler}
	\framesubtitle{Prepared Statements}

	% decrease font size for code listing
	\lstset{basicstyle=\tiny\ttfamily}

	\begin{itemize}
		\item Der \texttt{pdo\_mysql} Treiber verwendet nun standardmäßig vorbereitete Anweisungen.
		\item In TYPO3 < v10.2 werden \textit{emulierte vorbereitete Anweisungen} verwendet.
			Das bedeutet, dass alle zurückgegebenen Werte einer Abfrage Strings waren.
		\item Dieses Verhalten hat sich geändert und es werden vorbereitete Anweisungen verwendet, die native Datentypen zurückgeben.
		\item Zum Beispiel: Werte einer Spalte, die als Integer definiert sind, werden in PHP als \texttt{int} zurückgegeben.
		\item Diese Funktion kann durch Setzen der Option
			\texttt{PDO::ATTR\_EMULATE\_PREPARES} deaktiviert werden.

	\end{itemize}

\end{frame}

% ------------------------------------------------------------------------------
% Feature | 86967 | Allow fetching uid of a LazyLoadingProxy without loading the object first

\begin{frame}[fragile]
	\frametitle{Änderungen für Entwickler}
	\framesubtitle{Lazy Loading Proxy}

	% decrease font size for code listing
	\lstset{basicstyle=\tiny\ttfamily}

	\begin{itemize}
		\item Die Methode \texttt{getUid()} wurde der Klasse\newline
			\texttt{TYPO3\textbackslash
				CMS\textbackslash
				Extbase\textbackslash
				Persistence\textbackslash
				Generic\textbackslash
				LazyLoadingProxy} hinzugefügt.
		\item Dies ermöglicht es Entwicklern, die UID des betreffenden Objekts zu erhalten, ohne das Objekt von der Datenbank zu holen.

	\end{itemize}

\end{frame}

% ------------------------------------------------------------------------------
% Feature | 87380 | Introduce SiteLanguageAwareInterface to denote site language awareness

\begin{frame}[fragile]
	\frametitle{Änderungen für Entwickler}
	\framesubtitle{Site Language Awareness}

	% decrease font size for code listing
	\lstset{basicstyle=\tiny\ttfamily}

% A SiteLanguageAwareInterface with the methods setSiteLanguage(Entity\SiteLanguage $siteLanguage) and getSiteLanguage() has been introduced.
% The interface can be used to denote a class as aware of the site language.

	\begin{itemize}
		\item Ein \texttt{SiteLanguageAwareInterface} wurde eingefügt.
		\item Die Schnittstelle kann verwendet werden, um zu kennzeichnen, dass die Klasse die Webseitensprache berücksichtigt.
		\item Routing-Aspekte, die die Sprache der Website berücksichtigen, verwenden nun auch das
			\texttt{SiteLanguageAwareInterface}
			nebem dem \texttt{SiteLanguageAwareTrait}.
	\end{itemize}

\end{frame}

% ------------------------------------------------------------------------------
% Important | 89645 | Removed systemLog options

\begin{frame}[fragile]
	\frametitle{Änderungen für Entwickler}
	\framesubtitle{API Systemprotokoll}

	% decrease font size for code listing
	\lstset{basicstyle=\tiny\ttfamily}

	\begin{itemize}
		\item Die folgenden Optionen wurden aus der Standardkonfiguration von TYPO3 entfernt:

			\begin{itemize}\smaller
				\item \texttt{\$GLOBALS['TYPO3\_CONF\_VARS']['SYS']['systemLog']}
				\item \texttt{\$GLOBALS['TYPO3\_CONF\_VARS']['SYS']['systemLogLevel']}
			\end{itemize}\normalsize

		\item Autoren von Erweiterungen wird empfohlen, die Logging-API zu verwenden und die systemLog-Optionen zu entfernen.
	\end{itemize}

\end{frame}

% ------------------------------------------------------------------------------
% Feature | 89603 | Introduce native pagination for lists

\begin{frame}[fragile]
	\frametitle{Änderungen für Entwickler}
	\framesubtitle{Native Listen-Paginierung}

	% decrease font size for code listing
	\lstset{basicstyle=\tiny\ttfamily}

	\begin{itemize}
		\item Native Unterstützung für die Paginierung von Listen wie Arrays oder QueryResults von Extbase wurde eingeführt.
		\item Das \texttt{PaginatorInterface} definiert einen grundlegenden Methodensatz.
		\item Die Klasse \texttt{AbstractPaginator} enthält die Logik der Hauptseitenumbrüche.
		\item Dies ermöglicht es Entwicklern, allerlei Seitenumbrüchen zu implementieren.
\begin{lstlisting}
use TYPO3\CMS\Core\Pagination\ArrayPaginator;

$items = ['apple', 'banana', 'strawberry', 'raspberry', 'ananas'];
$currentPageNumber = 3;
$itemsPerPage = 2;

$paginator = new ArrayPaginator($itemsToBePaginated, $currentPageNumber, $itemsPerPage);
$paginator->getNumberOfPages(); // returns 3
$paginator->getCurrentPageNumber(); // returns 3
$paginator->getKeyOfFirstPaginatedItem(); // returns 5
$paginator->getKeyOfLastPaginatedItem(); // returns 5
\end{lstlisting}

	\end{itemize}

\end{frame}

% ------------------------------------------------------------------------------
% Deprecation | 89579 | ServiceChains require an array for excluded Service keys

\begin{frame}[fragile]
	\frametitle{Änderungen für Entwickler}
	\framesubtitle{Service-API}

	\begin{itemize}
		\item Das Argument \texttt{\$excludeServiceKeys} wird verwendet, um bestimmte Services bei der Verwendung einer Kette zu überspringen.
		\item Das Argument wurde in TYPO3 v10.2 von einer kommagetrennten Liste in ein Array geändert.
		\item Diese Änderung betrifft die Service API innerhalb der folgenden Komponenten:

			\begin{itemize}
				\item \texttt{GeneralUtility::makeInstanceService()}
				\item \texttt{ExtensionManagementUtility::findService()}
			\end{itemize}

		\item Das Übergeben einer kommagetrennten Liste funktioniert immer noch, wurde jedoch als \textbf{veraltet} markiert.

	\end{itemize}

\end{frame}

% ------------------------------------------------------------------------------
