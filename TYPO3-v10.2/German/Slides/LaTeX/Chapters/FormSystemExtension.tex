% ------------------------------------------------------------------------------
% TYPO3 Version 10.2 - What's New (German Version)
%
% @license	Creative Commons BY-NC-SA 3.0
% @link		http://typo3.org/download/release-notes/whats-new/
% @language	German
% ------------------------------------------------------------------------------

\section{Die Systemerweiterung "Form"}
\begin{frame}[fragile]
	\frametitle{Die Systemerweiterung "Form"}

	\begin{center}\huge{Kapitel 4:}\end{center}
	\begin{center}\huge{\color{typo3darkgrey}\textbf{Die Systemerweiterung "Form"}}\end{center}

\end{frame}

% ------------------------------------------------------------------------------
% System Extension "Form" - Summary

\begin{frame}[fragile]
	\frametitle{Die Systemerweiterung "Form"}
	\framesubtitle{Zusammenfassung}

	\small
		An der Systemerweiterung \textbf{"Form"} wurden mehrere Änderungen vorgenommen.
		Diese Änderungen betreffen sowohl Redakteure, als auch Integratoren und Entwickler.

		\vspace{0.2cm}

		Einige der Änderungen basieren sich auf Konzepten der TYPO3-Initiative Woche (T3INIT19).

	\normalsize

\end{frame}

% ------------------------------------------------------------------------------
% Important | 84221 | Restructuring of form setup

\begin{frame}[fragile]
	\frametitle{Die Systemerweiterung "Form"}
	\framesubtitle{Formulareinrichtung}

	\begin{itemize}
		\item Zuvor wurden drei Dateien verwendet: \texttt{BaseSetup.yaml}, \texttt{FormEditorSetup.yaml}, und \texttt{FormEngineSetup.yaml}.
		\item Dies wurde konsolidiert und in einer Datei zusammengefasst: \texttt{FormSetup.yaml}.
		\item Diese Datei enthält das Basis-Setup, inklusive dem Import der Konfiguration für Validatoren, Formularelemente und Finisher.
		\item Alle bisher verwendeten Vererbungen wurden aufgelöst, um die gesamte Konfiguration besser zu verstehen.
	\end{itemize}

\end{frame}

% ------------------------------------------------------------------------------
% Feature | 84203 | Unify form setup YAML loading

\begin{frame}[fragile]
	\frametitle{Die Systemerweiterung "Form"}
	\framesubtitle{YAML- Dateien}

	\begin{itemize}
		\item Die YAML-Dateien verwenden nun TYPO3-Core YAML Datei-Loader.
		\item Dadurch wurden folgende Funktionen ermöglicht:

			\begin{itemize}
				\item Der Import anderer YAML-Dateien über die Direktive \texttt{imports}.
				\item Das Ersetzen von \texttt{\%placeholders\%}.
			\end{itemize}

	\end{itemize}

\end{frame}

% ------------------------------------------------------------------------------
% Feature | 79445 | Add Multistep Wizard

\begin{frame}[fragile]
	\frametitle{Die Systemerweiterung "Form"}
	\framesubtitle{Der Multi-step Wizard}

	\begin{itemize}
		\item Ein neues JavaScript-Modul \texttt{MultiStepWizard} wurde eingeführt, das die folgenden Funktionen bietet:

			\begin{itemize}
				\item Navigation zu den vorherigen Schritten.
				\item Die Schritte unterstützen beschreibende Bezeichnungen wie "Start" oder "Ende" statt der numerischen Anzeige "Schritt x von y".
				\item Optimierte Konfigurationsstruktur.
			\end{itemize}

		\item Siehe \href{https://docs.typo3.org/c/typo3/cms-core/master/en-us/Changelog/10.2/Feature-79445-AddMultistepWizard.html}{das Änderungsprotokoll}
			für JavaScript-basierte Beispiele.

		\item Diese neue Funktion verbessert die Benutzererfahrung erheblich: Backend-Benutzer werden einen verbesserten Formularerstellungs-Assistenten bemerken.

	\end{itemize}

\end{frame}

% ------------------------------------------------------------------------------
% Feature | 89747 | Custom tables with record browser in forms

\begin{frame}[fragile]
	\frametitle{Die Systemerweiterung "Form"}
	\framesubtitle{Datensatzbrowser}

	% decrease font size for code listing
	\lstset{basicstyle=\tiny\ttfamily}

	\begin{itemize}
		\item Der Datensatzbrowser kann nun konfiguriert werden, um benutzerdefinierte Tabellen zu verwenden:
\begin{lstlisting}
TYPO3:
  CMS:
    Form:
      prototypes:
        standard:
          formElementsDefinition:
            MyCustomElement:
              formEditor:
                editors:
                  # ...
                  300:
                    identifier: myRecord
                    # ...
                    browsableType: tx_myext_mytable
                    propertyPath: properties.myRecordUid
                    # ...
\end{lstlisting}

	\end{itemize}

\end{frame}

% ------------------------------------------------------------------------------
% Feature | 89746 | Custom icon for record browser button in forms

\begin{frame}[fragile]
	\frametitle{Die Systemerweiterung "Form"}
	\framesubtitle{Datensatzbrowser}

	% decrease font size for code listing
	\lstset{basicstyle=\tiny\ttfamily}

	\begin{itemize}
		\item Die Schaltflächensymbole des Datensatzbrowsers sind auch konfigurierbar:
\begin{lstlisting}
TYPO3:
  CMS:
    Form:
      prototypes:
        standard:
          formElementsDefinition:
            MyCustomElement:
              formEditor:
                editors:
                  # ...
                  300:
                    identifier: contentElement
                    # ...
                    browsableType: tt_content
                    iconIdentifier: mimetypes-x-content-text
                    propertyPath: properties.contentElementUid
                    # ...
\end{lstlisting}

	\end{itemize}

\end{frame}

% ------------------------------------------------------------------------------
% Feature | 84713 | Access single values in form templates

\begin{frame}[fragile]
	\frametitle{Die Systemerweiterung "Form"}
	\framesubtitle{Datensatzbrowser}

	% decrease font size for code listing
	\lstset{basicstyle=\tiny\ttfamily}

	\begin{itemize}
		\item Ein neuer \textit{RenderFormValue-ViewHelper} ermöglicht Integratoren/Entwicklern den Zugriff auf einzelne Formularwerte in Vorlagen:
\begin{lstlisting}
<p>
  The following message was just sent by
  <formvh:renderFormValue renderable="{page.rootForm.elements.name}" as="formValue">
    {formValue.processedValue}
  </formvh:renderFormValue>:
</p>

<blockquote>
  <formvh:renderFormValue renderable="{page.rootForm.elements.message}" as="formValue">
    {formValue.processedValue}
  </formvh:renderFormValue>
</blockquote>
\end{lstlisting}

	\end{itemize}

\end{frame}

% ------------------------------------------------------------------------------
% Feature | 82706 | Render fieldset labels in form templates

\begin{frame}[fragile]
	\frametitle{Die Systemerweiterung "Form"}
	\framesubtitle{Fieldset-Labels}

	\begin{itemize}
		\item Das Element \texttt{Fieldset} ist nun in Vorlagen zugänglich.
		\item Standardmäßig betrifft dies das Formularelement \textbf{SummaryPage} sowie die Finisher \textbf{EmailToReceiver} und \textbf{EmailToSender}.
		\item Typischer Anwendungsfall:\newline
			\small
				Ein Formular mit einer Liefer- und einer Rechnungsadresse. Beide Abschnitte könnten ein Feld mit dem gleichen Namen haben, z.B. \texttt{street}.
				Es ist nun möglich, zwischen beiden Feldern durch die Verwendung von Fieldset-Labels zu unterscheiden.
			\normalsize

	\end{itemize}

\end{frame}

% ------------------------------------------------------------------------------
% Deprecation | 88238 | Allowed MIME types of FileUpload and ImageUpload

\begin{frame}[fragile]
	\frametitle{Die Systemerweiterung "Form"}
	\framesubtitle{Datei-Uploads}

	\begin{itemize}
		\item Vordefinierte \texttt{allowedMimeTypes} der folgenden Formularelemente wurden als \textbf{veraltet} gekennzeichnet:

			\begin{itemize}
				\item \texttt{FileUpload}
				\item \texttt{ImageUpload}
			\end{itemize}

		\item Alle gültigen MIME-Typen müssen in der Formulardefinition explizit aufgeführt werden \newline
			\smaller
				(vordefinierte MIME-Typen werden in TYPO3 v11 entfernt)
			\normalsize

		\item Integratoren können das neue Verhalten in TYPO3 v10 bereits mit einem Feature-Toggle aktivieren.

	\end{itemize}

\end{frame}

% ------------------------------------------------------------------------------
% Deprecation | 89742 | Form mixins

\begin{frame}[fragile]
	\frametitle{Die Systemerweiterung "Form"}
	\framesubtitle{Formular Mixins}

	\begin{itemize}
		\item Mixins wurden als \textbf{veraltet} markiert und sollten nicht mehr verwendet werden.
		\item Dies betrifft alle Vererbungen von \texttt{TYPO3.CMS.Form.mixins.*}.
		\item Migrationsmöglichkeiten:

			\begin{itemize}
				\item Binden Sie die wesentlichen Teile aus \texttt{TYPO3.CMS.Form.mixins.*} ein, oder
				\item migrieren Sie diese in benutzerdefinierte Mixins.
			\end{itemize}

	\end{itemize}

\end{frame}

% ------------------------------------------------------------------------------
