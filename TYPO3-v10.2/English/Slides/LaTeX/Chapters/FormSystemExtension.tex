% ------------------------------------------------------------------------------
% TYPO3 Version 10.2 - What's New (English Version)
%
% @author	Michael Schams <schams.net>
% @license	Creative Commons BY-NC-SA 3.0
% @link		http://typo3.org/download/release-notes/whats-new/
% @language	English
% ------------------------------------------------------------------------------

\section{System Extension "Form"}
\begin{frame}[fragile]
	\frametitle{System Extension "Form"}

	\begin{center}\huge{Chapter 4:}\end{center}
	\begin{center}\huge{\color{typo3darkgrey}\textbf{System Extension "Form"}}\end{center}

\end{frame}

% ------------------------------------------------------------------------------
% System Extension "Form" - Summary

\begin{frame}[fragile]
	\frametitle{System Extension "Form"}
	\framesubtitle{Summary}

	\small
		Several changes have been made to the system extension \textbf{"Form"}.
		These changes affect editors, integrators as well as developers.

		\vspace{0.2cm}

		Some of the changes are based on concepts developed during the TYPO3 Initiative Week (T3INIT19).

	\normalsize

\end{frame}

% ------------------------------------------------------------------------------
% Important | 84221 | Restructuring of form setup

\begin{frame}[fragile]
	\frametitle{System Extension "Form"}
	\framesubtitle{Form Setup}

	\begin{itemize}
		\item Three files were used previously: \texttt{BaseSetup.yaml}, \texttt{FormEditorSetup.yaml}, and \texttt{FormEngineSetup.yaml}.
		\item This has been streamlined and consolidated into one file now: \texttt{FormSetup.yaml}.
		\item This file contains the basic setup including imports of the configuration for validators, form elements and finishers.
		\item All previously used inheritances and mixins have been resolved which makes it very easy to understand the entire configuration.
	\end{itemize}

\end{frame}

% ------------------------------------------------------------------------------
% Feature | 84203 | Unify form setup YAML loading

\begin{frame}[fragile]
	\frametitle{System Extension "Form"}
	\framesubtitle{YAML Files}

	\begin{itemize}
		\item YAML files now use the TYPO3 core YAML file loader.
		\item This enabled features such as:

			\begin{itemize}
				\item Import of other YAML files via \texttt{imports} directive.
				\item Replacement of \texttt{\%placeholders\%}.
			\end{itemize}

	\end{itemize}

\end{frame}

% ------------------------------------------------------------------------------
% Feature | 79445 | Add Multistep Wizard

\begin{frame}[fragile]
	\frametitle{System Extension "Form"}
	\framesubtitle{Multi-step Wizard}

	\begin{itemize}
		\item A new JavaScript module \texttt{MultiStepWizard} has been introduced,
			that adds the following features:

			\begin{itemize}
				\item Navigation to previous steps.
				\item Steps support descriptive labels such as "Start" or "Finish", rather than the numerical indicator "Step x of y".
				\item Optimized configuration structure.
			\end{itemize}

		\item See \href{https://docs.typo3.org/c/typo3/cms-core/master/en-us/Changelog/10.2/Feature-79445-AddMultistepWizard.html}{ChangeLog}
			for JavaScript code examples.

		\item This new features improves the user experience significantly: backend users will notice an enhanced form creation wizard.

	\end{itemize}

\end{frame}

% ------------------------------------------------------------------------------
% Feature | 89747 | Custom tables with record browser in forms

\begin{frame}[fragile]
	\frametitle{System Extension "Form"}
	\framesubtitle{Record Browser}

	% decrease font size for code listing
	\lstset{basicstyle=\tiny\ttfamily}

	\begin{itemize}
		\item The record browser can now be configured to use custom tables:
\begin{lstlisting}
TYPO3:
  CMS:
    Form:
      prototypes:
        standard:
          formElementsDefinition:
            MyCustomElement:
              formEditor:
                editors:
                  # ...
                  300:
                    identifier: myRecord
                    # ...
                    browsableType: tx_myext_mytable
                    propertyPath: properties.myRecordUid
                    # ...
\end{lstlisting}

	\end{itemize}

\end{frame}

% ------------------------------------------------------------------------------
% Feature | 89746 | Custom icon for record browser button in forms

\begin{frame}[fragile]
	\frametitle{System Extension "Form"}
	\framesubtitle{Record Browser}

	% decrease font size for code listing
	\lstset{basicstyle=\tiny\ttfamily}

	\begin{itemize}
		\item The button icon of the record browser are now configurable:
\begin{lstlisting}
TYPO3:
  CMS:
    Form:
      prototypes:
        standard:
          formElementsDefinition:
            MyCustomElement:
              formEditor:
                editors:
                  # ...
                  300:
                    identifier: contentElement
                    # ...
                    browsableType: tt_content
                    iconIdentifier: mimetypes-x-content-text
                    propertyPath: properties.contentElementUid
                    # ...
\end{lstlisting}

	\end{itemize}

\end{frame}

% ------------------------------------------------------------------------------
% Feature | 84713 | Access single values in form templates

\begin{frame}[fragile]
	\frametitle{System Extension "Form"}
	\framesubtitle{Record Browser}

	% decrease font size for code listing
	\lstset{basicstyle=\tiny\ttfamily}

	\begin{itemize}
		\item A new \textit{RenderFormValue-ViewHelper} lets integrators/developers access single form values in templates:
\begin{lstlisting}
<p>
  The following message was just sent by
  <formvh:renderFormValue renderable="{page.rootForm.elements.name}" as="formValue">
    {formValue.processedValue}
  </formvh:renderFormValue>:
</p>

<blockquote>
  <formvh:renderFormValue renderable="{page.rootForm.elements.message}" as="formValue">
    {formValue.processedValue}
  </formvh:renderFormValue>
</blockquote>
\end{lstlisting}

	\end{itemize}

\end{frame}

% ------------------------------------------------------------------------------
% Feature | 82706 | Render fieldset labels in form templates

\begin{frame}[fragile]
	\frametitle{System Extension "Form"}
	\framesubtitle{Fieldset Labels}

	\begin{itemize}
		\item The section element \texttt{Fieldset} is now accessible in templates.
		\item By default this affects the \textbf{SummaryPage} form element as well as the \textbf{EmailToReceiver} and \textbf{EmailToSender} finishers.
		\item Typical use-case:\newline
			\small
				A form with a shipping and a billing address. Both sections could have a field with the same name, e.g. \texttt{street}.
				It is now possible to distinguish between both fields by using fieldset labels.
			\normalsize

	\end{itemize}

\end{frame}

% ------------------------------------------------------------------------------
% Deprecation | 88238 | Allowed MIME types of FileUpload and ImageUpload

\begin{frame}[fragile]
	\frametitle{System Extension "Form"}
	\framesubtitle{File Uploads}

	\begin{itemize}
		\item Predefined \texttt{allowedMimeTypes} of the following form elements have been marked \textbf{deprecated}:

			\begin{itemize}
				\item \texttt{FileUpload}
				\item \texttt{ImageUpload}
			\end{itemize}

		\item All valid MIME types must be explicitly listed in the form definition now\newline
			\smaller
				(predefined MIME types will be removed in TYPO3 v11)
			\normalsize

		\item Integrators can already activate the new behaviour in TYPO3 v10 by using a feature toggle.

	\end{itemize}

\end{frame}

% ------------------------------------------------------------------------------
% Deprecation | 89742 | Form mixins

\begin{frame}[fragile]
	\frametitle{System Extension "Form"}
	\framesubtitle{Form Mixins}

	\begin{itemize}
		\item Mixins have been marked as \textbf{deprecated} and should not be used anymore.
		\item This affects all inheritances from \texttt{TYPO3.CMS.Form.mixins.*}.
		\item Migration options:

			\begin{itemize}
				\item Embed the essential parts from \texttt{TYPO3.CMS.Form.mixins.*}, or
				\item migrate them to custom mixins.
			\end{itemize}

	\end{itemize}

\end{frame}

% ------------------------------------------------------------------------------
