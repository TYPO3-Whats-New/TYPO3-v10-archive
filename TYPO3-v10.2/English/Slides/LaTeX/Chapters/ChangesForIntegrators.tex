% ------------------------------------------------------------------------------
% TYPO3 Version 10.2 - What's New (English Version)
%
% @author	Michael Schams <schams.net>
% @license	Creative Commons BY-NC-SA 3.0
% @link		http://typo3.org/download/release-notes/whats-new/
% @language	English
% ------------------------------------------------------------------------------

\section{Changes for Integrators}
\begin{frame}[fragile]
	\frametitle{Changes for Integrators}

	\begin{center}\huge{Chapter 2:}\end{center}
	\begin{center}\huge{\color{typo3darkgrey}\textbf{Changes for Integrators}}\end{center}

\end{frame}

% ------------------------------------------------------------------------------
% Feature | 85592 | Add site title configuration to sites module

\begin{frame}[fragile]
	\frametitle{Changes for Integrators}
	\framesubtitle{Site Configuration (1)}

	\begin{itemize}

		\item The site title can now be configured in
			\textbf{SITE CONFIGURATION} $\rightarrow$ \textbf{Sites}.
		\item This lets integrators specify different site titles per language.
		\item The field in the template record is obsolete and has been marked as \textbf{deprecated}.
		\item The field \texttt{sys\_template.sitetitle} (database and TCA) will be removed in TYPO3 v11.
		\item The site title is used for the page title as well as for
			future \texttt{schema.org} integrations.
	\end{itemize}

\end{frame}

% ------------------------------------------------------------------------------
% Feature | 89398 | Support for environment variables in imports in site configurations

\begin{frame}[fragile]
	\frametitle{Changes for Integrators}
	\framesubtitle{Site Configuration (2)}

	% decrease font size for code listing
	\lstset{basicstyle=\tiny\ttfamily}

	\begin{itemize}

		\item It is now possible to use environment variables in imports of site configuration YAML files:
\begin{lstlisting}
imports:
  -
    resource: 'Env_%env("foo")%.yaml'
\end{lstlisting}

	\end{itemize}

\end{frame}

% ------------------------------------------------------------------------------
% Feature | 88102 | Frontend Login Form Via Fluid And Extbase

\begin{frame}[fragile]
	\frametitle{Changes for Integrators}
	\framesubtitle{Frontend Login (1)}

	\begin{itemize}

		\item TYPO3 v10.2 now includes an Extbase-version of the frontend login functionality.
		\item This solution has a few advantages:

			\begin{itemize}
				\item Modify the templates more easily.
				\item Send out HTML-based password recovery emails.
				\item Adjust and modify validators to enforce password restrictions.
			\end{itemize}

		\item The new Extbase plugin is available out-of-the-box for new installations.
		\item Existing TYPO3 instances will continue to use the old templates.
		\item Integrators can switch between the "old" and the "new" plugin by using a feature toggle.

	\end{itemize}

\end{frame}

% ------------------------------------------------------------------------------
% Feature | 88110 | Felogin extbase password recovery

\begin{frame}[fragile]
	\frametitle{Changes for Integrators}
	\framesubtitle{Frontend Login (2)}

	\begin{itemize}

		\item A password recovery form has been added as part of the Extbase plugin.
		\item Users can request a password change and will receive an email with a link which redirects them to the form.
		\item Default password validation rules:

			\begin{itemize}
				\item \texttt{NotEmptyValidator} - passwords cannot be empty.
				\item \texttt{StringLengthValidator} - passwords must have a minimum length.
			\end{itemize}

	\end{itemize}

\end{frame}

% ------------------------------------------------------------------------------
% Feature | 88110 | Felogin extbase password recovery

\begin{frame}[fragile]
	\frametitle{Changes for Integrators}
	\framesubtitle{Frontend Login (3)}

	% decrease font size for code listing
	\lstset{basicstyle=\tiny\ttfamily}

	\begin{itemize}
		\item These validation rules can be customized.
		\item For example:
\begin{lstlisting}
plugin.tx_felogin_login {
  settings {
    passwordValidators {
      10 = TYPO3\CMS\Extbase\Validation\Validator\AlphanumericValidator
      20 {
        className = TYPO3\CMS\Extbase\Validation\Validator\StringLengthValidator
        options {
          minimum = 12
          maximum = 32
        }
      }
      30 = \Vendor\MyExtension\Validation\Validator\MyCustomPasswordPolicyValidator
    }
  }
}
\end{lstlisting}

	\end{itemize}

\end{frame}

% ------------------------------------------------------------------------------
% Feature | 89526 | FeatureFlag: betaTranslationServer

\begin{frame}[fragile]
	\frametitle{Changes for Integrators}
	\framesubtitle{Localization Management Platform}

	\begin{itemize}

		\item \href{https://crowdin.com/}{crowdin} aims to replace the existing
			\href{https://translation.typo3.org/}{Pootle}
			solution as a localization/translation management platform.

		\item A feature toggle has been added in TYPO3 v10.2 that uses crowdin.com
			as the source for translations if enabled.

		\item Please note: this is in \textbf{beta status}.

		\item Read more about the
			\href{https://typo3.org/community/teams/typo3-development/initiatives/localization-with-crowdin/}{initiative}.

	\end{itemize}

	\begin{figure}
		\includegraphics[width=0.50\linewidth]{ChangesForIntegrators/crowdin-logo.png}
	\end{figure}

\end{frame}

% ------------------------------------------------------------------------------
% Feature | 89171 | Added possibility to have multiple sitemaps

\begin{frame}[fragile]
	\frametitle{Changes for Integrators}
	\framesubtitle{Multiple Sitemaps}

	% decrease font size for code listing
	\lstset{basicstyle=\tiny\ttfamily}

	\begin{itemize}

		\item It is now possible to configure multiple sitemaps.
		\item Syntax:
\begin{lstlisting}
plugin.tx_seo {
  config {
    <sitemapType> {
      sitemaps {
        <unique key> {
          provider = TYPO3\CMS\Seo\XmlSitemap\RecordsXmlSitemapDataProvider
          config {
            ...
          }
        }
      }
    }
  }
}
\end{lstlisting}

	\end{itemize}

\end{frame}

% ------------------------------------------------------------------------------
% Feature | 86759 | Support nomodule attribute for JavaScript includes

\begin{frame}[fragile]
	\frametitle{Changes for Integrators}
	\framesubtitle{HTML5 attribute \texttt{nomodule}}

	% decrease font size for code listing
	\lstset{basicstyle=\tiny\ttfamily}

	\begin{itemize}
		\item The HTML5 attribute \texttt{nomodule} is now supported when including JavaScript files in TypoScript.
\begin{lstlisting}
page.includeJSFooter.file = path/to/classic-file.js
page.includeJSFooter.file.nomodule = 1
\end{lstlisting}

		\item This attribute prevents a script from being executed in browsers that support module scripts.

		\item Read more about the standard in the
			\href{https://html.spec.whatwg.org/multipage/scripting.html#attr-script-nomodule}{specification}
			and about the concept of
			\href{https://hacks.mozilla.org/2015/08/es6-in-depth-modules/}{modules}.

	\end{itemize}

% <script type="module" src="path/to/file.js"></script>
% <script nomodule src="path/to/file/classic-file.js"></script>

\end{frame}

% ------------------------------------------------------------------------------
% Feature | 87798 | Provide a way to sort form lists in ext:form

\begin{frame}[fragile]
	\frametitle{Changes for Integrators}
	\framesubtitle{Sorting of Forms}

	% decrease font size for code listing
	\lstset{basicstyle=\tiny\ttfamily}

	\begin{itemize}
		\item Forms can now be sorted in either ascending or descending order.
		\item Two new settings were introduced: \texttt{sortByKeys} and \texttt{sortAscending}.
		\item Forms are initially sorted by their name and their file UID (ascending).
		\item To change the sorting, the following configuration needs to be added in the YAML configuration file:
\begin{lstlisting}
TYPO3:
  CMS:
    Form:
      persistenceManager:
        sortByKeys: ['name', 'fileUid']
        sortAscending: true
\end{lstlisting}

	\end{itemize}

\end{frame}

% ------------------------------------------------------------------------------
% Feature | 86918 | Add additional configuration for external link types in Linkvalidator

\begin{frame}[fragile]
	\frametitle{Changes for Integrators}
	\framesubtitle{Link Validator (1)}

	% decrease font size for code listing
	\lstset{basicstyle=\tiny\ttfamily}

	\begin{itemize}
		\item The Link Validator now supports additional configuration for external links.
		\item Values for \texttt{httpAgentUrl} and \texttt{httpAgentEmail} should be provided.
		\item Settings \texttt{headers}, \texttt{method} and \texttt{range} are advanced settings.
\begin{lstlisting}
mod.linkvalidator {
  linktypesConfig {
    external {
      httpAgentName = ...
      httpAgentUrl = ...
      httpAgentEmail = ...
      headers {
      }
      method = HEAD
      range = 0-4048
    }
  }
}
\end{lstlisting}

	\end{itemize}

\end{frame}

% ------------------------------------------------------------------------------
% Feature | 84990 | Add event for checking external links in RTE

\begin{frame}[fragile]
	\frametitle{Changes for Integrators}
	\framesubtitle{Link Validator (2)}

	\begin{itemize}
		\item Link Validator now marks broken \textbf{external} links in the RTE too.
		\item This feature was only available for internal links.
		\item It is recommended to run the Link Validator as a Scheduler task to regularly crawl for broken links.
	\end{itemize}

\end{frame}

% ------------------------------------------------------------------------------
