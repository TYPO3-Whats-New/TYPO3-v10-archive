% ------------------------------------------------------------------------------
% TYPO3 Version 10.2 - What's New (English Version)
%
% @author	Michael Schams <schams.net>
% @license	Creative Commons BY-NC-SA 3.0
% @link		http://typo3.org/download/release-notes/whats-new/
% @language	English
% ------------------------------------------------------------------------------

\section{Changes for Developers}
\begin{frame}[fragile]
	\frametitle{Changes for Developers}

	\begin{center}\huge{Chapter 3:}\end{center}
	\begin{center}\huge{\color{typo3darkgrey}\textbf{Changes for Developers}}\end{center}

\end{frame}

% ------------------------------------------------------------------------------
% Feature | 88950 | Add storeSession argument to Widget ViewHelpers

\begin{frame}[fragile]
	\frametitle{Changes for Developers}
	\framesubtitle{Widget ViewHelpers}

	% decrease font size for code listing
	\lstset{basicstyle=\smaller\ttfamily}

	\begin{itemize}
		\item Widget ViewHelpers set a session cookie in the frontend under certain circumstances.
		\item As this is not always desired (for example due to GDPR), this can be controlled now.
		\item A boolean \texttt{storeSession} has been introduced that lets developers enable/disable this feature.
\begin{lstlisting}
<f:widget.autocomplete
  for="name"
  objects="{posts}"
  searchProperty="author"
  storeSession="false" />
\end{lstlisting}

	\end{itemize}

\end{frame}

% ------------------------------------------------------------------------------
% Feature | 89577 | New PSR-14 based events for File Abstraction Layer
% Deprecation | 89577 | FAL SignalSlot handling migrated to PSR-14 events

\begin{frame}[fragile]
	\frametitle{Changes for Developers}
	\framesubtitle{PSR-14 Events in FAL}

	% decrease font size for code listing
	\lstset{basicstyle=\tiny\ttfamily}

	\begin{itemize}
		\item Approximately 40 new
			\href{https://www.php-fig.org/psr/psr-14/}{PSR-14}
			based Events have been introduced in the File Abstraction Layer (FAL).
		\item They replace existing Extbase Signal/Slots.
		\item Using the Signals continues to work (without producing any deprecation message!).
			However, the Signals in the FAL will likely be removed in TYPO3 v11.
		\item Extension authors are advised to migrate their code and use Events.
		\item Review the new PHP classes to learn more about PSR-14.
	\end{itemize}

\end{frame}


% ------------------------------------------------------------------------------
% Deprecation | 89733 | Signal Slots in Core Extension migrated to PSR-14 events

\begin{frame}[fragile]
	\frametitle{Changes for Developers}
	\framesubtitle{PSR-14 Events in the TYPO3 Core}

	\begin{itemize}
		\item A number of new PSR-14 Events replace Signal/Slots in the TYPO3 core:
			\newline

			\begin{itemize}\tiny
				\item \texttt{TYPO3\textbackslash
					CMS\textbackslash
					Core\textbackslash
					Imaging\textbackslash
					Event\textbackslash
					ModifyIconForResourcePropertiesEvent}
					\newline
				\item \texttt{TYPO3\textbackslash
					CMS\textbackslash
					Core\textbackslash
					DataHandling\textbackslash
					Event\textbackslash
					IsTableExcludedFromReferenceIndexEvent}
					\newline
				\item \texttt{TYPO3\textbackslash
					CMS\textbackslash
					Core\textbackslash
					DataHandling\textbackslash
					Event\textbackslash
					AppendLinkHandlerElementsEvent}
					\newline
				\item \texttt{TYPO3\textbackslash
					CMS\textbackslash
					Core\textbackslash
					Configuration\textbackslash
					Event\textbackslash
					AfterTcaCompilationEvent}
					\newline
				\item \texttt{TYPO3\textbackslash
					CMS\textbackslash
					Core\textbackslash
					Database\textbackslash
					Event\textbackslash
					AlterTableDefinitionStatementsEvent}
					\newline
				\item \texttt{TYPO3\textbackslash
					CMS\textbackslash
					Core\textbackslash
					Tree\textbackslash
					Event\textbackslash
					ModifyTreeDataEvent}
					\newline
				\item \texttt{TYPO3\textbackslash
					CMS\textbackslash
					Backend\textbackslash
					Backend\textbackslash
					Event\textbackslash
					SystemInformationToolbarCollectorEvent}
					\newline
			\end{itemize}

	\end{itemize}

\end{frame}

% ------------------------------------------------------------------------------
% Deprecation | 89718 | Legacy PageTSconfig parsing lowlevel API

\begin{frame}[fragile]
	\frametitle{Changes for Developers}
	\framesubtitle{TSconfig Parsing}

	% decrease font size for code listing
	\lstset{basicstyle=\tiny\ttfamily}

	\begin{itemize}
		\item Two new PHP classes have been introduced to load and parse PageTSconfig:
			\begin{itemize}\smaller
				\item \texttt{TYPO3\textbackslash
					CMS\textbackslash
					Core\textbackslash
					Configuration\textbackslash
					Loader\textbackslash
					PageTsConfigLoader}
				\item \texttt{TYPO3\textbackslash
					CMS\textbackslash
					Core\textbackslash
					Configuration\textbackslash
					Parser\textbackslash
					PageTsConfigParser}
			\end{itemize}

		\item For example:
\begin{lstlisting}
// Fetch all available PageTS of a page/rootline:
$loader = GeneralUtility::makeInstance(PageTsConfigLoader::class);
$tsConfigString = $loader->load($rootLine);

// Parse the string and apply conditions:
$parser = GeneralUtility::makeInstance(
  PageTsConfigParser::class, $typoScriptParser, $hashCache
);

$pagesTSconfig = $parser->parse($tsConfigString, $conditionMatcher);
\end{lstlisting}

	\end{itemize}

\end{frame}

% ------------------------------------------------------------------------------
% Important | 87518 | Use prepared statements for pdo_mysql per default

\begin{frame}[fragile]
	\frametitle{Changes for Developers}
	\framesubtitle{Prepared Statements}

	% decrease font size for code listing
	\lstset{basicstyle=\tiny\ttfamily}

	\begin{itemize}
		\item The \texttt{pdo\_mysql} driver uses prepared statements by default now.
		\item In TYPO3 < v10.2, \textit{emulated prepared statements} are used.
			This means, all returned values of a query were strings.
		\item This behavior has changed and prepared statements are used
			which return native data types.
		\item For example: values of a column defined as integer are returned in PHP as \texttt{int}.
		\item This feature can be deactivated by setting the option
			\texttt{PDO::ATTR\_EMULATE\_PREPARES} in your database connection.

	\end{itemize}

\end{frame}

% ------------------------------------------------------------------------------
% Feature | 86967 | Allow fetching uid of a LazyLoadingProxy without loading the object first

\begin{frame}[fragile]
	\frametitle{Changes for Developers}
	\framesubtitle{Lazy Loading Proxy}

	% decrease font size for code listing
	\lstset{basicstyle=\tiny\ttfamily}

	\begin{itemize}
		\item A method \texttt{getUid()} has been added to the class\newline
			\texttt{TYPO3\textbackslash
				CMS\textbackslash
				Extbase\textbackslash
				Persistence\textbackslash
				Generic\textbackslash
				LazyLoadingProxy}.
		\item This allows developers to fetch the UID of the proxied object without fetching the object from the database.

	\end{itemize}

\end{frame}

% ------------------------------------------------------------------------------
% Feature | 87380 | Introduce SiteLanguageAwareInterface to denote site language awareness

\begin{frame}[fragile]
	\frametitle{Changes for Developers}
	\framesubtitle{Denote Site Language Awareness}

	% decrease font size for code listing
	\lstset{basicstyle=\tiny\ttfamily}

% A SiteLanguageAwareInterface with the methods setSiteLanguage(Entity\SiteLanguage $siteLanguage) and getSiteLanguage() has been introduced.
% The interface can be used to denote a class as aware of the site language.

	\begin{itemize}
		\item A \texttt{SiteLanguageAwareInterface} has been introduced.
		\item The interface can be used to denote a class as aware of the site language.
		\item Routing aspects, that take the site language into account,
			are now using the \texttt{SiteLanguageAwareInterface}
			in addition to the \texttt{SiteLanguageAwareTrait}.
	\end{itemize}

\end{frame}

% ------------------------------------------------------------------------------
% Important | 89645 | Removed systemLog options

\begin{frame}[fragile]
	\frametitle{Changes for Developers}
	\framesubtitle{System Log API}

	% decrease font size for code listing
	\lstset{basicstyle=\tiny\ttfamily}

	\begin{itemize}
		\item The following options have been removed from TYPO3's default configuration:

			\begin{itemize}\smaller
				\item \texttt{\$GLOBALS['TYPO3\_CONF\_VARS']['SYS']['systemLog']}
				\item \texttt{\$GLOBALS['TYPO3\_CONF\_VARS']['SYS']['systemLogLevel']}
			\end{itemize}\normalsize

		\item Extension authors are adviced to use the Logging API and remove the systemLog options.
	\end{itemize}

\end{frame}

% ------------------------------------------------------------------------------
% Feature | 89603 | Introduce native pagination for lists

\begin{frame}[fragile]
	\frametitle{Changes for Developers}
	\framesubtitle{Native List Pagination}

	% decrease font size for code listing
	\lstset{basicstyle=\tiny\ttfamily}

	\begin{itemize}
		\item Native support for the pagination of lists such as arrays or QueryResults of Extbase has been introduced.
		\item The \texttt{PaginatorInterface} defines a basic set of methods.
		\item The \texttt{AbstractPaginator} class holds the main pagination logic.
		\item This enables developers to implement all kinds of paginators.
\begin{lstlisting}
use TYPO3\CMS\Core\Pagination\ArrayPaginator;

$items = ['apple', 'banana', 'strawberry', 'raspberry', 'ananas'];
$currentPageNumber = 3;
$itemsPerPage = 2;

$paginator = new ArrayPaginator($itemsToBePaginated, $currentPageNumber, $itemsPerPage);
$paginator->getNumberOfPages(); // returns 3
$paginator->getCurrentPageNumber(); // returns 3
$paginator->getKeyOfFirstPaginatedItem(); // returns 5
$paginator->getKeyOfLastPaginatedItem(); // returns 5
\end{lstlisting}

	\end{itemize}

\end{frame}

% ------------------------------------------------------------------------------
% Deprecation | 89579 | ServiceChains require an array for excluded Service keys

\begin{frame}[fragile]
	\frametitle{Changes for Developers}
	\framesubtitle{Service API}

	\begin{itemize}
		\item Argument \texttt{\$excludeServiceKeys} is used for skipping certain services when using a chain.
		\item The argument has been changed from a comma-separated list to an array in TYPO3 v10.2.
		\item This change affects the Service API within the following components:

			\begin{itemize}
				\item \texttt{GeneralUtility::makeInstanceService()}
				\item \texttt{ExtensionManagementUtility::findService()}
			\end{itemize}

		\item Passing a comma-separated list still works but has been marked as \textbf{deprecated}.

	\end{itemize}

\end{frame}

% ------------------------------------------------------------------------------
