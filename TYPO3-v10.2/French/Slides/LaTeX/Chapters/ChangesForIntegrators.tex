% ------------------------------------------------------------------------------
% TYPO3 Version 10.2 - What's New (French Version)
%
% @license	Creative Commons BY-NC-SA 3.0
% @link		http://typo3.org/download/release-notes/whats-new/
% @language	French
% ------------------------------------------------------------------------------

\section{Changements pour les intégrateurs}
\begin{frame}[fragile]
	\frametitle{Changements pour les intégrateurs}

	\begin{center}\huge{Chapitre 2~:}\end{center}
	\begin{center}\huge{\color{typo3darkgrey}\textbf{Changements pour les intégrateurs}}\end{center}

\end{frame}

% ------------------------------------------------------------------------------
% Feature | 85592 | Add site title configuration to sites module

\begin{frame}[fragile]
	\frametitle{Changements pour les intégrateurs}
	\framesubtitle{Configuration de site (1)}

	\begin{itemize}

		\item Le titre du site se configure dans
			\textbf{Gestionnaire de site} $\rightarrow$ \textbf{Sites}.
		\item Les intégrateurs peuvent spécifier un titre de site différent pour chaque langues.
		\item Le champ de l'enregistrement gabarit est obsolète et marqué \textbf{déprécié}.
		\item Le champ \texttt{sys\_template.sitetitle} (base et TCA) sera retiré de TYPO3 v11.
		\item Le titre du site est utilisé pour le titre de page et pour les intégrations futures de
			\texttt{schema.org}.
	\end{itemize}

\end{frame}

% ------------------------------------------------------------------------------
% Feature | 89398 | Support for environment variables in imports in site configurations

\begin{frame}[fragile]
	\frametitle{Changements pour les intégrateurs}
	\framesubtitle{Configuration de site (2)}

	% decrease font size for code listing
	\lstset{basicstyle=\tiny\ttfamily}

	\begin{itemize}

		\item Les variables d'environnement sont utilisables dans les importations des fichiers YAML de
			configuration de site~:
\begin{lstlisting}
imports:
  -
    resource: 'Env_%env("foo")%.yaml'
\end{lstlisting}

	\end{itemize}

\end{frame}

% ------------------------------------------------------------------------------
% Feature | 88102 | Frontend Login Form Via Fluid And Extbase

\begin{frame}[fragile]
	\frametitle{Changements pour les intégrateurs}
	\framesubtitle{Authentification frontend (1)}

	\begin{itemize}

		\item La fonctionnalité d'authentification frontend inclus une version Extbase dans TYPO3 v10.2.
		\item La solution présentes les avantages~:

			\begin{itemize}
				\item Aisance de modification du gabarit.
				\item Envoie d'email de récupération de mot de passe au format HTML.
				\item Ajustement et modification des validateurs pour forcer les restrictions sur les mots de passe.
			\end{itemize}

		\item Le plugin est directement disponible dans les nouvelles installations.
		\item Les instances TYPO3 existantes continuerons d'utiliser les anciens gabarits.
		\item Les intégrateurs peuvent basculer entre l'ancien et le nouveau plugin dans la configuration des fonctionnalités.

	\end{itemize}

\end{frame}

% ------------------------------------------------------------------------------
% Feature | 88110 | Felogin extbase password recovery

\begin{frame}[fragile]
	\frametitle{Changements pour les intégrateurs}
	\framesubtitle{Authentification frontend (2)}

	\begin{itemize}

		\item Le formulaire de récupération de mot de passe est intégré au plugin Extbase.
		\item Les utilisateurs peuvent demander un changement de mot de passe. Ils recevront
			un email avec un lien les redirigeant vers le formulaire.
		\item Règles de validation de mots de passe par défaut~:

			\begin{itemize}
				\item \texttt{NotEmptyValidator} - les mots de passe ne sont pas vides.
				\item \texttt{StringLengthValidator} - les mots de passe ont une longueur minimale.
			\end{itemize}

	\end{itemize}

\end{frame}

% ------------------------------------------------------------------------------
% Feature | 88110 | Felogin extbase password recovery

\begin{frame}[fragile]
	\frametitle{Changements pour les intégrateurs}
	\framesubtitle{Authentification frontend (3)}

	% decrease font size for code listing
	\lstset{basicstyle=\tiny\ttfamily}

	\begin{itemize}
		\item Ces règles de validation sont personnalisables.
		\item Par exemple~:
\begin{lstlisting}
plugin.tx_felogin_login {
  settings {
    passwordValidators {
      10 = TYPO3\CMS\Extbase\Validation\Validator\AlphanumericValidator
      20 {
        className = TYPO3\CMS\Extbase\Validation\Validator\StringLengthValidator
        options {
          minimum = 12
          maximum = 32
        }
      }
      30 = \Vendor\MyExtension\Validation\Validator\MyCustomPasswordPolicyValidator
    }
  }
}
\end{lstlisting}

	\end{itemize}

\end{frame}

% ------------------------------------------------------------------------------
% Feature | 89526 | FeatureFlag: betaTranslationServer

\begin{frame}[fragile]
	\frametitle{Changements pour les intégrateurs}
	\framesubtitle{Plateforme de gestion des traductions}

	\begin{itemize}

		\item L'objectif de \href{https://crowdin.com/}{crowdin} est de remplacer la solution
			\href{https://translation.typo3.org/}{Pootle} existante
			en tant que plateforme de gestion des traductions.

		\item Une nouveau paramètre dans la configuration des fonctionnalités est ajouté à TYPO3 v10.2 pour utiliser crowdin.com
			comme source de traduction.

		\item Remarque~: ceci est en \textbf{beta}.

		\item En savoir plus sur
			l'\href{https://typo3.org/community/teams/typo3-development/initiatives/localization-with-crowdin/}{initiative (en)}.

	\end{itemize}

	\begin{figure}
		\includegraphics[width=0.50\linewidth]{ChangesForIntegrators/crowdin-logo.png}
	\end{figure}

\end{frame}

% ------------------------------------------------------------------------------
% Feature | 89171 | Added possibility to have multiple sitemaps

\begin{frame}[fragile]
	\frametitle{Changements pour les intégrateurs}
	\framesubtitle{Sitemap multiples}

	% decrease font size for code listing
	\lstset{basicstyle=\tiny\ttfamily}

	\begin{itemize}

		\item Il est possible de définir de multiples sitemaps.
		\item Syntaxe~:
\begin{lstlisting}
plugin.tx_seo {
  config {
    <sitemapType> {
      sitemaps {
        <unique key> {
          provider = TYPO3\CMS\Seo\XmlSitemap\RecordsXmlSitemapDataProvider
          config {
            ...
          }
        }
      }
    }
  }
}
\end{lstlisting}

	\end{itemize}

\end{frame}

% ------------------------------------------------------------------------------
% Feature | 86759 | Support nomodule attribute for JavaScript includes

\begin{frame}[fragile]
	\frametitle{Changements pour les intégrateurs}
	\framesubtitle{Attribut HTML5 \texttt{nomodule}}

	% decrease font size for code listing
	\lstset{basicstyle=\tiny\ttfamily}

	\begin{itemize}
		\item L'attribut HTML5 \texttt{nomodule} est supporté lors de l'inclusion de fichiers JavaScript en TypoScript.
\begin{lstlisting}
page.includeJSFooter.file = path/to/classic-file.js
page.includeJSFooter.file.nomodule = 1
\end{lstlisting}

		\item L'attribut empêche l'exécution du script lorsque le navigateur supporte les scripts modulaires.

		\item En savoir plus sur le standard dans la
			\href{https://html.spec.whatwg.org/multipage/scripting.html#attr-script-nomodule}{spécification}
			et sur le concept de
			\href{https://hacks.mozilla.org/2015/08/es6-in-depth-modules/}{modules}.

	\end{itemize}

% <script type="module" src="path/to/file.js"></script>
% <script nomodule src="path/to/file/classic-file.js"></script>

\end{frame}

% ------------------------------------------------------------------------------
% Feature | 87798 | Provide a way to sort form lists in ext:form

\begin{frame}[fragile]
	\frametitle{Changements pour les intégrateurs}
	\framesubtitle{Tri des formulaires}

	% decrease font size for code listing
	\lstset{basicstyle=\tiny\ttfamily}

	\begin{itemize}
		\item Les formulaires peuvent être triés en ordre ascendant ou descendant.
		\item Deux options sont introduites~: \texttt{sortByKeys} et \texttt{sortAscending}.
		\item Les formulaires sont triés initialement par leur nom et l'identifiant de leur fichier (ascendant).
		\item Pour changer le tri, la configuration suivante doit être utilisée de la fichier de configuration YAML~:
\begin{lstlisting}
TYPO3:
  CMS:
    Form:
      persistenceManager:
        sortByKeys: ['name', 'fileUid']
        sortAscending: true
\end{lstlisting}

	\end{itemize}

\end{frame}

% ------------------------------------------------------------------------------
% Feature | 86918 | Add additional configuration for external link types in Linkvalidator

\begin{frame}[fragile]
	\frametitle{Changements pour les intégrateurs}
	\framesubtitle{Validateur de liens (1)}

	% decrease font size for code listing
	\lstset{basicstyle=\tiny\ttfamily}

	\begin{itemize}
		\item Le validateur de lien supporte des configurations supplémentaires pour les liens externes.
		\item Les valeurs pour \texttt{httpAgentUrl} et \texttt{httpAgentEmail} devraient être fournies.
		\item Les options \texttt{headers}, \texttt{method} et \texttt{range} sont pour l'usage avancée.
\begin{lstlisting}
mod.linkvalidator {
  linktypesConfig {
    external {
      httpAgentName = ...
      httpAgentUrl = ...
      httpAgentEmail = ...
      headers {
      }
      method = HEAD
      range = 0-4048
    }
  }
}
\end{lstlisting}

	\end{itemize}

\end{frame}

% ------------------------------------------------------------------------------
% Feature | 84990 | Add event for checking external links in RTE

\begin{frame}[fragile]
	\frametitle{Changements pour les intégrateurs}
	\framesubtitle{Validateur de liens (2)}

	\begin{itemize}
		\item Le validateur de lien marque les liens \textbf{externe} cassés dans le RTE.
		\item La fonction n'est disponible que pour les liens internes.
		\item Il est recommandé de mettre en place la tâche planifiée du validateur de lien pour
			chercher régulièrement les liens cassés.
	\end{itemize}

\end{frame}

% ------------------------------------------------------------------------------
