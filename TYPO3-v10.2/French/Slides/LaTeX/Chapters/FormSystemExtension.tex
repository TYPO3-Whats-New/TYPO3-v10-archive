% ------------------------------------------------------------------------------
% TYPO3 Version 10.2 - What's New (French Version)
%
% @license	Creative Commons BY-NC-SA 3.0
% @link		http://typo3.org/download/release-notes/whats-new/
% @language	French
% ------------------------------------------------------------------------------

\section{Extension système «~Form~»}
\begin{frame}[fragile]
	\frametitle{Extension système «~Form~»}

	\begin{center}\huge{Chapitre 4~:}\end{center}
	\begin{center}\huge{\color{typo3darkgrey}\textbf{Extension système «~Form~»}}\end{center}

\end{frame}

% ------------------------------------------------------------------------------
% Extension système «~Form~» - Summary

\begin{frame}[fragile]
	\frametitle{Extension système «~Form~»}
	\framesubtitle{Résumé}

	\small
		L'extension système \textbf{Form} a subi de nombreux changements.
		Ils affectent les éditeurs, les intégrateurs aussi bien que les développeurs.

		\vspace{0.2cm}

		Certains des changements sont basés sur des concepts développés durant la TYPO3 Initiative Week (T3INIT19).

	\normalsize

\end{frame}

% ------------------------------------------------------------------------------
% Important | 84221 | Restructuring of form setup

\begin{frame}[fragile]
	\frametitle{Extension système «~Form~»}
	\framesubtitle{Configuration du formulaire}

	\begin{itemize}
		\item Trois fichiers étaient utilisés~: \texttt{BaseSetup.yaml}, \texttt{FormEditorSetup.yaml}, et \texttt{FormEngineSetup.yaml}.
		\item Ils sont rationalisés et consolidés dans un seul fichier~: \texttt{FormSetup.yaml}.
		\item Ce fichier contient la configuration de base, incluant les importations de configuration pour les validateurs,
			les éléments de formulaire et les finaliseurs.
		\item Tous les héritages et incorporations (mixin) sont résolues, rendant la compréhension de la configuration plus aisée.
	\end{itemize}

\end{frame}

% ------------------------------------------------------------------------------
% Feature | 84203 | Unify form setup YAML loading

\begin{frame}[fragile]
	\frametitle{Extension système «~Form~»}
	\framesubtitle{Fichiers YAML}

	\begin{itemize}
		\item Les fichiers YAML utilisent le chargeur de YAML du noyau de TYPO3.
		\item Ces fonctionnalités sont donc accessibles~:

			\begin{itemize}
				\item Importation de fichiers YAML à l'aide de la directive \texttt{imports}.
				\item Remplacement des \texttt{\%substitutions\%}.
			\end{itemize}

	\end{itemize}

\end{frame}

% ------------------------------------------------------------------------------
% Feature | 79445 | Add Multistep Wizard

\begin{frame}[fragile]
	\frametitle{Extension système «~Form~»}
	\framesubtitle{Assistant multi-étapes}

	\begin{itemize}
		\item Le module JavaScript \texttt{MultiStepWizard} est introduit,
			ajoutant les fonctionnalités suivantes~:

			\begin{itemize}
				\item Navigation vers l'étape précédente.
				\item Libellés descriptifs tel que «~Début~» ou «~Fin~», au lieu de l'indicateur numérique «~Étape x de y~».
				\item Structure de configuration optimisée.
			\end{itemize}

		\item Voir le \href{https://docs.typo3.org/c/typo3/cms-core/master/en-us/Changelog/10.2/Feature-79445-AddMultistepWizard.html}{journal des changements (en)}
			pour des exemples de code JavaScript.

		\item Cette fonctionnalité améliore significativement l'expérience utilisateur~: les utilisateurs backend remarquerons
			un assistant de création de formulaire amélioré.

	\end{itemize}

\end{frame}

% ------------------------------------------------------------------------------
% Feature | 89747 | Custom tables with record browser in forms

\begin{frame}[fragile]
	\frametitle{Extension système «~Form~»}
	\framesubtitle{Explorateur d'enregistrements}

	% decrease font size for code listing
	\lstset{basicstyle=\tiny\ttfamily}

	\begin{itemize}
		\item L'explorateur d'enregistrement se configure pour utiliser des tables supplémentaires~:
\begin{lstlisting}
TYPO3:
  CMS:
    Form:
      prototypes:
        standard:
          formElementsDefinition:
            MyCustomElement:
              formEditor:
                editors:
                  # ...
                  300:
                    identifier: myRecord
                    # ...
                    browsableType: tx_myext_mytable
                    propertyPath: properties.myRecordUid
                    # ...
\end{lstlisting}

	\end{itemize}

\end{frame}

% ------------------------------------------------------------------------------
% Feature | 89746 | Custom icon for record browser button in forms

\begin{frame}[fragile]
	\frametitle{Extension système «~Form~»}
	\framesubtitle{Explorateur d'enregistrements}

	% decrease font size for code listing
	\lstset{basicstyle=\tiny\ttfamily}

	\begin{itemize}
		\item L'icône du bouton de l'explorateur d'enregistrements est personnalisable~:
\begin{lstlisting}
TYPO3:
  CMS:
    Form:
      prototypes:
        standard:
          formElementsDefinition:
            MyCustomElement:
              formEditor:
                editors:
                  # ...
                  300:
                    identifier: contentElement
                    # ...
                    browsableType: tt_content
                    iconIdentifier: mimetypes-x-content-text
                    propertyPath: properties.contentElementUid
                    # ...
\end{lstlisting}

	\end{itemize}

\end{frame}

% ------------------------------------------------------------------------------
% Feature | 84713 | Access single values in form templates

\begin{frame}[fragile]
	\frametitle{Extension système «~Form~»}
	\framesubtitle{Explorateur d'enregistrements}

	% decrease font size for code listing
	\lstset{basicstyle=\tiny\ttfamily}

	\begin{itemize}
		\item Le nouveau \textit{RenderFormValue-ViewHelper} permet aux intégrateurs et développeurs
			d'accéder à une valeur du formulaire dans les gabarits Fluid~:
\begin{lstlisting}
<p>
  The following message was just sent by
  <formvh:renderFormValue renderable="{page.rootForm.elements.name}" as="formValue">
    {formValue.processedValue}
  </formvh:renderFormValue>:
</p>

<blockquote>
  <formvh:renderFormValue renderable="{page.rootForm.elements.message}" as="formValue">
    {formValue.processedValue}
  </formvh:renderFormValue>
</blockquote>
\end{lstlisting}

	\end{itemize}

\end{frame}

% ------------------------------------------------------------------------------
% Feature | 82706 | Render fieldset labels in form templates

\begin{frame}[fragile]
	\frametitle{Extension système «~Form~»}
	\framesubtitle{Libellé des groupes}

	\begin{itemize}
		\item L'élément de groupe / section \texttt{Fieldset} est disponible dans les gabarits.
		\item Par défaut, sont affectés l'élément de formulaire \textbf{SummaryPage} et les finaliseurs \textbf{EmailToReceiver} et \textbf{EmailToSender}.
		\item Cas d'usage typique~:\newline
			\small
				Un formulaire avec une adresse d'expédition et une de facturation. Les deux sections peuvent avoir des
				champs de même nom, i.e. \texttt{rue}. La distinction de ces champs s'effectue en utilisant le libellé du groupe.
			\normalsize

	\end{itemize}

\end{frame}

% ------------------------------------------------------------------------------
% Deprecation | 88238 | Allowed MIME types of FileUpload and ImageUpload

\begin{frame}[fragile]
	\frametitle{Extension système «~Form~»}
	\framesubtitle{Chargement de fichiers}

	\begin{itemize}
		\item Les \texttt{allowedMimeTypes} prédéfinis pour les éléments de formulaire
			suivant sont marqués \textbf{dépréciés}~:

			\begin{itemize}
				\item \texttt{FileUpload}
				\item \texttt{ImageUpload}
			\end{itemize}

		\item Tous les types MIME valides doivent être explicitement listés dans le définition du formulaire\newline
			\smaller
				(les types MIME prédéfinis seront retirés de TYPO3 v11)
			\normalsize

		\item Les intégrateurs peuvent dès à présent activer le nouveau comportement dans la configuration
			des fonctionnalités de TYPO3 v10.

	\end{itemize}

\end{frame}

% ------------------------------------------------------------------------------
% Deprecation | 89742 | Form mixins

\begin{frame}[fragile]
	\frametitle{Extension système «~Form~»}
	\framesubtitle{Incorporation de formulaire}

	\begin{itemize}
		\item Les incorporation (mixins) sont marquées \textbf{dépréciées} et ne devraient plus être utilisées.
		\item Tous les héritage depuis \texttt{TYPO3.CMS.Form.mixins.*} sont affectés.
		\item Options de migration~:

			\begin{itemize}
				\item Intégrer les parties essentielles depuis \texttt{TYPO3.CMS.Form.mixins.*}, ou
				\item les migrer vers des incorporations personnalisées.
			\end{itemize}

	\end{itemize}

\end{frame}

% ------------------------------------------------------------------------------
