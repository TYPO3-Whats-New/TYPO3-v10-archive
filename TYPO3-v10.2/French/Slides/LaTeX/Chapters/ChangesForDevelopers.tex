% ------------------------------------------------------------------------------
% TYPO3 Version 10.2 - What's New (French Version)
%
% @license	Creative Commons BY-NC-SA 3.0
% @link		http://typo3.org/download/release-notes/whats-new/
% @language	French
% ------------------------------------------------------------------------------

\section{Changements pour les développeurs}
\begin{frame}[fragile]
	\frametitle{Changements pour les développeurs}

	\begin{center}\huge{Chapitre 3~:}\end{center}
	\begin{center}\huge{\color{typo3darkgrey}\textbf{Changements pour les développeurs}}\end{center}

\end{frame}

% ------------------------------------------------------------------------------
% Feature | 88950 | Add storeSession argument to Widget ViewHelpers

\begin{frame}[fragile]
	\frametitle{Changements pour les développeurs}
	\framesubtitle{Widget ViewHelpers}

	% decrease font size for code listing
	\lstset{basicstyle=\smaller\ttfamily}

	\begin{itemize}
		\item Les Widget ViewHelpers posent un cookie de session frontend dans certaines conditions.
		\item Comme ce n'est pas toujours souhaité (par exemple en raison de RGPD), c'est contrôlable.
		\item Un booléen \texttt{storeSession} est introduit permettant au développeur de contrôler cette fonction.
\begin{lstlisting}
<f:widget.autocomplete
  for="name"
  objects="{posts}"
  searchProperty="author"
  storeSession="false" />
\end{lstlisting}

	\end{itemize}

\end{frame}

% ------------------------------------------------------------------------------
% Feature | 89577 | New PSR-14 based events for File Abstraction Layer
% Deprecation | 89577 | FAL SignalSlot handling migrated to PSR-14 events

\begin{frame}[fragile]
	\frametitle{Changements pour les développeurs}
	\framesubtitle{Événements PSR-14 dans FAL}

	% decrease font size for code listing
	\lstset{basicstyle=\tiny\ttfamily}

	\begin{itemize}
		\item Environ 40 nouveaux événements basés sur
			\href{https://www.php-fig.org/psr/psr-14/}{PSR-14}
			sont introduits dans la couche d'abstraction des fichiers (FAL).
		\item Ils remplacent les Signal/Slots extbase existants.
		\item Les Signals sont toujours fonctionnels (sans message de dépréciation).
			Cependant, ces derniers sont susceptibles d'être retirés dans TYPO3 v11.
		\item Il est conseillé aux auteurs d'extension de migrer leur code et utiliser ces événements.
		\item Consultez les nouvelles classes PHP pour en savoir plus sur PSR-14.
	\end{itemize}

\end{frame}


% ------------------------------------------------------------------------------
% Deprecation | 89733 | Signal Slots in Core Extension migrated to PSR-14 events

\begin{frame}[fragile]
	\frametitle{Changements pour les développeurs}
	\framesubtitle{Événements PSR-14 dans le noyau TYPO3}

	\begin{itemize}
		\item Des événements PSR-14 remplacent les Signal/Slots dans le noyau TYPO3~:
			\newline

			\begin{itemize}\tiny
				\item \texttt{TYPO3\textbackslash
					CMS\textbackslash
					Core\textbackslash
					Imaging\textbackslash
					Event\textbackslash
					ModifyIconForResourcePropertiesEvent}
					\newline
				\item \texttt{TYPO3\textbackslash
					CMS\textbackslash
					Core\textbackslash
					DataHandling\textbackslash
					Event\textbackslash
					IsTableExcludedFromReferenceIndexEvent}
					\newline
				\item \texttt{TYPO3\textbackslash
					CMS\textbackslash
					Core\textbackslash
					DataHandling\textbackslash
					Event\textbackslash
					AppendLinkHandlerElementsEvent}
					\newline
				\item \texttt{TYPO3\textbackslash
					CMS\textbackslash
					Core\textbackslash
					Configuration\textbackslash
					Event\textbackslash
					AfterTcaCompilationEvent}
					\newline
				\item \texttt{TYPO3\textbackslash
					CMS\textbackslash
					Core\textbackslash
					Database\textbackslash
					Event\textbackslash
					AlterTableDefinitionStatementsEvent}
					\newline
				\item \texttt{TYPO3\textbackslash
					CMS\textbackslash
					Core\textbackslash
					Tree\textbackslash
					Event\textbackslash
					ModifyTreeDataEvent}
					\newline
				\item \texttt{TYPO3\textbackslash
					CMS\textbackslash
					Backend\textbackslash
					Backend\textbackslash
					Event\textbackslash
					SystemInformationToolbarCollectorEvent}
					\newline
			\end{itemize}

	\end{itemize}

\end{frame}

% ------------------------------------------------------------------------------
% Deprecation | 89718 | Legacy PageTSconfig parsing lowlevel API

\begin{frame}[fragile]
	\frametitle{Changements pour les développeurs}
	\framesubtitle{Traitement du TSconfig}

	% decrease font size for code listing
	\lstset{basicstyle=\tiny\ttfamily}

	\begin{itemize}
		\item Deux classes sont introduites pour charger et traiter le TSconfig de page~:
			\begin{itemize}\smaller
				\item \texttt{TYPO3\textbackslash
					CMS\textbackslash
					Core\textbackslash
					Configuration\textbackslash
					Loader\textbackslash
					PageTsConfigLoader}
				\item \texttt{TYPO3\textbackslash
					CMS\textbackslash
					Core\textbackslash
					Configuration\textbackslash
					Parser\textbackslash
					PageTsConfigParser}
			\end{itemize}

		\item Par exemple~:
\begin{lstlisting}
// Fetch all available PageTS of a page/rootline:
$loader = GeneralUtility::makeInstance(PageTsConfigLoader::class);
$tsConfigString = $loader->load($rootLine);

// Parse the string and apply conditions:
$parser = GeneralUtility::makeInstance(
  PageTsConfigParser::class, $typoScriptParser, $hashCache
);

$pagesTSconfig = $parser->parse($tsConfigString, $conditionMatcher);
\end{lstlisting}

	\end{itemize}

\end{frame}

% ------------------------------------------------------------------------------
% Important | 87518 | Use prepared statements for pdo_mysql per default

\begin{frame}[fragile]
	\frametitle{Changements pour les développeurs}
	\framesubtitle{Prepared Statements}

	% decrease font size for code listing
	\lstset{basicstyle=\tiny\ttfamily}

	\begin{itemize}
		\item Le pilote \texttt{pdo\_mysql} utilise les prepared statements par défaut.
		\item Avant TYPO3 v10.2, des \textit{prepared statements émulés} sont utilisés.
			En conséquence, les valeurs retournées par les requêtes étaient des chaînes de caractères.
		\item Ce comportement est changé et les prepared statements sont utilisés,
			retournant les types de donnée natifs.
		\item Par exemple~: les valeurs d'une colonne définie comme entier sont retournés en PHP comme \texttt{int}.
		\item La fonctionnalité est désactivable en ajoutant l'option
			\texttt{PDO::ATTR\_EMULATE\_PREPARES} à votre connexion à la base de données.

	\end{itemize}

\end{frame}

% ------------------------------------------------------------------------------
% Feature | 86967 | Allow fetching uid of a LazyLoadingProxy without loading the object first

\begin{frame}[fragile]
	\frametitle{Changements pour les développeurs}
	\framesubtitle{Mandataire de chargement différé}

	% decrease font size for code listing
	\lstset{basicstyle=\tiny\ttfamily}

	\begin{itemize}
		\item La méthode \texttt{getUid()} est ajoutée à la classe\newline
			\texttt{TYPO3\textbackslash
				CMS\textbackslash
				Extbase\textbackslash
				Persistence\textbackslash
				Generic\textbackslash
				LazyLoadingProxy}.
		\item Les développeurs peuvent donc récupérer l'identifiant d'un objet utilisant un mandataire
			sans devoir récupérer celui-ci depuis la base de données.

	\end{itemize}

\end{frame}

% ------------------------------------------------------------------------------
% Feature | 87380 | Introduce SiteLanguageAwareInterface to denote site language awareness

\begin{frame}[fragile]
	\frametitle{Changements pour les développeurs}
	\framesubtitle{Signifier la prise en charge de la langue de site}

	% decrease font size for code listing
	\lstset{basicstyle=\tiny\ttfamily}

% A SiteLanguageAwareInterface with the methods setSiteLanguage(Entity\SiteLanguage $siteLanguage) and getSiteLanguage() has been introduced.
% The interface can be used to denote a class as aware of the site language.

	\begin{itemize}
		\item L'interface \texttt{SiteLanguageAwareInterface} est introduite.
		\item Celle-ci permet de signifier qu'un classe prend en charge la langue du site.
		\item Les extensions du routage, qui prennent en compte la langue du site,
			utilisent donc \texttt{SiteLanguageAwareInterface}
			en plus de \texttt{SiteLanguageAwareTrait}.
	\end{itemize}

\end{frame}

% ------------------------------------------------------------------------------
% Important | 89645 | Removed systemLog options

\begin{frame}[fragile]
	\frametitle{Changements pour les développeurs}
	\framesubtitle{API de journalisation système}

	% decrease font size for code listing
	\lstset{basicstyle=\tiny\ttfamily}

	\begin{itemize}
		\item Les options de configurations suivantes sont retirées de la configuration par défaut~:

			\begin{itemize}\smaller
				\item \texttt{\$GLOBALS['TYPO3\_CONF\_VARS']['SYS']['systemLog']}
				\item \texttt{\$GLOBALS['TYPO3\_CONF\_VARS']['SYS']['systemLogLevel']}
			\end{itemize}\normalsize

		\item Il est conseillé aux auteurs d'extension d'utiliser l'API de journalisation et de retirer
			l'option systemLog.
	\end{itemize}

\end{frame}

% ------------------------------------------------------------------------------
% Feature | 89603 | Introduce native pagination for lists

\begin{frame}[fragile]
	\frametitle{Changements pour les développeurs}
	\framesubtitle{Pagination native de listes}

	% decrease font size for code listing
	\lstset{basicstyle=\tiny\ttfamily}

	\begin{itemize}
		\item Le support natif de la pagination de listes comme les tableaux ou
			les résultats de requête Extbase est introduit.
		\item La \texttt{PaginatorInterface} définie le jeu de méthodes de base.
		\item La classe \texttt{AbstractPaginator} contient la logique de pagination principale.
		\item Les développeurs peuvent donc implémenter de nouvelles paginations.
\begin{lstlisting}
use TYPO3\CMS\Core\Pagination\ArrayPaginator;

$items = ['apple', 'banana', 'strawberry', 'raspberry', 'ananas'];
$currentPageNumber = 3;
$itemsPerPage = 2;

$paginator = new ArrayPaginator($itemsToBePaginated, $currentPageNumber, $itemsPerPage);
$paginator->getNumberOfPages(); // returns 3
$paginator->getCurrentPageNumber(); // returns 3
$paginator->getKeyOfFirstPaginatedItem(); // returns 5
$paginator->getKeyOfLastPaginatedItem(); // returns 5
\end{lstlisting}

	\end{itemize}

\end{frame}

% ------------------------------------------------------------------------------
% Deprecation | 89579 | ServiceChains require an array for excluded Service keys

\begin{frame}[fragile]
	\frametitle{Changements pour les développeurs}
	\framesubtitle{API des services}

	\begin{itemize}
		\item L'argument \texttt{\$excludeServiceKeys} est utilisé pour ignorer certains services lors l'utilisation
			d'une séquence de services.
		\item Le format de l'argument est changé de liste séparée par la virgule en tableau dans TYPO3 v10.2.
		\item Ce changement impacte l'API des services dans ces composants~:

			\begin{itemize}
				\item \texttt{GeneralUtility::makeInstanceService()}
				\item \texttt{ExtensionManagementUtility::findService()}
			\end{itemize}

		\item Passer l'ancien format fonctionne toujours mais est marqué \textbf{déprécié}.

	\end{itemize}

\end{frame}

% ------------------------------------------------------------------------------
