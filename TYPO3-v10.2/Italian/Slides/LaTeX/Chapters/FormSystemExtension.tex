% ------------------------------------------------------------------------------
% TYPO3 Version 10.2 - What's New (Italian Version)
%
% @license	Creative Commons BY-NC-SA 3.0
% @link		http://typo3.org/download/release-notes/whats-new/
% @language	Italian
% ------------------------------------------------------------------------------

\section{Estensione di sistema "Form"}
\begin{frame}[fragile]
	\frametitle{Estensione di sistema "Form"}

	\begin{center}\huge{Capitolo 4:}\end{center}
	\begin{center}\huge{\color{typo3darkgrey}\textbf{Estensione di sistema "Form"}}\end{center}

\end{frame}

% ------------------------------------------------------------------------------
% System Extension "Form" - Summary

\begin{frame}[fragile]
	\frametitle{Estensione di sistema "Form"}
	\framesubtitle{Sommario}

	\small
		Sono state apportate diverse modifiche all'estensione di sistema \textbf{"Form"}.
		Queste modifiche riguardano editori, integratori e sviluppatori.

		\vspace{0.2cm}

		Alcuni dei cambiamenti si basano su concetti sviluppati durante la TYPO3 Initiative Week (T3INIT19).

	\normalsize

\end{frame}

% ------------------------------------------------------------------------------
% Important | 84221 | Restructuring of form setup

\begin{frame}[fragile]
	\frametitle{Estensione di sistema "Form"}
	\framesubtitle{Form Setup}

	\begin{itemize}
		\item In precedenza erano utilizzati tre file: \texttt{BaseSetup.yaml}, \texttt{FormEditorSetup.yaml} e \texttt{FormEngineSetup.yaml}.
		\item Questo è stato semplificato e consolidato in un unico file: \texttt{FormSetup.yaml}.
		\item Questo file contiene la configurazione di base che include le importazioni delle configurazioni per validatori, elementi del modulo e finisher.
		\item Tutte le eredità e i mixin utilizzati in precedenza sono stati risolti, il che rende molto facile comprendere l'intera configurazione.
	\end{itemize}

\end{frame}

% ------------------------------------------------------------------------------
% Feature | 84203 | Unify form setup YAML loading

\begin{frame}[fragile]
	\frametitle{Estensione di sistema "Form"}
	\framesubtitle{File YAML}

	\begin{itemize}
		\item I file YAML utilizzano ora il caricatore di file YAML del core di TYPO3.
		\item Questo abilita funzionalità come:

			\begin{itemize}
				\item Importazione di altri file YAML tramite la direttiva \texttt{imports}.
				\item Sostituzione di \texttt{\%placeholders\%}.
			\end{itemize}

	\end{itemize}

\end{frame}

% ------------------------------------------------------------------------------
% Feature | 79445 | Add Multistep Wizard

\begin{frame}[fragile]
	\frametitle{Estensione di sistema "Form"}
	\framesubtitle{Wizard Multi-step}

	\begin{itemize}
		\item Un nuovo modulo JavaScript \texttt{MultiStepWizard} è stato introdotto,
			questo aggiunge le seguenti funzionalità:

			\begin{itemize}
				\item Navigazione ai passaggi precedenti.
				\item I passaggi supportano etichette descrittive come "Inizio" o "Fine", anziché l'indicatore numerico "Step x di y".
				\item Struttura di configurazione ottimizzata.
			\end{itemize}

		\item Vedi il \href{https://docs.typo3.org/c/typo3/cms-core/master/en-us/Changelog/10.2/Feature-79445-AddMultistepWizard.html}{ChangeLog}
			per codice JavaScript di esempio.

		\item Queste nuove funzionalità migliorano notevolmente l'esperienza dell'utente: gli utenti di backend troveranno una procedure guidata per la creazione di form avanzati.

	\end{itemize}

\end{frame}

% ------------------------------------------------------------------------------
% Feature | 89747 | Custom tables with record browser in forms

\begin{frame}[fragile]
	\frametitle{Estensione di sistema "Form"}
	\framesubtitle{Navigazione dei record}

	% decrease font size for code listing
	\lstset{basicstyle=\tiny\ttfamily}

	\begin{itemize}
		\item La navigazione dei record può essere configurata per utilizzare tabelle personalizzate:
\begin{lstlisting}
TYPO3:
  CMS:
    Form:
      prototypes:
        standard:
          formElementsDefinition:
            MyCustomElement:
              formEditor:
                editors:
                  # ...
                  300:
                    identifier: myRecord
                    # ...
                    browsableType: tx_myext_mytable
                    propertyPath: properties.myRecordUid
                    # ...
\end{lstlisting}

	\end{itemize}

\end{frame}

% ------------------------------------------------------------------------------
% Feature | 89746 | Custom icon for record browser button in forms

\begin{frame}[fragile]
	\frametitle{Estensione di sistema "Form"}
	\framesubtitle{Navigazione dei record}

	% decrease font size for code listing
	\lstset{basicstyle=\tiny\ttfamily}

	\begin{itemize}
		\item L'icona del pulsante di navigazione dei record è ora configurabile:
\begin{lstlisting}
TYPO3:
  CMS:
    Form:
      prototypes:
        standard:
          formElementsDefinition:
            MyCustomElement:
              formEditor:
                editors:
                  # ...
                  300:
                    identifier: contentElement
                    # ...
                    browsableType: tt_content
                    iconIdentifier: mimetypes-x-content-text
                    propertyPath: properties.contentElementUid
                    # ...
\end{lstlisting}

	\end{itemize}

\end{frame}

% ------------------------------------------------------------------------------
% Feature | 84713 | Access single values in form templates

\begin{frame}[fragile]
	\frametitle{Estensione di sistema "Form"}
	\framesubtitle{Navigazione dei record}

	% decrease font size for code listing
	\lstset{basicstyle=\tiny\ttfamily}

	\begin{itemize}
		\item Un nuovo \textit{RenderFormValue-ViewHelper} permette agli integratori/sviluppatori l'accesso ai valori di un singolo form nei template:
\begin{lstlisting}
<p>
  The following message was just sent by
  <formvh:renderFormValue renderable="{page.rootForm.elements.name}" as="formValue">
    {formValue.processedValue}
  </formvh:renderFormValue>:
</p>

<blockquote>
  <formvh:renderFormValue renderable="{page.rootForm.elements.message}" as="formValue">
    {formValue.processedValue}
  </formvh:renderFormValue>
</blockquote>
\end{lstlisting}

	\end{itemize}

\end{frame}

% ------------------------------------------------------------------------------
% Feature | 82706 | Render fieldset labels in form templates

\begin{frame}[fragile]
	\frametitle{Estensione di sistema "Form"}
	\framesubtitle{Etichette dei Fieldset}

	\begin{itemize}
		\item L'elemento di sezione \texttt{Fieldset} è accessibile nei template.
		\item Di base questo influisce sull'elemento del form \textbf{SummaryPage} e sui finisher \textbf{EmailToReceiver} e \textbf{EmailToSender}.
		\item Caso d'uso tipico:\newline
			\small
				Un form con indirizzo di spedizione e di fatturazione. Entrambe le sezioni potrebbero avere un campo con lo stesso nome, es. \texttt{via}.
				E' possibile distinguere tra i due campi utilizzando le etichette dei fieldset.
			\normalsize

	\end{itemize}

\end{frame}

% ------------------------------------------------------------------------------
% Deprecation | 88238 | Allowed MIME types of FileUpload and ImageUpload

\begin{frame}[fragile]
	\frametitle{Estensione di sistema "Form"}
	\framesubtitle{Caricamento file}

	\begin{itemize}
		\item I predefiniti \texttt{allowedMimeTypes} dei seguenti elementi di form sono stati segnati come \textbf{deprecati}:

			\begin{itemize}
				\item \texttt{FileUpload}
				\item \texttt{ImageUpload}
			\end{itemize}

		\item Tutti i tipi di MIME validi devono essere indicati esplicitamente nella definizione del form\newline
			\smaller
				(i tipi di MIME predefiniti saranno rimossi in TYPO3 v11)
			\normalsize

		\item Gli integratori possono già attivare il nuovo comportamento in TYPO3 v10 attivando la funzione.

	\end{itemize}

\end{frame}

% ------------------------------------------------------------------------------
% Deprecation | 89742 | Form mixins

\begin{frame}[fragile]
	\frametitle{Estensione di sistema "Form"}
	\framesubtitle{Form Mixin}

	\begin{itemize}
		\item I mixin sono stati segnati come \textbf{deprecati} e non devono essere usati più.
		\item Questo interessa tutte le eredità di \texttt{TYPO3.CMS.Form.mixins.*}.
		\item Opzioni di migrazione:

			\begin{itemize}
				\item Incorpora le parti essenziali da \texttt{TYPO3.CMS.Form.mixins.*}, o
				\item li migra verso mixin personalizzati.
			\end{itemize}

	\end{itemize}

\end{frame}

% ------------------------------------------------------------------------------
