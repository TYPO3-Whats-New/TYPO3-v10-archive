% ------------------------------------------------------------------------------
% TYPO3 Version 10.2 - What's New (Italian Version)
%
% @license	Creative Commons BY-NC-SA 3.0
% @link		http://typo3.org/download/release-notes/whats-new/
% @language	Italian
% ------------------------------------------------------------------------------

\section{Modifiche per sviluppatori}
\begin{frame}[fragile]
	\frametitle{Modifiche per sviluppatori}

	\begin{center}\huge{Capitolo 3:}\end{center}
	\begin{center}\huge{\color{typo3darkgrey}\textbf{Modifiche per sviluppatori}}\end{center}

\end{frame}

% ------------------------------------------------------------------------------
% Feature | 88950 | Add storeSession argument to Widget ViewHelpers

\begin{frame}[fragile]
	\frametitle{Modifiche per sviluppatori}
	\framesubtitle{Widget ViewHelpers}

	% decrease font size for code listing
	\lstset{basicstyle=\smaller\ttfamily}

	\begin{itemize}
		\item I Widget ViewHelper impostano un cookie di sessione nel frontend in determinate circostanze.
		\item Poiché ciò non sempre è possibile (ad esempio per il GDPR), ora può essere controllato.
		\item E' stato introdotto un booleano \texttt{storeSession} che consente agli svluppatori di abilitare/disabilitare questa funzione.
\begin{lstlisting}
<f:widget.autocomplete
  for="name"
  objects="{posts}"
  searchProperty="author"
  storeSession="false" />
\end{lstlisting}

	\end{itemize}

\end{frame}

% ------------------------------------------------------------------------------
% Feature | 89577 | New PSR-14 based events for File Abstraction Layer
% Deprecation | 89577 | FAL SignalSlot handling migrated to PSR-14 events

\begin{frame}[fragile]
	\frametitle{Modifiche per sviluppatori}
	\framesubtitle{PSR-14 Events in FAL}

	% decrease font size for code listing
	\lstset{basicstyle=\tiny\ttfamily}

	\begin{itemize}
		\item Circa 40 nuovi eventi basati su
			\href{https://www.php-fig.org/psr/psr-14/}{PSR-14}
			sono stati introdotti nel File Abstraction Layer (FAL).
		\item Sostituiscono gli esistenti Signal/Slots Extbase.
		\item L'uso dei Signals continua a funzionare (senza creare nessun messaggio di deprecazione!).
			Tuttavia, i Signals nel FAL saranno probabilmente rimossi in TYPO3 v11.
		\item Si consiglia agli autori di estensioni di migrare il loro codice e utilizzare gli eventi.
		\item Esamina le nuove classi PHP per saperne di più sulla PSR-14.
	\end{itemize}

\end{frame}


% ------------------------------------------------------------------------------
% Deprecation | 89733 | Signal Slots in Core Extension migrated to PSR-14 events

\begin{frame}[fragile]
	\frametitle{Modifiche per sviluppatori}
	\framesubtitle{Eventi PSR-14 nel core TYPO3}

	\begin{itemize}
		\item Numerosi nuovi eventi PSR-14 sostituiscono Signal/Slots nel core TYPO3:
			\newline

			\begin{itemize}\tiny
				\item \texttt{TYPO3\textbackslash
					CMS\textbackslash
					Core\textbackslash
					Imaging\textbackslash
					Event\textbackslash
					ModifyIconForResourcePropertiesEvent}
					\newline
				\item \texttt{TYPO3\textbackslash
					CMS\textbackslash
					Core\textbackslash
					DataHandling\textbackslash
					Event\textbackslash
					IsTableExcludedFromReferenceIndexEvent}
					\newline
				\item \texttt{TYPO3\textbackslash
					CMS\textbackslash
					Core\textbackslash
					DataHandling\textbackslash
					Event\textbackslash
					AppendLinkHandlerElementsEvent}
					\newline
				\item \texttt{TYPO3\textbackslash
					CMS\textbackslash
					Core\textbackslash
					Configuration\textbackslash
					Event\textbackslash
					AfterTcaCompilationEvent}
					\newline
				\item \texttt{TYPO3\textbackslash
					CMS\textbackslash
					Core\textbackslash
					Database\textbackslash
					Event\textbackslash
					AlterTableDefinitionStatementsEvent}
					\newline
				\item \texttt{TYPO3\textbackslash
					CMS\textbackslash
					Core\textbackslash
					Tree\textbackslash
					Event\textbackslash
					ModifyTreeDataEvent}
					\newline
				\item \texttt{TYPO3\textbackslash
					CMS\textbackslash
					Backend\textbackslash
					Backend\textbackslash
					Event\textbackslash
					SystemInformationToolbarCollectorEvent}
					\newline
			\end{itemize}

	\end{itemize}

\end{frame}

% ------------------------------------------------------------------------------
% Deprecation | 89718 | Legacy PageTSconfig parsing lowlevel API

\begin{frame}[fragile]
	\frametitle{Modifiche per sviluppatori}
	\framesubtitle{Page TSconfig}

	% decrease font size for code listing
	\lstset{basicstyle=\tiny\ttfamily}

	\begin{itemize}
		\item Sono state introdotte due nuovi classi PHP per caricare e analizzare PageTSconfig:
			\begin{itemize}\smaller
				\item \texttt{TYPO3\textbackslash
					CMS\textbackslash
					Core\textbackslash
					Configuration\textbackslash
					Loader\textbackslash
					PageTsConfigLoader}
				\item \texttt{TYPO3\textbackslash
					CMS\textbackslash
					Core\textbackslash
					Configuration\textbackslash
					Parser\textbackslash
					PageTsConfigParser}
			\end{itemize}

		\item Ad esempio:
\begin{lstlisting}
// Fetch all available PageTS of a page/rootline:
$loader = GeneralUtility::makeInstance(PageTsConfigLoader::class);
$tsConfigString = $loader->load($rootLine);

// Parse the string and apply conditions:
$parser = GeneralUtility::makeInstance(
  PageTsConfigParser::class, $typoScriptParser, $hashCache
);

$pagesTSconfig = $parser->parse($tsConfigString, $conditionMatcher);
\end{lstlisting}

	\end{itemize}

\end{frame}

% ------------------------------------------------------------------------------
% Important | 87518 | Use prepared statements for pdo_mysql per default

\begin{frame}[fragile]
	\frametitle{Modifiche per sviluppatori}
	\framesubtitle{Statement predefiniti}

	% decrease font size for code listing
	\lstset{basicstyle=\tiny\ttfamily}

	\begin{itemize}
		\item Il driver \texttt{pdo\_mysql} utilizza statement predefiniti come impostazione predefinita.
		\item In TYPO3 < v10.2 sono utilizzate \textit{statement predefiniti emulati}.
			Questo significa, che tutti i valori restituiti da una query erano stringhe.
		\item Questo comportamento è cambiato e vengono utilizzate le statement predefiniti
			che restituiscono tipi di dati nativi.
		\item Ad esempio: i valori di una colonna definita come intero sono restituiti in PHP come \texttt{int}.
		\item Questa funzione può essere disattivata impostando l'opzione
			\texttt{PDO::ATTR\_EMULATE\_PREPARES} nella connessione al database.

	\end{itemize}

\end{frame}

% ------------------------------------------------------------------------------
% Feature | 86967 | Allow fetching uid of a LazyLoadingProxy without loading the object first

\begin{frame}[fragile]
	\frametitle{Modifiche per sviluppatori}
	\framesubtitle{Lazy Loading Proxy}

	% decrease font size for code listing
	\lstset{basicstyle=\tiny\ttfamily}

	\begin{itemize}
		\item Un metodo \texttt{getUid()} è stato aggiunto alla classe\newline
			\texttt{TYPO3\textbackslash
				CMS\textbackslash
				Extbase\textbackslash
				Persistence\textbackslash
				Generic\textbackslash
				LazyLoadingProxy}.
		\item Questo permette agli sviluppatori di recuperare l'UID dell'oggetto proxy senza recuperare l'oggetto dal database.

	\end{itemize}

\end{frame}

% ------------------------------------------------------------------------------
% Feature | 87380 | Introduce SiteLanguageAwareInterface to denote site language awareness

\begin{frame}[fragile]
	\frametitle{Modifiche per sviluppatori}
	\framesubtitle{Consapevolezza della lingua del sito}

	% decrease font size for code listing
	\lstset{basicstyle=\tiny\ttfamily}

% A SiteLanguageAwareInterface with the methods setSiteLanguage(Entity\SiteLanguage $siteLanguage) and getSiteLanguage() has been introduced.
% The interface can be used to denote a class as aware of the site language.

	\begin{itemize}
		\item Una \texttt{SiteLanguageAwareInterface} è stata introdotta.
		\item L'interfaccia può essere utilizzata per indicare una classe consapevole della lingua del sito.
		\item Gli aspetti di routing, che tengono in considerazione la lingua del sito,
			utilizzano ora \texttt{SiteLanguageAwareInterface}
			oltre a \texttt{SiteLanguageAwareTrait}.
	\end{itemize}

\end{frame}

% ------------------------------------------------------------------------------
% Important | 89645 | Removed systemLog options

\begin{frame}[fragile]
	\frametitle{Modifiche per sviluppatori}
	\framesubtitle{System Log API}

	% decrease font size for code listing
	\lstset{basicstyle=\tiny\ttfamily}

	\begin{itemize}
		\item Le seguenti opzioni sono state rimosse dalla configurazione di base di TYPO3:

			\begin{itemize}\smaller
				\item \texttt{\$GLOBALS['TYPO3\_CONF\_VARS']['SYS']['systemLog']}
				\item \texttt{\$GLOBALS['TYPO3\_CONF\_VARS']['SYS']['systemLogLevel']}
			\end{itemize}\normalsize

		\item Si consiglia agli autori di estensioni di utilizzare le API di Logging e di rimuovere le opzioni di systemLog.
	\end{itemize}

\end{frame}

% ------------------------------------------------------------------------------
% Feature | 89603 | Introduce native pagination for lists

\begin{frame}[fragile]
	\frametitle{Modifiche per sviluppatori}
	\framesubtitle{Paginazione di elenchi nativa}

	% decrease font size for code listing
	\lstset{basicstyle=\tiny\ttfamily}

	\begin{itemize}
		\item E' stato introdotto il supporto nativo per la paginazione di elenchi come array o QueryResults di Extbase.
		\item Il \texttt{PaginatorInterface} definisce un insieme di metodi di base.
		\item La classe \texttt{AbstractPaginator} contiene la logica principale di paginazione.
		\item Questo permette agli sviluppatori di implementare tutti i tipi di paginazione.
\begin{lstlisting}
use TYPO3\CMS\Core\Pagination\ArrayPaginator;

$items = ['apple', 'banana', 'strawberry', 'raspberry', 'ananas'];
$currentPageNumber = 3;
$itemsPerPage = 2;

$paginator = new ArrayPaginator($itemsToBePaginated, $currentPageNumber, $itemsPerPage);
$paginator->getNumberOfPages(); // returns 3
$paginator->getCurrentPageNumber(); // returns 3
$paginator->getKeyOfFirstPaginatedItem(); // returns 5
$paginator->getKeyOfLastPaginatedItem(); // returns 5
\end{lstlisting}

	\end{itemize}

\end{frame}

% ------------------------------------------------------------------------------
% Deprecation | 89579 | ServiceChains require an array for excluded Service keys

\begin{frame}[fragile]
	\frametitle{Modifiche per sviluppatori}
	\framesubtitle{API di servizio}

	\begin{itemize}
		\item Il parametro \texttt{\$excludeServiceKeys} è utilizzato per saltare determinati servizi quando si utilizza una catena.
		\item L'argomento è stato modificato da un elenco separato da virgole in un array in TYPO3 v10.2.
		\item Questa modifica ha effetto sulle API di servizio nei seguenti componenti:

			\begin{itemize}
				\item \texttt{GeneralUtility::makeInstanceService()}
				\item \texttt{ExtensionManagementUtility::findService()}
			\end{itemize}

		\item Il passaggio di un elenco separato da virgole funziona ancora ma è stato contrassegnato come \textbf{deprecato}.

	\end{itemize}

\end{frame}

% ------------------------------------------------------------------------------
