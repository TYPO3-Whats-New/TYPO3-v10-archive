% ------------------------------------------------------------------------------
% TYPO3 Version 10.2 - What's New (Italian Version)
%
% @license	Creative Commons BY-NC-SA 3.0
% @link		http://typo3.org/download/release-notes/whats-new/
% @language	Italian
% ------------------------------------------------------------------------------

\section{Modifiche per integratori}
\begin{frame}[fragile]
	\frametitle{Modifiche per integratori}

	\begin{center}\huge{Capitolo 2:}\end{center}
	\begin{center}\huge{\color{typo3darkgrey}\textbf{Modifiche per integratori}}\end{center}

\end{frame}

% ------------------------------------------------------------------------------
% Feature | 85592 | Add site title configuration to sites module

\begin{frame}[fragile]
	\frametitle{Modifiche per integratori}
	\framesubtitle{Configuratore del sito (1)}

	\begin{itemize}

		\item Il titolo del sito può essere configurato in
			\textbf{SITE CONFIGURATION} $\rightarrow$ \textbf{Sites}.
		\item Questo permette agli integratori di definire titoli del sito differenti per lingua.
		\item Il campo nel record di template è obsoleto ed è stato segnato come \textbf{deprecato}.
		\item Il campo \texttt{sys\_template.sitetitle} (database e TCA) sarà rimosso in TYPO3 v11.
		\item Il titolo del sito è utilizzato sia per il titolo della pagina che per integrazioni
			future per \texttt{schema.org}.
	\end{itemize}

\end{frame}

% ------------------------------------------------------------------------------
% Feature | 89398 | Support for environment variables in imports in site configurations

\begin{frame}[fragile]
	\frametitle{Modifiche per integratori}
	\framesubtitle{Configuratore del sito (2)}

	% decrease font size for code listing
	\lstset{basicstyle=\tiny\ttfamily}

	\begin{itemize}

		\item E' possibile utilizzare le variabili di ambiente nelle importazioni dei file YAML per la configurazione del sito:
\begin{lstlisting}
imports:
  -
    resource: 'Env_%env("foo")%.yaml'
\end{lstlisting}

	\end{itemize}

\end{frame}

% ------------------------------------------------------------------------------
% Feature | 88102 | Frontend Login Form Via Fluid And Extbase

\begin{frame}[fragile]
	\frametitle{Modifiche per integratori}
	\framesubtitle{Frontend Login (1)}

	\begin{itemize}

		\item TYPO3 v10.2 include ora una versione Extbase per le funzionalità di login nel frontend.
		\item Questa soluzione presenta alcuni vantaggi:

			\begin{itemize}
				\item Modifica più facile dei template.
				\item Invio di email per recupero password basate su HTML.
				\item Regolazione e modifica dei validatori per imporre restrizioni sulla password.
			\end{itemize}

		\item Il nuovo plugin Extbase è disponibile per le nuove installazioni.
		\item Le istanze esistenti di TYPO3 continueranno a utilizzare i vecchi template.
		\item Gli integratori possono alternare tra il "vecchio" e il "nuovo" plugin usando un interruttore funzione.

	\end{itemize}

\end{frame}

% ------------------------------------------------------------------------------
% Feature | 88110 | Felogin extbase password recovery

\begin{frame}[fragile]
	\frametitle{Modifiche per integratori}
	\framesubtitle{Frontend Login (2)}

	\begin{itemize}

		\item E' stato aggiunto un modulo di recupero password come parte del plugin Extbase.
		\item Gli utenti possono richiedere il recupero della password e riceveranno un'email con un link che li indirizza al modulo.
		\item Regole predefinite di convalida password:

			\begin{itemize}
				\item \texttt{NotEmptyValidator} - le password non possono essere vuote.
				\item \texttt{StringLengthValidator} - le password devono avere una lunghezza minima.
			\end{itemize}

	\end{itemize}

\end{frame}

% ------------------------------------------------------------------------------
% Feature | 88110 | Felogin extbase password recovery

\begin{frame}[fragile]
	\frametitle{Modifiche per integratori}
	\framesubtitle{Frontend Login (3)}

	% decrease font size for code listing
	\lstset{basicstyle=\tiny\ttfamily}

	\begin{itemize}
		\item Queste regole di validazione possono essere personalizzate.
		\item Ad esempio:
\begin{lstlisting}
plugin.tx_felogin_login {
  settings {
    passwordValidators {
      10 = TYPO3\CMS\Extbase\Validation\Validator\AlphanumericValidator
      20 {
        className = TYPO3\CMS\Extbase\Validation\Validator\StringLengthValidator
        options {
          minimum = 12
          maximum = 32
        }
      }
      30 = \Vendor\MyExtension\Validation\Validator\MyCustomPasswordPolicyValidator
    }
  }
}
\end{lstlisting}

	\end{itemize}

\end{frame}

% ------------------------------------------------------------------------------
% Feature | 89526 | FeatureFlag: betaTranslationServer

\begin{frame}[fragile]
	\frametitle{Modifiche per integratori}
	\framesubtitle{Piattaforma di gestione delle traduzioni}

	\begin{itemize}

		\item \href{https://crowdin.com/}{crowdin} mira a sostituire la soluzione esistente
			\href{https://translation.typo3.org/}{Pootle}
			come piattaforma di localizzazione / traduzione.

		\item In TYPO3 v10.2 è stata aggiunta una funzione che permette di abilitare crowdin.com
			come sorgente delle traduzioni.

		\item Nota: Questo è in \textbf{stato beta}.

		\item Leggi di più al riguardo dell'
			\href{https://typo3.org/community/teams/typo3-development/initiatives/localization-with-crowdin/}{iniziativa}.

	\end{itemize}

	\begin{figure}
		\includegraphics[width=0.50\linewidth]{ChangesForIntegrators/crowdin-logo.png}
	\end{figure}

\end{frame}

% ------------------------------------------------------------------------------
% Feature | 89171 | Added possibility to have multiple sitemaps

\begin{frame}[fragile]
	\frametitle{Modifiche per integratori}
	\framesubtitle{Sitemap multiple}

	% decrease font size for code listing
	\lstset{basicstyle=\tiny\ttfamily}

	\begin{itemize}

		\item E' possibile configurare sitemap multiple.
		\item Sintassi:
\begin{lstlisting}
plugin.tx_seo {
  config {
    <sitemapType> {
      sitemaps {
        <unique key> {
          provider = TYPO3\CMS\Seo\XmlSitemap\RecordsXmlSitemapDataProvider
          config {
            ...
          }
        }
      }
    }
  }
}
\end{lstlisting}

	\end{itemize}

\end{frame}

% ------------------------------------------------------------------------------
% Feature | 86759 | Support nomodule attribute for JavaScript includes

\begin{frame}[fragile]
	\frametitle{Modifiche per integratori}
	\framesubtitle{attributo HTML5 \texttt{nomodule}}

	% decrease font size for code listing
	\lstset{basicstyle=\tiny\ttfamily}

	\begin{itemize}
		\item L'attributo HTML5 \texttt{nomodule} è supportato quando si includono file JavaScript in TypoScript.
\begin{lstlisting}
page.includeJSFooter.file = path/to/classic-file.js
page.includeJSFooter.file.nomodule = 1
\end{lstlisting}

		\item Questo attributo impedisce l'esecuzione di uno script nei browser che supportano gli script del modulo.

		\item Puoi leggere di più al riguardo delle
			\href{https://html.spec.whatwg.org/multipage/scripting.html#attr-script-nomodule}{specifiche}
			dello standard e sul concetto di
			\href{https://hacks.mozilla.org/2015/08/es6-in-depth-modules/}{modulo}.

	\end{itemize}

% <script type="module" src="path/to/file.js"></script>
% <script nomodule src="path/to/file/classic-file.js"></script>

\end{frame}

% ------------------------------------------------------------------------------
% Feature | 87798 | Provide a way to sort form lists in ext:form

\begin{frame}[fragile]
	\frametitle{Modifiche per integratori}
	\framesubtitle{Ordine dei moduli}

	% decrease font size for code listing
	\lstset{basicstyle=\tiny\ttfamily}

	\begin{itemize}
		\item I moduli possono essere ordinati in ordine crescente o decrescente.
		\item Sono state introdotte due nuove impostazioni: \texttt{sortByKeys} e \texttt{sortAscending}.
		\item I moduli sono inizialmente ordinati per nome e UID del file (crescente).
		\item Per modificare l'ordinamento, è necessario aggiungere la seguente impostazione nel file di configurazione YAML:
\begin{lstlisting}
TYPO3:
  CMS:
    Form:
      persistenceManager:
        sortByKeys: ['name', 'fileUid']
        sortAscending: true
\end{lstlisting}

	\end{itemize}

\end{frame}

% ------------------------------------------------------------------------------
% Feature | 86918 | Add additional configuration for external link types in Linkvalidator

\begin{frame}[fragile]
	\frametitle{Modifiche per integratori}
	\framesubtitle{Validatore link (1)}

	% decrease font size for code listing
	\lstset{basicstyle=\tiny\ttfamily}

	\begin{itemize}
		\item Il validatore di link supporta una configurazione aggiuntiva per link esterni.
		\item Devono essere forniti i valori per \texttt{httpAgentUrl} e \texttt{httpAgentEmail}.
		\item \texttt{Headers}, \texttt{method} e \texttt{range} sono impostazioni avanzate.
\begin{lstlisting}
mod.linkvalidator {
  linktypesConfig {
    external {
      httpAgentName = ...
      httpAgentUrl = ...
      httpAgentEmail = ...
      headers {
      }
      method = HEAD
      range = 0-4048
    }
  }
}
\end{lstlisting}

	\end{itemize}

\end{frame}

% ------------------------------------------------------------------------------
% Feature | 84990 | Add event for checking external links in RTE

\begin{frame}[fragile]
	\frametitle{Modifiche per integratori}
	\framesubtitle{Validatore link (2)}

	\begin{itemize}
		\item Il validatore di link segnala i link \textbf{esterni} interrotti anche nel RTE.
		\item Questa funzione era disponibile solo per i link interni.
		\item Si consiglia di eseguire la validazione dei link come task dello scheduler per individuare i link interrotti.
	\end{itemize}

\end{frame}

% ------------------------------------------------------------------------------
